%%%%%%%%%%%%%%%%%%%%%%%%%%%%%%%%%%%%%%%%%%%%%%%%%%%%%%%%%%%%%%%%%%%%%%%%%%%%%%%
\chapter{Perception auditive consciente et ses corrélats neuronaux}
\label{chapitre1}
\noindent \hrulefill \\
%%%%%%%%%%%%%%%%%%%%%%%%%%%%%%%%%%%%%%%%%%%%%%%%%%%%%%%%%%%%%%%%%%%%%%%%%%%%%%%

Nous présentons dans ce premier chapitre les mécanismes et processus à l'œuvre lors de la perception auditive consciente et l'analyse de la scène auditive par les êtres humains. 
Ces processus peuvent donner lieu à des phénomènes de masquage auditif, d'intérêt particulier en neurosciences cognitives puisqu'ils permettent de comparer les états perceptifs d'objets masqués vis-à-vis d'objets non-masqués. 
En permettant une telle comparaison, ils fournissent un moyen d'étudier les corrélats neuronaux de la perception auditive consciente, et notamment de son caractère multistable. 

%%%%%%%%%%%%%%%%%%%%%%%%%%%%%%%%%%%%%%%%%%%%%%%%%%%%%%%%%%%%%%%%%%%%%%%%%%%%%%%
\section{Système auditif et traitement de l'information auditive}
\label{systemeauditif}
%%%%%%%%%%%%%%%%%%%%%%%%%%%%%%%%%%%%%%%%%%%%%%%%%%%%%%%%%%%%%%%%%%%%%%%%%%%%%%%

Les informations en provenance du monde qui nous entoure sont captées par de multiples systèmes sensoriels et perceptifs (visuel, auditif, gustatif, olfactif et somatosensoriel). 
Le système auditif humain est capable d'isoler et d'identifer de façon robuste les indices acoustiques conduisant à la reconnaissance d'un très grand nombre de sources sonores. 
Le traitement de l'information auditive passe par une série de processus et mécanismes débutant au niveau du système auditif périphérique jusqu'à atteindre le système auditif central.

%%%%%%%%%%%%%%%%%%%%%%%%%%%%%%%%%%%%%%%%%%%%%%%%%%%%%%%%%%%%%%%%%%%%%%%%%%%%%%%
\subsection{Système auditif périphérique}
\label{systemeauditifperif}
%%%%%%%%%%%%%%%%%%%%%%%%%%%%%%%%%%%%%%%%%%%%%%%%%%%%%%%%%%%%%%%%%%%%%%%%%%%%%%%

Le terme «son» fait référence à l’ensemble des vibrations, ou ondes sonores, auxquelles l’oreille humaine est sensible. 
Un son est entendu lorsque les vibrations de l'air ambiant atteignent le tympan et le mettent en mouvement dans des conditions d'amplitude et de fréquence telles que cette stimulation mécanique induit une sensation auditive. 
Les sons se caractérisent par leur intensité, leur fréquence et leur durée. 
L'intensité détermine le caractère faible ou fort du son et l'amplitude des vibrations acoustiques est mesurée en décibel (dB), avec un niveau de pression acoustique de référence de 20 $\mu$Pa lorsque cette amplitude est exprimée en dB SPL (dB Sound Pressure Level). 
La fréquence d'un son représente usuellement le nombre de vibrations acoustiques par seconde distinguant ainsi les sons graves des sons aigus et se mesure en Hertz (Hz). 
Elle définit la hauteur du son : plus la fréquence de la vibration est faible, plus le son est grave, et inversement, plus la fréquence de vibration est élevée, plus le son est aigu. 
Pour recueillir puis transformer ces sons, l’homme dispose d’un organe sensoriel évolué, constitué de trois parties aux rôles distincts : l'oreille externe, l'oreille moyenne et l'oreille interne (Figure \ref{fig:chap2systemeauditifperif}).

\begin{figure*}[!t]
\center
\includegraphics[width=\columnwidth]{illustrations/utilisé/système_auditif_périphérique.png}
\caption[Système Auditif Périphérique]{Système Auditif Périphérique. L’homme dispose d’un organe sensoriel évolué, constitué de trois parties aux rôles distincts : l'oreille externe (OE), l'oreille moyenne (OM) et l'oreille interne (OI). Adapté de Kristen Wienandt Marzeion - Medical Illustration Sourcebook.}
\label{fig:chap2systemeauditifperif}
\end{figure*}

Premièrement, l’oreille externe est constituée du pavillon et du conduit auditif externe. 
Elle est limitée par une membrane souple, la membrane tympanique (ou tympan), qui la sépare de l'oreille moyenne. 
De par sa configuration anatomique, l'oreille externe se comporte comme une antenne acoustique : le pavillon, associé au volume crânien, diffracte les ondes sonores et les canalise vers le conduit auditif externe \citep{isnard2016efficacite}. 
Cet ensemble assure une protection mécanique du tympan et joue un rôle de résonateur qui modifie sélectivement l'amplitude et la phase des ondes acoustiques \citep{albouy2013behavioral}. 
Deuxièmement, l’oreille moyenne est la cavité aérienne située entre l’oreille externe et l’oreille interne. 
Cette caisse tympanique est fermée sur l'oreille externe par la membrane du tympan, et sur l'oreille interne par l'intermédiaire des fenêtres ronde et ovale. 
Elle communique avec le milieu extérieur via la trompe d'Eustache pour assurer l'équilibre de pression de part et d'autre du tympan, nécessaire au fonctionnement optimal de la chaîne tympano-ossiculaire \citep{hasselmann2017codage}. 
L'oreille moyenne joue un double rôle de transmission et de protection. 
Lorsque le tympan vibre sous l'action des variations de pression transportées par l'onde sonore, il met en mouvement les trois os de la chaîne (le marteau, l’enclume et l’étrier). 
Ce complexe tympano-ossiculaire se comporte comme un levier, qui transforme les oscillations du tympan en un mouvement de piston de la platine de l'étrier sur la fenêtre ovale \citep{lorenzi2016audition}. 
Troisièmement, l'oreille interne renferme deux organes assurant chacun une fonction sensorielle : l’audition par la cochlée et l’équilibration par le vestibule. 

\begin{figure*}[!t]
\center
\includegraphics[width=\columnwidth]{illustrations/utilisé/cochlee.jpg}
\caption[Organe cochléaire]{Organe cochléaire. La cochlée est l'organe de l'audition assurant la conversion du signal mécanique vibratoire provenant de l'oreille moyenne en un signal électrochimique qui est ensuite véhiculé par le nerf auditif jusqu'au cortex. Adapté de Kristen Wienandt Marzeion - Medical Illustration Sourcebook.}
\label{fig:chap2cochlee}
\end{figure*}

La cochlée, organe de l'audition, assure la conversion du signal vibratoire provenant de l'oreille moyenne en un signal bioélectrique qui est ensuite véhiculé par le nerf auditif jusqu'au cortex (Figure \ref{fig:chap2cochlee}). 
La cochlée est recouverte d'une paroi osseuse présentant deux ouvertures vers l'oreille moyenne : la fenêtre ovale et la fenêtre ronde. 
La fenêtre ovale est fermée par une membrane en contact permanent avec la plaque de l'étrier. 
En conséquence, la fenêtre ovale reçoit les vibrations mécaniques des osselets, puis les transmet aux liquides de l'oreille interne. 
La fenêtre ronde joue le rôle de membrane d'expansion pour le liquide de l'oreille interne. 
De par son élasticité, elle absorbe les variations de volume induites par les mouvements de la fenêtre ovale et facilite la propagation de l’onde le long de la cochlée \citep{lorenzi2016audition}. 
Lorsque les liquides de l'oreille interne sont mis en mouvement par les vibrations de l'étrier, l'onde formée se propage de la fenêtre ovale et croît le long de la membrane basilaire pour atteindre un pic maximal à une distance donnée \citep{isnard2016efficacite}. 
Les liquides étant par nature incompressibles, la membrane de la fenêtre ronde se déforme, permettant ainsi aux vibrations d’évoluer sans contraintes \citep{hasselmann2017codage}. 
L’emplacement du pic de résonance dans la cochlée dépend de la fréquence de la vibration. 
Les sons de basses fréquences font entrer en résonance l'apex de la cochlée (endroit où la membrane basilaire est large et fine). 
Les sons de hautes fréquences font entrer en résonance la base de la cochlée (endroit où la membrane basilaire est étroite et épaisse). 
Cette structure définit l'organisation tonotopique de l'organe cochléaire, en référence à la localisation du pic de résonnance dans la cochlée en fonction de la fréquence de vibration \citep{moore2012introduction}. 

\begin{figure*}[!t]
\center
\includegraphics[width=\columnwidth]{illustrations/utilisé/organe_de_corti.jpg}
\caption[Cellules cilliées de l'organe cochléaire]{Organe de Corti et cellules cilliées. L’organe de Corti est l’organe neurosensoriel de la cochlée et comporte un ensemble de cellules sensorielles spécifiques (les cellules ciliées). Adapté de Kristen Wienandt Marzeion - Medical Illustration Sourcebook.}
\label{fig:chap2organedecorti}
\end{figure*}

Ancré sur la membrane basilaire, l’organe de Corti représente l’organe neurosensoriel de la cochlée (Figure \ref{fig:chap2organedecorti}). 
L'organe comporte un ensemble de cellules sensorielles spécifiques (les cellules ciliées), des fibres nerveuses qui y sont connectées, et des structures de support, toutes indispensables au maintien de l’intégrité du système \citep{hasselmann2017codage, lorenzi2016audition}. 
Les cellules ciliées, dénommées ainsi par la présence de stéréocils ancrés sur leur pôle apical sont le siège de la transduction mécano-électrique. 
Cette transduction mécano-électrique correspond à la transformation de la vibration de la membrane en message nerveux. 
Les cellules ciliées externes sont pourvues de propriétés contractiles améliorant la sensibilité et la sélectivité fréquentielle de la cochlée.
Elles sont au nombre d'environ $12500$ chez l’homme et sont disposées en trois rangées sur la partie externe du tunnel de Corti. 
Bien qu’elles bénéficient de propriétés mécano-transductives, elles ne semblent pas convoyer d’informations sonores. 
En revanche, elles jouent un rôle essentiel dans un mécanisme actif d'amplification locale de vibration de la membrane basilaire. 
Sous l’action des vibrations, la membrane tectoriale oscille, entraînant les stéréocils des cellules ciliées qui y sont ancrés. 
Les cellules ciliées internes sont, quant-à-elles, responsables de la transduction de l'information sonore. 
Elles sont au nombre d'environ $3500$ chez l’homme et sont disposées en une rangée sur la partie interne du tunnel de Corti. 
Véritables cellules sensorielles, elles assurent au moyen de la transduction, la transformation des oscillations mécaniques en signal bioélectrique \citep{albouy2013behavioral}. 

Si deux sons de fréquences différentes sont joués en même temps, l'auditeur peut souvent entendre deux sons séparés plutôt qu'un son combiné. 
Lorsque les signaux sont perçus comme un son combiné, cela signifie qu'ils résident dans une même bande critique.
Cet effet serait dû au filtrage cochléaire \citep{moore1995hearing, moore2012auditory, moore2012introduction}. 
Le concept de bande critique auditive, introduit par Harvey Fletcher en $1933$ \citep{fletcher1933loudness} et affiné en $1940$ \citep{fletcher1940auditory}, décrit la largeur de bande de fréquences du «filtre auditif» créée par la cochlée. 
La bande critique est la bande de fréquences auditives dans laquelle deux sons vont interférer l'un avec l'autre, sous forme de masquage et de modulation d'amplitude (Figure \ref{fig:chap2formefiltresauditifs}). 
L'organe cochléaire vibre en présence d’un son et chaque point est sensible à une certaine fréquence pour laquelle l’amplitude de la vibration est maximale. 
Pour une fréquence donnée, la membrane basilaire est excitée au point précis, entraînant une vibration de plus faible amplitude aux fréquences voisines. 
Si la membrane basilaire est excitée à une certaine fréquence (par exemple par un masqueur), une nouvelle excitation à une fréquence proche (par exemple un son cible) ne sera alors perçue que si elle surmonte la vibration induite par la première. 
On admet communément que ce chevauchement entre excitations fournit la base des phénomènes de masquage auditif \citep{moore2012auditory, moore2012introduction}. 

\begin{figure*}[!t]
\center
\includegraphics[width=0.64\columnwidth]{illustrations/utilisé/filtres_auditifs_Moore_1995.jpg}
\includegraphics[width=0.35\columnwidth]{illustrations/utilisé/forme_filtres_auditifs_Moore_1995.jpg}
\includegraphics[width=0.64\columnwidth]{illustrations/utilisé/patterns_masquage_Moore_1995.jpg}
\includegraphics[width=0.35\columnwidth]{illustrations/utilisé/patterns_excitation_Moore_1995.jpg}
\caption[Formes et patterns d'excitations des filtres auditifs]{(Haut Gauche) Illustration schématique de la technique utilisée par \cite{patterson1976auditory} pour déterminer la forme du filtre auditif. Le seuil d'un signal sinusoïdal est mesurée comme une fonction de la largeur d'un banc spectral dans le masqueur bruit. La quantité de bruit passant à travers le filtre auditif centré à la fréquence du signal est proportionnel aux aires ombragées. (Haut Droite) Forme du filtre auditif centré à 1 kHz, représenté pour des niveaux de sons en entrée allant de $20$ à $90$~dB SPL. Le niveau à la sortie du filtre est représenté comme une fonction de la fréquence. Du coté des faibles fréquences, le filtre devient progressivement moins raide avec l'augmentation du niveau d'intensité. Du coté des hautes fréquences, la raideur du filtre augmente avec les niveaux d'intensité croissant. Enfin, à des niveaux d'intensité sonores modéré, on remarque une certaine symétrie sur l'échelle fréquentielle linéaire utilisée. (Bas Gauche) Patterns de masquage pour un bruit masqueur centré à $410$~Hz. Chaque courbe montre l'élèvation du seuil d'un signal sinusoïdal comme une fonction de la fréquence du signal. Le niveau d'intensité du bruit global pour chaque courbe est indiqué. (Bas Droite) Illustration représentant comment le pattern d'excitation d'une sinusoïde de 1 kHz peut être dérivée en calculant les sorties des filtres auditifs comme une fonction de leur fréquence centrale. Le panel du haut montre cinq filtres auditifs, centrés à différentes fréquences tandis que le panel du bas montre le pattern d'excitation calculé. Adapté de \cite{glasberg1990derivation} et \cite{moore1995frequency}.}
\label{fig:chap2formefiltresauditifs}
\end{figure*}

Conduisant à l'organisation tonotopique de la sensibilité aux gammes de fréquences le long de la membrane, cette résolution fréquentielle a été modélisée comme un ensemble de filtres passe-bande juxtaposés appelés «filtres auditifs». 
Ainsi, les filtres auditifs sont associés à des points le long de la membrane basilaire et déterminent la sélectivité fréquentielle cochléaire, et donc la discrimination de l'auditeur entre les différents sons. 
Ils sont non-linéaires, dépendent de l'intensité et leur largeur de bande diminue de la base au sommet de la cochlée lorsque l'accord sur la membrane basilaire passe des hautes aux basses fréquences. 
La bande critique auditive a servi de base à l’élaboration de nombreux modèles auditifs. 
La modélisation fonctionnelle du concept de bande critique se fait usuellement à l’aide d’un banc de filtres visant à reproduire l’excitation mécanique de la membrane basilaire \citep{gnansia2005modele}. 
Le signal de chaque canal du banc de filtres correspond alors à la déformation de la membrane basilaire à un endroit donné. 
À partir de l’observation des phénomènes de masquage, il est possible de déterminer la forme de ces filtres, leur largeur, et leur disposition dans le banc (Figure \ref{fig:chap2formefiltresauditifs}). 
La largeur de bande rectangulaire équivalente (ERB) est une notion associée aux filtres auditifs et montre la relation entre filtre auditif, fréquence et largeur de bande critique. 
L'ERB transmet la même quantité d'énergie que le filtre auditif auquel il correspond et montre comment il change en fonction de la fréquence d'entrée. 
Selon \cite{glasberg1990derivation}, l'ERB peut-être approximée, à de faibles niveaux sonores, par l'équation suivante : $ERB(f) = 24,7 \times (4,37 f / 1000 + 1)$, où l'ERB est en Hz et f est la fréquence centrale en Hz. 
Chaque ERB équivaudrait environ à 0,9 mm sur la membrane basilaire \citep{glasberg1990derivation}. 
L'ERB peut ainsi être convertie en une échelle qui se rapporte à la largeur de bande et indique la position du filtre auditif le long de la membrane basilaire. 
Par exemple, une ERB de 3,36 correspond à une fréquence à l'extrémité apicale de la membrane basilaire alors qu'une ERB de 38,9 correspond à la base et une valeur de 19,5 se situe à mi-chemin entre les deux. 

Ainsi, un son complexe comprend différentes composantes fréquentielles, chacune provoquant un pic dans le modèle de vibration à un endroit spécifique de la membrane basilaire dans la cochlée. 
Les composantes fréquentielles sont ensuite codées indépendamment sur le nerf auditif qui transmet les informations sonores au cerveau. 
Ce codage individuel ne se produit que si les composantes fréquentielles sont suffisamment différentes en fréquence. 
Autrement, elles se trouvent dans la même bande critique et sont codées au même endroit et sont perçues comme un seul son au lieu de deux.
De cette manière, le système auditif périphérique humain transmet le signal acoustique à l'oreille interne via une transformation complexe, avant que le signal résultant ne soit encodé en potentiels d'action dans le nerf auditif. 
Le signal électrique induit est ensuite transmis par le nerf auditif au cerveau via les voies auditives ascendantes, en passant par plusieurs relais auditifs successifs appartenant au système auditif central.

%%%%%%%%%%%%%%%%%%%%%%%%%%%%%%%%%%%%%%%%%%%%%%%%%%%%%%%%%%%%%%%%%%%%%%%%%%%%%%%
\subsection{Système auditif central}
\label{systemeauditifcentral}
%%%%%%%%%%%%%%%%%%%%%%%%%%%%%%%%%%%%%%%%%%%%%%%%%%%%%%%%%%%%%%%%%%%%%%%%%%%%%%%

Le système auditif central présente une organisation permettant le transfert, le codage et l’analyse des potentiels d’action d’origine périphérique. 
La complexité du système auditif central s’appuie sur une richesse cellulaire croissante, de la périphérie vers les centres supérieurs (Figure \ref{fig:chap2voieauditive}). 
L'audition, comme toute autre modalité sensorielle, possède une voie et des centres primaires, totalement dédiés à cette fonction, et des voies non-primaires où convergent l'ensemble des modalités. 
Les messages auditifs sont transmis au cerveau par ces deux types de voies (Figure \ref{fig:chap2voieauditive2}). 
La voie auditive primaire est une voie courte possèdant trois ou quatre relais, rapide du fait de ses fibres myélinisées, transportant exclusivement les messages de la cochlée et aboutissant au cortex auditif primaire. 
La voie non-primaire regroupe les différents messages sensoriels envoyés simultanément au cerveau et permet de sélectionner le type d'information à traiter en priorité. 
Le premier relais de la voie auditive primaire se produit dans les noyaux cochléaires du tronc cérébral, qui reçoivent les axones du nerf auditif. 
Le deuxième relais majeur dans le tronc cérébral se situe dans le complexe olivaire supérieur. 
En quittant ce relais, un troisième neurone transporte le message jusqu'au niveau du colliculus inférieur du mésencéphale. 
Un dernier relais, avant le cortex, se produit dans le corps géniculé médian du thalamus. 
Le dernier neurone de la voie auditive primaire relie le thalamus au cortex auditif.
À ce stade, le message, décodé lors de son passage dans les neurones précédents de la voie, est reconnu, mémorisé et peut être intégré dans une réponse volontaire. 

\begin{figure*}[!t]
\center
\includegraphics[width=\columnwidth]{illustrations/utilisé/auditory_pathway_4.png}
\caption[Voies auditives et cheminement de l'information auditive]{Voies auditives et cheminement de l'information auditive à travers la structure hiérarchique de traitement de l'information auditive. Les flèches jaunes indiquent le trajet des informations sonores. Adapté de Kristen Wienandt Marzeion - Medical Illustration Sourcebook.}
\label{fig:chap2voieauditive}
\end{figure*}

Le passage par ces relais auditifs affine la sélectivité fréquentielle grâce à des neurones accordés à différentes fréquences \citep{isnard2016efficacite}. 
Des mécanismes d'inhibition progressifs, issus du système auditif central, sont responsables de cet affinement de l'accord en fréquence \citep{zhang1997corticofugal}. 
Par ailleurs, les relations de voisinage fréquentiel sont préservées à chaque étape du traitement auditif jusque dans la représentation tonotopique du cortex auditif primaire. 
Cette préservation est analogue aux cartes topographiques des autres systèmes sensoriels comme la rétinotopie dans le cortex visuel ou la somatotopie dans le cortex somatosensoriel \citep{king2009unraveling, lyon1996auditory, rauschecker1998cortical}. 
Cependant, la structure précise du système auditif reste incertaine car elle présente davantage de relais avant d'atteindre le cortex, et donc de pré-traitements \citep{king2009unraveling}. 
Ainsi, certaines caractéristiques auditives sont décodées tôt dans les voies auditives, comme l'intensité, la fréquence, ou les indices de localisation. 
Au contraire, c'est seulement au niveau du cortex auditif que se situent les fonctions plus complexes, en particulier celle de la discrimination auditive \citep{bathellier2012discrete}. 
Le cortex auditif constitue le dernier relais des voies auditives et est considéré comme le centre intégrateur principal des informations auditives. 
Comme d’autres structures sensorielles, le cortex auditif est organisé de manière hiérarchique. 
Structure complexe et lieu d’interactions entre informations auditives, il intègre également des informations dont la source est cognitive ou provenant d'autres modalités sensorielles \citep{lorenzi2016audition}. 

C'est à la surface du lobe temporal, au niveau du gyrus de Heschl (Figure \ref{fig:chap2gyrusheschl}) que le cortex auditif est divisé en trois grandes régions. 
La première, le cœur, est constituée de l’aire primaire et de deux régions plus rostrales ou antérieures. 
La deuxième, composée des aires secondaires, qui se présentent sous la forme d’une première ceinture entourant le cœur ; cette ceinture est scindée en plusieurs sous-régions selon leur position (latérale, médiale, rostrale et caudale). 
La troisième, composée des aires associatives, lesquelles viennent entourer les aires secondaires. 
L’organisation histologique du cortex auditif est laminaire, avec 6 couches parallèles, traditionnellement notées de I à VI, de la surface vers la profondeur. 
L’ensemble de ces aires sont inter-connectées entre elles et interagissent fortement avec les autres régions impliquées dans les grandes fonctions cognitives telles que l’attention, la planification, la mémoire, etc. 
Il existe également des liaisons inter-hémisphériques, via le corps calleux et la commissure blanche antérieure, offrant des possibilités de mécanismes d’intégration supplémentaires. 

\begin{figure*}[!t]
\center
\includegraphics[width=0.8\columnwidth]{illustrations/utilisé/voieauditive2.jpg}
\caption[Intégration comportementale de l'information auditive]{Voies auditives et intégration comportementale de l'information auditive à travers la structure hiérarchique de traitement de l'information auditive. Adapté du site \textit{www.cochlea.eu}.}
\label{fig:chap2voieauditive2}
\end{figure*}

Le cortex auditif présente ainsi une organisation tonotopique complexe. 
Les enregistrements par microélectrodes ou par imagerie cérébrale fonctionnelle ont permis de spécifier le comportement du cortex auditif en réponse à des stimuli de nature variée. 
L’aire auditive primaire, procède au décodage des aspects temporels, fréquentiels et énergétiques de stimuli simples \citep{recanzone1993plasticity, riecke2007hearing, steinschneider2008spectrotemporal}. 
Les aires secondaires ont un rôle dans l'intégration des informations primaires \citep{rauschecker1995processing}. 
Elles permettent également d'initier l’identification des informations, la planification des réponses appropriées, le stockage des informations, ou encore l’attention à porter à une information pertinente \citep{riecke2009hearing}. 
Elles projettent aussi vers des centres impliqués dans les autres modalités sensorielles, la planification, le raisonnement, l’attention, la réponse motrice, ou encore la mémoire (cortex préfrontal et hippocampe). 
En retour, ces différentes structures sont en mesure d’influer sur les aires secondaires et primaires, afin de moduler l’activité du système sensoriel sous-jacent. 

Au cours des deux dernières décennies, de nombreuses études ont suggéré que les voies de traitement du cortex auditif sont organisées de façon duale \citep{ahveninen2006task, alain2001and, albouy2013behavioral, altmann2007processing, 
anourova2001evidence, belin2000voice, bidet2005dynamics, bushara1999modality, clarke2005and, garell2013functional, rauschecker1998cortical, rauschecker2009maps, recanzone2011perception, tardif2008interactions, zatorre2004sensitivity}. 
En utilisant des approches anatomiques et fonctionnelles, ces voies de traitement ont été définies comme les flux auditifs dorsaux et ventraux. 
Le flux dorsal implique des connexions entre les régions temporales postérieures et le cortex préfrontal via des relais dans les régions pariétales. 
Au contraire, le flux ventral implique des projections des zones sensorielles auditives primaires vers les zones corticales temporales et frontales antérieures. 
Comme dans le système visuel, il est supposé que la voie pariéto-dorsale sous-tend le traitement spatial auditif («où») tandis que la voie temporo-ventrale est impliquée dans l'identification de motifs ou d'objets complexes («quoi») \citep{kaas1999and, rauschecker2009maps}. 

\begin{figure*}[!t]
\center
\includegraphics[width=0.8\columnwidth]{illustrations/utilisé/gyrus_heschl.png}
\caption[Gyrus de Heschl]{Gyrus de Heschl situé au niveau du cortex auditif primaire et localisé dans le lobe temporal. Adapté de \cite{friederici2011brain}.}
\label{fig:chap2gyrusheschl}
\end{figure*}

Des études récentes vont au-delà de cette vision et supposent que des processus plus complexes peuvent être soutenus par ces deux voies. 
En effet, les interactions fronto-temporo-pariétales sont recrutées par de multiples fonctions cognitives telles que l'attention ou la mémoire \citep{naghavi2005common}. 
On peut observer des faisceaux de projection séparés antérieurs (ventral) et postérieurs (dorsal) au moyen de méthodes de traçage anatomique des voies chez le primate \citep{kaas1999and, rauschecker1997serial, romanski1999auditory}. 
De tels faisceaux ont aussi pu être observés en utilisant l'imagerie par tenseur de diffusion directement chez l'homme \citep{catani2008diffusion, frey2008dissociating}. 
Ces études ont suggéré qu'il pourrait y avoir des voies ventrales et dorsales de la substance blanche reliant les zones auditives aux cortex frontal et pariétal \citep{catani2008diffusion, croxson2005quantitative, frey2008dissociating}. 
Le rôle spécifique de chaque voie est encore flou et controversé \citep{albouy2013behavioral, rauschecker2009maps}. 
Il est cependant reconnu que le gyrus temporal supérieur, les régions frontales inférieures et le cortex pariétal sont impliqués dans le traitement de la parole, de la musique et du langage. 
Faire abstraction de ces évidences et attribuer une fonction exclusivement spatiale au flux auditif dorsal et une fonction d'identification d'objets au flux ventral ne serait pas approprié \citep{rauschecker2009maps}. 

Ainsi, de son passage du système auditif périphérique jusqu'au système auditif central, l'information sonore d'un stimulus auditif en provenance de l'environnement suit un tracé qui tend fondamentalement à rendre la perception auditive finement dépendante de l'intégrité des voies de traitement hiérarchique de l'information auditive. 
En conséquence, cette structure de traitement de l'information met en jeu un ensemble de mécanismes et de processus intégratifs, lesquels permettent d'aboutir in fine à la capacité des êtres humains à analyser une scène auditive complexe. 

%%%%%%%%%%%%%%%%%%%%%%%%%%%%%%%%%%%%%%%%%%%%%%%%%%%%%%%%%%%%%%%%%%%%%%%%%%%%%%%
\section{Analyse de la scène auditive et masquage auditif}
\label{analysesceneauditiveetmasquage}
%%%%%%%%%%%%%%%%%%%%%%%%%%%%%%%%%%%%%%%%%%%%%%%%%%%%%%%%%%%%%%%%%%%%%%%%%%%%%%%

%%%%%%%%%%%%%%%%%%%%%%%%%%%%%%%%%%%%%%%%%%%%%%%%%%%%%%%%%%%%%%%%%%%%%%%%%%%%%%%
\subsection{Analyse de la scène auditive}
\label{analysesceneauditive}
%%%%%%%%%%%%%%%%%%%%%%%%%%%%%%%%%%%%%%%%%%%%%%%%%%%%%%%%%%%%%%%%%%%%%%%%%%%%%%%

\begin{figure*}[!t]
\center
\includegraphics[width=0.66\columnwidth]{illustrations/utilisé/cocktail_party_problem.jpg}
\caption[Situation de Cocktail Party]{Dessin d'une situation typique du problème de «Cocktail party». Dans une telle situation, une personne doit sélectivement «écouter» une source sonore distincte parmi un ensemble de sources sonores actives. La capacité à maintenir une conversation avec une personne lors d'une situation de cocktail party a souvent été utilisée pour illustrer la ségrégation de flux auditifs. Adapté de \cite{alain2000selectively}.}
\label{fig:chap2cocktailpartyproblem}
\end{figure*}

Dans notre environnement quotidien, notre système auditif est confronté à des mélanges complexes de sons provenant de diverses sources. 
Sur la base de ces mélanges, la perception auditive remplit un certain nombre de fonctions essentielles : i) informer sur l’environnement et les objets qui nous entourent, ii) alerter et détecter, iii) reconnaître les sources, iv) appréhender l'espace, et v) communiquer.

Dans des scènes relativement calmes, nous pouvons entendre des flux auditifs individuels avec facilité, parfois même sans en avoir conscience \citep{sussman2007role}. 
Au contraire, dans des scènes bruyantes, où les bruits de fond masquent partiellement les sons d'intérêts, cette tâche est plus complexe et plus exigeante. 
Les informations auditives pertinentes ne peuvent être récupérées que si le système réussit à décomposer le mélange en unités perceptives significatives appelées «flux» \citep{bregman1990auditory}. 
L'une des fonctions principales du système auditif est de décomposer le mélange complexe de sons arrivant aux oreilles en une représentation qui isole les sources sonores individuelles \citep{dykstra2011neural}. 
Ce processus, appelé analyse de la scène auditive, est considéré comme crucial pour la survie et la communication. 
Les capacités de compréhension de la parole et de perception musicale reposent sur cette capacité du système auditif à d'abord séparer les informations entrantes en composantes distinctes. 
L'incapacité à séparer les informations acoustiques peut ainsi entraîner un groupement illusoire de sons provenant de sources multiples, ce qui peut provoquer un résultat négatif dans de nombreuses situations, comme par exemple l'absence de perception de l'alarme auditive par le pilote. 

À tout moment de notre expérience auditive, il peut y avoir de multiples objets présents dans la scène auditive, pouvant ou non, correspondre à des sources sonores présentes dans l'environnement acoustique. 
Les objets auditifs sont des constructions perceptivement bien définies. 
Malgré leurs équivalents visuels, il est difficile de donner un sens intuitif à cette notion dans la modalité auditive \citep{ahveninen2006task, alain2000selectively, dyson2010auditory, kubovy2001auditory, schnupp2013neural}. 
Une représentation bidimensionnelle d'un objet auditif peut être définie comme le produit de mécanismes de groupement selon les dimensions de fréquence et de temps \citep{bizley2013and, griffiths2004auditory, griffiths2012auditory}. 
Définir des limites claires entre des objets auditifs simultanés et séparer les informations qui appartiennent à un objet auditif spécifique du reste de la scène auditive est une tâche extrêmement difficile. 
Plusieurs principes de groupement, basés sur l'analyse des modèles auditifs dans l'espace temps-fréquence, ont été proposés pour la classification des indices acoustiques perceptuels dans une scène auditive complexe \citep{bizley2013and, griffiths2004auditory}.  

\begin{figure*}[!t]
\center
\includegraphics[width=0.66\columnwidth]{illustrations/utilisé/ABA_stimuli_percept.jpg}
\caption[Paradigme de streaming auditif ABA]{Spectrogramme schématique de stimuli utilisés pour étudier le streaming auditif. Une succession de tonalités avec deux fréquences différentes, A et B, est présentée ("Stimulus"). Le sujet peut percevoir soit un flux unique avec un rythme de "galop" (ABA-ABA-ABA illustré par les lignes vertes reliant le A et le B dans "Percept"), soit deux flux réguliers (A-A-A et B-B-B, illustrés par la ligne bleue reliant les tonalités A et la ligne rouge reliant les tonalités B). Le percept peut alterner entre les deux interprétations. Adapté de \cite{schwartz2012multistability}.}
\label{fig:chap2paradigmestreamingauditifABA}
\end{figure*}

Le groupement de composantes auditives peut faire émerger la perception d'un flux spécifique et se base ainsi sur une organisation cohérente des entrées auditives. 
La séparation d'une scène auditive en plusieurs flux, ou ségrégration des flux auditifs est plus largement connu sous le nom d'effet «Cocktail party» \citep{cherry1953some}, présenté sur la Figure \ref{fig:chap2cocktailpartyproblem}.
\cite{cherry1953some} a mené des expériences d'attention dans lesquelles les participants écoutaient deux messages différents d'un seul haut-parleur\footnote{Au début des années 1950, une grande partie des premières recherches sur l'attention ont porté sur les problèmes rencontrés par les contrôleurs aériens. 
À cette époque, les contrôleurs recevaient les messages des pilotes par haut-parleurs dans la tour de contrôle. 
Entendre les voix mélangées de nombreux pilotes sur un seul haut-parleur rendait la tâche du contrôleur très difficile \citep{sorkin1983human}.}
 en même temps et essayaient de les séparer. 
Ses travaux ont révélé que la capacité à séparer les sons du bruit de fond est affectée par de nombreuses variables, telles que le sexe du locuteur, la direction d'où provient le son, la hauteur du son et la vitesse de la parole.
De cette façon, une tâche de cocktail-party est une tâche perceptive complexe, pouvant ainsi être facilitée par des indices informatifs dans le stimulus acoustique le long des dimensions temporelles et spectrales \citep{akram2015neural}. 

Au niveau d'analyse le plus élémentaire, l'organisation perceptive de la scène auditive commence par les attributs les plus fondamentaux des objets auditifs \citep{dykstra2011neural}. 
Ces attributs comprennent la hauteur, l'intensité, la durée, le timbre et la position dans l'espace, et sont largement indépendants l'un de l'autre sur le plan perceptif. 
De nombreuses études ont été réalisées sur les indices de groupement perceptif et la ségrégation des flux \citep{bregman1994auditory, carlyon2004brain, darwin1997auditory}. 
Les événements auditifs, variations temporelles brusques de l'énergie sonore, peuvent être groupés dans le temps en flux si les attributs perceptifs constituant les événements successifs sont suffisamment similaires les uns aux autres. 
Le principe de similarité des caractéristiques établit que les sons sont plus susceptibles d'avoir été émis par la même source s'ils ont des caractéristiques acoustiques similaires \citep{moore2002factors, moore2012properties}. 
Le groupement ou la séparation d'événements auditifs successifs en objets auditifs est connu sous le nom de «streaming auditif» \citep{bregman1994auditory}. 

Le célèbre paradigme de streaming auditif ABA a été étudié pour la première fois dans les années 1970 par Bregman et ses collègues \citep{bregman1971primary}. 
Il consiste à présenter à un auditeur des triplets de la forme ABA-ABA, où A et B sont des sons de courte durée (généralement des sons purs) et le tiret un espace silencieux \citep{van1975temporal, van1977minimum}. 
Lorsque la séparation des fréquences est faible ou que la vitesse de présentation est lente, les sons sont perçus comme étant groupés et comme provenant d'une seule source sonore («fusion»). 
Lorsque la séparation des fréquences est importante ou que la vitesse de présentation est rapide, les sons A et B sont perçus comme étant séparés, chacun formant son propre flux auditif («ségrégation»). 
La Figure \ref{fig:chap2paradigmestreamingauditifABA} montre les perceptions communes typiques de l'auditeur dans l'espace paramétrique défini par la séparation des fréquences et le taux de présentation dans un paradigme de streaming. 
Lorsque le groupe de tonalités est intégré (c'est-à-dire un flux), on entend un rythme de galop distinctif segmenté en triolets successifs à trois tonalités se produisant environ toutes les demi-secondes \citep{dykstra2011neural}. 
Lorsque le groupe de tonalités est ségrégé, la perception d'un rythme galopant est perdue ; ce que l'on peut entendre à sa place sont deux rythmes isochrones, l'une des tonalités A à un rythme moitié moins élevé que celui des tonalités B. 
En outre, le rapport le plus courant parmi les auditeurs qui entendent deux flux est celui d'un avant-plan et d'un arrière-plan perceptifs, l'un et l'autre étant occupés par les tonalités A ou les tonalités B à un moment donné. 

Le paradigme de streaming auditif a fait l'objet d'un nombre important d'études chez l'humain et chez différents animaux depuis plusieurs décennies \citep{anstis1985adaptation, bee2010neural, carlyon2003cross, dykstra2011widespread, kashino2007dynamics, sussman1999investigation}. 
Cependant, ce paradigme n'est pas le seul paradigme à avoir permis l'étude des mécanismes associés à l'analyse de la scène auditive, la ségrégation des flux, l'organisation cohérente et le groupement des objets auditifs. 
En effet, à l'opposé de la ségrégation des flux, le système peut échouer à ségréger correctement les flux et ainsi permettre l'apparition de phénomènes de masquage auditif comme révélé par la perception d'un flux unique. 

%%%%%%%%%%%%%%%%%%%%%%%%%%%%%%%%%%%%%%%%%%%%%%%%%%%%%%%%%%%%%%%%%%%%%%%%%%%%%%%
\subsection{Les différentes types de masquage auditif}
\label{masquageauditif}
%%%%%%%%%%%%%%%%%%%%%%%%%%%%%%%%%%%%%%%%%%%%%%%%%%%%%%%%%%%%%%%%%%%%%%%%%%%%%%%

Le problème de détecter un son provenant d’une source sonore particulière devient beaucoup plus difficile lorsque les sons d’autres sources indépendantes viennent se chevaucher dans le temps \citep{kidd2008informationalreview}. 
L'échec de la détection d'un son du fait de la présence d’autres sons correspond aux phénomènes de masquage auditif \citep{delgutte1990physiological, fletcher1940auditory, moore1995hearing, wegel1924auditory}. 
Au cours de ces phénomènes perceptifs, lorsqu'un son cible audible --- au-dessus du seuil de détection --- n'est pas perçu, il est dit «masqué». 
L'autre son, masquant est quant à lui appelé «masqueur». 
La recherche en psychoacoustique a longtemps cherché à quantifier le degré d’interférence qui résulte de la compétition entre sources sonores distinctes. 
Usuellement, le masquage auditif a été étudié en présentant au sujet des tonalités pures (une cible et une masquante) en cherchant à obtenir les seuils de masquage. 
Le seuil non-masqué est le plus petit niveau d'intensité acoustique de la cible qui peut être perçu sans qu'un masqueur soit présent. 
Le seuil masqué, au contraire, est le plus petit niveau d'intensité de la cible qui peut être perçu lorsqu'il est combiné à un masqueur spécifique. 
La quantité de masquage représente la différence entre les seuils masqués et non masqués (en dB SPL, Figure \ref{fig:chap2schemamasqueurcible}).
Il existe deux types de masquages selon les processus en jeu : le masquage énergétique et le masquage informationnel. 

\begin{figure*}[!t]
\center
\includegraphics[width=0.3\columnwidth]{illustrations/utilisé/masquage_auditif.png}
\includegraphics[width=0.55\columnwidth]{illustrations/utilisé/masqueurs_multi_tonalités.jpg}
\caption[Seuil masqué et masqueurs multi-tonalités à fréquence aléatoire]{(Gauche) Procédure de quantification du phénomène de masquage auditif. Le seuil non-masqué est le plus petit niveau d'intensité acoustique de la cible qui peut être perçu sans qu'un masqueur soit présent, tandis que le seuil masqué est le plus petit niveau d'intensité de la cible qui peut être perçu lorsqu'il est combiné à un masqueur spécifique. La quantité de masquage représente la différence entre les seuils masqués et non masqués. Adapté de \cite{gelfand2017hearing}. (Droite) Deux masqueurs de fréquences aléatoires dans une expérimentation de masquage multi-tonalités. À gauche est présenté la cible plus le masqueur et à droite seulement le masqueur est présenté. En traits gris pointillés apparaît la région fréquentielle protégée, ne contenant pas de tonalités du masqueur. Adapté de \cite{kidd2008informationalreview}.}
\label{fig:chap2schemamasqueurcible}
\end{figure*}

%%%%%%%%%%%%%%%%%%%%%%%%%%%%%%%%%%%%%%%%%%%%%%%%%%%%%%%%%%%%%%%%%%%%%%%%%%%%%%%
\subsubsection{Masquage énergétique}
\label{masquageenergetique}
%%%%%%%%%%%%%%%%%%%%%%%%%%%%%%%%%%%%%%%%%%%%%%%%%%%%%%%%%%%%%%%%%%%%%%%%%%%%%%%

La plus ancienne forme de masquage auditif étudiée est le «masquage énergétique» (ME), nommé aussi masquage périphérique. 
Le phénomène de ME illustre les limites de la sélectivité fréquentielle du traitement de l’information sonore dans la cochlée. 
Dans le ME, une cible et un masqueur présentent des caractéristiques spectrales qui se chevauchent \citep{delgutte1990physiological}. 
Le masqueur produit un pattern d’excitations dans la cochlée qui vient supprimer l’activité de la cible, de sorte que ce dernier n’est pas en mesure d'être représenté dans le nerf auditif. 
Le masqueur vient ainsi élever localement le seuil d’audition faisant que le son cible nécessite alors plus d’énergie pour être perçu. 
La détectabilité d'un son est dès lors majoritairement déterminée par une petite partie de l'énergie du bruit située à proximité spectrale ou temporelle de la tonalité \citep{fletcher1940auditory, green1966signal, penner1973critical}. 
Ce phénomène est directement lié à l'existence des filtres cochléaires (voir Section \ref{systemeauditifperif}). 
En effet, comme un masqueur tombant dans le même filtre que la cible provoque un effet de masquage, on peut tester la sélectivité fréquentielle du système auditif en expérimentant de telles conditions. 
Les variations de seuil utilisées pour évaluer la sélectivité sont attribuées aux différences de quantité du ME. 
La quantité de masquage est la plus importante lorsque le masqueur et la cible ont la même fréquence et diminue à mesure que la fréquence du signal s'éloigne de la fréquence du masqueur.
Si un signal et un masqueur sont présentés simultanément, alors seules les fréquences du masqueur qui se situent dans la bande critique contribuent au masquage du signal. 
L'efficacité du masqueur à élever le seuil de détection du signal dépend donc de la fréquence du signal et de la fréquence du masqueur. 
De cette manière, dans un environnement sonore, l'auditeur doit détecter et reconnaître les sons pertinents (cibles) intégrés dans un contexte acoustique contenant les sons de nombreuses autres sources non pertinentes (masqueurs). 
Cette détection implique une ségrégation des sources sonores distinctes et amène l'auditeur à engager une organisation perceptive cohérente des flux présentés, laquelle est fonction de la tâche demandée. 

%%%%%%%%%%%%%%%%%%%%%%%%%%%%%%%%%%%%%%%%%%%%%%%%%%%%%%%%%%%%%%%%%%%%%%%%%%%%%%%
\subsubsection{Masquage informationnel}
\label{masquageinformationnel}
%%%%%%%%%%%%%%%%%%%%%%%%%%%%%%%%%%%%%%%%%%%%%%%%%%%%%%%%%%%%%%%%%%%%%%%%%%%%%%%

La détection d'un signal auditif bien au-dessus du seuil peut également être réduite sans présenter un quelconque chevauchement dans la représentation tonotopique cochléaire. 
Ce phénomène a été qualifié de «masquage informationnel» (MI) \citep{durlach2003note, kidd2008informationalreview, pollack1975auditory, watson1976factors}. 
Le MI est présenté pour la première fois dans la littérature par Irwin Pollack en 1975 \citep{pollack1975auditory}. 
Cette forme de masquage se produit à un niveau plus élevé dans le système auditif comparativement au ME \citep{lutfi1989informational, leek1991informational, pollack1975auditory, watson1981role}. 
Cependant, \cite{durlach2006auditory} a soulevé la difficile problématique de spécifier ce que signifie périphérie et chevauchement dans le masquage auditif.
Le MI à un certain niveau peut vraisemblablement apparaître comme du ME à un niveau supérieur, via un chevauchement entre cible et masqueur dans plusieurs canaux plus centraux. 
Cela amenant donc à ce que toute forme de masquage observée soit du ME s'il est examiné à un niveau suffisamment élevé \citep{durlach2006auditory}. 
Néanmoins, le MI peut disparaître après un entraînement extensif, ne reflétant donc pas une perte d'information dans le système auditif périphérique \citep{oxenham2003informational}.
Le MI auditif est donc plus largement défini comme une élévation du seuil de performance de détection de la cible qui ne pourrait être expliqué par les facteurs qui contribuent au ME \citep{kidd2008informationalreview}. 
Depuis maintenant plusieurs décennies, la recherche en psychoacoustique a étudié l'influence de composantes du signal sur cet effet de masquage et donc sur l'élévation des seuils de performance. 
Ainsi, plusieurs facteurs sont susceptibles de venir influencer la quantité de masquage produite. 

%%%%%%%%%%%%%%%%%%%%%%%%%%%%%%%%%%%%%%%%%%%%%%%%%%%%%%%%%%%%%%%%%%%%%%%%%%%%%%%
\subsection{Facteurs influençant le masquage informationnel}
\label{facteursmasquage}
%%%%%%%%%%%%%%%%%%%%%%%%%%%%%%%%%%%%%%%%%%%%%%%%%%%%%%%%%%%%%%%%%%%%%%%%%%%%%%%

L'incertitude associée au masqueur et la similarité dans les propriétés acoustiques de la cible et du masqueur sont les deux facteurs principaux les plus influents identifiés dans le MI et qui impliquent des manipulations différentes des signaux \citep{durlach2003informational, kidd2008informationalreview, neff1987masking, watson2005some}. 
\cite{watson2005some} distingue ainsi l'effet de la variation entre les essais des caractéristiques du stimulus («MI basé sur l'incertitude») et l'effet de la similarité entre la cible et le masqueur («MI basé sur la similarité»). 
En conséquence, cela implique des processus fondamentalement différents aboutissant au phénomène perceptif de MI, hautement variable entre les individus \citep{oxenham2003informational}. 

%%%%%%%%%%%%%%%%%%%%%%%%%%%%%%%%%%%%%%%%%%%%%%%%%%%%%%%%%%%%%%%%%%%%%%%%%%%%%%%
\subsubsection{Incertitude}
\label{masquageinformationnelincertitude}
%%%%%%%%%%%%%%%%%%%%%%%%%%%%%%%%%%%%%%%%%%%%%%%%%%%%%%%%%%%%%%%%%%%%%%%%%%%%%%%

Le MI a tout d'abord attiré l'attention grâce aux travaux de Charles Watson et de ses collègues qui ont étudié la discrimination de tonalités sous variation de leurs caractéristiques \citep{watson1975factors, watson1976factors}. 
Ils ont cherché à déterminer l'effet de l'incertitude associée à la discrimination des changements de caractéristiques d'éléments de patterns cibles (\textit{e.g.}, un changement de la fréquence ou de l'intensité). 
De grandes différences dans la capacité des auditeurs à discerner des changements dans un élément cible ont été observées en manipulant l'incertitude des tonalités du masqueur. 
L'incertitude, telle que considérée ici, repose sur les modifications aléatoires des différentes composantes spectrales et temporelles. 
Elle est mesurée par l'information contenue dans le signal sur la base des contraintes statistiques qui régissent les variations aléatoires et donc des lois sous-jacentes. 
Comme la variation aléatoire est une caractéristique inévitable des environnements acoustiques naturels, il est essentiel de comprendre comment sont interprétés les sons en fonction de cette variation. 
En effet, la détection de tonalités dans du bruit ou dans un complexe multi-tonalités est gravement dégradée lorsque seules quelques composantes spectrales du bruit/complexe sont présentées au hasard à chaque essai. 
Globalement, \cite{watson1976factors} ont montré que i) l'incertitude du masqueur a un effet plus négatif sur la détection du signal que l'incertitude du signal cible, ii) la quantité de MI augmente avec le degré d'incertitude du masqueur et iii) l'incertitude du masqueur et celle de la cible interagissent, rendant l'effet de l'incertitude sur la cible (fréquence ou position temporelle) plus faible à moins que l'incertitude du masqueur soit faible également. 

\begin{figure*}[!t]
\center
\includegraphics[width=0.26\columnwidth]{illustrations/utilisé/quantité_masquage_nombre_composantes_3.png}
\includegraphics[width=0.36\columnwidth]{illustrations/utilisé/quantité_masquage_nombre_composantes.png}
\includegraphics[width=0.36\columnwidth]{illustrations/utilisé/quantité_masquage_nombre_composantes_2.png}
\caption[Résultats typiques d'une expérimentation de masquage avec masqueur multi-tonalités]{Résultats typiques d'une expérimentation de masquage avec masqueur multi-tonalités. (Gauche) Quantité de masquage totale (en dB, en ordonnée) en fonction du nombre de composantes fréquentielles du masqueur (en abscisse). Adapté de \cite{neff1987masking}. (Milieu) Reproduction des résultats de \cite{neff1987masking} par \cite{oh1998nonmonotonicity}. (Droite) Quantité de masquage totale en fonction du nombre de composantes fréquentielles du masqueur pour les auditeurs individuels dans chaque groupe d'âge : enfants (cercles remplis) et adultes (cercles ouverts). Les fonctions moyennes sont indiquées par les lignes pleines et les lignes pointillées avec des moyennes marquées par * pour les enfants et les adultes, respectivement. La ligne en pointillé représente les données moyennes obtenues pour 11 adultes dans les rapports précédents de \cite{oh1998nonmonotonicity}. Adapté de \cite{oh2001children}.}
\label{fig:chap2resultatsmasquageinformationnel}
\end{figure*}

Les travaux de Robert Lutfi et coll. dans les années 90 sur les tâches de discrimination ont porté sur la relation entre MI, variation aléatoire du signal et quantité d'information associée à cette variation \citep{lutfi1989informational, lutfi1990informational, lutfi1992informational}. 
Dans une tâche de discrimination d'échantillon \citep{lutfi1989informational}, c'est la valeur de la différence à discriminer qui varie aléatoirement d'un essai à l'autre. 
Cette variation fournit des informations potentielles sur les statistiques des signaux présentés. 
Selon la théorie de la détection du signal (TDS), l'observateur, sur la base de ce qu'il sait des propriétés statistiques du signal et du bruit, maximise la probabilité de reporter correctement la présence d'un signal cible dans du bruit. 
Selon Robert Lutfi, le terme information peut être utilisé dans le cadre des approches par masquage via la théorie de l'information pour être synonyme de variation aléatoire du signal. 
Ainsi, un son très variable d’un instant à l’autre est susceptible de transmettre une grande quantité d’information \citep{lutfi1988interpreting, lutfi1989informational, lutfi1990much, lutfi1990informational, lutfi1992informational}. 
La quantité exacte d'information finalement extraite du signal dépend toutefois de la capacité de l'auditeur à réduire l'incertitude associée à la variation. 
En effet, c’est le degré élevé de variabilité associé aux sons qui différencie les exigences informationnelles de la plupart des tâches de discriminations. 

Depuis qu'elles ont rendu compte de ces résultats pour la première fois, les études de psychoacoustique ont largement adopté une approche paramétrique du problème de la détection de signaux dans le MI.
L'objectif étant généralement d'identifier les déterminants critiques de la performance en observant l'effet de diverses manipulations physiques des signaux cibles et masqueurs sur la performance. 
Ce n'est que plus tard, que \cite{neff1987masking} ont introduit le paradigme de «masquage multi-tonalités», dans lequel de l'incertitude est créée au sein du masqueur par la présentation d'un ensemble de tonalités à fréquences aléatoires. 
Ce paradigme de masquage multi-tonalités \citep{neff1987masking} est présenté sur la Figure \ref{fig:chap2schemamasqueurcible} à droite en axes Intensité-Fréquence avec un exemple schématique des stimuli typiquement utilisés. 
On observe deux tirages de masqueurs multi-tonalités, un dans chaque intervalle d'observation (gauche et droite). 
Dans celui de gauche, une tonalité cible est intégrée dans une région protégée. 
La région protégée ou «région fréquentielle protégée» est une gamme de fréquences entourant le signal dans laquelle ne se trouve aucune composantes du masqueur. 
L'objectif de la région fréquentielle protégée est par conséquent de limiter le ME \citep{neff1988effective}. 

À l'origine, \cite{neff1987masking} s'intéressaient à la question de savoir combien de composantes fréquentielles étaient nécessaires pour créer du «bruit» pour l'auditeur. 
Dans leur expérience, ils ont examiné les seuils pour des signaux cibles de $250$, $1000$ et $4000$~Hz présentés simultanément avec des masqueurs comportant $1$ à $100$ composantes sinusoïdales. 
Le nombre de composantes de masquage était constant tout au long d’un bloc d’essais, mais les fréquences des composantes étaient tirées au hasard dans une plage de $5000$~Hz pour chaque présentation. 
Pensant que les composantes du masqueur s'approchent rarement suffisamment près du signal pour le masquer énergétiquement, les auteurs s'attendaient à très peu de masquage pour les masqueurs composés d'un petit nombre de composantes. 
De façon surprenante, de grandes quantités de masquage (plus de $50$~dB pour les cibles à $1000$ et $4000$~Hz) ont été trouvées pour des masqueurs composés de seulement dix composantes (Figure \ref{fig:chap2resultatsmasquageinformationnel}).
Le masquage maximal n'a pas été observé pour le masqueur composé de toutes les composantes (\textit{i.e.}, bruit gaussien), mais pour un masqueur composé d'environ $10$ à $20$ composantes fréquentielles. 

Ainsi, des masqueurs de quelques tonalités à fréquence aléatoire peuvent produire un important MI (basé sur l'incertitude) si les fréquences sont tirées à partir d’un large intervalle fréquentiel et modifiées à chaque essai. 
Ces résultats ont pu être reproduits par la suite \citep{neff1988effective, neff1993informational, oh1998nonmonotonicity}.
Ces études psychoacoustiques ont donc révélé que la variation entre les essais des caractéristiques du stimulus présente des effets négatifs sur la détection diminuant ainsi la capacité des auditeurs à percevoir des signaux auditifs. 
L'incapacité perceptive des auditeurs associée à la variation entre les essais des caractéristiques du stimulus correspondrait alors à un MI basé sur l'incertitude. 

%%%%%%%%%%%%%%%%%%%%%%%%%%%%%%%%%%%%%%%%%%%%%%%%%%%%%%%%%%%%%%%%%%%%%%%%%%%%%%%
\subsubsection{Similarité}
\label{masquageinformationnelsimilarité}
%%%%%%%%%%%%%%%%%%%%%%%%%%%%%%%%%%%%%%%%%%%%%%%%%%%%%%%%%%%%%%%%%%%%%%%%%%%%%%%

La similarité de sons ou d'éléments de sons est également susceptible d'affecter l'organisation des flux perceptifs \citep{dickerson2014did, kidd2002similarity, lee2011evaluation}. 
Identifier une cible, un changement dans cette cible ou même indiquer où ce changement s'est produit, peut être difficile lorsque le son cible est présenté simultanément avec des sons en compétition. 
L'importance de la similarité entre les sons présentés aura un effet profond sur la capacité de l'auditeur à détecter le son cible attendu. 
Les sons présentant des caractéristiques similaires sont généralement perçus comme un seul flux. 
Au contraire, les sons présentant des différences de caractéristiques sont généralement séparés perceptivement en flux distincts. 
Une grande similarité entre un masqueur et une cible favorise le MI (\textit{i.e.}, réduit la détection), tandis qu'une faible similarité entre eux diminue drastiquement le MI (\textit{i.e.}, favorise la détection) \citep{watson2005some}. 
De cette façon, la capacité de ségrégation perceptive des flux se fonde également sur un MI basé sur la similarité. 
Les sons du masqueur et de la cible peuvent être similaires sur plusieurs dimensions, qui correspondent aux différentes propriétés acoustiques des sons (durée, fréquence, intensité, etc...). 

\begin{figure*}[!t]
\center
\includegraphics[width=0.45\columnwidth]{illustrations/utilisé/spectrogrammes_masqueurs_burst.png}
\includegraphics[width=0.54\columnwidth]{illustrations/utilisé/spectrogrammes_masqueurs_mbs_mdb.png}
\caption[Masqueurs à «salves» multiples similaires (MBS) et différentes (MDB)]{Masqueurs à «salves» multiples similaires (MBS) et différentes (MDB). 
(Gauche) Spectrogrammes sonores du masqueur et de la cible pour MBD (lignes en gras : $1$~kHz), le nombre de salves du masqueur variant de $1$ (salve unique) à $8$ (rangée supérieure) et l'intervalle entre les salves variant pour quatre salves du masqueur (rangée inférieure). Adapté de \cite{kidd2003multiple}.
(Droite) Spectrogrammes sonores de masqueurs avec cibles (lignes en gras : $1$~kHz) à rafales multiples similaires (MBS, rangée du haut) et à rafales multiples différentes (MBD, rangée du bas). Dans la colonne de gauche (S pour Similar), la fréquence cible varie/reste constante dans le temps de manière similaire aux fréquences du masqueur dans le temps, alors que dans la colonne de droite (D pour Dissimilar), la fréquence cible dans le temps diffère des fréquences du masqueur dans le temps. Adapté de \cite{durlach2003informational}.}
\label{fig:chap2spectrogrammesmasqueursburst}
\end{figure*}

\cite{kidd2003multiple} ont étudié si une modification de la similarité entre la cible et le masqueur pouvait affecter la quantité de masquage obtenue en utilisant deux types de masqueurs multi-tonalités distincts (Figure \ref{fig:chap2spectrogrammesmasqueursburst} Gauche). 
Dans un premier cas, les fréquences au sein du masqueur étaient tirées aléatoirement sur chaque fenêtre temporelle (\textit{i.e.}, une salve) tout au long de la séquence.
Dans un second cas, le masqueur était composé de plusieurs salves de tonalités contraintes de se situer dans des zones de fréquence au fûr et à mesure de la progression de la séquence de salves. 
Ainsi, le masqueur était composé d'un ensemble de «flux» à bande étroite (\textit{i.e.}, les salves). 
De cette manière, la similarité vis-à-vis de l'ensemble des cibles correspondait à la disposition du masqueur en flux à bande étroite. 
Un degré élevé d'incertitude quant aux fréquences de la cible et du masqueur était également présent, de sorte que le sujet devait surveiller toute la gamme de fréquences pour localiser la cible. 
Du fait de ce degré élevé d'incertitude au niveau des fréquences, les flux du masqueur pouvaient vraisemblablement ressembler à un modèle de cible, du moins au début de la séquence de salves, augmentant ainsi la probabilité de confusion avec la cible, ou de perte d'information due à une mauvaise orientation de l'attention. 
Les résultats de cette étude ont indiqué que les masqueurs constitués d'ensembles de flux à bande étroite (masqueurs à salves multiples similaires, MBS) ont produit beaucoup plus de masquage que les masqueurs dans lequel les tonalités  n'étaient pas reliées entre elles d'une salve à l'autre (masqueurs à salves multiples différentes, MDB). 
Les auteurs ont souligné que cette augmentation du masquage était une conséquence de la plus grande similarité entre la cible et le masqueur \citep{kidd2003multiple}. 

Une autre étude a cherché à démontrer que la quantité de MI peut être considérablement réduit en diminuant la similarité entre la cible et le masqueur \citep{durlach2003informational}. 
Les auteurs ont mené cinq expériences dans lesquelles le seuil de détection d'une tonalité cible présentée simultanément à des masqueurs multi-tonalités a été mesuré pour deux conditions. 
Dans un cas, la cible a été construite de manière à être «similaire» au masqueur ; dans l'autre cas, elle a été construite de manière à être «différente» du masqueur. 
La similarité cible-masqueur variait dans des dimensions telles que la durée, la localisation, le changement de fréquence et la cohérence spectro-temporelle. 
Le masqueur spécifique variait selon les expériences, mais était constant pour les deux conditions. 
La Figure \ref{fig:chap2spectrogrammesmasqueursburst} (Droite) présente plusieurs spectrogrammes correspondants à différents modes de présentation des stimuli en termes de similarité cible-masqueur lors d'une tâche de MI multi-tonalités. 
Les résultats ont montré qu'une forte diminution de la quantité de masquage était observée pour la condition dissimilaire par rapport à la condition similaire entre la cible et le masqueur \citep{durlach2003informational}. 

Récemment, \cite{dickerson2015sound} ont également appuyé ces résultats en montrant que la performance à une tâche de discrimination de changement des caractéristiques des signaux était fortement influencée par le degré de similarité entre la cible et les sources non-pertinentes.
Ainsi, la diminution de la similarité (ou augmentation de la dissimilarité) dans les propriétés acoustiques entre la cible et le masqueur tend à favoriser la ségrégation de flux auditifs distincts et ainsi à réduire les effets de masquage associée à l'incertitude du stimulus.

%%%%%%%%%%%%%%%%%%%%%%%%%%%%%%%%%%%%%%%%%%%%%%%%%%%%%%%%%%%%%%%%%%%%%%%%%%%%%%%
\subsubsection{Variabilité inter-individuelle}
\label{masquageinformationnelvariabilité}
%%%%%%%%%%%%%%%%%%%%%%%%%%%%%%%%%%%%%%%%%%%%%%%%%%%%%%%%%%%%%%%%%%%%%%%%%%%%%%%

Finalement, la quantité de masquage est hautement spécifique à chaque auditeur et cette variabilité inter-individuelle, que l'on peut comprendre comme la capacité d'un auditeur à réduire l'incertitude associée à la variation des caractéristiques des signaux, est un aspect important des phénomènes de masquage \citep{durlach2003informational, neff1993informational, oxenham2003informational}. 
On peut faire l'hypothèse qu'une telle variation dégrade la représentation interne du masqueur de l'auditeur et par conséquent la sensibilité à une tonalité cible ajoutée au masqueur est diminuée. 
La comparaison des résultats individuels pour des auditeurs dans ces expériences est à même de fournir des informations supplémentaires importantes sur la nature du MI.
En effet, les différences dans la capacité des auditeurs à discerner des changements dans un élément cible permettent de catégoriser les auditeurs en fonction de leur seuil.
Dans leur étude, \cite{neff1993informational} ont souligné une forte variabilité inter-individuelle comme tous les auditeurs ne montraient pas d’effets importants de l’incertitude de la fréquence du masqueur. 
Pour les auditeurs affectés par l’incertitude du masqueur, la quantité de MI était comprise entre 20 et 30 dB et pouvait être supérieure en fonction de propriétés de masquage spécifiques telles que l'intensité et le nombre de composants. 
Par conséquent, ils ont adopté le terme «seuil haut» pour désigner les auditeurs qui affichaient constamment de grandes quantités de MI malgré un entraînement poussé. 
Au contraire, ils ont adopté le terme «seuil bas» pour désigner l'autre groupe, qui a montré des effets beaucoup moins importants de l'incertitude du masqueur. 
De plus, ces différences inter-individuelles apparaissaient beaucoup plus importantes dans des conditions cible-masqueur similaires par rapport aux conditions dissimilaires \citep{durlach2003informational}. \\

Pour conclure, deux paradigmes ont été en mesure de nous renseigner et d'améliorer notre compréhension des phénomènes de perception auditive d'un signal sonore. 
En fait, dans la modalité auditive, un certain continuum apparaît entre la ségrégation auditive et le masquage auditif \citep{chang2016detection, lutfi2012detection, lutfi2013information}. 
À une extrémité de ce continuum, la ségrégation est révélée par une perception cohérente de deux flux distincts : «AAA/BBB/AAA» pour le paradigme de streaming auditif et «cible/masqueur» pour le masquage auditif. 
Au contraire, à l'autre extrémité, le masquage s'établit avec l'absence de ségrégation des flux, révélée par la perception d'un flux unique : «ABABABAB» pour le paradigme de streaming auditif et «masqueur (comprenant la cible)» pour le masquage auditif. 
Il est clair que la détection d'une tonalité intégrée dans un masqueur multi-tonalités à fréquences aléatoires peut être très faible même lorsque les masqueurs ont peu d'énergie dans la région fréquentielle du signal.
Contrairement au ME induit dans le système auditif périphérique par le chevauchement des réponses neurales à la cible et au masqueur, les mécanismes neuronaux et localisations propres au MI sont encore mal compris \citep{shinn2008object}. 
Une hypothèse serait l'existence d'un goulot d'étranglement du traitement de l'information dans le cerveau \citep{overath2007information, gutschalk2008neural} évitant au stimulus d'accéder à la conscience perceptive. 
Cette hypothèse repose sur des aspects de limitation de ressources cognitives et attentionnelles et de transmission de l'information à l'échelle cérébrale. 

De cette manière, le MI n’est pas un phénomène unique mais peut être considéré comme le résultat des actions de toutes les étapes de traitement de l’information au-delà de la périphérie auditive. 
Il apparaît ainsi intimement lié aux phénomènes de groupement perceptif, de ségrégation de source, d'attention, de mémoire et de capacités de traitements cognitifs plus générales \citep{kidd2008informationalreview}. 
On comprend des expériences qui ont cherché à mesurer la quantité de masquage et l'impact sur la détection lors de la manipulation des caractéristiques des stimuli que le MI est déterminé par une variété de facteurs. 
On peut citer d'un coté : le nombre, l'intensité et les fréquences des tonalités du masqueur et de la cible qui contribuent à l'incertitude du masqueur et de la cible, et d'un autre coté, la similarité acoustique entre le masqueur et la cible. 
L'incertitude associée au masqueur et la similarité cible-masqueur sont les deux facteurs dominants sur la quantité de masquage produite et donc sur la performance de la détection de cible. 
Néanmoins, cette influence dominante de ces facteurs sur la performance des auditeurs et ainsi sur la conscience perceptive ne se limite pas à des aspects purement psychoacoustiques. 
Au-delà de ces aspects comportementaux, la conscience perceptive vis-à-vis d'un objet de l'environnement peut être étudiée à un niveau neural afin de mieux comprendre le substrat neuronal qui lui est associé. 
Pour y parvenir, il est nécessaire de définir plus précisément ce que l'on entend par perception consciente et quels sont les moyens dont on dispose aujourd'hui pour permettre l'étude de son substrat et/ou corrélat neuronal. 

%%%%%%%%%%%%%%%%%%%%%%%%%%%%%%%%%%%%%%%%%%%%%%%%%%%%%%%%%%%%%%%%%%%%%%%%%%%%%%%
\section{Perception consciente et modèles de conscience}
\label{perceptionconscientetheorie}
%%%%%%%%%%%%%%%%%%%%%%%%%%%%%%%%%%%%%%%%%%%%%%%%%%%%%%%%%%%%%%%%%%%%%%%%%%%%%%%

%%%%%%%%%%%%%%%%%%%%%%%%%%%%%%%%%%%%%%%%%%%%%%%%%%%%%%%%%%%%%%%%%%%%%%%%%%%%%%%
\subsection{Conscience, perception consciente et multistabilité du percept}
\label{conscienceetperceptionconsciente}
%%%%%%%%%%%%%%%%%%%%%%%%%%%%%%%%%%%%%%%%%%%%%%%%%%%%%%%%%%%%%%%%%%%%%%%%%%%%%%%

\begin{figure*}[!t]
\center
\includegraphics[width=0.6\columnwidth]{illustrations/utilisé/theme_conscience.jpg}
\caption[Différents thèmes de recherche sur la conscience et proposition sur la manière dont ils s'articulent]{Différents thèmes de recherche sur la conscience, et proposition sur la manière dont ils s'articulent. Image réadaptée de celle de \cite{khamassi2021neurosciences} : Figure 6.1.}
\label{fig:chap2themeconscience}
\end{figure*}

La «conscience» est un terme polysémique et c'est la décomposition en sous-composantes de cette notion protéiforme qui permet de proposer des définitions opérationnelles, et ainsi de l'étudier de façon scientifique. 
La conscience semble désigner l'ensemble de ce que l'on peut saisir personnellement de sa propre vie mentale, et par extension ce que l'on devine de celle des autres \citep{khamassi2021neurosciences}. 
La conscience se réfère ainsi principalement à l'expérience phénoménale elle-même, et accessoirement à des aspects de cette expérience tels que sa globalité, son sentiment de propriété personnelle, la capacité de rapporter son contenu verbalement ou d'une autre manière, et la conscience d'être conscient, appelée «métacognition» \citep{ward2004attention}. 
La Figure \ref{fig:chap2themeconscience} présente la distinction des principaux thèmes associés au mot conscience. 
Actuellement, on s'accorde à diviser les phénomènes conscients en «états de conscience» et en «contenus de conscience» \citep{aru2012distilling, dehaene2006conscious, khamassi2021neurosciences}. 

Une première distinction repose sur l'utilisation du terme «conscience» dans un sens intransitif pour désigner l'«état de conscience», comme lorsque le praticien se demande si un patient est conscient ou inconscient suite à un accident. 
La recherche sur les états de conscience va s'intéresser aux phénomènes neurophysiologiques qui distinguent l'état d'éveil conscient d'autres états. 
Les états de conscience font référence à différents niveaux soutenus d'éveil ou de veille tels que le sommeil, la méditation, le coma et d'autres états de conscience altérés tels que l'anesthésie \citep{farthing1992psychology}. 
Cette condition de l'état de conscience détermine l'émergence de la conscience dans un sens transisitf cette fois, c'est-à-dire, prendre conscience de quelque chose, un contenu comme une sensation, une émotion ou un souvenir. 

Le «contenu de conscience» réfère alors à des expériences momentanées telles que celle d'un carré rouge brièvement présenté sur un écran d'ordinateur ou celle du bruit soudain d'une alarme sonore. 
De cette façon, c'est l'«accès conscient» qui va représenter l'élément fondamental de cette conscience transitive, c'est-à-dire, la saisie d'un contenu particulier à un moment donné. 
Une définition «opérationnelle» de l'accès conscient permet l'investigation scientifique : un contenu conscient est un contenu dont le sujet rapporte qu'il en est conscient. 
Cette avancée conceptuelle relativement simple va être à la base des études ayant pour objectif de comparer l'activité cérébrale dans des situations où un même stimulus sensoriel est rapporté consciemment ou non. 
Le contenu de conscience peut également s'étendre dans le temps comme lorsqu'on observe une œuvre d'art peinte pendant une période prolongée \citep{eklund2019electrophysiological}. 
Le tableau dans son ensemble fait partie de l'expérience, car on se focalise sur différentes parties du tableau. 
Ainsi, à un niveau plus global, le flux de conscience se comprend comme l'enchaînement et l'articulation des différents contenus dans une structure plus complexe et plus soutenue dans le temps \citep{khamassi2021neurosciences}. 

De manière importante, une fonction essentielle de la perception est de détecter un signal d'intérêt parmi d'autres signaux potentiellement distrayants, et de rendre son contenu accessible à la conscience. 
La perception consciente nous permet dès lors de nous intéresser plus spécifiquement au contenu de conscience disponible au travers du fonctionnement de ses réseaux neuronaux qui apparaît être lié au stimulus. 
Les fondements neuronaux de la perception consciente sont essentiels pour comprendre notre expérience du monde \citep{block2007consciousness, crick1998consciousness, dehaene2011experimental}. 
Pour comprendre comment le cerveau génère des expériences, il faut identifier «l'ensemble minimal de processus neuronaux qui, ensemble, sont suffisants pour l'expérience consciente d'un contenu particulier» \citep{aru2012distilling}. 
Ces processus neuronaux en relation avec la génération du contenu de conscience lié à l'expérience sont communément appelés «corrélats neuronaux de la conscience» (CNC). 

Une étape importante dans l'étude des corrélats neuronaux de la perception consciente a été l'utilisation de stimuli ambigüs et multistables \citep{leopold1996activity}. 
De tels stimuli sont physiquement identiques mais peuvent être perçus de différentes manières par le sujet. 
La perception multistable représente ainsi la tendance d'un stimulus physique donné à susciter deux ou plusieurs perceptions distinctes mais stables. 
Elle peut se produire selon diverses modalités sensorielles, notamment la vision, la somatosensation et surtout l'audition \citep{dykstra2011neural}. 
Un exemple de multistabilité perceptive dans la modalité auditive est le processus de ségrégation des flux auditifs comme dans les paradigmes d'étude de streaming auditif et de MI. 
Streaming auditif et MI sont ainsi caractérisés par un phénomène de perception consciente bistable comme le stimulus peut donner lieu à une représentation intégrée (un flux où la cible et le bruit sont fusionnés) ou ségrégée (deux flux où la cible et le bruit sont distincts). 

La multistabilité présente donc les deux caractéristiques suivantes \citep{schwartz2012multistability} : (i) les stimuli ont plus d'une organisation perceptive plausible et (ii) ces organisations ne sont pas compatibles entre elles. 
D'un point de vue neural, ces stimuli multistables sont utiles pour isoler les réponses cérébrales directement liées à l'expérience perceptive d'une personne, appelé «contenu neural de la conscience» \citep{gazzaniga2009cognitive}. 
En partant de l'hypothèse que les processus sensoriels sont principalement couplés aux propriétés des stimuli physiques, les processus neuronaux qui diffèrent entre les différentes perceptions ont été considérés comme de forts candidats pour les corrélats neuronaux de la perception consciente. 
En comparant les réponses à un même stimulus pour différentes expériences perceptives des mêmes stimuli, on peut dissocier l'activité cérébrale liée directement au stimulus de l'activité cérébrale liée à la perception. 
Depuis maintenant plusieurs décennies, les nombreuses expériences basées sur la recherche de corrélats neuronaux de la perception consciente ont permis l'élaboration de modèles de la conscience. 

%%%%%%%%%%%%%%%%%%%%%%%%%%%%%%%%%%%%%%%%%%%%%%%%%%%%%%%%%%%%%%%%%%%%%%%%%%%%%%%
\subsection{Les modèles de la conscience}
\label{modelesconscience}
%%%%%%%%%%%%%%%%%%%%%%%%%%%%%%%%%%%%%%%%%%%%%%%%%%%%%%%%%%%%%%%%%%%%%%%%%%%%%%%

Il existe aujourd'hui différents modèles théoriques basés sur des données expérimentales qui visent à expliquer l'émergence d'un percept à la conscience \citep{dehaene2011experimental, kleiner2020mathematical, sattin2021theoretical, seth2009explanatory, tagliazucchi2013sleep, taylor2011review}. 
Un modèle de conscience est une description théorique mettant en relation les propriétés cérébrales de la conscience avec les propriétés phénoménales de la conscience \citep{seth2007models}. 
Les propriétés cérébrales peuvent être de l'ordre d'une activité électrique irrégulière rapide ou d'une activation cérébrale locale. 
Au contraire, les propriétés phénoménales de la conscience s'apparentent aux «qualia», qui sont généralement définis comme des perspectives à la première personne, ou plus spécifiquement les unités d'une scène consciente. 
Les modèles de conscience doivent être distingués des corrélats neuronaux de la conscience \citep{crick1990towards}. 
L'identification de corrélations entre des aspects de l'activité cérébrale et des aspects de la conscience limite et encadre, dans une certaine mesure, la spécification de modèles neurobiologiquement plausibles. 
Néanmoins, de telles corrélations ne fournissent pas en elles-mêmes de liens explicatifs entre l'activité neurale et la conscience \citep{seth2007models}. 
Les modèles de conscience sont pertinents précisément dans la mesure où ils comprennent des éléments informationnels qui proposent des liens explicatifs entre les propriétés neuronales et les propriétés phénoménales \citep{dehaene2011experimental, kleiner2020mathematical, sattin2021theoretical, tagliazucchi2013sleep}. 
Plusieurs modèles sont considérés ici pour rendre compte de liens explicatifs entre les propriétés cérébrales et les propriétés phénoménales de la conscience. 

%%%%%%%%%%%%%%%%%%%%%%%%%%%%%%%%%%%%%%%%%%%%%%%%%%%%%%%%%%%%%%%%%%%%%%%%%%%%%%%
\subsubsection{Théorie de l'espace de travail global}
\label{theorieespacedetravailglobal}
%%%%%%%%%%%%%%%%%%%%%%%%%%%%%%%%%%%%%%%%%%%%%%%%%%%%%%%%%%%%%%%%%%%%%%%%%%%%%%%

La théorie de l'espace de travail global («Global Workspace», GW) de \cite{baars1993cognitive} est un modèle de conscience qui a inspiré une variété de modèles connexes. 
L'idée centrale est que le contenu cognitif conscient est globalement disponible pour divers processus cognitifs, notamment l'attention, la mémoire et le rapport verbal \citep{baars1993cognitive}. 
La notion de «disponibilité globale» est proposée pour expliquer l'association entre conscience et processus cognitifs intégratifs comme l'attention, la prise de décision et la sélection d'actions. 
Comme la disponibilité globale apparaît nécessairement limitée à un seul flux de contenu, la théorie de l'espace de travail global peut naturellement expliquer la nature sérielle de l'expérience consciente \citep{seth2007models}. 
\cite{baars1993cognitive} décrit cette théorie en termes d'architecture de «tableau noir», dans laquelle des modules de traitement séparés et quasi-indépendants s'interfacent avec une ressource centralisée disponible dans un réseau global. 
Un tel espace de travail global, constitué de connexions cortico-corticales longue distance, assimile d'autres processus en fonction de leur importance tandis que d'autres processeurs activés automatiquement n'entrent pas dans cet espace. 

Plus tard, \cite{dehaene1998neuronal}, puis \cite{dehaene2003neuronal} ont proposé une implémentation neuronale d'une architecture d'espace de travail global, appelée «espace de travail neuronal global» («Global Neuronal Workspace», GNW). 
S'appuyant sur la proposition antérieure de Bernard Baars, la théorie GNW développée par Stanislas Dehaene et ses collègues soutient que la conscience découle d'une architecture à capacité limitée qui est conçue de manière adaptative pour extraire les informations pertinentes d'une variété de systèmes mentaux et les rendre largement disponibles à des fins telles que l'encodage linguistique, le stockage en mémoire, la planification et la prise de décision \citep{dehaene2001towards, dehaene2011experimental, dehaene2014toward, dehaene2014consciousness}. 
Dans ce modèle, les stimuli sensoriels mobilisent les neurones excitateurs avec des axones cortico-corticaux longue distance, ce qui conduit à la genèse d'un modèle d'activité global parmi les neurones de l'espace de travail. 
Un tel modèle global peut inhiber des modèles d'activité alternatifs parmi les neurones de l'espace de travail. 
Cette inhibition empêche ainsi le traitement conscient de stimuli alternatifs \citep{dehaene1998neuronal, dehaene2003neuronal}. 

Lors de la perception, une quantité massive d'informations est traitée inconsciemment par des mécanismes spécialisés qui fonctionnent en parallèle. 
Certaines de ces informations sont toutefois sélectionnées comme étant particulièrement pertinentes pour les objectifs actuels propres à la tâche de l'individu, et elles franchissent alors le seuil de l'accès conscient et entrent dans l'espace de travail global pour un partage flexible \citep{kemmerer2015we}. 
Selon \cite{dehaene2014consciousness} : «cette disponibilité globale des informations est précisément ce que nous expérimentons subjectivement comme un état conscient» p.168. 
Le modèle suppose ainsi qu'une augmentation progressive de la visibilité du stimulus devrait s'accompagner d'une transition soudaine de l'espace de travail neuronal vers un schéma d'activité. 
Lors de la pleine conscience d'un stimulus, le schéma d'activité d'intégration globale correspond alors à un phénomène nommé «ignition» \citep{dehaene2003neuronal}.

Dans ce modèle théorique du GNW, la prise de conscience correspond à une étape tardive et optionnelle du traitement cérébral d'un stimulus, se produisant après les étapes purement sensorielles.
Elle correspond ainsi à la mise en commun des informations sensorielles au sein d'un plus vaste réseau incluant des aires supra-modales, permettant notamment de maintenir cette information plus longtemps, de la mémoriser explicitement, de la rapporter verbalement par une action motrice, et plus généralement de l'intégrer à la planification de nos actions. 
L'accès à ces différentes fonctions cognitives à travers la mise en place de cet «espace de travail» cérébral fait écho aux propriétés psychologiques typiquement associées au traitement conscient, permettant de relier de manière cohérente la description psychologique du traitement conscient, les mécanismes cérébraux sous-jacents et leurs effets cognitifs \citep{khamassi2021neurosciences}. 

%%%%%%%%%%%%%%%%%%%%%%%%%%%%%%%%%%%%%%%%%%%%%%%%%%%%%%%%%%%%%%%%%%%%%%%%%%%%%%%
\subsubsection{Théorie des récurrences locales}
\label{theorierecurrencelocale}
%%%%%%%%%%%%%%%%%%%%%%%%%%%%%%%%%%%%%%%%%%%%%%%%%%%%%%%%%%%%%%%%%%%%%%%%%%%%%%%

Plusieurs modèles de ce type existent mais l'un d'entre eux a eu un impact considérable sur la tentative de fonder la conscience sur la base des processus de récurrences à l'échelle cérébrale \citep{lamme2000distinct,lamme2003visual, lamme2006towards}. 
Selon \cite{lamme2003visual}, la combinaison entre des perspectives psychologiques et des perspectives neuronales a permis d'établir une distinction entre l'attention et la conscience. 
Ces deux phénomènes pourraient dès lors être définis de manière orthogonale comme des processus neuronaux entièrement distincts. 
L'attention, à travers la sélection attentionnelle, représente la «manière dont le traitement sensorimoteur est modifié par l'état actuel du réseau neuronal, façonné par des facteurs génétiques, l'expérience et les événements récents» \citep{lamme2003visual}. 
La conscience, à travers l'expérience phénoménale, présente une origine différente et proviendrait de l'interaction récurrente entre des groupes de neurones \citep{lamme2000distinct, lamme2003visual, lamme2006towards}. 

Le cerveau contient une grande quantité de récurrences : tout module cérébral bien connecté à une autre zone du cerveau aura des connexions réciproques avec celle-ci. 
Cela est particulièrement visible dans la hiérarchie des systèmes visuels, où l'on constate que la hiérarchie va dans les deux sens. 
Selon l'ampleur à laquelle les interactions récurrentes entre les aires visuelles intègrent des interactions avec les aires liées à l'action ou à la mémoire, la conscience évolue d'une conscience phénoménale à une conscience d'accès \citep{lamme2003visual}. 
Cette évolution dépend des mécanismes de sélection de l'attention, qui influencent à la fois le balayage ascendant (par «feedforward») et les interactions récurrentes. 
D'autres mécanismes déterminent toutefois si les neurones s'engageront dans des interactions récurrentes, et donc si le traitement passera d'un état inconscient à un état conscient. 
Les stimuli conscients atteignent un niveau de traitement dépassant la détection initiale des caractéristiques, où au moins une première interprétation perceptive cohérente de la scène est réalisée. 
Ici, la liaison de certaines caractéristiques d'un objet, telles que sa couleur et sa forme, nécessiterait de l'attention, alors que d'autres combinaisons de caractéristiques seraient détectées de manière préattentive. 
Le niveau conscient serait alors, avant que l'attention ne soit allouée, constitué de caractéristiques provisoirement liées\footnote{Le problème de liaison (ang, «binding problem») représente le problème fondamental de savoir comment l'unité de la perception consciente est obtenue par les activités distribuées du système nerveux central \citep{revonsuo1999binding}. De manière plus précise, le problème de liaison est un terme utilisé à l'interface entre les neurosciences, les sciences cognitives et la philosophie de l'esprit, qui a plusieurs significations. En premier, il y a le problème de la ségrégation : il s'agit d'un problème pratique qui consiste à savoir comment le cerveau sépare les éléments dans des schémas complexes d'entrées sensorielles afin de les attribuer à des «objets» discrets. En second, il y a le problème de la combinaison : le problème de la façon dont les objets, l'arrière-plan et les caractéristiques abstraites ou émotionnelles sont combinées en une seule expérience.} \citep{lamme2003visual}. 

Une telle connectivité sur la base de processus de récurrences a été utilisée pour développer de nombreux modèles efficaces de traitement de la vision \citep{lamme2000distinct, lamme2003visual, lamme2006towards, pollen2003explicit, taylor2011review}. 
Cependant, la manière détaillée dont la conscience pourrait ainsi être créée n'est pas claire, car aucun modèle neuronal spécifique associé de la conscience et basé sur la récurrence, n'a été proposé afin de générer l'expérience consciente elle-même et de la tester. 
La récurrence neuronale dans la connectivité entre n'importe quelle paire de modules cérébraux est une partie importante du traitement cérébral. 
En effet, puisque la récurrence dans la connectivité semble être la règle plutôt qu'un cas isolé entre deux zones cérébrales, on s'attend donc à ce qu'elle soit importante pour être incluse dans la création de la conscience en tant que partie d'une architecture cérébrale globale \citep{taylor2011review}. 
Les diverses approches de la récurrence ont conduit à des modèles spécifiques de traitement des formes dans la vision et à une meilleure compréhension des difficultés de ces modèles qui ne supposent qu'un simple traitement visuel à action directe. 
Selon \cite{lamme2006towards}, la dichotomie neuronale entre «récurrent$=$conscient» et «feedforward$=$inconscient» donne lieu à des prédictions testables pour les expériences comportementales. 
Si, en effet, tous les processus récurrents partagent la caractéristique de la phénoménalité et la tendance à induire une plasticité synaptique, on peut prédire que l'apprentissage suivra les aspects phénoménaux des stimuli (par exemple, le son), plutôt que leurs caractéristiques physiques (par exemple, la fréquence), même lorsque ce qui est appris n'est pas rapportable. 

%%%%%%%%%%%%%%%%%%%%%%%%%%%%%%%%%%%%%%%%%%%%%%%%%%%%%%%%%%%%%%%%%%%%%%%%%%%%%%%
\subsubsection{Théorie de l'hypothèse du noyau dynamique}
\label{theoriehypothesenoyaudynamique}
%%%%%%%%%%%%%%%%%%%%%%%%%%%%%%%%%%%%%%%%%%%%%%%%%%%%%%%%%%%%%%%%%%%%%%%%%%%%%%%

Dans ce modèle théorique, l'apparition d'une scène consciente particulière constitue une «discrimination hautement informative» \citep{edelman2000reentry, tononi1998consciousness}. 
Cette théorie a été développée sur la base d'une approche par modélisation de la conscience en se focalisant plutôt sur ses aspects fonctionnels et informationnels à l'échelle neuronale. 
Plus précisèment, le modèle développé dans ce contexte s'appuie fondamentalement sur la théorie de la sélection des groupes neuronaux ou «darwinisme neuronal», qui représente une théorie sélective (\textit{i.e.}, de sélection neuronale) du développement et des fonctions du cerveau \citep{edelman1987neural, edelman2003naturalizing}. 

Certaines observations expérimentales ont fortement suggéré que le système thalamo-cortical est crucial pour la conscience. 
La décharge neuronale dans un sous-ensemble distribué de neurones thalamo-corticaux se produirait en réponse à tout percept conscient. 
Giulio Tononi et Gerald Edelman ont suggéré qu'un tel groupe de neurones dispose des propriétés suivantes \citep{edelman2000reentry, tononi1998consciousness} : 
\begin{itemize}
\item[$\bullet$] il est «fonctionnel», c'est-à-dire qu'il est constitué de neurones ayant des interactions plus fortes entre eux qu'avec le reste du système thalamo-cortical, 
\item[$\bullet$] il est hautement «différencié», c'est-à-dire qu'il admet un large répertoire de modèles d'activité différents possibles. 
\end{itemize}
Ces deux propriétés découlent des deux principales caractéristiques phénoménologiques de la conscience, c'est-à-dire sa nature unifiée et indécomposable, et sa grande informabilité.

Dans ce modèle théorique, une scène consciente est «intégrée» : chaque scène consciente est vécue d'une manière intrinsèque «tout-en-un». 
Mais, elle est également à la fois «différenciée» : chaque scène consciente est unique. 
Les expériences conscientes représentent ainsi ces discriminations hautement informatives --- intégrées et ségrégées --- \citep{edelman2003naturalizing} et afin de contribuer au contenu conscient, un neurone ou un groupe de neurones doit participer à ce groupe fonctionnel thalamo-cortical, appelé «noyau dynamique». 
Au sein de ce groupe, des interactions neuronales ré-entrantes produisent une succession d'états métastables\footnote{La métastabilité fait ici référence à la propriété de certains systèmes à exprimer des transitions entre des états de moindres stabilités vers des états de plus fortes stabilités du à des perturbations significatives de leurs variables d'états (\textit{i.e.}, les variables qui définissent leurs états).} différenciés mais unitaires \citep{edelman2000reentry, edelman2003naturalizing, tononi1998consciousness}. 
Ainsi, le terme noyau dynamique souligne la nature fluctuante et transitoire du groupe fonctionnel : sa configuration est supposée changer constamment (à une échelle temporelle de centaines de millisecondes) pour s'adapter au flux continu de la perception et de la conscience. 

Par conséquent, une erreur catégorique serait de discuter de l'endroit du cerveau où la conscience a lieu ou des neurones qui participent à sa génération, puisque la conscience est un processus dynamique qui se réorganise constamment et qui comprend toujours un groupe fonctionnel thalamo-cortical hautement différencié \citep{edelman2000reentry, edelman2003naturalizing, tagliazucchi2013sleep, tononi1998consciousness}.
L'une des caractéristiques notables du modèle de noyau dynamique de la conscience est l'existence d'une mesure quantitative de «complexité neuronale» $C_N$ \citep{tononi1994measure}, dont des valeurs élevées accompagneraient la conscience. 
La complexité neuronale mesure ainsi la quantité avec laquelle la dynamique d'un système neuronal est à la fois intégrée et différenciée. 
Il apparaît que la structure cérébrale disposant de boucles ré-entrantes à l'échelle du système thalamo-cortical semble idéalement adaptée pour produire une dynamique de grande complexité neurale \citep{sporns2000connectivity, sporns2000theoretical}.

%%%%%%%%%%%%%%%%%%%%%%%%%%%%%%%%%%%%%%%%%%%%%%%%%%%%%%%%%%%%%%%%%%%%%%%%%%%%%%%
\subsubsection{Théorie de l'information intégrée de la conscience}
\label{theorieinformationintegree}
%%%%%%%%%%%%%%%%%%%%%%%%%%%%%%%%%%%%%%%%%%%%%%%%%%%%%%%%%%%%%%%%%%%%%%%%%%%%%%%

La théorie de l'information intégrée de la conscience (TII) a été développée par l'équipe de G. Tononi et ses collaborateurs au cours des deux dernières décennies \citep{balduzzi2008integrated, tononi2003measuring, tononi2004information, tononi2011integrated, tononi2016integrated, oizumi2014phenomenology}. 
Fondamentalement issue de la théorie de l'hypothèse du noyau dynamique précédemment citée \citep{edelman2000reentry, tononi1998consciousness}, la TII a été développée principalement par G. Tononi comme une approche mathématique permettant de mesurer à la fois la quantité et la qualité de la conscience non seulement dans les organismes biologiques tels que nous-mêmes, mais aussi, du moins en principe, dans les dispositifs artificiels tels que les robots \citep{oizumi2014phenomenology, tononi2004information, tononi2008consciousness, tononi2012integrated, tononi2015consciousness}. 

Le cerveau présente un fonctionnement ségrégé à de multiples niveaux d'organisation et une activité cérébrale globalement intégrée à de multiples échelles allant du neurone seul jusqu'à des connexions à longue distance du réseau cérébral. 
Un tel couplage entre ségrégation fonctionnelle et intégration globale construit un socle fonctionnel pour les processus et mécanismes associés à la conscience. 
Selon la théorie de l'hypothèse du noyau dynamique, l'intégration à travers des interactions ré-entrantes cortico-corticales est un élément clé des expériences conscientes \citep{edelman2000reentry, edelman2003naturalizing, tononi1998consciousness}. 
En s'appuyant sur les idées issues de l'hypothèse du noyau dynamique et de la théorie de l'information \citep{tononi2003measuring, tononi2004information, tononi2008consciousness}, G. Tononi propose la thèse que le cerveau est sans doute le lieu de la conscience. 
Même si l'on est indisposé à considérer le cerveau comme un «générateur» de conscience, il est difficile de contester la preuve que le cerveau joue un rôle très important dans notre expérience consciente, dont la nature exacte reste un mystère. 
G. Tononi pose ainsi quatre questions fondamentales concernant le cerveau et la conscience :

\begin{itemize}
\item[$\bullet$] Pourquoi notre conscience est-elle générée par certaines parties du cerveau (les réseaux thalamo-corticaux) plutôt que par d'autres (par exemple le cervelet) ?
\item[$\bullet$] Pourquoi notre conscience est-elle «plus» dans l'état éveillé que dans l'état de rêve, de coma, ou autres états végétatifs et certains états anesthésiques ?
\item[$\bullet$] Quelles sont les conditions qui déterminent si un système est conscient ou non ?
\item[$\bullet$] Qu'est-ce qui contribue aux qualités spécifiques de notre expérience consciente ? 
\end{itemize} 

Pour répondre à ces questions et «mesurer» la conscience, G. Tononi utilise des outils issus de la théorie de l'information en se basant sur les propriétés phénoménologiques de la conscience. 
Les deux principales propriétés phénoménologiques qu'il met en avant sont la «différenciation» et l'«intégration».
Selon la TII, la conscience est définie comme la capacité d'un système à intégrer des informations \citep{balduzzi2008integrated, tononi2011integrated, tononi2016integrated, oizumi2014phenomenology}. 
Un système est dès lors considéré comme capable d'intégrer de l'information dans la mesure où il dispose d'un large répertoire d'états et que les états de chaque élément sont causalement dépendants des états des autres éléments. 
La capacité inhérente d'un système (par exemple, un réseau cérébral) à différencier un grand nombre d'états disponibles (propriété de différenciation), tout en étant indécomposable en un sous-ensemble de sous-systèmes indépendants (propriété d'intégration) constitue la conscience. 

La TII se base sur le fait qu'un état conscient est simultanément différencié --- chaque expérience est unique dans la mesure où elle exclut un très grand nombre de possibilités alternatives --- et intégré --- chaque expérience comprend une «scène» unifiée perçue selon une perspective particulière --- et donc que toute scène consciente constitue une discrimination hautement informative. 
Elle prédit donc que des degrés variables de conscience seront associés à des degrés variables de différenciation et d'intégration dans le système thalamo-cortical humain. 
La conscience est ainsi associée à un modèle d'activité neuronale différenciée dans des circuits thalamo-corticaux distribués et il est essentiel qu'une intégration par des interactions ré-entrantes entre ces circuits ait lieu.

Toutefois, la TII va au-delà de la théorie de l'hypothèse du noyau dynamique en généralisant ses principes afin d'établir une méthode fondée sur des propriétés visant à évaluer la conscience dans le monde physique. 
Pour cela, elle propose une nouvelle mesure de «quantité de conscience» générée par un système. 
Cette mesure canonique, $\Phi$, est définie comme la quantité d'information causalement efficace qui peut être intégrée à travers le maillon le plus faible d'un système \citep{tononi2003measuring, tononi2004information}. 
Une distinction importante entre $\Phi$ et $C_N$ est que $\Phi$ mesure les interactions causales dirigées au sein d'un système alors que $C_N$ mesure sa dynamique d'intégration et de différenciation. 
Selon la TII, la conscience telle que mesurée par $\Phi$ est caractérisée comme étant une «disposition» ou une «potentialité». 
Le contenu d'une scène consciente donnée est alors spécifié par la valeur, à un moment donné, des variables qui servent de médiateurs aux interactions informationnelles au sein du système. 
Une autre caractéristique de la TII est que $\Phi$ est proposée comme condition suffisante pour la conscience, de sorte que tout système ayant un $\Phi$ suffisamment élevé, qu'il soit biologique ou non biologique, serait conscient. 
La TII présente ainsi un cadre mathématique pour évaluer la qualité et la quantité de la conscience \citep{oizumi2014phenomenology, tononi2012integrated, tononi2015integrated, tononi2015consciousness} que nous aborderons plus en détail dans la Section \ref{integrationinformationmasquageinformationnel}. \\

Pour conclure, alors que les concepts fondateurs des neurosciences cognitives ont considéré que les systèmes perceptifs étaient soutenus par des zones cérébrales isolées, les théories récentes vont au-delà de ce point de vue. 
En effet, la conscience perceptive serait soutenue par des systèmes fonctionnels organisés sur la base de réseaux d'interactions dynamiques entre différentes zones cérébrales. 
En état d'éveil, le cerveau intègre en continu un flux d'information dynamique en provenance des objets physiques de son environnement qu'il doit traiter et se représenter comme objets cohérents. 
Ces objets physiques présentent un contenu informationnel --- sur la base de leurs caractéristiques structurelles --- qui leur est propre, et dont les caractéristiques informationnelles vont être traitées à des étapes spécifiques de la chaîne intégrative de traitement de l'information afin d'aboutir à une représentation mentale cohérente de l'objet perçu. 
Les différents modèles de conscience et de perception consciente présentés s'ancrent sur des modèles empirico-théoriques et sur des résultats provenant d'études de neuroimagerie structurelle et fonctionnelle. 
De telles études ont permis à ces différents modèles de tisser des liens explicatifs entre propriétés cérébrales et propriétés phénoménales de la conscience et de donner lieu à des hypothèses sur l'activité génératrice de tels liens. 
Dans certains des modèles de la conscience présentés, par exemple, la TII ou encore l'hypothèse du noyau dynamique, on retrouve ces notions de dynamique et d'intégration de l'information à l'échelle cérébrale ainsi que des suggestions de représentations de comment les processus de différenciation et de ségrégation entre les états cérébraux émergent pour aboutir à une perception consciente cohérente. 
En effet, au delà de son aspect purement dynamique, l'activité cérébrale consiste en une activité globalement intégrée et fonctionnellement ségrégée à plusieurs niveaux s'étendant du neurone jusqu'aux réseaux d'aires cérébrales. 
De cette manière, les modèles de la conscience ont permis, permettent et permettront l'étude des corrélats neuronaux de la perception auditive consciente et de la dynamique inhérente à ses mécanismes et processus à l'échelle cérébrale. 
De tels modèles couplés aux paradigmes d'analyse de la scène auditive comme celui du streaming auditif et du MI multi-tonalités, offrent donc l'opportunité de permettre l'étude de la dynamique de la construction du percept conscient associé à un stimulus auditif et de ses corrélats neuronaux. 

%%%%%%%%%%%%%%%%%%%%%%%%%%%%%%%%%%%%%%%%%%%%%%%%%%%%%%%%%%%%%%%%%%%%%%%%%%%%%%%
\section{Perception auditive consciente et ses corrélats neuronaux}
\label{perceptionauditiveconscienteetcorrelats}
%%%%%%%%%%%%%%%%%%%%%%%%%%%%%%%%%%%%%%%%%%%%%%%%%%%%%%%%%%%%%%%%%%%%%%%%%%%%%%%

%%%%%%%%%%%%%%%%%%%%%%%%%%%%%%%%%%%%%%%%%%%%%%%%%%%%%%%%%%%%%%%%%%%%%%%%%%%%%%%
\subsection{Étude de la perception auditive consciente}
\label{perceptionauditiveconsciente}
%%%%%%%%%%%%%%%%%%%%%%%%%%%%%%%%%%%%%%%%%%%%%%%%%%%%%%%%%%%%%%%%%%%%%%%%%%%%%%%

La perception auditive consciente consiste en une capacité de l'individu à construire de manière consciente un percept auditif qui est lié à une stimulation acoustique externe. 
Elle implique des processus d'analyse de la scène auditive, de ségrégation des flux auditifs et d'organisation perceptive auditive. 
La mise en œuvre dans l'architecture neuronale du système auditif de ces différents processus est loin d'être aussi bien comprise que les phénomènes perceptifs eux-mêmes \citep{albouy2013behavioral, bee2008cocktail, bidet2009neurophysiological, micheyl2007role, shamma2010behind, shamma2011temporal, snyder2007toward}.
Pour aller au-delà de ces aspects et permettre une compréhension plus approfondie des substrats neuronaux de la conscience, il est nécessaire de déterminer quelles représentations mentales, dans le flux du traitement de l'information, atteignent ou non la conscience. 

Précédemment, deux paradigmes expérimentaux ont été décrits comme couramment utilisés afin d'étudier la perception auditive consciente : le paradigme de streaming auditif (Figure \ref{fig:chap2paradigmestreamingauditif} A) et le paradigme de MI multi-tonalités (Figure \ref{fig:chap2paradigmestreamingauditif} B). 
Ils permettent la production d'un percept auditif bistable chez l'auditeur, pouvant donner lieu à une activité neuronale différenciée. 
Cette différence dans l'activité neuronale entre stimulus perçu et non-perçu est hautement informative sur les processus et mécanismes en lien avec l'accès conscient du stimulus. 
En permettant des variations paramétriques des stimuli, les deux paradigmes permettent donc de produire des réactions comportementales différentes et de générer des schémas d'activité neuronale variants. 

Les études de neuroimagerie portant sur la perception auditive consciente au moyen de ces paradigmes ont rendu une caractérisation plus fine de la différence des signaux neuronaux associés à la construction de percepts auditifs.
La relation entre le masquage auditif et la ségrégation des flux auditifs a notamment été observée dans des études psychophysiques ainsi que lors d'études d'imagerie non-invasives sur des sujets humains \citep{sussman2001auditory, winkler2003newborn, micheyl2007hearing, sheft2008method, snyder2012attention}. 
D'autres études ont également utilisé des techniques de neuroimagerie fonctionnelle pour étudier la perception auditive chez l'humain \citep{cusack2005intraparietal, deike2004auditory, deike2010active, gutschalk2005neuromagnetic, gutschalk2007human, kondo2009involvement, schadwinkel2010activity, schadwinkel2011transient, snyder2006effects, snyder2007sequential, sussman1999investigation, wilson2007cortical}. 
Les techniques EEG/MEG et d'IRMf (Imagerie par Résonance Magnétique fonctionnelle) ont été utilisées comme méthodes non-invasives pour étudier les corrélats neuronaux de la ségrégation des flux auditifs \citep{akram2015neural, denham2006role, micheyl2007hearing, melcher2009auditory, gutschalk2014functional}. 

Au moyen de ces différents outils de neuroimagerie fonctionnelle, il est devenu possible d'étudier plus en détails la dynamique associée à la ségrégation des flux auditifs chez l'humain et ainsi de mieux comprendre les corrélats neuronaux de la perception auditive consciente à travers l'accès conscient d'un stimulus auditif. 
L'étude de la perception auditive consciente à travers ses corrélats neuronaux a premièrement été réalisée à partir d'études des évènements et potentiels du signal électrophysiologique liés à la perception auditive. 

\begin{figure*}[!t]
\center
\includegraphics[width=0.8\columnwidth]{illustrations/utilisé/auditorysceneanalysis.jpg}
\caption[Spectrogrammes de deux types de stimuli utilisés pour étudier l'analyse de la scène auditive]{Spectrogrammes de deux types de stimuli pour l'étude de l'analyse de la scène auditive. (A) Paradigme de streaming auditif ABA-ABA : lorsque deux tonalités qui diffèrent peu en fréquence (panel du bas) sont présentés aux sujets, ces derniers déclarent entendre un seul flux. À l'inverse, lorsque la différence de fréquence est importante et que les sons graves et aigus sont désynchronisés (panel du haut), les auditeurs déclarent entendre deux flux réguliers. (B) Paradigme de MI multi-tonalités : bien que les tonalités cibles bleues soient faciles à discriminer visuellement du fond multi-tonalités, les auditeurs ne les entendent pas toujours, donnant ainsi un percept différent en fonction de si le flux de tonalités cible a été ségrégé ou non de celui du masqueur multi-tonalités. Adapté de \cite{dykstra2013psychophysics}.} 
\label{fig:chap2paradigmestreamingauditif}
\end{figure*}

%%%%%%%%%%%%%%%%%%%%%%%%%%%%%%%%%%%%%%%%%%%%%%%%%%%%%%%%%%%%%%%%%%%%%%%%%%%%%%%
\subsection{Corrélats neuronaux de la perception auditive consciente}
%%%%%%%%%%%%%%%%%%%%%%%%%%%%%%%%%%%%%%%%%%%%%%%%%%%%%%%%%%%%%%%%%%%%%%%%%%%%%%%

\subsubsection{Potentiels liés à la perception auditive consciente}

Une approche commune à l'étude des corrélats neuronaux de la perception auditive se base sur des techniques d'analyses de potentiels liés à l'évènement auditif. 
L'EEG humain spontané se produit en l'absence de stimuli sensoriels spécifiques, mais peut être facilement modifié par de tels stimuli. 
Les ERP («Event-Related Potentials») sont des «potentiels électriques générés par le cerveau qui sont liés à des événements internes ou externes spécifiques tels que des stimuli, des réponses ou des décisions» \citep{luck2014introduction}. 
Les ERP se produisent à des périodes de latence longues par rapport aux stimuli et sont généralement associés à un état endogène du cerveau. 
Un seul ERP, usuellement enregistré au moyen d'un EEG du cuir chevelu, a une amplitude de 0,5 à 15 $\mu$V, beaucoup plus faible que l'EEG de fond spontané (100 $\mu$V) le rendant par conséquent non-visible à l'œil nu dans le signal EEG. 
Afin de démêler et de révéler l'ERP spécifique pertinent de l'EEG de fond non pertinent (\textit{i.e.}, spontané), la technique d'ERP est basée sur le principe mathématique de sommation. 
Cela consiste à faire la moyenne de centaines de répétitions synchronisées dans le temps d'une même condition expérimentale afin d'atténuer les activités qui ne sont pas liées à l'événement. 
La forme d'onde résultante contient une série de pics positifs et négatifs, appelés composants, censés refléter l'activité --- les potentiels post-synaptiques --- dans les générateurs cérébraux sous-jacents. 
Dans une expérience, les ERPs consistent en une forme d'onde contenant une série de composantes caractéristiques, se produisant généralement moins de 500 millisecondes après la présentation du stimulus \citep{king2014characterizingthesis}. 
Ces composants sont désignées par des acronymes (\textit{e.g.}, «Error Related Negativity», ERN) ou par une lettre indiquant la polarité (N=négatif, P=positif), suivie d'un nombre indiquant la latence en millisecondes à partir du début du stimulus. 
Par exemple, l'onde N100 est un pic négatif survenant environ 100 ms après le stimulus et l'onde P300 un pic positif survenant environ 300 ms après le stimulus.
L'amplitude, la latence et la covariance (dans le cas de sites d'électrodes multiples) de chaque composante peuvent alors être étudiées, en relation avec une tâche cognitive ou sans tâche \citep{nunez2007electroencephalogram}. 
Alors que certains potentiels sont facilement obtenus par la répétition de stimuli (\textit{e.g.}, le potentiel N100, suscité par la perception de stimuli auditifs), d'autres potentiels sont suscités par des paradigmes plus complexes. 

\begin{figure*}[!t]
\center
\includegraphics[width=0.7\columnwidth]{illustrations/utilisé/figureEEGerpwaveforms.jpg}
\caption[Extraction de la forme d'onde ERP de l'EEG]{Extraction de la forme d'onde ERP de l'EEG en cours d'enregistrement. (Haut) La réponse spécifique ERP à chaque stimulus est trop faible par rapport à l'EEG global. Les segments ERPs suivants chaque stimulus sont alors extraits. (Bas) Les stimuli ($1,\ldots,N$) sont présentés et on isole les ERPs de l'EEG global pour en faire la moyenne afin de créer la forme d'onde «grande-moyenne» de l'ERP. Les composantes ERPs du signal EEG deviennent alors visibles et on peut ensuite recueillir leur amplitude et leur latence. Adapté de \cite{luck2000event}.}
\label{fig:figure5extractionerp}
\end{figure*}

Une des premières études EEG en audition utilisait une tâche de détection dans laquelle les sujets devaient détecter des tonalités présentées à leur seuil de détection \citep{hillyard1971evoked}. 
Trois sujets pour lesquels l'EEG a été enregistré à partir d'une électrode centrale (\textit{i.e.}, le vertex) devaient détecter des sons à hauteur de 50\% de probabilité de présence qui étaient joués avec du bruit blanc en arrière-plan. 
Les grandes moyennes ERP des signaux EEG ont été calculées à partir d'essais de présence des sons (une pour les sons détectés et une pour les sons non-détectés). 
L'ERP pour les sons détectés a montré une onde N100 pour un sujet et une onde P300 pour les trois sujets. 
Plus tard, ces résultats ont été reproduits et étendus à d'autres sujets dans plusieurs études \citep{parasuraman1980brain, paul1972evoked, squires1973vertex}. 
\cite{paul1972evoked} ont enregistré l'EEG au niveau du vertex, au fur et à mesure que les tonalités étaient jouées au seuil de détection. 
Ils ont ainsi trouvé que l'amplitude de la P300 était positivement corrélée à la confiance de la détection. 
\cite{squires1973vertex} ont également enregistré l'EEG au niveau du vertex, en mesurant la confiance de la détection à l'aide d'une échelle de notation, au fur et à mesure que les tonalités au seuil de détection étaient jouées. 
Ils ont observé que la N100 et la P300 étaient plus grandes et avaient une latence plus courte pour les tonalités qui ont été détectées avec une confiance élevée par rapport à celles avec une confiance faible. 
Les auteurs ont aussi rapporté que la N100 était liée à la force du signal conformément à la TDS \citep{macmillan2004detection, squires1973vertex}. 
En conséquence, bien que le signal était physiquement identique à chaque répétition d'un signal, la force perçue du signal par l'auditeur variait. 
Certains sons étaient perçus comme étant plus forts que d'autres, et l'amplitude et la latence de la N100 étaient liées à l'intensité subjective perçue du son \citep{picton1977evoked, squires1973vertex}. 
Pour déterminer si l'amplitude de la N100 augmentait avec la puissance du signal, \cite{squires1973vertex} ont inclus une tâche passive dans laquelle l'intensité des tonalités était variée. 
Les résultats ont confirmé que la N100 devenait plus grande et avait une latence plus précoce lorsque le niveau de tonalité était augmenté. 
Les tonalités détectées avec la plus grande confiance dans la tâche active ont provoqué une amplitude et une latence de la N100 égales à celles provoquées par une tonalité plus forte de $8$~dB dans la tâche passive. 
Cela pourrait dès lors suggérer que l'intensité subjective perceptive peut être reflétée par l'amplitude et la latence de la N100 \citep{eklund2019electrophysiological}. 

Plus tard, \cite{parasuraman1980brain} ont enregistré l'EEG sur les électrodes centrales pendant que les sujets effectuaient une tâche de détection où des sons de hauteurs différentes étaient joués au seuil de la détection. 
Dans cette tâche, les sujets devaient d'abord fournir des scores de confiance, puis identifier la hauteur du son. 
Les résultats ont montré que, quelle que soit la hauteur, les valeurs de la N100 et de la P300 étaient plus élevées pour les sons détectés avec un niveau de confiance élevé que pour ceux détectés avec un niveau de confiance faible. 
Les tonalités détectées et correctement identifiées ont montré une P300 plus grande que les tonalités détectées qui ont été incorrectement identifiées. 
En mesurant la confiance, \cite{parasuraman1980brain} ont ainsi trouvé la même relation entre confiance et P300 que dans les études précédentes \citep{paul1972evoked, squires1973vertex}. 
De plus, comme la N100 n'a pas été affectée par la performance de détection, elle apparaissaît être liée à la force du signal perçu. 

% La négativité de discordance (MMN, «mismatch negativity») est une composante ERP liée à un évènement non-attendu (non-pertinent) à l'écoute d'un signal auditif \citep{naatanen2001perception, naatanen2011auditory}. 
% \cite{sussman1999investigation} ont été les premiers à utiliser le paradigme de streaming auditif en mesurant simultanément l'activité neurale à la recherche de cette composante. 
% Ils ont enregistré les ERPs alors que les sujets devaient ignorer des séquences de sons alternés qui variaient dans leur rythme de présentation. 
% Lorsque la vitesse de présentation était lente --- condition connue pour produire la perception de sons individuels entendus successivement (fusion) --- des sons déviants peu fréquents n'évoquaient pas de MMN. 
% Lorsque la vitesse de présentation était rapide --- condition connue pour produire la perception de flux auditifs séparés (fission) --- les mêmes sons déviants peu fréquents évoquaient une réponse MMN. 
% Cela indiquait que la détection de changement se produisait au sein, mais pas entre, les flux auditifs \citep{sussman1999investigation}. 
% Plus tard, \cite{sussman2007role} ont enregistré les ERPs évoquées par des trains de tonalités tandis que l'attention des auditeurs était simultanément détournée vers une tâche difficile de détection du changement d'intensité d'un bruit. 
% La ségrégation était favorisée par l'augmentation de la séparation fréquentielle entre les sous-ensembles de tonalités, dans lesquels un certain nombre de tonalités d'intensité différente étaient présentes. 
% Lorsque la séquence était perçue comme deux flux séparés, cela provoquait une MMN dans la réponse neurale \citep{sussman2007role}, suggèrant qu'une attention soutenue n'est pas nécessaire pour initier la ségrégation des flux auditifs. 

De cette manière, ces premières études sur la perception auditive consciente ont montré que des composantes ERPs du signal EEG étaient observables lors de la détection de tonalités lors de tâches perceptives. 
Un message important que l'on tire de ces études ERPs est que les composantes permettent, à un certain degré, d'être corrélées avec la perception conscience par les sujets de tonalités cibles dans une tâche de détection ainsi que leur niveau de confiance quant à cette détection. 
Néanmoins, ce que l'on garde à l'esprit ici, c'est la nécessité de procéder à la sommation sur l'ensemble des essais pour obtenir in fine des formes d'onde grande-moyenne liées à la présentation du stimulus. 
De façon intéressante, ces études ont ensuite ouvert la voie pour analyser plus spécifiquement les localisations en lien avec les mécanismes et processus associés à la perception auditive consciente. 

\subsubsection{Localisations et évènements liés à la perception auditive consciente}

De nombreuses études de neuroimagerie ont révélé un rôle du cortex auditif dans l'analyse de la scène auditive \citep{fishman2001neural, gutschalk2005neuromagnetic, micheyl2005perceptual, middlebrooks2013spatial} et des corrélats neuronaux de la ségrégation des flux auditifs ont été trouvés au niveau des premiers stades de traitement du cortex auditif \citep{dykstra2011neural, dykstra2016neural, gutschalk2008neural, konigs2012functional}. 
Lors de la perception consciente, l'activité de l'IRMf au sein du cortex auditif et des régions préfrontales était augmentée lors de transitions perceptives \citep{knapen2011role}. 
Ces transitions perceptives ont également été étudiées en utilisant des enregistrements par microélectrodes chez des animaux de laboratoire \citep{bee2004primitive, bee2010neural, fay1998auditory, fay2000spectral, fishman2001neural, fishman2004auditory, itatani2009auditory, kanwal2003neurodynamics, kashino2012functional, micheyl2005perceptual, pressnitzer2008perceptual, schul2006auditory}. 
Des centres neuronaux ont aussi été impliqués lors des transitions à d'autres niveaux de la hiérarchie de traitement, tels que le noyau cochléaire \citep{pressnitzer2008perceptual} et le sillon intra-pariétal \citep{cusack2005intraparietal, teki2011brain}. 

Des corrélats seraient également disponibles dans plusieurs zones supra-modales du cerveau \citep{cusack2005intraparietal} et d'autres études ont aussi rapporté une plus grande activité au sein d'un réseau fronto-pariétal durant la perception auditive consciente \citep{eriksson2007similar, eriksson2017activity, giani2015detecting}. 
Les résultats de plusieurs de ces études suggèrent que des corrélats de la bistabilité auditive peuvent être trouvés dans les cortex auditif primaire et secondaire \citep{gutschalk2005neuromagnetic, gutschalk2008neural}. 
Cependant, trop peu de données sont encore actuellement disponibles sur le rôle des zones situées en dehors du cortex auditif dans l'analyse de la scène auditive \citep{albouy2013behavioral, cusack2005intraparietal, kondo2009involvement, pressnitzer2008perceptual}. 
Le degré auquel différents réseaux et différentes aires cérébrales seraient nécessairement requis pour la perception auditive consciente reste actuellement peu documenté \citep{dykstra2017roadmap, eklund2019electrophysiological, wiegand2018cortical}. 
Globalement, la perception consciente semble reposer sur l'activation d'un réseau cérébral fronto-temporo-pariétal chez l'humain \citep{demertzi2013consciousness}. 

\begin{figure*}[!t]
\center
\includegraphics[width=\columnwidth]{images_articles/Gutschalk_2008_1.jpg}
\includegraphics[width=\columnwidth]{images_articles/Gutschalk_2008_2.jpg}
\caption[Corrélats neuraux ERPs de la conscience perceptive auditive dans le MI]{(Haut Gauche et Milieu) Spectrogramme de stimulus de MI, en 18 bandes de fréquences (allant de 239 Hz à 5000 Hz). La cible (noire) était un tonalité répétitive (ici 1000 Hz) et le masqueur présentait un intervalle entre ses tonalités d'une durée moyenne de 200 ms (gauche) ou 800 ms (droite). (Haut Droite) Probabilités moyennes de détection et de fausses détections sur le temps parmi les auditeurs pour les 200 ms (cercles remplis) et les 800 ms (cercles ouverts). (Bas Gauche) Localisation des dipôles ARN dans le cortex auditif. (Bas Droite) Grande-moyenne des formes d'onde ERP. On observe à droite, la composante ARN entre 50 et 250 ms révélée pour les cibles détectées. Adapté de \cite{gutschalk2008neural}.}
\label{fig:chap2gutschalk2008}
\end{figure*}

Dans une étude, \cite{gutschalk2008neural} ont combiné un paradigme de MI avec des enregistrements MEG chez l'humain pour étudier le rôle du cortex auditif dans le MI. 
La tâche était de détecter un flux de tonalités cibles se répétant régulièrement dans un contexte de tonalités masquantes tirées aléatoirement en temps et en fréquence. 
Les stimuli étaient similaires à ceux utilisés dans les études précédentes sur le MI \citep{neff1987masking, kidd2003multiple} sauf que les tonalités du masqueur n'étaient pas synchronisées avec les tonalités de la cible. 
Les auteurs ont comparé les réponses MEG provoquées par les tonalités cibles détectées à celles des tonalités non détectées. 

Une forte négativité en provenance du cortex auditif secondaire a été observée dans un intervalle de 50 à 250 msec suivant l'apparition des tonalités cibles lorsque l'auditeur détectait ces tonalités (Figure \ref{fig:chap2gutschalk2008}). 
Cette réponse longue latence était forte lorsque les auditeurs ont reporté entendre la cible mais n'était pas mesurable lorsque les auditeurs ont échoué à détecter la cible. 
Les auteurs ont appelé cette forme d'onde la «négativité reliée à la conscience» (ARN, «Awareness Related Negativity»). 
L'ARN apparaitraît ainsi reliée à la conscience perceptive puisqu'elle n'était pas observée pour des cibles non-détectées. 
Les auteurs ont alors suggéré que le MI peut apparaître entre les étapes de traitement précoce et tardif (50-250 msec) dans le cortex auditif. 
Comme il a été observé que l'amplitude de la N100 est plus grande lorsque les tonalités étaient détectées par les sujets \citep{squires1973vertex}, les auteurs ont également suggéré que la conscience auditive proviendrait d'une étape de traitement similaire à celle de la N100 dans le cortex auditif en émergeant du cortex auditif plutôt que du tronc cérébral ou de structures corticales supra-modales de niveaux supérieurs. 

Cependant, dans l'étude de \cite{gutschalk2008neural}, la cartographie exacte de l'activité MEG à des parties distinctes du cortex auditif était limitée et ne pouvait dissocier les générateurs dans les aires cérébrales. 
Ainsi, cela ne permettait pas de savoir si la différence d'activité observée entre les cibles détectées et non-détectées était uniquement confinée aux aires auditives secondaires, ou si elle était déjà observée dans le cortex auditif primaire. 
Afin de déterminer l'étape dans le cortex auditif où l'activité covarie avec la perception auditive consciente, une autre étude \citep{wiegand2012correlates} a combiné IRMf et MEG en adaptant le paradigme de MI employé par \cite{gutschalk2008neural}. 
Ils ont observé que l'activité dans le cortex auditif était plus forte pour les cibles détectées que pour les cibles non-détectées (Figure \ref{fig:chap2wiegandgutschalk2012} Gauche). 
Une activation généralisée a été observée dans le cortex auditif pour les tonalités cibles détectées, reproduisant les résultats de \cite{gutschalk2008neural}.
La comparaison entre les cibles détectées et non détectées a également révélé une activité confinée au gyrus de Heschl médian, siège du cortex auditif primaire. 
Cela suggère que l'activité liée à la perception consciente implique le cortex auditif primaire et n'est pas limitée à l'activité dans les zones secondaires. 
Cette étude a donné une preuve que l'activité dans le cortex auditif primaire est reliée à la perception auditive consciente et ne comprend pas seulement une représentation des paramètres du stimulus physique. 

Dans une autre étude MEG sur le MI, basée sur une analyse par modélisation causale dynamique\footnote{La modélisation causale dynamique est une méthode d'analyse et d'interprétation des données issue de la neuroimagerie fonctionnelle. 
Le but de la modélisation causale dynamique est de déduire l'architecture causale de systèmes dynamiques couplés ou distribués \citep{marreiros2010dynamic}. 
Il s'agit d'une procédure de comparaison bayésienne de modèles sur la façon dont les données de séries temporelles ont été générées. 
Elle a été développée et appliquée principalement pour estimer le couplage entre les régions du cerveau et la façon dont ce couplage est influencé par les conditions expérimentales.}, \cite{giani2015detecting} ont montré que l'ARN proviendrait plus spécifiquement de processus récursifs dans le cortex auditif. 
Lors de chaque essai, les participants ont indiqué s'ils avaient détecté une paire de tonalités présentées de manière séquentielle (la cible) qui était intégrée dans un masqueur multi-tonalités. 
L'ARN n'est apparue que pour la deuxième des deux tonalités, c'est-à-dire une tonalité avant le rapport comportemental des participants sur la cible. 
Au contraire, elle a été observée deux tonalités avant le rapport comportemental dans l'étude précédente de \cite{gutschalk2008neural} qui a utilisé une série de $12$ tonalités. 
Les auteurs ont analysé l'activité MEG pour les détections et les omissions, séparément pour la première et la deuxième tonalité de la paire de cibles. 
Seule la seconde tonalité a provoqué la négativité à $150$~ms, suggérant ainsi une ségrégation de la paire de tonalités par rapport au masqueur multi-tonalités. 
Une P300 a été la seule composante qui a été amplifiée de manière significative pour les deux tonalités, lorsqu'elles ont été détectées, montrant ainsi son lien avec la conscience perceptive \citep{giani2015detecting}. 
L'analyse par modélisation causale dynamique a permis d'indiquer que l'ARN sous-tendant la ségrégation des flux auditifs était principalement due à des changements dans la connectivité intrinsèque des cortex auditifs. 
En revanche, la réponse P300 comme signature de la conscience perceptive, reposait sur les interactions entre les cortex pariétal et auditif. 
Une détection réussie reposerait ici sur un traitement récurrent entre les zones corticales auditives et pariétales d'ordre supérieur. 
Il apparaît ainsi que la conscience perceptive dans le MI émerge dans une cascade complexe de traitement neuronal qui s'accumule sur plusieurs tonalités cibles. 
Un lien a donc été établi entre conscience perceptive auditive et ARN, composante située entre $50$-$250$~ms, localisée dans le cortex auditif. 
Cette étude vient appuyer de manière empirique les modèles théoriques de la conscience basés sur le traitement récurrent de l'information au sein de zones cérébrales localisées \citep{lamme2000distinct, lamme2003visual, lamme2006towards}. 

\begin{figure*}[!t]
\center
\includegraphics[width=0.49\columnwidth]{images_articles/Wiegand_Gutschalk_2012_1.jpg}
\includegraphics[width=0.49\columnwidth]{images_articles/Wiegand_Gutschalk_2012_5.jpg}
\caption[Formes d'ondes MEG évoquées en réponse à des tonalités détectées sous MI]{Adaptation du paradigme de MI par \cite{wiegand2012correlates}. (Gauche) Spectrogramme du stimulus de MI. (Droite) Formes d'ondes MEG pour chacune des tonalités cibles détectées (ligne pleine) et non détectées (ligne pointillée). Adapté de \cite{wiegand2012correlates}.}
\label{fig:chap2wiegandgutschalk2012}
\end{figure*}

Afin de localiser encore plus finement la perception conscience au sein des aires cérébrales, \cite{dykstra2016neural} ont combiné des enregistrements EEG intracrâniens (iEEG) invasifs chez des patients épileptiques en neurochirurgie avec une tâche de MI. 
L'objectif était de mieux caractériser les corrélats neuronaux de la conscience perceptive auditive avec des reports perceptifs, essai par essai, de sons cibles intégrés dans des masqueurs multi-tonalités aléatoires. 
Les résultats ont montré que les cibles détectées ont provoqué une activité focale précoce dans le cortex auditif et dans le gyrus temporal supérieur, zone contenant le gyrus de Heschl. 
Cette activité comprenait à la fois une forte activité dans la bande de fréquence gamma ainsi qu'une forte négativité entre $100$ et $200$~ms. 
Cette activité était largement réduite voir absente pour les cibles non-détectées. 
Une large composante P300 a également été révélée avec des générateurs dans le cortex temporo-frontal et temporo-latéral. 
Selon \cite{dykstra2016neural}, il n'est pas évident de savoir si ces réponses reflètent bien une perception auditive consciente, par opposition à ce qui serait plutôt lié à des processus de traitements pré/post-perceptuel sur la base de l'orientation de l'attention. 
En effet, la distinction entre attention et conscience soulignée dans les modèles de traitements par récurrence \citep{lamme2000distinct, lamme2003visual, lamme2006towards}, rappelle qu'il n'est pas forcément simple de comprendre le substrat neuronal spécifique de la conscience perceptive sur la base de ce type de traitements pré/post-perceptuel qui peuvent lui être associés. 

\subsubsection{Processus «top-down» / «bottom-up» et orientation de l'attention}

Comme l'attention est un terme général qui désigne de multiples processus cognitifs \citep{petersen2012attention}, la relation entre attention et ARN reste obscure. 
Selon \cite{ward2004attention}, l'attention est le processus par lequel les organismes sélectionnent un sous-ensemble d'informations disponibles sur lequel ils se concentrent pour les traiter et les intégrer. 
On considère généralement que l'attention comporte au moins trois aspects : l'orientation, le filtrage et la recherche, et qu'elle peut se concentrer sur une seule source d'information ou être répartie entre plusieurs. 
La manière la plus simple de choisir entre plusieurs stimuli est d'orienter nos récepteurs sensoriels vers un ensemble de stimuli et de les éloigner d'un autre. 
L'attention agit comme un filtre, en extrayant davantage d'informations des stimuli sur lesquels on insiste et en supprimant l'extraction d'informations des stimuli sur lesquels on n'insiste pas. 
L'attention est étroitement liée à la conscience comme toutes deux sont intégratives, mais aussi sélectives. 
L'attention est généralement conceptualisée comme l'augmentation du rapport signal/bruit à la fois par l'inhibition du traitement des stimuli sans attention et par le renforcement du traitement des stimuli avec attention. 

Dans leur étude, \cite{gutschalk2008neural} ont également suggéré que les ressources de traitement limitées dans le cortex auditif seraient une cause de masquage, ne permettant donc pas l'accès à la conscience de la cible auditive. 
Les processus attentionnels «top-down» ou «descendants» sont des processus conduits par la tâche ou dépendants du contexte, tels que l'attention sélective.
Les processus «bottom-up» ou «ascendants» sont des processus conduits par la saillance des caractéristiques des stimuli et sous-tendant la ségrégation des flux. 
Une fois qu'une certaine charge de traitement est atteinte et que des mécanismes ascendants bottom-up et des mécanismes descendants top-down interagissent, cela peut biaiser la compétition entre flux auditifs. 
Dans la modalité auditive, orienter volontairement l'attention conduit à renforcer l'activité neurale à la fois dans le cortex auditif mais également dans les cortex frontal et pariétal \citep{eklund2019electrophysiological}. 
Préalablement, certaines études avaient montré qu'indicer l'attention vers ou loin d'une cible auditive en module la perception consciente \citep{leek1991informational, richards2004cuing}. 

Dans une autre étude, \cite{elhilali2009interaction} ont cherché à manipuler l'attention des participants sur les différentes caractéristiques de la scène auditive, 
Ils ont étudié comment les processus top-down et bottom-up interagissent pour permettre l'analyse d'une scène auditive complexe en utilisant un paradigme de MI. 
Dans une première tâche, les participants devaient détecter un déviant fréquentiel dans le signal cible («tâche cible»). 
Dans une seconde tâche, ils devaient détecter une élongation temporelle des tonalités du signal masqueur («tâche masqueur», Figure \ref{fig:chap2elhilali2009}). 
Cette étude a montré trois principaux résultats. 

\begin{figure*}[!t]
\center
\includegraphics[width=0.48\columnwidth]{images_articles/Elhilali_2009_1.jpg}
\includegraphics[width=0.48\columnwidth]{images_articles/Elhilali_2009_2.jpg}
\includegraphics[width=0.48\columnwidth]{images_articles/Elhilali_2009_3.jpg}
\includegraphics[width=0.48\columnwidth]{images_articles/Elhilali_2009_4.jpg}
\caption[Interactions «bottom-up» et «top-down» sous MI]{(Haut Gauche) Spectrogramme schématique de stimulus employé par \cite{elhilali2009interaction}. Dans la «tâche cible», les participants devaient détecter un décalage fréquentiel ($\Delta F$) déviant dans les tonalités cibles répétitives. Dans la «tâche masqueur», les participants devaient détecter une élongation temporelle ($\Delta T$) soudaine des tonalités du masqueur. (Haut Droite) Résultats des performances comportementales pour les tâches cibles et masqueur, mesurées par l'indice de performance d' en fonction de la largeur de la zone fréquentielle protégée. (Bas Gauche) Réponse neurale normalisée au rythme cible par participant (barres individuelles) et par tâche (rouge pour la tâche cible, bleu pour la tâche masqueur). La réponse neurale était calculée comme le rapport entre la puissance de la réponse neurale au rythme de la cible ($4$~Hz) et la puissance moyenne de l'activité neurale de fond. Le fond rose clair (respectivement, bleu clair) représente la moyenne sur les participants pour la tâche cible (respectivement, tâche de masqueur). (Bas Droite) Réponse neurale normalisée au rythme cible, et performance comportementale, en fonction du temps passé dans la tâche cible, moyennée sur les participants. Adapté de \cite{elhilali2009interaction}.}
\label{fig:chap2elhilali2009}
\end{figure*}

Premièrement, les processus attentionnels interagissent avec les paramètres physiques du stimulus et peuvent agir pour renforcer des caractéristiques particulières qui seraient attendues dans la scène auditive. 
Selon les auteurs, les résultats montrent que l'attention auditive module fortement la représentation neuronale soutenue de la cible. 
Plus précisément, l'attention soutenue était corrélée à une augmentation maintenue du signal neuronal variant dans le temps. 
La représentation neuronale modulée était localisée dans le cortex auditif primaire, fournissant une preuve supplémentaire de l'implication des mécanismes neuronaux du cortex auditif primaire dans l'analyse de la scène auditive.
Deuxièmement, les données ont révélé que le renforcement de la saillance acoustique (dirigée par les processus ascendants), entraînant une augmentation de la détectabilité perceptive, était également corrélée à une augmentation de la puissance et de la cohérence soutenues du signal neuronal. 
Dans ce cas, l'augmentation du signal neuronal se produisait indépendamment de la tâche exécutée, mais avec des conséquences comportementales différentes : dans la tâche cible, elle entraînait une augmentation de la performance, mais dans la tâche masqueur, une diminution (par interférence). 
Troisièmement, les données ont montré un biais de l'hémisphère gauche dans la représentation neurale de la cible, pour la tâche cible, suggérant un rôle fonctionnel de l'hémisphère gauche dans l'attention sélective. 
En revanche, pour la tâche masqueur, le biais hémisphérique dans la représentation neurale de la cible (non-détectée) était inversé vers la droite. 

Ces résultats mettent en évidence une interaction finement couplée entre la représentation neurale au plus bas niveau et la représentation cognitive au niveau supérieur des objets auditifs dans la ségrégation des flux auditifs. 
Dans l'ensemble, un résultat significatif de cette étude est qu'elle démontre non seulement un couplage fort entre la représentation neuronale mesurée d'un signal et sa manifestation perceptive, mais aussi qu'elle place la source de ce couplage au niveau du cortex sensoriel. 
Ainsi, la représentation neuronale du percept est encodée à l'aide des mécanismes du cortex sensoriel axés sur les caractéristiques, mais elle est façonnée de manière durable par des projections axées sur l'attention provenant d'aires de niveau supérieur. 
Un tel cadre pourrait sous-tendre des mécanismes généraux d'organisation de la «scène» dans n'importe quelle modalité sensorielle.
Cela suggère que des mécanismes neuronaux sous-jacents au rôle de l'attention top-down dans la construction des flux perceptuels serait à même d'impliquer des projections top-down agissant en conjonction avec le stimulus physique comme régulateurs ou horloges pour les patterns de décharges de populations neuronales dans le cortex auditif. 

Ainsi, comme cela a été montré précédemment \citep{leek1991informational, richards2004cuing}, l'orientation volontaire de l'attention peut faciliter la perception consciente de cibles acoustiques en présence de MI \citep{elhilali2009interaction}. 
Sur la base de preuves expérimentales, l'attention apparaît donc moduler le traitement cortical dans l'analyse de la scène auditive \citep{elhilali2009interaction, gutschalk2008neural, molloy2019auditory}. 
En conséquence, cela rend très peu probable qu'une sortie du MI --- une ségrégation des flux --- puisse être étudiée en l'absence totale d'attention \citep{gartner2021auditory}. 
Certains aspects de l'attention sont toutefois plus étroitement liés à la pertinence de la tâche vis-à-vis des stimuli cibles \citep{elgueda2019state, huang2019associations, knyazeva2020representation}. 

Il ressort de ces différents éléments que le couplage entre attention et conscience perceptive n'est pas simple et qu'une distinction orthogonale entre ces deux notions comme envisagée par les modèles de traitements par récurrence \citep{lamme2000distinct, lamme2003visual, lamme2006towards} est relativement difficile à établir. 
Un argument contre une interprétation purement basée sur l'attention serait le fait qu'une activité similaire est observée dans le cortex auditif secondaire lorsqu'un seul flux est présenté au sujet \citep{dykstra2016neural, gutschalk2008neural}, alors qu'aucune activité majeure n'est observée dans les zones situées en dehors du cortex auditif \citep{wiegand2018cortical}. 
Le couplage entre attention et conscience perceptive aurait lieu en particulier dans des situations de concurrence perceptive, où il existe de multiples interprétations perceptives possibles \citep{desimone1995neural, gartner2021auditory}. 
Le même réseau dans le cortex auditif pourrait très bien être actif pendant la perception d'un seul flux, sans nécessiter d'attention sélective. 
Néanmoins, en considérant premièrement que l'attention et la conscience perceptive partagent certaines ressources et certains réseaux, et deuxièmement que la conscience perceptive repose sur une fine interaction entre des processus attentionnels et des processus de saillance des caractéristiques, on peut en déduire que l'interconnexion qui les caractérise est fondamentalement basée et structurée sur des mécanismes neuronaux engagés dans des processus de récurrence de traitements de l'information au sein d'une structure hiérarchique de zones distribuées. 

\subsubsection{Intégration et traitement cérébral distribué}

Dans certains modèles théoriques comme celui de l'hypothèse du noyau dynamique ou celui de l'information intégrée, l'intégration de l'information au sein de zones distribuées du système est considérée comme un processus clé pour générer des expériences conscientes. 
En conséquence, il devrait y avoir une élévation du niveau d'activité au sein des «prétendus» corrélats neuronaux de la conscience lorsque les exigences en matière d'intégration augmentent. 
Une étude a testé cette hypothèse selon laquelle l'activité cérébrale liée à la conscience reflèterait l'intégration \citep{eriksson2017activity}. 
L'auteur a utilisé un protocole de MI et a mesuré par IRMf comment l'activité cérébrale variait en fonction de la demande d'intégration nécessaire à la tâche. 
L'intégration des caractéristiques requises pour percevoir la cible était manipulée en comparant des tonalités simples récurrentes avec des triolets de tonalités harmoniques (Figure \ref{fig:chap2erikkson2017}). 
L'audition de triolets nécessite une intégration supplémentaire dans l'espace des fréquences et sur une fenêtre temporelle plus longue que pour les tonalités simples. 
La difficulté de perception a également été manipulée en réduisant la distance dans l'espace des fréquences entre les cibles et les tonalités du masqueur.
De cette manière, \cite{eriksson2017activity} a manipulé la difficulté perceptive et l'intégration associées aux différents stimuli et a suggéré que la perception consciente de la cible devait s'associer à une activité accrue dans les régions temporales et frontales supérieures, comme l'activité au sein de telles régions est associée à de possibles corrélats neuronaux de la perception auditive consciente. 

\begin{figure*}[!t]
\center
\includegraphics[width=0.7\columnwidth]{images_articles/Eriksson_2017_1.jpg}
\caption[Une partie de l'activité des corrélats neuraux de la conscience reflète l'intégration.]{(Gauche) Représentation graphique des stimuli auditifs employés par \cite{eriksson2017activity}. Une tonalité cible a été présentée à plusieurs reprises dans chaque essai, masquée par des tonalités dans des fréquences supérieures et inférieures à la cible. L'intégration des caractéristiques requises pour percevoir la cible a été manipulée en comparant des tonalités simples récurrentes (A, C) avec des triolets de tonalités harmoniques (B, D ; encerclés). La difficulté de perception a été manipulée en réduisant la région spectrale protégée (AB contre CD ; zones grises). Adapté de \cite{eriksson2017activity}.}
\label{fig:chap2erikkson2017}
\end{figure*}

L'activité des sillons intra-pariétaux postérieur et inférieur du lobe pariétal était modulée par les demandes d'intégration et une interaction entre l'intégration et la difficulté perceptive était révélée dans la jonction temporo-pariétale. 
Selon des recherches antérieures, ces deux régions sont fortement impliquées dans les processus attentionnels \citep{buschman2015behavior, cabeza2012cognitive}. 
Ce résultat vient, dans une certaine mesure, s'ajouter à ceux d'études antérieures en IRMf et EEG ayant révélé une dissociation entre l'activité neuronale liée à l'attention et la perception consciente \citep{koivisto2007meaning, koivisto2006independence, schurger2008distinct, tsubomi2009connectivity, wyart2008neural}. 
Contrairement aux recherches précédentes sur la ségrégation des flux auditifs \citep{kondo2009involvement}, aucun changement significatif du signal BOLD (Blood Oxygen Level Dependant, le signal dépendant du niveau d'oxygène sanguin) n'était trouvé dans le cortex auditif primaire associé à l'identification de la cible. 
Par contre, un changement significatif du signal BOLD était trouvé dans le gyrus temporal postéro-supérieur lorsque les participants ont identifié le flux de tonalités répétées. 
De façon critique, seule l'activité du cortex pariétal gauche a augmenté de manière significative en fonction des demandes croissantes d'intégration. 
\cite{eriksson2017activity} en a conclu que le niveau d'intégration requis pour percevoir consciemment un pattern cible vient moduler de manière significative l'activité dans le cortex pariétal. 

Cette étude apparaît comme une preuve supplémentaire que l'activité du cortex pariétal augmente lorsque l'organisation perceptive donne lieu à deux flux auditifs plutôt qu'un seul. 
En effet, \cite{cusack2005intraparietal} avait précédemment suggéré que le cortex pariétal est important pour l'organisation perceptive et plus particulièrement pour la liaison des différentes caractéristiques des objets auditifs. 
Cette modulation apparaît de manière cohérente avec les suggestions précédentes selon lesquelles les processus intégratifs sont importants pour générer des expériences conscientes. 
L'intégration et le traitement cérébral distribué peuvent donc être à même d'expliquer une partie de l'activité neuronale associée aux expériences conscientes lors de la perception auditive de tonalités cibles. 

En outre, \cite{eriksson2017activity} a manipulé les caractéristiques du signal cible en faisant varier le pattern de tonalités cibles (tonalité unique et triolets de tonalités) pour étudier la demande d'intégration au niveau cérébral. 
La structure du pattern de tonalités cibles de la stimulation apparaît ainsi présenter un effet sur la réponse neuronale en lien avec la perception auditive consciente. 
Cette structuration caractéristique du pattern de la cible vien également influencer, au même titre que l'incertitude du masqueur et la similarité cible-masqueur, la perception de la cible. 

\subsubsection{Taux de tonalités cibles et réponses neurales}

L'organisation structurelle du pattern de tonalités cibles, à travers la densité de tonalités, vient s'ajouter comme un facteur supplémentare ayant une influence critique sur la conscience perceptive de la cible auditive. 
Les taux de tonalités cibles, aussi appelés «taux de modulation» ont un impact important sur la performance de détection. 
Le nombre de tonalités par seconde, appartenant au pattern cible, et la forme de ce pattern est susceptible de moduler l'activité au sein des régions cérébrales associées à la conscience perceptive \citep{eriksson2017activity}. 
Les taux de $2$ à $10$~Hz sont considérés comme essentiels dans le groupement des indices physiques et perceptifs dans une scène acoustique complexe et la formation de flux auditifs \citep{kowalski1996analysis, miller2002spectrotemporal, moore2002factors}. 
L'étude de \cite{elhilali2009interaction} sur l'orientation des processus attentionnels avec les tâches cible et masqueur avait utilisé un taux de tonalités cibles «lent» de $4$~Hz. 
Plus tard, \cite{akram2014investigating} ont employé un taux de tonalités plus rapide de $7$~Hz dans un paradigme similaire (Figure \ref{fig:chap2akram2014}) afin d'explorer les réponses neurales et comportementales. 
Il ont fait l'hypothèse que les enregistrements MEG obtenus refléteraient directement la dépendance comportementale du streaming auditif sur les taux de tonalités de la cible. 

\begin{figure*}[!t]
\center
\includegraphics[width=0.40\columnwidth]{images_articles/Akram_2014_1.jpg}
\includegraphics[width=0.32\columnwidth]{images_articles/Akram_2014_2.jpg}
\includegraphics[width=0.40\columnwidth]{images_articles/Akram_2014_3.jpg}
\includegraphics[width=0.40\columnwidth]{images_articles/Akram_2014_4.jpg}
\caption[Tâche d'attention en fonction de la variation des paramètres du stimulus sous MI]{(Haut Gauche) Illustration du type de stimuli employé par \cite{akram2014investigating}. Une séquence régulière de tonalités (en rouge) est placée dans un fond de tonalités (masqueurs, en jaune) distribués de façon aléatoire (en temps et en fréquence) et protégée par une région spectrale (région en vert). Dans la «tâche cible», les sujets devaient détecter une tonalité à fréquence décalée se produisant de manière aléatoire (flèche rouge). Dans la «tâche masqueur», les sujets devaient détecter une élongation de toutes les tonalités constitutives du masqueur dans une fenêtre de temps de $500$ ms (flèches bleues). (Haut Droite) Performance en fonction de la variation de la zone spectrale protégée. (Bas Gauche) Performance en fonction du taux de modulation pour une gamme étendue de $2$ à $10$~Hz, par pas de $2$~Hz. (Bas Droite) Performance en fonction du temps par taux de modulation. Adapté de \cite{akram2014investigating}.}
\label{fig:chap2akram2014}
\end{figure*}

De manière quelque peu similaire à \cite{elhilali2009interaction}, \cite{akram2014investigating} ont manipulé l'orientation de l'attention des sujets sur différentes caractéristiques de la scène «en direction» ou «à l'écart» de la cible. 
Les deux tâches (cible et masqueur) ont ainsi nécessité une attention particulière, mais à des caractéristiques spectrales et temporelles différentes de la scène auditive. 
Selon les auteurs, comme les stimuli présentés dans les deux tâches étaient identiques, toute différence dans la représentation neurale du son devait être le résultat d'une modulation de l'attention. 
Afin de vérifier que la réponse neurale à la cible a servi comme un indicateur du streaming, les auteurs ont corrélé les réponses neurales avec la performance comportementale sous une variété de paramètres du stimulus (taux de tonalité cible, fréquence de tonalité cible et région fréquentielle protégée) et d'états attentionnels (changer l'objectif de la tâche tout en maintenant le même stimulus). 
Dans toutes les conditions qui ont facilité la ségrégation cible-masqueur dans la réponse comportementale, les réponses neurales MEG ont aussi changé de manière consistante avec le comportement. 
De cette manière, le fait de s'attendre au flux cible a causé une augmentation significative dans la puissance et la cohérence de phase des réponses dans l'enregistrement des canaux corrélés avec une augmentation de la performance comportementale des auditeurs. 
Les réponses neurales ont également augmenté comme la zone de protection (région spectrale protégée) s'élargissait et comme le taux de tonalités cibles augmentait. 
En outre, lorsque le taux de tonalités cibles augmentait, la construction de la réponse neurale était significativement plus rapide, reflétant une construction accélérée des percepts de ségégration des flux auditifs \citep{elhilali2009interaction}. 

Un autre aspect important de la ségrégation des flux auditifs associée à la perception du stimulus concerne la base sur laquelle le report explicite conscient du sujet est réalisée. 
En effet, si l'on oriente l'attention du sujet vers une caractéristique précise du stimulus, cela consiste quasiment à réaliser une tâche de détection de changement dans la structure du stimulus. 
Si maintenant, on demande au sujet de reporter sa perception d'un élément du stimulus qu'il considère comme étant la cible d'intérêt, cela pose nécessairement la question de savoir quelle est la caractéristique principale sur laquelle porter son attention afin de bien réussir la tâche. 
La décision du sujet de reporter si oui ou non, il a perçu ce qu'on lui a «appris» être la cible, dépendra des caractéristiques structurelles familières entre ce qui représente une cible et ce qui pourrait potentiellement représenter une cible. 
Il apparaît donc important de considérer la manière dont le signal cible émerge à la conscience de l'auditeur sur la base des caractéristiques structurelles du signal. 
Sur la base de ces différents travaux ayant manipulé des structures de stimuli différentes dans le MI \citep{akram2014investigating, elhilali2009interaction, eriksson2017activity, giani2015detecting, gutschalk2008neural, wiegand2012correlates}, il apparaît relativement difficile de comprendre le lien entre la représentation mentale du sujet vis-à-vis du signal d'intérêt et le report explicite conscient du sujet vis-à-vis de ce qu'il pense avoir perçu. 

\subsubsection{Report de la perception du flux global ou de chaque tonalité ?}

D'un coté, une limite de l'approche des tâches de détection de signal est qu'elle nécessite généralement qu'une réponse motrice soit donnée au moment de la perception, signifiant un report explicite pouvant engager des processus neuronaux au-delà de ceux nécessaires à la perception consciente \citep{aru2012distilling, wiegand2018cortical}. 
Pour cela, \cite{wiegand2018cortical} ont cherché à étudier les corrélats neuronaux de la perception auditive consciente avec et sans report explicite en IRMf. 
Dans une première expérience, des patterns de tonalités réguliers ont été présentés comme cibles sous MI, et les participants ont reporté leurs perceptions sur chaque essai. 
Dans une deuxième expérience, des patterns de tonalités réguliers ont été présentés sans masquage, tandis que les participants ont (i) fixé passivement une cible visuelle, (ii) effectué une tâche visuelle supplémentaire, et (iii) reporté la présence du pattern auditif cible uniquement. 
Cette deuxième expérience a été réalisée pour étudier les composantes post-perceptives potentielles liées à la tâche de perception auditive de la cible, en employant une configuration similaire avec un indice de réponse visuelle, mais où le schéma de stimulus auditif a été présenté sans le masqueur multi-tonalités. 

Les résultats de la première expérience viennent confirmer l'implication d'un réseau fronto-pariétal pour les cibles détectées par rapport aux cibles manquées. 
Les auteurs ont identifié le sillon temporal supérieur et le sillon pré-central inférieur en tant que constituants potentiels de ce réseau soutenant la perception auditive consciente en coordination avec le cortex auditif. 
Il a été suggéré précédemment que l'activation d'un tel réseau fronto-pariétal distribué était nécessaire pour la perception consciente \citep{dehaene2011experimental, eriksson2007similar}. 
D'autres, au contraire, ont soutenu qu'au moins une partie de cette activité est liée à des tâches de report \citep{aru2012distilling, pitts2014gamma, peter2005visibility, tsuchiya2015no} plutôt qu'à la perception consciente, en soi. 
\cite{wiegand2018cortical} ont trouvé que d'autres zones du cerveau, notamment le sillon pré-central supérieur, le sillon intra-pariétal et le cortex insulaire, ont également été actives pour les cibles détectées par rapport aux cibles manquées. 
Pour les auteurs, puisque ces zones n'étaient pas actives pour des patterns de tonalités similaires sans masquage, elles seraient moins susceptibles d'être des constituants d'un réseau nécessaire à la perception auditive consciente. 
Ces zones pourraient néanmoins être nécessaires pour une modulation attentionnelle du cortex auditif qui surmonterait le MI produit par le masqueur multi-tonalités. 
Dans ce cas, le sillon pré-central supérieur, le sillon intra-pariétal et le cortex insulaire seraient alors nécessaires pour la perception sous une situation de MI, mais pas pour la perception auditive consciente en général.

D'un autre coté, la relation exacte entre ARN et perception auditive consciente de tonalités cibles dans un masqueur multi-tonalités est encore confuse \citep{dykstra2017electrophysiological, snyder2015testing}. 
Il n'est pas clair, en effet, si l'ARN est liée à la perception de tonalités individuelles ou si elle est plus directement liée à la perception du flux de tonalités cible global. 
Selon \cite{giani2015detecting}, l'ARN reflèterait uniquement la perception d'un flux auditif global plutôt que la perception des seules tonalités du flux. 
Récemment, \cite{gartner2021auditory} ont cherché à montrer que l'ARN reflèterait en fait un traitement renforcé lié à la tâche de détection plutôt qu'une sensibilisation perceptive au flux de tonalités. 
La dissociation de ces deux aspects est difficile, en particulier lorsque l'auditeur connaît la cible et y répond immédiatement pour indiquer sa perception du stimulus ambigü. 
Dans le paradigme de MI utilisé classiquement, la cible peut être identifiée par une répétition de tonalités identiques. 
De plus, elle peut être identifiée aussi grâce à l'intervalle temporel constant entre les tonalités de la cible. 
Cette configuration de stimulus implique que les participants savent donc immédiatement qu'ils ont détecté les tonalités pertinentes pour la tâche. 
Cette pertinence pour la tâche pourrait alors déterminer le traitement des tonalités cibles dans le cortex auditif. 

Afin de mieux comprendre cet aspect, \cite{gartner2021auditory} ont étudié si l'activité neuronale était plus étroitement liée à la tâche de détection ou plutôt à la perception des tonalités. 
Ils ont utilisé des cibles qui ne pouvaient être identifiées qu'avec des indices fournis pendant l'essai. 
Comme les participants n'étaient informés de la cible qu'après que ses tonalités aient été présentées, le rôle de l'identification immédiate de la cible sur la génération de l'ARN pouvait être étudié. 
Les résultats ont montré que les contrastes cibles détectées vs manquées ne différaient que lorsque les participants identifiaient correctement l'ensemble de la séquence cible. 
Au contraire, ils ne différaient pas lorsqu'ils identifiaient la fréquence de tonalité sur la base d'indices post-stimulus. 
De cette manière, selon \cite{gartner2021auditory}, la conscience perceptive du flux sonore global serait liée directement à l'activité neurale au sein du cortex auditif, comme reflétée par l'ARN. \\

Finalement, le substrat neuronal de l'accès conscient d'un stimulus sensoriel auditif n'est clairement pas déterminé à l'heure actuelle. 
Malgré de nombreux efforts expérimentaux et théoriques, les signatures neurales du traitement conscient restent une question encore largement ouverte \citep{sitt2014large}. 
Quelle est leur localisation temporelle intrinsèque vis-à-vis des stimuli ?
Se trouvent-elles dans des réponses précoces \citep{koivisto2006independence, melloni2007synchronization} ou tardives \citep{gaillard2009converging, sergent2005timing} aux stimulations sensorielles ? 
Quel est le type de traitements sous-jacents à l'émergence des phénomènes de conscience ? 
Pour qu'un sujet prenne conscience du stimulus, certaines théories récentes suggèrent la nécessité d'instancier des boucles de rétroaction à grande échelle \citep{del2007brain, del2009causal, libedinsky2011role}. 
Ces boucles sont-elles réalisées entre des zones sensorielles primaires et des zones intégratives fronto-pariétales \citep{eriksson2007similar, eriksson2017activity, giani2015detecting}. 
La perception consciente dépend-elle d'un échange massif d'informations entre des sites corticaux distants \citep{gaillard2009converging, lumer1999covariation} ?
Au contraire, la conscience perceptive émergerait-elle plutôt uniquement d'une activité récurrente localisée \citep{lamme2000distinct, lamme2003visual, lamme2006towards, pins2003neural} ? 
Ou encore, serait-elle directement associée à une intégration globale de l'information dans le cerveau \citep{tononi2008consciousness} ? 

%%%%%%%%%%%%%%%%%%%%%%%%%%%%%%%%%%%%%%%%%%%%%%%%%%%%%%%%%%%%%%%%%%%%%%%%%%%%%%%
\section{Conclusion}
\label{conclusionchapitre1}
%%%%%%%%%%%%%%%%%%%%%%%%%%%%%%%%%%%%%%%%%%%%%%%%%%%%%%%%%%%%%%%%%%%%%%%%%%%%%%%

Finalement, les aspects généraux des mécanismes et processus associés à l'analyse de la scène auditive, à la ségrégation des flux auditifs et à l'organisation perceptive auditive aboutissant à la perception auditive consciente ont été présentés. 
Les paradigmes d'analyse de la scène auditive tels que le streaming et le masquage auditif offrent une opportunité spécifique d'étudier la dynamique de la construction d'un percept auditif. 
Streaming et masquage auditif sont construits à partir de caractéristiques structurelles, pour lesquelles certaines études ont mesuré l'influence sur la ségrégation des flux.  
Le paradigme de masquage informationnel apparaît présenter un ensemble de paramètres plus riche et relativement plus complexe à manipuler. 
Toutefois, des limites émergent concernant les études de la perception auditive consciente dans le MI chez les êtres humains. 
Aujourd'hui, l'influence catégorique des paramètres construisant les stimuli sur la construction du percept auditif n'est pas clairement établie. 
En effet, il ressort que malgré le nombre important d'études sur le MI, aucune n'a cherché à étudier globalement les paramètres des stimuli dans leur ensemble pour caractériser la construction du percept auditif. 

En outre, il ressort des études de neuroimagerie que la dynamique de la perception auditive consciente sur la base des caractéristiques du stimulus n'a pas été suffisamment étudiée dans le cadre du MI. 
D'un coté, l'IRMf a permis d'obtenir une localisation spatiale des processus et mécanismes en lien avec l'activité de zones cérébrales associées à la perception auditive consciente. 
Néanmoins, en raison de la faible résolution temporelle de l'IRMf, cela ne suffit pas pour caractériser la dynamique des traitements informationnels associée à la construction du percept auditif. 
D'un autre coté, les formes d'onde associées aux évènements liés à la perception consciente des tonalités du flux cible apparaîssent informatives sur une unique dimension temporelle. 
Cette dimension représente la variation d'amplitude du potentiel électrique à la surface du scalp (les ERPs) et donc de la synchronisation spatio-temporelle des décharges d'assemblées neuronales issues de populations cérébrales locales. 
Plusieurs études sur le MI ont mis en corrélation une composante ERP négative --- l'ARN --- à la perception consciente des tonalités du flux cible. 

Un message important tiré des études ERPs est le degré de corrélation entre ces composantes ERPs et la perception consciente de tonalités cibles des sujets dans une tâche de détection. 
Les composantes ERPs du signal EEG sont ainsi un premier niveau de signatures électrophysiologiques disponibles et facilement accessibles par analyse du signal. 
Les ERPs représentent des moyennes réalisées sur de nombreux essais et il est relativement difficile d'établir un suivi quant-à la construction dynamique d'un percept conscient. 
En outre, les composantes ERPs ne donnent pas suffisamment d'information sur le contenu informationnel du signal cérébral. 
En ce sens, les ERPs représentent des concepts associés à des aspects différents de ceux issus des théories de la conscience, en termes de dimensions d'analyse qu'elles proposent. 

Dans ce chapitre, au-delà de certains corrélats neuronaux associés à la perception auditive consciente, nous avons également présenté des modèles permettant d'étudier la conscience. 
Les corrélats ne sont qu'une étape dans la compréhension des mécanismes neuronaux associés à la prise de conscience et il apparaît clairement nécessaire d'explorer comment de tels phénomènes se construisent sur le temps. 
Malgré de nombreuses études sur les corrélats neuronaux de la perception auditive consciente, un constat est que très peu d'études ont cherché à caractériser plus finement la dynamique de la construction du percept auditif. 
En effet, très peu d'études se sont intéressées spécifiquement à la quantité d'information contenu et transmise par les populations neuronales lors de la perception consciente. 

Un manque existe dans la littérature sur la façon dont on peut caractériser la dynamique de la construction d'un percept auditif et comment on peut quantifier l'information associée à une telle dynamique transitoire à l'échelle cérébrale. 
Nous souhaitons donc davantage caractériser la conscience perceptive sur la base de l'accès conscient chez l'être humain au moyen d'expérimentations de MI et d'EEG. 
Pour y parvenir, nous allons utiliser une méthodologie de caractérisation de la dynamique cérébrale associée à la prise de conscience. 
Nous avons désormais besoin de présenter les méthodes qui vont nous permettre de caractériser au mieux la dynamique cérébrale associée à la construction d'un percept auditif.

%%%%%%%%%%%%%%%%%%%%%%%%%%%%%%%%%%%%%%%%%%%%%%%%%%%%%%%%%%%%%%%%%%%%%%%%%%%%%%%
\clearpage\null\newpage
%%%%%%%%%%%%%%%%%%%%%%%%%%%%%%%%%%%%%%%%%%%%%%%%%%%%%%%%%%%%%%%%%%%%%%%%%%%%%%%
