%%%%%%%%%%%%%%%%%%%%%%%%%%%%%%%%%%%%%%%%%%%%%%%%%%%%%%%%%%%%%%%%%%%%%%%%%%%%%%%
\chapter{Discussion générale, limites et perspectives}
\label{chapitre6}
%\addcontentsline{toc}{chapter}{Discussion}
%%%%%%%%%%%%%%%%%%%%%%%%%%%%%%%%%%%%%%%%%%%%%%%%%%%%%%%%%%%%%%%%%%%%%%%%%%%%%%%

\noindent \hrulefill \\

%%%%%%%%%%%%%%%%%%%%%%%%%%%%%%%%%%%%%%%%%%%%%%%%%%%%%%%%%%%%%%%%%%%%%%%%%%%%%%%
\section{Rappel des objectifs de la thèse}
%%%%%%%%%%%%%%%%%%%%%%%%%%%%%%%%%%%%%%%%%%%%%%%%%%%%%%%%%%%%%%%%%%%%%%%%%%%%%%%

Le but de ce travail de thèse était de caractériser la dynamique cérébrale associée à la perception auditive consciente afin de pouvoir apporter des solutions pratiques au problème de l'absence de détection d'une alarme sonore par un pilote. 
Pour y parvenir, nous nous sommes intéressés à des aspects théoriques et fondamentaux de la perception auditive consciente et à sa caractérisation pragmatique et pratique. 
Nous avons considéré la problématique de caractérisation de la dynamique de l'accès conscient d'un stimulus auditif pertinent au milieu d'un environnement acoustique complexe. 
Nous avons plus spécifiquement étudié les corrélats neuronaux de la conscience (CNC) auditive au moyen de l'électroencéphalographie dans le masquage informationnel chez le sujet adulte humain sain. 

Une première partie de ce travail a été consacrée à implémenter un protocole expérimental permettant d'étudier la dynamique de l'accès conscient d'un stimulus auditif. 
Détecter une alarme sonore dans un cockpit par le pilote est assimilable à un problème de ségrégation des flux auditifs tel qu'il existe dans le cas des situations de cocktail party. 
Dans de tels contextes, le système auditif humain doit procéder à une ségrégation des différents flux pour aboutir à une organisation cohérente des représentations perceptives. 
Le phénomène de masquage informationnel, résultat d'une absence de ségrégation du flux issu de la cible d'intérêt par les voies de traitement de l'information auditive, est particulièrement pertinent dans le cadre des études comparatives. 
En effet, il permet la comparaison entre les états perceptifs des objets masqués, non-perçus, vis-à-vis des objets non-masqués, perçus par le sujet. 
Nous avons donc modélisé expérimentalement la situation opérationnelle par un protocole de MI afin de pouvoir aborder méthodiquement le problème de caractérisation de la dynamique de l'accès conscient de la cible auditive. 

Notre premier objectif expérimental a été une exploration des différents paramètres du MI afin de déterminer une situation adéquate pour l'étude des corrélats neuronaux de la perception auditive consciente. 
Dans une première étude (étude I), nous avons fait varier les paramètres influençant la dynamique de la perception consciente afin d'obtenir leurs effets sur la qualité et la vitesse de la construction du percept auditif.
Cette étude était nécessaire pour obtenir les conditions expérimentales favorables à un décours temporel compatible avec les mesures utilisées par la suite pour caractériser les CNC auditive.

Une deuxième partie de ce travail a ensuite été consacrée à caractériser les CNC auditive en étudiant la dynamique neuronale associée au traitement de l'information issu de la construction du percept auditif. 
Nous avons étudié les mécanismes de traitements informationnels observables à l'échelle cérébrale macroscopique en utilisant l'électroencéphalographie au travers de la situation expérimentale de MI. 
Pour cela, nous avons utilisé deux approches. 
La première approche visait à caractériser la dynamique cérébrale au moyen de mesures des caractéristiques statistiques du signal électrophysiologique. 
La deuxième approche visait à caractériser la conscience sur la base de l'utilisation de mesures de l'état de conscience issues de la théorie de l'information intégrée de la conscience. 

Notre second objectif expérimental a donc été de tester l'efficacité de ces deux approches sur la base de la capacité des mesures à rendre compte de la dynamique associée à l'accès conscient d'un percept auditif. 
Pour cela, les outils issus des théories de l'information (TI) et de l'information intégrée (TII) nous ont donné plusieurs possibilités. 
Premièrement, nous avons pu quantifier le contenu informationnel et le degré de complexité associés aux signaux neuronaux issus de l'activité cérébrale lors de la perception auditive conscience. 
Deuxièmement, nous avons pu quantifier la transmission et l'échange d'information entre les différentes zones cérébrales à l'échelle macroscopique lors de la perception auditive conscience. 
Troisièmement, nous avons pu quantifier le degré auquel l'information est intégrée à l'échelle cérébrale macroscopique dans la perception auditive consciente. 
Nous avons étudié ces aspects dans une deuxième étude (étude II) impliquant de l'EEG et fondée sur les ensembles de combinaisons de paramètres recueillis à partir de la première étude (étude I). 

%%%%%%%%%%%%%%%%%%%%%%%%%%%%%%%%%%%%%%%%%%%%%%%%%%%%%%%%%%%%%%%%%%%%%%%%%%%%%%%
\section[Influence des paramètres de la cible et du masqueur (Étude I)]{Influence des paramètres de la cible et du masqueur sur la dynamique de la perception auditive consciente dans le masquage informationnel (Étude I)}
%%%%%%%%%%%%%%%%%%%%%%%%%%%%%%%%%%%%%%%%%%%%%%%%%%%%%%%%%%%%%%%%%%%%%%%%%%%%%%%

L'objectif de cette première étude était de déterminer, au travers de trois expériences indépendantes (Exp I, II et III), l'influence de paramètres déterminants du MI.
Ces paramètres étaient : la similarité temporelle entre la cible et le masqueur, le taux de répétition de la cible et l'incertitude du masqueur.
Leur influence était évaluée sur la performance et la dynamique de la prise de conscience perceptive liée à la détection d'une cible dans un paradigme de MI multi-tonalités. 
Les caractéristiques acoustiques des stimuli et leurs relations ont été manipulés afin d'obtenir les paramètres de second ordre du stimulus que sont la similarité temporelle cible-masqueur et l'incertitude du masqueur. 
Les temps de détection ont été analysés en utilisant des modèles de survie avec effets-mixtes (modèle de fragilité). 
Ce type de modèles nous a permis de prendre en compte à la fois les caractéristiques temporelles des données de temps de réaction ainsi que leur degré de variabilité inter-individuelle d'une manière quantitative. 

L'effet de la similarité temporelle cible-masqueur nous a permis de conclure que la conscience perceptive est généralement améliorée lorsque la similarité est faible. 
En effet, il a été montré précédemment qu'une similarité fréquentielle et temporelle élevée diminue fortement la performance de la détection \citep{durlach2003informational, kidd2002similarity}. 
Notre étude complète ces observations et montre également que l'effet de la similarité temporelle entre la cible et le masqueur est asymétrique. 
La cible est plus facile à détecter lorsque ses tonalités sont plus longues que celles du masqueur. 
Au contraire, la durée de tonalités du masqueur a un effet faible, voire nul, si elle est plus longue que celle de la cible. 
Ce résultat suggère que la durée de la cible et du masqueur est un facteur important dans la ségrégation des flux auditifs. 

Dans une étude, \cite{bendixen2013different} ont étudié l'influence de la similarité et de la prévisibilité du stimulus sur les rapports perceptifs lors d'une tâche de streaming. 
Les participants devaient indiquer continuellement s'ils percevaient une seule séquence cohérente ou deux flux de sons simultanés lors de la présentation de séquences de sons. 
La similarité était basée sur l'association de sons ayant des caractéristiques fréquentielles similaires dans le temps. 
La prévisibilité correspondait à l'association de sons qui se suivent de manière prévisible. 
Les données ont confirmé que la similarité et la prévisibilité sont toutes deux utilisées comme indices dans la ségrégation des flux, mais qu'elles exercent leur influence à différents stades de l'analyse de la scène auditive. 
Seule la similarité des caractéristiques semblait contribuer à la première étape putative de l'analyse de la scène auditive, au cours de laquelle des organisations sonores alternatives sont découvertes \citep{bregman1994auditory}. 
La similarité des caractéristiques et la prévisibilité semblaient toutes deux contribuer à la deuxième étape de l'analyse de la scène auditive, au cours de laquelle la compétition entre les organisations sonores alternatives a lieu. 
Les auteurs ont donc suggéré que la similarité et la prévisibilité impliquent des mécanismes au moins partiellement distincts et agissent sur des étapes différentes de l'analyse de la scène auditive \citep{bendixen2013different}.

La détection de la cible a également été améliorée par des taux de répétition de la cible plus élevés. 
Il a été montré précédemment que des taux de répétition de la cible rapides sont associés à une meilleure détection \citep{xiang2010competing, akram2014investigating}. 
Dans certains cas, le taux de répétition de la cible au-delà d'une valeur critique (environ $40$~Hz) peut provoquer la fusion des sons en un flux facilement détectable par les sujets. 
Ce paramètre représente un facteur essentiel dans le regroupement des indices physiques et perceptifs dans une scène acoustique complexe \citep{moore2002factors}. 
Dans les Expériences~II et~III, les effets observés du taux de répétition de la cible sur la dynamique de la conscience perceptive sont en accord avec la littérature.
La durée de tonalité cible et le taux de répétition de la cible, peuvent agir de concert pendant toute la durée du silence entre les cibles successives. 
De plus petites durées de silence entre les cibles, associées à un taux de répétition des cibles plus élevé ou à une durée des cibles plus longue (pour un taux de répétition des cibles fixe), favorisent l'accumulation des flux auditifs.

La détection de la cible est affectée de manière différente par l'incertitude spectrale et l'incertitude temporelle et nous avons montré que la conscience perceptive est facilitée lorsque le nombre de fréquences par octave est faible. 
En effet, l'incertitude du masqueur a été manipulée en utilisant le nombre de fréquences par octave et l'intervalle moyen entre les tonalités. 
Pour un nombre donné de fréquences par octave, la conscience perceptive est améliorée pour les intervalles inter-tonalité les plus longs. 
Précédemment, il a été montré que l'incertitude fréquentielle des composantes individuelles du masqueur était prédominante dans la production de l'effet de masquage \citep{neff1995individual, neff1988effective}. 

Nous avons observé que l'incertitude temporelle du masqueur, quantifiée par l'entropie de la distribution des tonalités, n'est pas linéairement liée à la facilitation de la conscience perceptive. 
Cette propriété statistique du masqueur n'explique donc pas directement les changements dans la dynamique de la conscience perceptive. 
Précédemment, des caractéristiques statistiques des stimuli comme l'incertitude ont été utilisées pour expliquer les performances de la détection \citep{chang2016detection, lutfi2013information}. 
Une fonction qui associerait la performance de la détection à la divergence d'information des séquences de tonalités permettrait d'établir une relation entre le streaming et le masquage auditif \citep{chang2016detection}.
Cela pourrait signifier qu'une fonction unique de la séparation statistique des séquences de tonalité décrit les effets des deux facteurs sur ces deux phénomènes. 
Streaming et masquage de tonalités seraient alors les deux limites opposées d'une fonction commune de la divergence d'information. 

Dans notre étude, nous avons cependant observé que l'incertitude n'est pas forcément la meilleure caractéristique pour expliquer les changements dans le traitement auditif. 
Au contraire, une caractéristique acoustique de second ordre du masqueur --- la densité spectro-temporelle, définie comme le nombre de tonalités par seconde et par octave --- peut expliquer les changements dans la dynamique de la conscience perceptive d'une manière plus directe (voir Figure~\ref{fig:HvsSTD}). 
Une fonction commune de la divergence d'information entre la cible et le masqueur pourrait être associée à de telles caractéristiques de densités du stimulus. 
Il serait alors intéressant d'étudier les attributs sonores qui construisent ces caractéristiques afin de mieux comprendre comment ces propriétés statistiques peuvent être partagées entre la cible et le masqueur. 
Cela suggère que les attributs physiques des stimuli auditifs contribuent à la création de caractéristiques de second ordre du stimulus qui s'associent dans les représentations perceptives. 

Deux phénomènes complémentaires contribuent à la ségrégation des flux et donc à la levée du masquage : d'une part, les propriétés acoustiques du masqueur et de la cible et leurs différences statistiques qui contribuent à la difficulté de la tâche et d'autre part, la façon dont le système auditif intègre le contenu informationnel du stimulus et extrait les informations pertinentes afin de former le percept de la cible et qui conduit à la conscience perceptive. 
Dans l'audition, la bistabilité perceptive peut être difficile à relier à la perception ordinaire, qui est généralement non ambiguë. 
La dynamique à long terme de la bistabilité perceptive consiste en des alternances entre des percepts mutuellement exclusifs et le cerveau est capable d'extraire des représentations invariantes dans le temps des sons présentés. 
Pour former une représentation cohérente des objets auditifs, le système auditif doit exploiter les différences statistiques de la structure temporelle des signaux pour séparer la figure du fond \citep{lutfi2013information}.
Ces représentations sont robustes au caractère aléatoire inhérent aux sources sonores du monde réel, tout en restant flexibles pour s'adapter à un environnement dynamique \citep{skerritt2018detecting}. 
Utilisant un modèle perceptif de traitement prédictif, \cite{skerritt2018detecting} ont proposé que le cerveau collecte des statistiques d'ordre supérieur sur les dépendances temporelles entre les différents sons des séquences sonores. 
Leur modèle a montré des corrélations entre la performance de la tâche et les différences individuelles dans la perception. 
De cette manière, on pourrait suggérer que la fonction de divergence d'information entre la cible et le masqueur soit fondée sur des statistiques collectées sur la base de dépendances temporelles entre la cible et le masqueur. 
Les dépendances temporelles présentant une plus grande divergence d'information seraient à même de défavoriser le masquage et ainsi de favoriser la ségrégation des flux auditifs. 

La ségrégation des flux auditifs peut être considérée comme un analogue de la ségrégation figure-fond \citep{teki2011brain}, et est généralement très flexible pour les scènes auditives \citep{kleinschmidt2002human, knapen2011role, lumer1998neural, sterzer2007neural}. 
Sous réserve d'un contrôle volontaire, un auditeur peut choisir de se concentrer sur une source sonore particulière et de faire passer le reste au second plan \citep{steele2016buildup}. 
Cette capacité de modifier l'organisation perceptive de la scène par une attention sélective et un contrôle volontaire est un élément fondamental de la bistabilité perceptive dans l'audition. 
Dans le cas du streaming, la recherche originale de \cite{van1975temporal} montre une certaine capacité des sujets à contrôler l'une ou l'autre organisation perceptive lorsqu'elle est ambiguë, en leur donnant pour instruction de «maintenir» le percept jusqu'à ce que les paramètres du stimulus deviennent trop extrêmes. 
Il faut toutefois distinguer cette capacité de celle qui consiste à changer le centre d'attention pour mettre un autre objet auditif au premier plan dans des conditions d'écoute \citep{akram2014investigating, elhilali2009interaction}. 
Dans ce cas, les objets auditifs sont déjà formés en ensembles distincts d'éléments sonores, et l'auditeur doit simplement choisir entre les objets. 
Dans le paradigme de \cite{van1975temporal}, il apparaît que l'auditeur doit choisir comment regrouper les éléments sonores. 

Le comportement de la dynamique d'organisation perceptive doit néanmoins être remis dans le contexte de l'environnement auditif dans lequel est intégré le sujet. 
Un environnement réel composé d'évènements sonores naturels est un environnement actif sur une échelle de temps prolongée, et n'est pas juste un environnement de laboratoire confiné à des présentations isolées de stimuli individuels. 
En effet, dans les situations naturelles, l'existence de multiples indices, spectraux, de localisation, d'intensité, de modulation, etc..., réduit la probabilité d'ambiguïté dans l'organisation des représentations perceptives. 
Les conditions d'écoute réelles s'apparentent davantage à des perturbations d'un état pseudo-stable qu'à des essais répétés de présentations de stimuli séparés par un silence \citep{steele2016buildup}. 
La plupart des cas de perception auditive ordinaire démontrent ainsi la stabilité des organisations perceptives des signaux auditifs en représentations d'objets.
Pour certains stimuli ambigüs ou de longue durée, un conflit perceptif peut apparaître entre l'intégration et la ségrégation, la perception alternant ainsi entre intégration et ségrégation. 
Au départ, toutes les caractéristiques du stimulus semblent intégrées et puis, quelques secondes après, un percept ségrégé peut émerger : la ségrégation des caractéristiques acoustiques en flux peut ainsi prendre plusieurs secondes. 
Les auditeurs sont capables d'ajuster leur intervalle d'intégration temporelle lors de la détection de signaux de durées incertaintes ou inatendues \citep{dai1995detecting}. 
De cette manière, la dynamique de l'accumulation d'évidence associée à un percept est différente après des changements d'attention par rapport à son accumulation depuis le début de l'essai \citep{cusack2004effects}. 

Il a été proposé que l'intégration auditive des informations acoustiques soit modélisée comme un processus d'accumulation de preuves \citep{barniv2015auditory, nguyen2020buildup}. 
Un tel modèle peut rendre compte de l'accumulation au cours du temps des flux auditifs et donc de la dynamique de la construction de la perception consciente. 
Le streaming peut être utilisé pour montrer l'augmentation au cours du temps du fractionnement perceptif d'événements sonores possédant des caractéristiques acoustiques différentes en flux distincts \citep{anstis1985adaptation, cusack2004effects, pressnitzer2006temporal}.
Le changement de la probabilité d'organisation perceptive en fonction du temps que les observateurs mettent pour reporter un percept peut alors être quantifiée comme une fonction d'accumulation. 
Lorsque la moyenne des reports du sujet sur la ségrégation des flux est calculée sur des essais répétés, la fonction d'accumulation est alors obtenue pour représenter la probabilité de ségrégation. 

Certains modèles perceptifs ont proposé des systèmes qui accumulent des preuves sur une période de temps jusqu'à ce qu'ils atteignent un seuil de décision \citep{gold2007neural}. 
Cependant, dans le cas d'un stimulus ambigu, il n'est pas évident de savoir ce qui doit exactement «s'accumuler» au fil du temps, car aucune interprétation particulière n'est correcte sans ambiguïté. 
\cite{steele2016buildup} a présenté un modèle statistique dans lequel la perception alterne pendant un essai entre différents groupements, comme dans la bistabilité perceptive, avec des durées de dominance aléatoires et indépendantes échantillonnées à partir de différentes distributions de probabilité spécifiques à la perception. 
Ce modèle permet de décrire la dynamique à court terme de l'accumulation observée sur des essais courts relativement à la statistique des durées de perception pour les deux organisations perceptuelles alternées.
De plus, ce modèle décrit bien les fonctions d'accumulation et les alternances dans les simulations de modèles de réseaux neuronaux composés de populations sélectives de percepts en concurrence par inhibition mutuelle. 
\cite{steele2016buildup} suggère à partir des résultats de ce modèle de commutation statistique que l'accumulation n'est pas une caractéristique nécessaire pour produire la construction du percept. 
L'augmentation graduelle de la probabilité d'un percept au fil du temps pourrait refléter la dynamique de la commutation entre des percepts ayant des durées aléatoires indépendantes et un état initial donné. 

Nous avons également trouvé différents niveaux de variabilité entre les sujets dans les trois expériences. 
Le modèle de fragilité de l'expérience II a révélé un effet de la fragilité, montrant un niveau élevé d'hétérogénéité entre les sujets dans la dynamique de la conscience perceptive. 
Aucun effet de la fragilité n'a été observé dans les expériences I et III, caractérisant une plus grande homogénéité entre les sujets dans ces conditions expérimentales. 
Précédemment, les études sur le MI ont expliqué cette variabilité inter-individuelle comme un reflet possible de la manière dont les sujets traitent les paramètres du stimulus \citep{kidd2008informationalreview, oxenham2003informational}. 
Des différences de traitement dans les caractéristiques du stimulus pourraient alors favoriser ou défavoriser la ségrégation du flux associé à la cible de ceux du masqueur. 
Par conséquent, cela pourrait amener les auditeurs à présenter des stratégies individuelles idéales ou non idéales \citep{kidd2008informationalreview}. 
Les différences observées dans les stratégies de traitement seraient attribuées à des combinaisons spécifiques de propriétés du masqueur et de la cible de chaque expérience. 
L'interaction entre l'incertitude du masqueur, la similarité temporelle cible-masqueur et le taux de répétition de la cible dans les plages utilisées dans l'expérience II conduirait ainsi à des stratégies plus variables que celles des expériences I et III, visible de par l'effet de la fragilité dans le modèle. 

Finalement, cette première étude montre que la similarité temporelle cible-masqueur, le taux de répétition de la cible et l'incertitude du masqueur, propriétés du stimulus dans le MI, modulent la dynamique de la perception consciente. 
L'étude de l'effet des paramètres du stimulus et de leurs interactions permet ainsi de mieux comprendre l'évolution dans le temps du risque de détection de la cible et de mieux déterminer la prédictibilité de la détection du signal. 
Cette étude montre ainsi que l'utilisation des modèles de survie pour l'analyse de la dynamique associée à la construction d'un percept est largement pertinente pour mieux comprendre le décours temporel associé à la perception consciente en général et permettre une investigation plus approfondie de ses corrélats neuronaux. 
De plus, l'étude de la perception consciente auditive pourrait bénéficier de la combinaison des modèles de survie, d'accumulation de preuves ou de commutations statistiques avec l'enregistrement de l'activité électrophysiologique à différentes échelles dans des tâches de MI pour déchiffrer la relation entre la perception auditive consciente et sa dynamique neuronale. 
Il serait ainsi pertinent de réaliser une étude plus systématique sur la base de modèles de survie et d'un ensemble plus grand de paramètres dans des situations expérimentales variées afin d'étudier de manière plus spécifique la pertinence des modèles d'accumulation ou de commutation. 

%%%%%%%%%%%%%%%%%%%%%%%%%%%%%%%%%%%%%%%%%%%%%%%%%%%%%%%%%%%%%%%%%%%%%%%%%%%%%%%
\section[Caractérisation électrophysiologique de la perception consciente (Étude II)]{Caractérisation électrophysiologique de la dynamique de la perception auditive consciente dans le masquage informationnel (Étude II)}
%%%%%%%%%%%%%%%%%%%%%%%%%%%%%%%%%%%%%%%%%%%%%%%%%%%%%%%%%%%%%%%%%%%%%%%%%%%%%%%

L'objectif de cette deuxième étude était de caractériser les CNC auditive en étudiant la dynamique neuronale associée au traitement de l'information issu de la construction d'un percept auditif. 
Étudier la conscience perceptive consiste classiquement à lier les reports subjectifs à l'activité neuronale afin d'identifier l'ensemble minimal d'événements et de mécanismes neuronaux suffisants pour l'émergence d'un percept conscient.  
Par conséquent, une fois un signal neuronal identifié comme un corrélat potentiel de la conscience, on essaye usuellement de dissocier la conscience et le CNC proposé de deux manières distinctes \citep{dembski2021perceptual}. 
Premièrement, le CNC peut-il être clairement observé lorsque les sujets ne perçoivent pas consciemment le stimulus ? 
Deuxièmement, un sujet peut-il percevoir consciemment un stimulus même si le CNC est absent ? 
De cette façon, un véritable CNC devrait être présent lorsqu'un stimulus est expérimenté consciemment et devrait être absent lorsque le même stimulus n'est pas expérimenté consciemment. 
La recherche sur les CNC adopte donc l'une des deux approches suivantes : 

\begin{itemize}
\item[$\bullet$] soit elle cible les mécanismes neuronaux qui déterminent le contenu d'une expérience phénoménale spécifique (\textit{e.g.}, l'audition d'une alarme), connus sous le nom de \emph{CNC spécifique} au contenu, 
\item[$\bullet$] soit elle cible les substrats neuronaux qui soutiennent la capacité à avoir des expériences conscientes indépendamment du contenu spécifique, connus sous le nom de \emph{CNC complet}. 
\end{itemize}

Pour étudier les CNC spécifiques au contenu, on enregistre généralement l'activité neuronale de sujets conscients tout en présentant des stimuli qui sont parfois perçus consciemment et parfois non perçus. 
Pour étudier les CNC complets, on cherche usuellement à les isoler en comparant les états de veille, de sommeil et d'anesthésie chez des volontaires sains ou en observant des patients souffrant de divers troubles de la conscience. 
Nous avons donc étudié les CNC spécifiques au contenu dans la modalité auditive en utilisant plusieurs approches parmi lesquelles on trouve certaines des méthodes de quantification des CNC complets. 
Dans cette étude, nous avons utilisé un ensemble de mesures distinctes provenant d'approches différentes, afin de refléter leur capacité à indiquer la perception consciente de la cible auditive. 
Ce type d'étude s'inscrit dans une approche méthodologique générale visant à caractériser le signal cérébral sur la base de mesures disponibles dans la littérature afin d'expliquer les mécanismes et les processus à l'oeuvre lors des phénomènes de perception consciente. 
C'est en adoptant ce type d'étude comparative dans des situations expérimentales variées que pourront être trouvées les meilleures manières de caractériser et d'indiquer les phénomènes de perception consciente lors de situations pratiques. 

%%%%%%%%%%%%%%%%%%%%%%%%%%%%%%%%%%%%%%%%%%%%%%%%%%%%%%%%%%%%%%%%%%%%%%%%%%%%%%%
\subsection{Analyses des potentiels reliés à l'évènement}
%%%%%%%%%%%%%%%%%%%%%%%%%%%%%%%%%%%%%%%%%%%%%%%%%%%%%%%%%%%%%%%%%%%%%%%%%%%%%%%

Le premier objectif consistait à étudier les corrélats neuronaux associés à la perception auditive consciente en analysant les composantes ERPs du signal EEG dans le MI. 
Le but de l'observation des ERPs était de «valider» les données car, comme les analyses de formes d'ondes nécessitent plusieurs essais pour la comparaison cibles perçues vs cibles non-perçues, les ERPs ne peuvent pas être envisagées comme des CNC utilisables en réponse à la problématique de l'absence de détection d'une alarme auditive. 
Nous avons donc d'abord cherché à reproduire les principaux résultats de la littérature sur les CNC auditive \citep{dykstra2016neural, gartner2021auditory, giani2015detecting, gutschalk2008neural, wiegand2012correlates}. 
Ces corrélats apparaissent principalement sous la forme d'une négativité temporale à environ $200$-$250$~ms et d'une positivité centro-pariétale tardive, après $300$~ms. 

D'une part, l'ARN a été présentée comme le corrélat neuronal de la perception auditive consciente observé au niveau du cortex auditif lors de la perception de tonalités du signal cible \citep{gutschalk2008neural, eklund2019auditory, konigs2012functional, gartner2021auditory, wiegand2012correlates}. 
Il s'agit d'une onde négative dont l'amplitude est plus grande pour les tonalités cibles détectées que pour les tonalités cibles non-détectées. 
D'autre part, la P300 a été présentée comme le corrélat neuronal de la conscience perceptive sur la base d'un traitement intégratif global des caractéristiques du stimulus \citep{dehaene2006conscious, del2007brain, sergent2004neural, sergent2005timing}. 
Il s'agit d'une onde positive dont l'amplitude est plus grande et la latence plus courte pour les tonalités détectées que pour les tonalités cibles non-détectées \citep{dykstra2016neural, giani2015detecting, parasuraman1980brain, paul1972evoked, squires1973vertex}. 
Ces deux composantes que sont la négativité, dite «précoce», qui serait spécifique à la modalité, et la positivité dite «tardive», indépendante de la modalité (P3b ou P300 tardive) ont été présentées comme possibles CNC. 

Plus d'une dizaine d'études ont montré que la P300 tardive est absente en dépit de la vision, de l'audition ou de la sensation consciente de stimuli, l'excluant de facto comme un véritable CNC \citep{dembski2021physiological}. 
En revanche, aucune preuve suffisante d'une dissociation entre les ondes négatives précoces et la conscience n'a été rapportée jusqu'à présent. 
Nous avons donc choisi d'étudier en premier ces deux composantes du fait de leur rôle prétendument associé à la conscience perceptive. 
Bien que la P300 ait été exclue par certains de l'ensemble des vrais CNC, nous l'avons tout de même étudié puisqu'elle apporte un aperçu des mécanismes de traitement intégratif de l'information au niveau cérébral. 

Nous avons trouvé une large déflexion négative sur les électrodes C5, F6 et F7 pour la première tonalité (B1) avant le report lorsque la cible a été détectée. 
Lors de la présentation d'une cible auditive composée de 10 tonalités avec le masqueur, seule la première tonalité (B1) avant la détection pour les cibles détectées a suscité cette onde négative dans une fenêtre de $250$ à $350$~ms. 
De par sa localisation spatiale et temporelle, cette onde négative localisée au niveau du lobe temporal gauche pour les électrodes C5 et F7, et droit pour l'électrode F6, semble largement correspondre à une onde ARN et sa localisation pourrait être attribuée au cortex auditif. 
De plus, l'amplitude de cette onde négative était maximale dans les sites les plus antérieurs du cuir chevelu, ce qui est cohérent avec la littérature \citep{eklund2019auditory}, tant que l'on associe l'activité des électrodes F6 et F7 situées à l'avant du lobe temporal comme pouvant provenir des cortex auditifs. 
Ainsi, détecter la cible auditive a significativement diminué les valeurs d'amplitude de l'ARN pour la première tonalité B1 dans la fenêtre $250-350$~ ms comparativement au fait de ne pas détecter la cible auditive. 

Nous n'avons observé aucune forme d'onde ARN pour la deuxième tonalité B2 avant le report lorsque les cibles ont été détectées au niveau des cortex auditifs. 
Nous avons observé que les valeurs d'amplitude suscitées par la première tonalité B1 se sont montrées significativement inférieures à celles suscitées par la seconde tonalité B2, pour laquelle aucune ARN n'a été reportée. 
Cela semble être cohérent avec le fait que la consigne donnée aux participants était de n'appuyer sur le bouton qu'une fois qu'ils étaient suffisamment sûrs d'avoir détecté une cible tonale régulière, et donc un pattern d'au moins deux tonalités cibles. 
Dans l'étude de \cite{giani2015detecting}, une onde négative de type ARN a été corrélée avec la conscience uniquement en réponse à la deuxième tonalité de la paire de cibles, mais pas à la première. 
Selon les auteurs, cela indiquait que la négativité de la conscience auditive reflète le processus de séparation d'un stimulus auditif d'un fond multi-tonalités plutôt que la perception consciente du stimulus. 
Cependant, la cible ne pouvait être identifiée consciemment qu'après que les deux tonalités aient été jouées. 

Si la perception de la première tonalité est une condition préalable à la détection de la cible, elle n'est pas suffisante et par conséquent, aucune activité neuronale liée à la détection de la cible ne devrait être observée pour la première tonalité. 
Néanmoins, la deuxième tonalité perçue à l'échelle du système auditif pourrait représenter une cause de la modification des patterns de transmission de l'information. 
Il est possible que les processus liés à l'ARN codent plutôt pour le contexte sonore, pouvant aller d'un événement isolé à une tonalité parmi de multiples similaires dans un fond multi-tonalités, reflétant un continuum allant du saillant au masqué \citep{gartner2021auditory}.
Les résultats de notre étude montrent que de larges formes d'ondes négatives de type ARN peuvent être observées sur la première tonalité avant le report au niveau des cortex auditifs lorsque que la cible a été perçue par le sujet. 
Cette étude permet d'apporter un autre argument expérimental sur la potentialité de la composante ARN comme un corrélat neuronal EEG de la perception consciente d'un signal auditif.

Dans notre analyse, la sélection de la latence et des pics d'amplitude négative a été faite en fonction des caractéristiques a priori de la composante ARN selon les études précédentes. 
Il serait intéressant d'étudier plus précisément la latence temporelle d'apparition de l'ARN à plusieurs niveaux, comme cela n'a pas été systématiquement étudié dans la littérature. 
Il serait également intéressant d'étudier plus spécifiquement l'augmentation en amplitude de l'ARN en fonction des tonalités avant la détection de sorte à pouvoir observer les éventuels effets transitoires perceptifs. 
Dans notre cas, on voit que la différence est très importante entre la tonalité B2 et la tonalité B1. 
Peut-être pourrions-nous nous attendre à observer une augmentation progressive des valeurs d'amplitude de l'ARN en fonction des tonalités de la cible. 
Sur les quatre tonalités étudiées, seulement la tonalité avant le report perceptif a révélé une différence significative entre les cibles détectées et omises. 
Une étude plus approfondie augmenterait l'étendue de tonalités étudiées et chercherait à ce que le report perceptif soit encore plus éloigné du début de l'essai afin de pouvoir analyser un nombre plus conséquent de tonalités cibles. 
Tout comme l'ARN, il est notamment intéressant de constater l'important différentiel de valeur d'amplitudes dans les fenêtres temporelles considérées entre les tonalités B2 et B1, c'est-à-dire les deux tonalités précédant le report. 

Enfin, nous avons observé, que pour quatre électrodes centro-sagitalles (FCz, Fz, CPz et Pz), la détection de la cible a augmenté les valeurs d'amplitude de la P300 de la première tonalité avant la détection (B1) comparativement aux cibles n'ayant pas été détectées dans une fenêtre temporelle allant de $250$ à $350$~ms. 
Aucune autre augmentation n'a été observée pour les autres tonalités (B2, A1 et A2).
Classiquement, l'onde P300 présente une latence d'environ $250$-$600$~ms et est composée d'au moins deux sous-composantes distinctes, une P3a fronto-centrale précoce et une P3b pariétale maximale plus tardive \citep{dembski2021perceptual}. 
La P3a se produit dans les états d'inconscience et reflète des processus attentionnels automatiques, dirigés par le stimulus, comme lorsqu'un stimulus inattendu attire involontairement l'attention.
La P3b, quant-à elle, est le plus souvent sollicitée dans des contextes expérimentaux au cours de tâches impliquant la discrimination de cibles peu fréquentes et est supposée refléter le stockage du contenu dans la mémoire de travail, les transformations stimulus-réponse, la mise à jour du contexte, la catégorisation du stimulus et la perception consciente \citep{dehaene2011experimental, dembski2021perceptual, luck2014introduction, sergent2005timing}. 
Des études récentes viennent contredire le rôle de la P3b dans la perception consciente et tendent à montrer cette composante plutôt comme un potentiel processus post-perceptuel que comme un processus d'intégration consciente \citep{cohen2020distinguishing, fishman2021learning, pitts2014gamma, pitts2014isolating, tsuchiya2015no}. 
Dans la plupart des études, les stimuli critiques sont usuellement directement liés à la tâche, de sorte que les contrastes entre conscients et inconscients pourraient bien inclure des différences dans le traitement post-perceptif en plus des différences dans la perception consciente en soi \citep{tsuchiya2015no}. 

Dans leur étude, \cite{giani2015detecting} ont trouvé que la P300 était amplifiée de manière significative pour les deux tonalités, lorsqu'elles ont été détectées et que la P300 comme marqueur de la perception consciente semblait reposer sur des interactions entre les cortex pariétal et auditif. 
Les auteurs ont suggéré que la perception auditive consciente de la paire de tonalités reposait sur un traitement récurrent entre les zones corticales auditives et pariétales d'ordre supérieur. 
Dans leurs analyses, \cite{dykstra2016neural} ont trouvé une large composante P300 pour les cibles détectées, associée à des générateurs dans le cortex temporo-frontal et temporo-latéral. 
Dans notre étude, les générateurs seraient assez étendus puisque allant de la zone fronto-centrale antérieure jusqu'à une zone pariétale plus postérieure. 
De cette manière, on observe des valeurs plus élevées de P300 au niveau de l'axe sagittal cérébral du sujet. 
C'est vrai qu'il est intéressant de considérer la diffusion du potentiel électrique selon un axe sagittal en réponse à la détection d'une cible régulière comme un potentiel marqueur de l'intégration des caractéristiques du flux de la cible auditive, lequel pourrait sous-tendre sa ségrégation vis à vis des caractéristiques du flux du masqueur \citep{dykstra2016neural, giani2015detecting}. 
Cela suggérerait de pouvoir considérer la P300 comme un potentiel marqueur de la ségrégation des flux auditifs à l'échelle de l'activité cérébrale macroscopique. 

%%%%%%%%%%%%%%%%%%%%%%%%%%%%%%%%%%%%%%%%%%%%%%%%%%%%%%%%%%%%%%%%%%%%%%%%%%%%%%%
\subsection{Contenu informationnel et complexité du signal EEG}
%%%%%%%%%%%%%%%%%%%%%%%%%%%%%%%%%%%%%%%%%%%%%%%%%%%%%%%%%%%%%%%%%%%%%%%%%%%%%%%

Le deuxième objectif consistait à étudier le contenu informationnel et la complexité du signal neuronal afin de pouvoir caractériser la perception consciente de la cible auditive dans le MI. 
Nous avons étudié la performance de 12 mesures d'entropie et de complexité pour évaluer l'effet de la perception auditive consciente sur l'information et la complexité associées à la dynamique de l'activité cérébrale. 
Nous avons utilisé l'approche usuelle qui consiste à comparer la différence des valeurs de mesure entre les cibles perçues et les cibles non perçues, avant et après la perception ainsi qu'une procédure d'agrégation topographique de ces valeurs sur l'ensemble des électrodes de clusters afin de pouvoir disposer de mesures moyennes associées à des aires cérébrales distinctes. 
Nous avons supposé qu'une perception réussie de la cible auditive augmenterait le degré d'information et de complexité contenu dans les signaux neuronaux associés à un réseau cérébral fronto-temporo-pariétal fonctionnel. 
Nous avons également supposé qu'une augmentation progressive du degré d'information et de complexité serait observée dans les signaux neuronaux issus de ces aires cérébrales lors de la perception consciente. 

Les résultats montrent que la perception auditive consciente a augmenté les valeurs des mesures d'entropie et de complexité (sauf celles de l'exposant $\alpha$) dans le cluster fronto-central. 
Lors de la perception de la cible auditive, toutes les mesures d'entropie et de complexité (excepté l'exposant $\alpha$) ont présenté des valeurs supérieures pour les électrodes les plus latérales FC5 et FC6. 
L'entropie de permutation (PeEn) et la dimension fractale de Petrosian (PFD) étaient les seules mesures à présenter des valeurs plus élevées dans le cluster FC et moins élevées dans la plupart des autres clusters pour les cibles détectées. 
De cette manière, l'entropie de permutation (PeEn) et la dimension fractale de Petrosian (PFD) ont permis de discriminer les cibles détectées des cibles non-détectées pour la plupart des clusters (sauf clusters central et temporal) pour PeEn et pour tous les clusters pour PFD.

Au sein du cluster fronto-central, les mesures d'entropie ont globalement présenté des valeurs plus élevées avant la référence temporelle pour les cibles détectées. 
Dans les mesures de complexité, avant la référence, seules les dimensions fractales de Higuchi (HFD) et de Petrosian (PFD) ont montré des fenêtres temporelles avec des valeurs supérieures pour les cibles détectées, tandis qu'après la référence, HFD, PFD et l'exposant de Hurst ont montré des fenêtres significatives. 
Pour certaines mesures, des pics de valeurs ont pu être observés juste après la référence temporelle, lorsque les cibles ont été détectées. 
Cette augmentation n'était cependant pas visible pour les cibles non-détectées.
Dans les mesures d'entropie, nous avons observé cette hausse de manière qualitative pour l'entropie de décomposition en valeurs singulières (SvEn) environ $400$~ms après le report. 
Globalement, lorsque les cibles étaient détectées, les valeurs des différentes mesures d'entropie étaient plus élevées avant le report. 
C'est à partir du report qu'une diminution était ensuite visible avec des chevauchements entre les mesures pour les cibles détectées et celles non-détectées.
Dans les mesures de complexité, l'exposant de Hurst et le paramètre de complexité de Hjorth ont exprimé un large pic environ $400$~ms après avoir reporté la perception de la cible. 
Cependant, les analyses ont révélé que seul l'exposant de Hurst présentait une variation significative avec des valeurs supérieures juste après la référence pour les cibles détectées comparativement à celles manquées.

Ainsi, lorsqu'une cible auditive est perçue par le sujet, les signaux issus de l'aire cérébrale fronto-centrale présentent une augmentation de leur contenu en information et en complexité. 
Cela suggère ici que la perception auditive consciente est associée à l'augmentation du contenu en information et du degré de complexité dans le signal neuronal des aires fronto-centrales latérales.
Les électrodes les plus latérales du cluster fronto-central (FC5 et FC6), adjacentes à la zone cérébrale temporale laissent supposer que cette augmentation en information et en complexité est associée au cortex auditif. 
L'augmentation en contenu informationnel avec le rapprochement topographique du lobe temporal et du cortex auditif pourrait nous laisser suggérer ici un argument en faveur des mécanismes de conscience associés aux boucles de traitement récurrent \citep{lamme2000distinct, lamme2003visual, lamme2006towards}. 
Il a été suggéré que le traitement local récurrent dans les zones sensorielles reflète l'expérience de l'information sensorielle, et que le traitement global récurrent reflète l'expérience de la mise en mémoire de travail de l'information et l'expérience de la décision de ce qu'il faut faire avec l'information \citep{eklund2019electrophysiological, lamme2000distinct, lamme2003visual, lamme2006towards}.
Des expériences purement phénoménales générées par des boucles locales récurrentes dans les zones sensorielles pourraient être associées à des modifications du contenu en information au sein de ces zones. 
Un lien pourrait alors être établi entre la génération d'une composante ARN au niveau des lobes temporaux et l'augmentation en contenu d'information et de complexité au niveau de ces zones lorsque la perception de tonalités d'une cible auditive émerge chez le sujet. 

Notre analyse sur le contenu informationnel et la complexité du signal EEG dans le MI permet également d'exposer un lien entre la ségrégation des flux auditifs et la conscience perceptive dans la modalité auditive. 
Tant que l'on associe les électrodes latérales du cluster fronto-central aux cortex auditifs, on peut supposer que la conscience perceptive dans le MI reflétée par une ségrégation réussie des flux de la cible et du masqueur suscite une hausse du contenu en information et en complexité du signal EEG issu de cette zone cérébrale. 
Globablement, le système auditif est capable de garder de l’information acoustique en mémoire à court terme et l’information stockée est disponible pour le rappel et le calcul à plus ou moins long-terme \citep{shen2017auditory}. 
Il est donc en mesure d’évaluer la consistance de l’information acoustique sur le temps et de grouper les composantes consistantes en un même courant perceptuel. 
Il est possible que la hausse de valeurs observée soit une conséquence de la mise en place de traitements récurrents au sein des cortex auditifs ayant pour rôle le maintien consistant de l'information acoustique sur le temps. 
Cela pourrait peut-être expliquer les valeurs supérieures pour certaines des mesures d'entropie et de complexité lorsque les cibles auditives ont été détectées comparativement à leur non-détection. 

En fait, l'analyse de la scène auditive est liée non seulement à l'extraction des caractéristiques du stimulus et à la formation et la sélection des objets perceptifs, mais aussi à l'attention sélective, à la liaison perceptive et à la conscience \citep{elhilali2009interaction, kaya2017modelling, kondo2017auditory, moore2012properties, pressnitzer2006temporal, snyder2012attention}. 
Comme l'analyse de scène auditive consiste à analyser un mélange de sons afin de parvenir à des perceptions correspondantes aux sources sonores individuelles \citep{bregman1994auditory}, il suffit de quelques centaines de millisecondes pour activer une grande cascade de régions cérébrales différentes, chacune effectuant une transformation différente de l'entrée sensorielle \citep{groen2017contributions}. 
Dans le MI, le processus de ségrégation des flux auditifs conduisant à la perception de la cible auditive se base sur une cascade de traitements de l'information et d'interactions entre les processus top-down et bottom-up \citep{elhilali2009interaction}. 
La hausse de valeurs du contenu en information et en complexité du signal neuronal observée pourrait également consister en un reflet des traitements progressifs issus de cette cascade dans les cortex auditifs puisque ceux-ci représentent un centre d'intégration de l'information auditive. 
Ainsi, le pic de valeurs observé pour l'exposant de Hurst, mesure de dépendance à long-terme du signal, suggère que la conscience perceptive reflétée par le report perceptif provoque une augmentation dans la complexité du signal neuronal, qui pourrait être un signe de maintien et de consistance de l’information acoustique sur le temps. 

Il est intéressant de lier ces résultats avec ceux sur les composantes ERPs précédemment étudiées. 
Les ondes négative (ARN) et positive (P300) ont été trouvées notamment au niveau des régions antéro-temporales (ARN) et sagittales (P300) environ $300$-$500$~ms après le stimulus. 
Dans la région fronto-centrale, nous avons observé que certaines mesures d'entropie et de complexité (SvEn, exposant de Hurst et complexité de Hjorth) présentaient un pic (significatif pour l'exposant de Hurst) à environ $400$~ms. 
Cette augmentation des valeurs de complexité pourrait être associée aux augmentations des amplitudes des ondes négative (ARN) ou positive (P300) diffuses. 
Comme à l'échelle cérébrale macroscopique, le courant diffuse à la surface du scalp, cette supposition paraît largement plausible. 
Ainsi, la variation de l'amplitude des formes d'ondes évoquées par le stimulus pourrait s'associer à une variation de la complexité du signal neuronal. 
Au-delà du corrélat neuronal ERP, nous mettons en évidence ici un corrélat informationnel sur la base des valeurs de complexité estimées au moyen des caractéristiques statistiques du signal neuronal. 

La manière classique d'extraire des informations du signal EEG pour l'évaluation des états de conscience est d'étudier les changements dans le comportement oscillatoire de l'EEG grâce à des calculs effectués dans le domaine fréquentiel.  
Les mesures de complexité cérébrale permettent de surmonter certaines des limites des mesures usuelles quantifiant la puissance de l'EEG du scalp. 
Des valeurs de complexité semblables à celles de l'état d'éveil peuvent être détectées même lorsque des ondes lentes de forte amplitude dominent l'EEG spontané chez certains patients peu conscients \citep{casarotto2016stratification}. 
Dans le cadre des études sur les états de conscience, l'évaluation de la sensibilité et de la spécificité des mesures de complexité dans différentes conditions permet d'estimer comment une mesure se rapproche des processus neuronaux pertinents pour la présence ou l'absence de conscience. 

Globalement, les études ayant évalué les performances des mesures de complexité du signal cérébral ont fourni une quantification précise dans différents contextes cliniques et se sont accordées sur une même conclusion : la complexité est plus élevée dans les conditions où la conscience est présente et plus faible dans celles où elle est perdue \citep{sarasso2021consciousness}. 
On s'attend donc usuellement à une distribution plus complexe de l'activation neuronale dans un cerveau conscient avec une activité cérébrale dans la conscience présentant une structure et une évolution dans le temps plus complexes que dans les états non-conscients. 
Ce degré de cohérence dans différentes conditions est supérieur à celui caractérisant d'autres catégories de mesures précédemment citées et étudiées, comme les ERPs. 
Lorsqu'elles sont directement comparées, les mesures de complexité surpassent largement les ERPs, comme la P3b, dans la détection des patients peu conscients, ces derniers étant caractérisés par une sensibilité moindre \citep{sarasso2021consciousness, sitt2014large}. 
De même, la complexité reste élevée chez les sujets conscients pendant le sommeil paradoxal ou les hallucinations à la kétamine \citep{casarotto2016stratification, farnes2020increased}, alors que les ERPs associés à des stimuli déviants disparaissent généralement \citep{bravermanova2018psilocybin, strauss2015disruption}. 
Dans l'ensemble, il semble que les mesures de complexité soient plus fiables que d'autres mesures, offrant sans doute non seulement une meilleure précision diagnostique, mais aussi un guide potentiel pour identifier, parmi les nombreuses facettes de l'activité cérébrale, les propriétés centrales qui sont plus pertinentes pour la conscience \citep{kreuzer2017eeg}.
En lien avec nos résultats, il semblerait que l'exposant de Hurst, en tant que mesure de complexité à l'échelle cérébrale d'un contenu de conscience, représente un indicateur caractéristique pertinent qui nécéssiterait des études plus spécifiques dans le cadre de l'évaluation de états de perception consciente du stimulus. 

Des approches plus récentes visant à extraire des informations de l'EEG à différents niveaux d'anesthésie ont utilisé les paramètres non-linéaires qui reflètent le contenu en information, la complexité et/ou la prévisibilité du signal. 
Ces approches permettent d'extraire des informations non-linéaires du signal, alors que les mesures linéaires comme l'entropie spectrale ou l'exposant de Hurst ne permettent pas de détecter ces non-linéarités \citep{anier2010entropy, jordan2009detection, kreuzer2017eeg}. 
Comparées aux approches spectrales, les mesures univariées comme l'entropie approximée (ApEn) et l'entropie de permutation (PeEn) se sont révélées plus performantes pour distinguer l'EEG enregistré pendant la conscience de celui enregistré pendant l'inconscience et pour refléter différents niveaux d'anesthésie générale \citep{bruhn2000approximate, jordan2008electroencephalographic, liang2015eeg}. 
Dans notre analyse, les valeurs de l'entropie de permutation (PeEn) étaient plus élevées dans le cluster fronto-central et moins élevées dans la plupart des autres clusters lors de la perception auditive consciente. 
Ainsi, cela reflète sa capacité à pouvoir extraire des caractéristiques non-linéaires du signal neuronal associées à la conscience perceptive de la cible auditive dans le MI et suggère par conséquent qu'elle nécéssite des études plus approfondies sur son potentiel à représenter un corrélat informationnel de la perception auditive consciente à l'échelle cérébrale. 
De cette manière, les mesures d'entropie et de complexité, en tant que reflets de différents types de propriétés et caractéristiques des signaux cérébraux étudiés demandent à être davantage utilisées en pratique afin de permettre le développement d'algorithmes pertinents. 
Pour compléter ce travail, il sera nécessaire de réaliser une comparaison pragmatique des algorithmes sur la base de leur rapidité de calcul, ainsi que leur fiabilité et leur robustesse dans l'estimation des valeurs.

%%%%%%%%%%%%%%%%%%%%%%%%%%%%%%%%%%%%%%%%%%%%%%%%%%%%%%%%%%%%%%%%%%%%%%%%%%%%%%%
\subsection{Transmission de l'information}
%%%%%%%%%%%%%%%%%%%%%%%%%%%%%%%%%%%%%%%%%%%%%%%%%%%%%%%%%%%%%%%%%%%%%%%%%%%%%%%

Le troisième objectif consistait à étudier la transmission de l'information à l'échelle cérébrale lors de la perception consciente de la cible auditive dans le MI. 
Nous avons étudié cela au moyen de deux mesures de dépendance linéaire (covariance et corrélation) et deux mesures de dépendance non-linéaire (information mutuelle et entropie de transfert). 
Nous avons supposé que la perception auditive consciente serait caractérisée par des mécanismes de transfert d'information plus importants entre les cortex frontal, temporal et pariétal, avec notamment un flux antéro-postérieur plus élevé. 

Les résultats montrent que les signaux issus des électrodes situées au niveau du lobe temporal des deux hémisphères covarient ensemble linéairement lorsque les sujets ont détecté les cibles auditives.
L'activité EEG de l'aire antéro-frontale gauche covariait de manière significative avec des zones centrales et centro-pariétales lors de la perception de la cible auditive. 
Le transfert d'information dirigé, mis en évidence par l'entropie de transfert, était également supérieur lors de la perception auditive consciente. 
Les électrodes pariétales postérieures de l'hémisphère droit (P6 et P8) étaient des cibles vers lesquelles étaient dirigés des flux d'information. 
L'information arrivant au niveau de ce cluster pariétal provenait notamment de plusieurs zones contra-latérales telles que les aires temporale, fronto-centrale, centrale ou encore pariétale. 
Nos résultats suggèrent que les électrodes pariétales P6 et P8 sont associées à un centre pariétal postérieur fonctionnel dont le rôle serait corrélé à la perception auditive consciente. 

Précédemment, l'activité associée à un réseau fronto-temporo-pariétal a été considérée comme un substrat neuronal essentiel dans les mécanismes associés à la perception auditive consciente \citep{demertzi2013consciousness, dykstra2017roadmap, eklund2019electrophysiological, eriksson2007similar, eriksson2017activity, giani2015detecting, wiegand2018cortical}. 
Des réponses neuronales au sein de l'activité du réseau fronto-pariétal accompagnent notamment les transitions entre les états perceptifs, mettant en évidence le rôle fonctionnel de ce réseau sur les mécanismes qui sous-tendent les organisations perceptives au cours de la perception bistable \citep{knapen2011role}.
Des réponses transitoires ont également été observées dans un réseau latéral droit du cortex fronto-pariétal au moment des transitions perceptives, suggérant un possible reflet des mécanismes d'initiation neuronale des transitions \citep{kleinschmidt2002human, lumer1998neural, sterzer2007neural}. 
En outre, le cortex pariétal postérieur, recevant de nombreuses entrées multisensorielles, a été associé à une multitude de fonctions reliant la perception à la planification et à l'action \citep{andersen2009intention}.
Il a aussi été suggéré que l'activité du cortex pariétal est essentielle à l'organisation perceptive et plus particulièrement à la liaison des différentes caractéristiques des objets auditifs \citep{cusack2005intraparietal}. 
L'activité du cortex pariétal est augmentée lorsque l'organisation perceptive donne lieu à deux flux auditifs plutôt qu'un seul et a également été modulée par des demandes d'intégration liées au stimulus auditif dans le MI, notamment lorsque les sujet ont identifié la cible \citep{eriksson2017activity}. 
De cette manière, l'activité cérébrale au sein du réseau fronto-pariétal a globalement été associée à des changements de la conscience perceptive se produisant pendant les transitions entre les percepts. 

Dans notre analyse, nous avons observé qu'une zone pariétale postérieure et une zone antéro-frontale droites étaient la cible de flux d'information lorsque la cible auditive a été perçue. 
Le transfert d’information vers la première zone vient appuyer le rôle des aires pariétales droites dans leur capacité à associer les représentations sensorielles et perceptives liées à l’intégration des caractéristiques des stimuli. 
En outre, les flux d'information dirigés vers la seconde zone viennent appuyer le rôle de l'activation d'un réseau fronto-pariétal en réponse aux fluctuations des transitions perceptives et des représentations des flux auditifs.
L'attractivité des flux d'information en direction du cortex pariétal postérieur fait écho à l'étude de \cite{pereira2021evidence} dans laquelle la détection du stimulus était liée à une augmentation du taux de décharges des neurones et de l'activité ECoG enregistrée dans le cortex pariétal postérieur, ainsi qu'à une augmentation de la réponse EEG de surface. 
Les auteurs ont conclu que les changements graduels dans la dynamique neuronale pendant l'accumulation de preuves étaient liés à la conscience perceptive et au monitoring perceptif \citep{pereira2021evidence}.
Les preuves pourrait s'accumuler dans le cortex pariétal postérieur jusqu'à ce qu'elles franchissent un seuil, déclenchant ainsi la coalescence de plusieurs réseaux encapsulés en un seul réseau responsable de la diffusion des signaux neuronaux dans tout le cerveau.

De cette façon, le cortex pariétal postérieur aurait un rôle de déclencheur de l'embrasement neuronal sous-jacent à la conscience perceptive. 
Ce mécanisme proposé sur la base d'un processus en tout ou rien impliquant un seuil serait compatible avec plusieurs approches théoriques \citep{windey2015consciousness} et notamment la théorie de l'espace de travail neuronal global, pour laquelle il est nécessaire que l'activité neuronale correspondante soit diffusée globalement dans le cortex afin que le stimulus soit perçu consciemment \citep{mashour2020conscious}. 
La diffusion globale serait déclenchée lorsqu'un processus (inconscient) d'accumulation de preuves atteint un seuil similaire aux processus physiologiques qui sous-tendent la prise de décision. 
Les résultats de notre analyse viennent appuyer cette hypothèse selon laquelle les neurones situés au niveau du cortex pariétal postérieur pourraient correspondre à un centre fonctionnel accumulant de l'information en provenance de zones cérébrales diverses, reflété par une augmentation significative du transfert d'information vers les électrodes pariétales pour les cibles perçues. 

En outre, \cite{bidet2007mecanismes} a étudié les mécanismes neurophysiologiques impliqués dans la perception auditive consciente en considérant la ségrégation des flux et l'extraction d'attributs fréquentiels et spatiaux en utilisant des réponses d'enregistrements EEG de surface ou intracérébraux. 
Elle a montré que des mécanismes d'attention auditive contrôlés par un réseau fronto-pariétal ont facilité la sélection d'un son dans un mélange acoustique en augmentant les réponses corticales aux informations pertinentes et en diminuant celles aux sons distracteurs \citep{bidet2007mecanismes}. 
Une étude récente a montré que certaines zones frontales pourraient faire partie intégrante du réseau central pour l'accès conscient spontané, même en l'absence d'une tâche manifeste, comme l'attestent les activations tardives et soutenues dans les sources frontales inférieures pendant de l'écoute passive \citep{sergent2021bifurcation}.
Couplé à nos résultats, cela vient soutenir l'hypothèse que les zones antéro-frontales et pariétales sont des centres de connexions importants dans le cadre de la sélection des informations accumulées et que la ségrégation des flux peut se faire à partir d'un traitement intégratif de haut niveau des attributs fréquentiels et spatiaux dans ces zones, sur la base de processus attentionnels et conscients. 
Néanmoins, la dissociation entre l'activité frontale et pariétale et leur lien respectif avec la perception consciente reste insuffisamment décrite par les modèles neuronaux actuels de la conscience. 
En particulier, il n'est pas clair si ces deux régions jouent un rôle similaire dans la sélection, le maintien et la diffusion de l'information dans et à travers le cortex ou, au contraire, si elles sont spécialisées dans l'une de ces fonctions \citep{king2014characterizingthesis}.

Dans un protocole de MI utilisant des composantes de langage plutôt que des tonalités, \cite{szalardy2019neuronal} ont montré que la détection de la cible n'est pas améliorée avec des masqueurs présentant une intensité acoustique faible (\textit{i.e.}, plus douce), mais que c'est la suppression des composantes distractives (\textit{i.e.}, les masqueurs) qui devient moins efficace. 
Ces résultats semblent suggérer que des distracteurs moins intenses nécessitent plus de ressources cognitives \citep{szalardy2019neuronal}.
Contrairement au ME induit dans le système auditif périphérique par le chevauchement des réponses neurales à la cible et au masqueur, les mécanismes neuronaux et localisations propres au MI sont encore mal compris \citep{shinn2008object} et cela avait conduit à l'hypothèse de l'existence d'un goulot d'étranglement du traitement de l'information dans le cerveau \citep{overath2007information, gutschalk2008neural}. 
Dans ce goulot d'étranglement, une certaine «quantité putative» viendrait limiter la ségrégation des flux auditifs et ainsi provoquer le masquage informationnel d'ordre plus «central». 
Cette hypothèse repose largement sur des aspects de limitation de ressources cognitives et attentionnelles et de transmission de l'information à l'échelle cérébrale, et fournit une contrainte limitante au niveau informationnel qui serait à l'origine de l'incapacité du stimulus d'accéder à la conscience perceptuelle.  
En utilisant des mesures de transfert d'information dirigé, il serait intéressant d'étudier dans quelle mesure ce goulot d'étranglement de l'information pourrait être caractérisé au sein des patterns d'activation dans le réseau fronto-temporo-pariétal afin de montrer la présence d'un tel goulot qui pourrait être lié au seuil d'accumulation d'évidence de l'information à l'échelle cérébrale \citep{barniv2015auditory, nguyen2020buildup, pereira2021evidence}. 

%%%%%%%%%%%%%%%%%%%%%%%%%%%%%%%%%%%%%%%%%%%%%%%%%%%%%%%%%%%%%%%%%%%%%%%%%%%%%%%
\subsection{Intégration de l'information}
%%%%%%%%%%%%%%%%%%%%%%%%%%%%%%%%%%%%%%%%%%%%%%%%%%%%%%%%%%%%%%%%%%%%%%%%%%%%%%%

Le quatrième objectif consistait à utiliser une méthodologie fondée sur une approche théorique de la conscience pour caractériser la dynamique de l'activité cérébrale lors de la perception consciente de la cible auditive.
Nous avons étudié le décours temporel de l’intégration de l’information pour les cibles perçues et non-perçues, avant et après la référence temporelle, avec un décalage temporel usuellement nécessaire pour le calcul de ces mesures. 
Nous avons supposé que la perception de la cible auditive susciterait un niveau plus élevé d'information intégrée par rapport à l'absence de perception. 
Nous avons également supposé qu'une hausse progressive de la quantité d'information intégrée dans le temps serait observée jusqu'à atteindre le report explicite conscient par le sujet lors de l'appui-bouton, symbolisant ainsi l'augmentation du niveau d'intégration de l'information jusqu'au seuil de l'accès conscient. 

Les résultats montrent que la perception auditive consciente a augmenté l'intégration de l'information dans les aires cérébrales temporales sur la base d'une augmentation des valeurs de la redondance d'information estimée par $\Phi^{MI}$. 
En effet, les valeurs d'information intégrée par redondance $\Phi^{MI}$ étaient plus élevées pour les cibles détectées comparativement aux cibles manquées dans le cluster temporal.
Dans ce cluster, des fenêtres temporelles significatives ont également été observées avant la référence temporelle entre les cibles détectées et celles manquées pour cette mesure $\Phi^{MI}$. 
Les valeurs de $\Phi^{MI}$ étaient supérieures avant la référence temporelle lors de la perception auditive consciente. 
Néanmoins, c'est une diminution progressive au cours du temps des valeurs de $\Phi^{MI}$ qui a été observée lors de la perception auditive consciente. 
Dès le report perceptif, une diminution importante de la quantité d'information intégrée par redondance au cours du temps a également pu être observée jusqu'à ce que les valeurs de $\Phi^{MI}$ dépassent celles des cibles non-perçues. 
Nos résultats montrent que le report perceptif du sujet lors de la perception auditive consciente a provoqué une chute de l'information intégrée sur la base des valeurs de redondance de l'information dans le lobe temporal chez l'humain. 

Ce travail fournit une caractérisation spécifique de la dynamique de l'accès conscient de la cible auditive à l'échelle cérébrale macroscopique à travers l'intégration de l'information au niveau de la zone sensorielle auditive. 
L'augmentation du niveau de redondance de l'information lors de la conscience perceptive d'une cible auditive représente un argument corrélationnel pour la théorie du traitement récurrent dans laquelle l'activité précoce dans les zones sensorielles primaires correspond étroitement à la conscience phénoménale. 
Cependant, les valeurs d'information intégrée par redondance supérieures lors de la conscience perceptive de la cible auditive dans l'aire cérébrale sagittale peut souligner le rôle d'intégration des caractéristiques du stimulus à un degré plus élevé de traitement et peut également venir soutenir les théories de l'information intégrée et de l'espace de travail neuronal global. 

Dans la théorie de l'espace de travail neuronal global, les informations sont sélectionnées sur la base de leur pertinence pour les objectifs de la tâche, et franchissent le seuil de l'accès conscient pour entrer dans l'espace de travail global pour un partage flexible \citep{dehaene2003neuronal,kemmerer2015we}. 
Cette mise en commun des informations sensorielles au sein d'un plus vaste réseau incluant des aires supra-modales, serait ce qui permet de maintenir l'information plus longtemps, de la mémoriser explicitement, de la reporter, et plus généralement de l'intégrer à la planification des actions \citep{dehaene2014consciousness}.
Selon cette théorie, la perception consciente correspond à une étape tardive et optionnelle du traitement cérébral d'un stimulus, se produisant après les étapes purement sensorielles \citep{khamassi2021neurosciences}.
Nos résultats ont montré que l'aire sagittale est associée à une augmentation des valeurs de l'information intégrée par redondance lors de la perception auditive consciente. 
À ce moment, le fait que les valeurs d'information intégrée par redondance soient supérieures dans le cluster temporal par rapport au cluster sagittal pourrait aussi signifier que le traitement récurrent de l'information est plus intense dans les aires sensorielles avant d'être relayé au sein de l'espace de travail global pour permettre le partage flexible, ce qui se refléterait par une activité plus tardive dans la zone sagittale. 

En outre, dans la théorie de l'espace de travail neuronal, la composante P300, en tant que reflet de la conscience perceptive, a été associée à une intégration globale des caractéristiques du stimulus lors de la perception consciente. 
Dans notre étude, les valeurs supérieures d'amplitude de l'onde P300 et d'information intégrée par redondance au niveau de l'aire sagittale pour la perception consciente pourraient vraisemblablement indiquer que ces deux marqueurs sont liés et représentent des formes de traitements récurrents associés à l'intégration des caractéristiques du stimulus à l'échelle cérébrale macroscopique. 
Ces mécanismes d'intégration partagés sur plusieurs aires cérébrales lors de l'accès conscient de la cible auditive font principalement écho à l'activation sous-jacente à la perception auditive consciente d'un réseau centro-temporal, qui pourrait représenter des processus spécifiques du réseau plus général fronto-temporo-pariétal impliqué dans les phénomènes conscients. 
On pourrait penser à des activations spécifiques du réseau centro-temporal amenant à favoriser les représentations des informations sensorielles associées à la ségrégation des flux auditifs et à initier leur partage au sein de l'espace de travail global. 

Nous avons trouvé une diminution significative des valeurs d'information intégrée par redondance dès le report de la cible dans l'aire temporale, et qualitative au niveau de l'aire sagittale. 
Le cortex auditif possède une activité associée aux traitements intégratifs de l'information liée au stimulus sonore et il serait possible que cette diminution soit due à un ralentissement des mécanismes de décodage des informations sensorielles et de l'intégration des informations liées aux différents flux appartenant au stimulus. 
Ces résultats pourraient être mis en relation avec les résultats de modélisation de l'étude de \cite{sergent2005dynamique}, dans laquelle un accès à la conscience semblait correspondre au franchissement d'un seuil au-delà duquel il devient possible de reporter explicitement la perception du stimulus cible.
Le franchissement de ce seuil correspondrait à une transition tardive et rapide dans l’activité neuronale évoquée par le stimulus, et engageant une série de processus neuronaux spécifiques du traitement conscient. 
Les auteurs ont proposé que la dynamique discontinue observée empiriquement pourrait reposer sur la mise en œuvre de mécanismes d'amplification et de maintien de l’activité au sein d’un réseau d’aires cérébrales distantes. 
Cet état relativement stable de communication globale permettrait alors de rendre les informations perceptives accessibles pour un ensemble de traitements, notamment pour le report explicite. 
La diminution des mécanismes de redondance de l'information dans l'aire temporale peut nous amener à une identification possible des étapes successives du traitement perceptif du stimulus et de la transition vers l’accès conscient dans le réseau fronto-temporo-pariétal \citep{eriksson2007similar, eriksson2017activity, giani2015detecting}.
Il serait alors intéressant de cibler les succesions de transmission de l'information au sein de ce réseau en adoptant par exemple une approche de modélisation causale dynamique couplée à des mesures de transmission de l'information, pour essayer de dissocier ces différentes étapes de transition vers l'accès conscient. 

Dans la TII, le passage du traitement inconscient au traitement conscient est marqué par une augmentation massive de l'intégration de l'information. 
Dans le cadre de la percepion consciente, on peut largement supposer que le degré de complexité issu de cette intégration est généré dans certaines parties du cerveau mais pas dans d'autres \citep{sarasso2021consciousness}. 
Cela consiste en un problème supplémentaire à définir les frontières du «sous-ensemble» d'éléments pertinents et la mesure à laquelle ces frontières sont variables dans le cadre de la perception consciente de stimuli.  
Dans la formulation originale de l'hypothèse du noyau dynamique, une stratégie était définie pour identifier, sur la base de dépendances statistiques le groupe fonctionnel générant la complexité pertinente \citep{tononi1998consciousness}.
Cette stratégie a été précisée par une perspective causale dans la dernière formulation de la TII \citep{oizumi2014phenomenology}. 
Le problème se résume à trouver les frontières qui incluent le sous-ensemble d'éléments qui génèrent plus de complexité que tout autre sous-ensemble, plus petit ou plus grand. 

Malgré ce difficile problème propre à la TII, notre travail essaie d'apporter des réponses à la détermination des différents sous-ensembles générateurs de complexité et de leurs frontières. 
L'activité des aires sensorielles auditives exprime un traitement redondant d'information qui permet de rendre compte d'un certain niveau d'intégration de l'information à l'échelle du cortex auditif. 
La zone cérébrale sagittale est associée à une hausse de cette redondance de l'information lors de la conscience perceptive. 
Les aires fronto-centrales montrent une complexité accrue lors de cette conscience de la cible auditive. 
Enfin, l'aire pariétale postérieure droite condense un centre informationnel attracteur des flux d'informations en provenance des aires contra-latérales. 
Ces résultats suggèrent simplement de voir la conscience perceptive comme un phénomène multi-facettes, donnant lieu à des représentations physiques distinctes mais liées qui s'inscrivent dans de multiples substrats neuronaux à l'échelle cérébrale. 

Pour élucider les substrats neuronaux de la conscience, il est d'abord nécessaire de déterminer quelles représentations mentales, dans le flux du traitement de l'information, atteignent ou non la conscience. 
À notre connaissance, les études originales ayant menées au développement de la TII (et encore avant, de l'hypothèse du noyau dynamique) ont été réalisées dans le cadre des études sur les états de conscience des sujets humains. 
Aujourd'hui, il est largement accepté que les états de conscience sont médiés par des mécanismes panmodaux ou intermodaux \citep{bachmann2020commentary}. 
L'intégrité des circuits thalamo-corticaux semble absolument nécessaire pour qu'un état de conscience puisse émerger. 
On comprend de cela que la mesure à laquelle l'information est intégrée dans de tels circuits est apparu pour les précurseurs/théoriciens de la TII un gage de qualité, voire une «potentialité» à exprimer la conscience humaine. 
En fait, la TII cherche à expliquer la conscience humaine sur la base de postulats théoriques et de recherches empiriques, mais elle va bien au-delà de cela.
La TII pose un postulat généralisable à un ensemble de systèmes du monde réel dans lequel les organismes biologiques sont largement représentés. 

Les fondements de la TII sont les éléments de la théorie de l'information classique. 
La TII cherche à établir une façon de quantifier la communication dans un système sur la base de ses propriétés (ségrégation, différentation, intégration). 
Cette communication est plus précisément comprise comme une mise à disposition dans un espace commun de l'information et la manière dont l'information va circuler dans cet espace et être contrainte par les propriétés du système dans son état actuel va représenter la quantité potentielle à estimer. 
Comme des processus supramodaux communs pourraient sous-tendre la perception consciente dans différentes modalités sensorielles, il pourrait apparaître pertinent que l'augmentation de l'intégration d'information dans les clusters temporal et sagittal découverts ici soit en lien avec la généralisation de l'information disponible concernant le stimulus.
La TII pourrait nous permettre de savoir si les mécanismes associés aux traitements et à l'émergence des contenus de conscience sont universels (panmodaux/intermodaux) ou spécifiques à chaque modalité sensorielle. 

Finalement, dans un système aussi complexe que le cerveau humain, on s'attend lors de phénomènes conscients à pouvoir disposer d'un vaste répertoire de structures de complexité informationnelle générées par les différents sous-ensembles. 
Ces phénomènes sont susceptibles de générer des états distincts de ce répertoire qui dépendent de la capacité des nombreux modules fonctionnellement spécialisés du système thalamo-cortical à interagir rapidement et efficacement. 
En considérant les transitions perceptuelles dans le MI, avec le fait que pour un même stimulus le sujet perçoit certaines tonalités cibles et n'en perçoit pas d'autres, l'aspect transitoire et changeant de telles structures représentationnelles est une perspective hautement considérable dans l'étude des mécanismes associés à la perception consciente du stimulus. 
Déterminer quelle activité à l'échelle microscopique est liée aux représentations purement sensorielles, et laquelle est plus étroitement liée à la perception, et quels autres composants sont nécessaires dans le réseau produisant la conscience perceptive auditive est un objectif pour de futures études. 

Mesurer l'intégration de l'information à plusieurs échelles et chercher à en spécifier les structures d'information conceptuelle intégrée lors de paradigmes de stimulation multistables tels que le streaming auditif ou le MI, pourrait nous aider à mieux comprendre les transitions d'organisation perceptive et les commutations entre les états perceptuels alternatifs et ainsi à caractériser la dynamique de la construction de la perception consciente. 
Cette approche n'en est qu'à ses débuts, et des preuves expérimentales de la discrimination des états perceptifs conscients ainsi qu'une description adéquate des structures informationnelles à partir de données de séries temporelles réelles est nécessaire \citep{de2019fractal}.
Globalement, nous pensons que nos résultats vont dans ce sens et apportent un argument expérimental pour que la théorie de l'information intégrée et les mesures qu'elle propose soient mises à l'épreuve de manière plus rigoureuse et élargie dans le cadre des phénomènes de perception consciente. 

%%%%%%%%%%%%%%%%%%%%%%%%%%%%%%%%%%%%%%%%%%%%%%%%%%%%%%%%%%%%%%%%%%%%%%%%%%%%%%%
\section{Limites et perspectives associées à ce travail de thèse}
%%%%%%%%%%%%%%%%%%%%%%%%%%%%%%%%%%%%%%%%%%%%%%%%%%%%%%%%%%%%%%%%%%%%%%%%%%%%%%%

\subsection{Sur le protocole psychoacoustique et le masquage informationnel}

L'absence de screening audiométrique lors de nos expériences est une première limite de ces études. 
Nous n'avons considéré que des sujets qui ont reporté avoir une audition normale dans la population générale, sans nous en être assurés objectivement, ce qui peut être problématique dans le cadre d'une étude spécifiquement psychoacoustique. 
Il est raisonnable de supposer que les personnes souffrant d'une déficience auditive périphérique peuvent présenter une susceptibilité différente au MI ou une utilisation différente des indices pour sortir du masquage, par rapport aux personnes ayant une audition normale \citep{amiri2020overview}. 
La question serait de savoir dans quelle mesure ces différences sont attribuées à des processus impliqués dans les mécanismes périphériques tels qu'une sensibilité réduite, une compression auditive ou des filtres auditifs plus larges. 
Néanmoins, il a été montré que la déficience auditive neurosensorielle n'augmente pas la susceptibilité des individus au MI et que les personnes malentendantes peuvent même avoir moins de MI que les personnes ayant une audition normale \citep{alexander2004informational, micheyl2000informational}. 
De cette manière, le fait que les sujets de nos expériences soient agés entre $18$ et $38$ ans laisse penser que des troubles auditifs sont peu probables. 

Le faible nombre de sujets impliqués dans les groupes indépendants étudiés dans les trois expériences (n$=13$-$14$) est une autre limite. 
Bien que nous montrons que les analyses de modélisation de survie soient utiles et pertinentes pour caractériser l'influence spécifique des paramètres de la stimulation sur la conscience perceptive, les groupes recrutés sont de faibles tailles comparativement à d'autres études purement psychoacoustiques. 
Néanmoins, la puissance des analyses de survie réside principalement dans le nombre d'évènements observés, lequel était suffisament élevé dans nos expériences. 

Le fait que nous n'avons utilisé qu'une seule ERB de part et d'autre du signal cible (et non deux comme dans \cite{gutschalk2008neural}) pourrait être une limite dans le sens où cela pourrait contribuer à laisser une part de masquage énergétique dans l'étude. 
Bien que nous ayons vu que les phénomènes de masquage énergétique et de masquage informationnel sont d'ordre différent, la présence de masquage énergétique dans le cadre d'une étude spécifique au MI peut rendre plus difficile l'interprétation des résultats. 
Néanmoins, puisque le MI est un phénomène auditif encore mal compris et que des chevauchements puissent être à même de se produire entre ME et MI \citep{durlach2006auditory}, on peut supposer que la taille de l'ERB utilisée dans nos études ne soit pas un aspect contraignant sur l'interprétation de nos résultats, mais consisterait cependant en un paramètre supplémentaire à étudier. 

D'un autre coté, les tonalités cibles étaient présentées au même niveau que les tonalités individuelles du masqueur, c'est-à-dire un rapport de niveau cible/masqueur de $0$~dB \citep{dykstra2016neural}. 
De ce fait, les cibles peuvent s'avérer moins saillantes lorsque présentées simultanément avec des masqueurs de fortes densités spectro-temporelles. 
Ainsi, les futures études prendront le soin de considérer plus spécifiquement le rapport signal/bruit lors de la création des stimuli. 
Également, nous voyons l'absence de manipulation du paramètre de niveau d'intensité acoustique des signaux cible et masqueur comme une limite supplémentaire. 
Bien qu'ayant été relativement bien étudié précédemment dans la littérature, l'influence de l'intensité acoustique pourra être étudiée dans le cadre d'un paradigme comme le notre au moyen d'analyses par modèles de survie et par modèles d'accumulation. 

De plus, bien que s'étant appuyés sur certaines des valeurs de paramètres échantillonnées dans la littérature, nous avons utilisé un protocole modifié du paradigme de MI classiquement employé par \cite{dykstra2016neural, gutschalk2008neural, neff1987masking, wiegand2012correlates, wiegand2018cortical}.
En effet, nous nous sommes basés sur une approche couplant distribution fréquentielle et distribution temporelle des tonalités du masqueur pour construire nos stimuli. 
Les tonalités du masqueur ont ainsi été créées sur la base de séquences spectrales (\textit{i.e.}, le nombre de fréquences par octave) plutôt que sur la base du nombre de tonalités par fenêtre temporelle comme dans les situations usuelles. 
Néanmoins, cette différence nous a permis d'étudier ces aspects et ne constitue pas une vraie limite de notre travail. 

\subsection{Sur le report de la perception consciente}

L'utilisation de mesures de reports subjectifs supplémentaires permettant d'évaluer la métacognition (\textit{i.e.}, la performance métacognitive) est une perspective très intéressante à ce travail. 
La métacognition représente la capacité introspective sur ses propres états mentaux et a été principalement caractérisée à travers des reports de confiance dans divers types de tâches. 
La supra-modalité de la métacognition semble reposer sur des estimations de confiance supra-modales et sur des signaux de décision partagés entre les modalités sensorielles. 
Dans un protocole comme le notre, on pourrait demander au sujet de renseigner son degré de certitude/confiance quant-à sa réponse donnée au cours de l'essai \citep{khamassi2021neurosciences, pereira2021evidence}. 
Ce faisant, le report explicite serait dans ce contexte obligatoire. 
Récemment, l'utilisation des approches comparatives classiques pour délimiter les corrélats neuronaux de la conscience a été vivement critiquée \citep{aru2012distilling, tsuchiya2016no}
Elle serait à même de confondre les mécanismes cognitifs et neuronaux associés à l'expérience phénoménale en soi de ceux associés à la déclaration de l'expérience phénoménale (\textit{i.e.}, son report explicite) \citep{aru2012distilling}. 
Une grande partie de l'activité neuronale, accompagnant la perception consciente d'un stimulus, ne refleterait pas la conscience phénoménale en soi, mais plutôt un ou plusieurs processus antérieurs ou postérieurs, tels que l'attention, la mémoire, la prise de décision et le report explicite \citep{dembski2021perceptual}. 
Chercher à établir des liens entre le report explicite, la performance métacognitive et les mesures électrophysiologiques serait très intéressant dans l'étude des mécanismes cérébraux associés à la perception consciente chez l'être humain. 
En effet, dans le cadre de l'étude de la conscience de la situation en aéronautique notamment \citep{endsley1999level, endsley2017here}, ces mesures de performance métacognitive sont pertinentes puisqu'elles permettent de prendre en compte la variabilité des degrés de confiance que le pilote peut avoir. 

Le report explicite synchronisé du sujet à chaque tonalité du signal cible pourrait également apparaître comme une perspective très intéressante à ce travail. 
Demander au sujet de réaliser une tâche de synchronisation sensorimotrice (\textit{i.e.}, ici de tapping) chaque fois qu'il perçoit une tonalité de la cible renseigne sur la ségrégation des flux auditifs. 
Dans ce cas, le sujet doit simplement appuyer sur le presse-bouton à chaque tonalité de la cible qu'il détecte. 
Dans cette approche, la bistabilité perceptive disponible lors du MI pourrait être étudiée de manière plus approfondie, comme dans les études des reports alternés lors du streaming auditif. 
Les probabilités de ségrégation dans le streaming et le MI auditif étant modélisées à partir de fonctions d'accumulation, les modèles d'accumulation d'évidence couplés aux modèles de survie peuvent ainsi permettre de mieux renseigner sur les processus de l'organisation perceptive dans le système auditif. 
En couplant de tels modèles à un paradigme de synchronisation sensori-motrice et en étudiant l'activité EEG, cela pourrait fournir des indications pertinentes sur les corrélats neuronaux macroscopiques de la perception auditive consciente au-delà de ce que nous avons pu fournir dans ce travail. 
En associant les fonctions d'accumulation d'évidence aux fonctions de risque de la détection avec du tapping sensorimoteur, on pourrait mieux comprendre en quoi la perception du flux cible global se différencie d'une perception d'évidence accumulée sur les tonalités successives de la cible auditive. 

\subsection{Sur le protocole EEG et les mesures utilisées}

Dans cette seconde étude, nous n'avons pas réalisé une étude approfondie des paramètres rentrant en jeu dans le calcul des estimateurs des différentes mesures (pour celles dépendant de paramètres). 
Au contraire, nous nous sommes basés sur les valeurs recommandées dans les études précédentes, ce qui est le cas usuel dans la littérature. 
Cependant, une étude rigoureuse chercherait nécessairement à évaluer l'influence de l'ensemble des paramètres dans le comportement en sortie des valeurs observables des mesures. 
De plus, nous n'avons pas réalisé de comparaison des temps de calcul des différentes mesures d'entropie et de complexité. 
Bien que certaines mesures ont montré des procédures d'estimations rapides (exposant de Hurst et paramètres de Hjorth), ce n'est pas le cas de toutes (comme l'exposant fractal $\alpha$).

Tous les calculs nécessaires à l'obtention des différentes mesures (entropie, complexité et information intégrée) ont été réalisés sur un serveur de calcul au sein des locaux de l'ONERA, disposant de 40 cœurs physiques. 
Pour les mesures d'information intégrée, les temps de calcul obtenus étaient très longs pour un nombre d'électrodes restreint et pour une taille d'échantillons temporels faible (voir Table~\ref{fig:table5tempsdecalculinfointegree}). 
Malheureusement, le temps de calcul nécessaire à l'obtention de chacune des mesures n'a pas été déterminé avec précision. 
C'est une limitation qui peut se résoudre simplement dans de futures expérimentations. 
L'importance de la durée de temps de calcul est donc à moduler ici en fonction de la mesure calculée. 

Les temps de calcul ont été obtenus sur deux ensembles de données séparés (Hits et Miss) pour les quatre métriques et correspondent pour chaque ensemble au temps obtenu pour le calcul de tous les essais dans l'ensemble. 
Pour les cibles détectées, nous avons reporté une moyenne de $1.72$ heure par essai (soit $103$ min), tandis que pour les cibles non-détectées la moyenne était de $1.03$ heure par essai (soit $61.92$ min). 
Si on «suppose» ici un temps de calcul approximativement équivalent pour chacune des quatre mesures, on obtient une moyenne de $0.43$ heure par essai (soit $25.8$ min) pour les cibles détectées et une moyenne de $0.25$ heure par essai (soit $15.45$ min) pour les cibles non-détectées. 
Pour un essai de $6$ secondes, le temps pour obtenir le décours temporel de l'intégration d'information sur ces $6$ secondes approcherait les $1500$ secondes pour un essai cible détectée et $900$ secondes pour un essai cible non-détectée. 
Ces résultats montrent une nouvelle fois que l'utilisation en temps réél de mesures d'information intégrée souffrent de problèmes d'application non-négligeables. 

Comme les signaux physiologiques sont très sensibles au bruit, la plupart des pipelines de traitement comprennent une étape de prétraitement avant l'extraction des caractéristiques afin d'améliorer le rapport signal/bruit.
Pour cette raison, les différentes étapes des pipelines d'une interface homme-machine (IHM) se doivent d'être fonctionnelles en termes de temps de calcul \citep{roy2020can}. 
Cependant, la plupart des recherches menées aujourd'hui sur les IHM impliquent rarement une estimation en ligne et elle est généralement réalisée hors ligne. 
Ce problème d'utilisation pratique est à nuancer avec les performances de calcul actuellement développées à partir d'outils technologiques de plus en plus petits associé aux systèmes de pointe. 
En fait, ce que l'on peut finalement envisager à partir de nos résultats des mesures d'information intégrée, c'est que sur la base de systèmes de neuroimagerie portatifs de petite dimension \citep{somon2020unobtrusive}, nous pourrions nous pencher sur la meilleure façon de calculer de telles mesures à partir de l'échantillonnage temporel du signal d'activité cérébrale. 
Un dispositif de ce genre, basé sur de nouvelles architectures matérielles de calcul \citep{nuno2021real}, estimerait en temps réél, les différentes «sous-mesures» (les informations mutuelles conditionnelles notamment) nécessaires à l'obtention des mesures globales d'information intégrées de manière à pouvoir rendre compte du processus d'intégration de l'information à une échelle localisée. 

Dans ce travail, l'objectif n'était pas d'implémenter des procédures de traitement en temps-réel mais plutôt de trouver des indicateurs à l'échelle cérébrale EEG de la conscience perceptive d'une cible auditive. 
Dans le cadre des implémentations actuelles des IHM en aéronautique, l'objectif est de pouvoir mettre en œuvre un système neuro-adaptatif permettant d'estimer les états cognitifs et mentaux en temps réel \citep{verdiere2019approche}. 
Un cockpit neuro-adaptatif tiendrait compte de ces états associés au contenu de conscience de l'opérateur afin d'adapter de façon dynamique le comportement du système IHM et des boucles d'automatisation qui en ressortent. 
L'un des principaux obstacles au développement des technologies neuro-adaptatives est que leur précision pour détecter de tels états spécifiques est encore trop faible.
Cela est largement dû au fait que la connaissance sur les mécanismes et processus neuronaux associés à ces états «prédéfinis» est loin d'être élucidée. 
C'est pourquoi, en plus de s'attacher à développer des technologies neuro-adaptatives pratiques, il est nécessaire en parallèle de continuer à caractériser à un niveau fondamental les mécanismes cognitifs des états et contenus de conscience d'intérêt. 

Nos choix ont été portés vers, d'une part, des mesures issues de la théorie de l'information et des mesures dérivées de complexité du signal et, d'autre part, des mesures issues de la théorie de l'information intégrée de la conscience. 
Cependant, de nombreuses mesures sont disponibles et ont été développées dans le cadre de théories spécifiques ou de contextes expérimentaux particuliers.  
En fait, la littérature actuelle regorge de mesures, potentiellement ou non applicables au domaine de l'électrophysiologie humaine. 
Les différentes mesures étudiées au cours de ce travail peuvent être regroupées, classées et catégorisées en fonction de ces caractéristiques et propriétés qu'elles mesurent. 
Bien sûr, il est largement possible dans le cadre de l'étude des processus conscients que les différentes mesures reflètent toutes différentes facettes d'un même mécanisme neuronal cohérent \citep{king2014characterizing}. 

Les résultats que nous avons trouvé permettent de soutenir l'idée que différents types de mesures peuvent légitimement être étudiées dans le cadre de de la résolution de problématiques associées à des situations opérationnelles. 
Selon toute vraisemblance, ce n'est qu'en étudiant précisément ces différentes mesures dans le cadre de protocoles expérimentaux, que nous pourrons déterminer lesquelles sont les plus appropriées à répondre à des problématiques spécifiques. 
Dans notre cas, nous avons pu travailler, non pas sur la résolution directe d'une problématique opérationnelle, mais sur le substrat neuronal associé à cette problématique, lequel permet en retour une considération plus appronfondie des futures implémentations technologiques conduisant à une solution directe. 
Ainsi, le point fort de ce travail est qu'il a essayé de mettre en lien une situation opérationnelle pratique ayant soulevée une problématique dont le substrat neuronal peut être étudié de manière analytique (\textit{i.e.}, la perception auditive consciente) avec la mise en place d'une approche méthodologique d'analyse du substrat au moyen d'outils théoriques de mesures des propriétés dynamiques et informationnelles du système.

L’ensemble de ces travaux apportent des arguments supplémentaires pour soutenir que la perception auditive consciente est associée à une hausse des traitements informationnels cérébraux. 
La dynamique de ces traitements d'information représente un indice fort pour la caractérisation de la conscience perceptive et de la construction de l'accès conscient d'un stimulus. 
Ces indicateurs seront à tester de manière écologique et directe pour l'implémentation de technologies de monitoring neuro-adaptatives permettant la présentation de neurofeedbacks \citep{nuno2021real}. 
L'utilisation de mesures de complexité comme par exemple l'exposant de Hurst dans le cadre de stratégies en boucle fermée, visant à indiquer par un neurofeedback que le pilote, n'a pas détecté l'alarme auditive, est une solution envisageable aux problématiques avancées. 
De la même manière, un enregistrement en temps réél de l'activité cérébrale au niveau des zones temporales à partir d'un appareillage technologique adéquat, et le calcul simultané de l'information intégrée serait un outil précieux, étendant ses perspectives d'utilisation au-delà du seul champ de l'aéronautique. 
L'utilisation d'une mesure d'information intégrée telle que $\Phi^{MI}$ dans un neurocockpit s'avérerait utile dans la mesure où l'enregistrement des données cérébrales issues du cortex temporal serait suivi directement par un prétraitement des données pour enfin aboutir à son estimation en «temps-réel». 

À partir de nos travaux, nous ne pouvons que soutenir les recommandations de \cite{verdiere2019approche} consistant à considérer la simplicité des métriques comme un précieux avantage dans la conception des systèmes neuro-adaptatifs. 
Nous en avons une preuve évidente avec les mesures théoriques de l'information intégrée, prenant un temps considérable pour leur estimation. 
Au contraire, des mesures simples basées sur les caractéristiques du signal telles que l'exposant de Hurst ou les paramètres de Hjorth sont rapides à estimer. 
En ce sens, il est préconisable d'éviter autant que possible la complexité, tant pour des raisons conceptuelles que pratiques. 
Néanmoins, nous avons vu que la capacité des métriques ne se valent pas toutes pour décrire les propriétés du système étudié. 

Finalement, la perspective la plus importante de ce travail de thèse et, dans la continuité des aspects pratiques qui en découlent, serait d'étudier la discrimination des états perceptifs associés à la perception auditive consciente au sein du MI sur la base de l'apprentissage statistique. 
Tout marqueur physiologique, ou plus généralement un vecteur de marqueurs, est fiable pour un état (ou contenu de conscience) donné si les valeurs de ce marqueur caractérisent bien l'état (ou le contenu) d'intérêt. 
En d'autres termes, la distribution probabiliste sous-jacente d'un tel vecteur est connue (ou supposée) pour être significativement différente selon l'état ou le contenu de conscience défini du sujet. 
À partir de méthodes de classification statistique, il est possible de calculer une fonction de prédiction basée sur un ensemble de données qui contient des vecteurs de valeurs de caractéristiques.
Cette fonction permet d'associer l'état ou le contenu de conscience le plus plausible à tout nouveau vecteur, et pas seulement à ceux présents dans l'ensemble de données.
Dans le cas de la classification de données physiologiques, les caractéristiques sont les mesures physiologiques obtenues au préalable et la sortie souhaitée une condition portant sur un état ou un contenu de conscience. 
De futures analyses pourront appliquer ces techniques de classification afin de permettre la discrimination statistique des états perceptifs et ainsi de renforcer les résultats issus des analyses sur les différentes mesures recueillies, pour caractériser et discriminer la perception auditive consciente. 

%%%%%%%%%%%%%%%%%%%%%%%%%%%%%%%%%%%%%%%%%%%%%%%%%%%%%%%%%%%%%%%%%%%%%%%%%%%%%%%
\clearpage\null\newpage
%%%%%%%%%%%%%%%%%%%%%%%%%%%%%%%%%%%%%%%%%%%%%%%%%%%%%%%%%%%%%%%%%%%%%%%%%%%%%%%
