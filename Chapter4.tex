%%%%%%%%%%%%%%%%%%%%%%%%%%%%%%%%%%%%%%%%%%%%%%%%%%%%%%%%%%%%%%%%%%%%%%%%%%%%%%%
\chapter[Étude 1 : Influence des paramètres de la cible et du masqueur dans le MI]{Étude 1 : Influence des paramètres de la cible et du masqueur sur la dynamique de la perception auditive consciente sous masquage informationnel}
\label{chapitre4}
\noindent \hrulefill \\
%%%%%%%%%%%%%%%%%%%%%%%%%%%%%%%%%%%%%%%%%%%%%%%%%%%%%%%%%%%%%%%%%%%%%%%%%%%%%%%

Le masquage informationnel a été étudié en utilisant la détection d'une cible auditive intégrée dans un masqueur multi-tonalités aléatoire. 
La construction du percept auditif de la cible est influencée par les propriétés du masqueur et de la cible. 
La plupart des études traitant des performances de discrimination négligent la dynamique de la conscience perceptive. 
Cette première étude vise à examiner de manière spécifique la dynamique de la conscience perceptive en utilisant des modèles de survie dans un paradigme de masquage informationnel en manipulant l'incertitude du masqueur, la similarité cible-masqueur et le taux de répétition de la cible. 
En accord avec les études précédentes, elle montre qu'un taux élevé de répétition de la cible, une faible similarité entre le masqueur et la cible et une faible incertitude du masqueur facilitent la détection de la cible. 
Dans le contexte des modèles d'accumulation de preuves, ces résultats peuvent être interprétés par des changements dans les paramètres d'accumulation. 
La description probabiliste de la conscience perceptive fournit un point de référence pour le choix des paramètres de la cible et du masqueur afin d'examiner la dynamique cognitive et neuronale sous-jacente de la conscience perceptive. \\

L'objectif principal de cette étude est d'examiner comment l'incertitude du masqueur, la similarité temporelle cible-masqueur et le taux de répétition de la cible peuvent influencer la dynamique de la conscience perceptive d'une cible intégrée dans un masqueur multi-tonalités. 
Pour atteindre cet objectif, nous avons conçu trois expériences (simplement désignées I, II et III) qui échantillonnent différentes propriétés du masqueur et de la cible selon les valeurs utilisées dans des études antérieures \citep{akram2014investigating, elhilali2009interaction, elhilali2009temporal, giani2015detecting, gutschalk2008neural, wiegand2012correlates, wiegand2018cortical} afin de fournir un ensemble relativement complet de conditions expérimentales pertinentes. 
L'incertitude du masqueur a été manipulée en utilisant l'intervalle inter-tonalités moyen du masqueur et le nombre de fréquences par octave et caractérisée par l'entropie de la distribution des tonalités. 
La similarité entre la cible et le masqueur a été manipulée en utilisant les durées des tonalités du masqueur et de la cible et caractérisée par leur différence de durée. 
Enfin, le taux de répétition de la cible a également été manipulé en raison de son effet sur le flux auditif et de son effet supposé sur la dynamique de la conscience perceptive. 
Pour ces trois expériences, en plus de l'étude des performances de détection, nous avons utilisé l'analyse par modèle de survie pour étudier la dynamique de la conscience perceptive.

%%%%%%%%%%%%%%%%%%%%%%%%%%%%%%%%%%%%%%%%%%%%%%%%%%%%%%%%%%%%%%%%%%%%%%%%%%%%%%%
\section{Matériel et Méthodes}
\label{chapitre4materielmethodes}
%%%%%%%%%%%%%%%%%%%%%%%%%%%%%%%%%%%%%%%%%%%%%%%%%%%%%%%%%%%%%%%%%%%%%%%%%%%%%%%

Cette étude a été approuvée par le comité d'éthique sous la référence : IRB00011835-2020-06-09-253. 

%%%%%%%%%%%%%%%%%%%%%%%%%%%%%%%%%%%%%%%%%%%%%%%%%%%%%%%%%%%%%%%%%%%%%%%%%%%%%%%
\subsection{Participants}
\label{chapitre4particpants}
%%%%%%%%%%%%%%%%%%%%%%%%%%%%%%%%%%%%%%%%%%%%%%%%%%%%%%%%%%%%%%%%%%%%%%%%%%%%%%%

Tous les sujets ont été recrutés à l'Université d'Aix-Marseille sur le campus Saint-Charles. 
Un nouveau groupe de sujets a été recruté pour chaque expérience. 
Quatorze sujets ($7$ femmes) ont participé à l'expérience I. 
Trois sujets étaient gauchers (S1, S9 et S11). 
Les sujets étaient âgés de $18$ à $32$~ans, avec une moyenne de $24$~ans et un écart-type de $3$~ans. 
Quatorze sujets ($5$ femmes) ont participé à l'expérience II. 
Deux sujets étaient gauchers (S2 et S9). 
Leur âge variait de $18$ à $38$~ans, avec une moyenne de $24$~ans et un écart-type de $5$~ans. 
Enfin, treize sujets ($6$ femmes) ont participé à l'expérience III. 
Deux sujets étaient gauchers (S3 et S7). 
Les âges allaient de $18$ à $38$ ans avec une moyenne de $24$~ans et un écart-type de $5$~ans. 
Tous les sujets avaient une audition normale et n'avaient aucun antécédent de déficience auditive ou neurologique. 
La participation aux expériences n'a pas été rémunérée et s'est faite sur la base du volontariat, suite à un appel à participation. 
Les sujets ont été informés de l'objectif de l'étude et chaque volontaire a donné son consentement écrit pour participer.

%%%%%%%%%%%%%%%%%%%%%%%%%%%%%%%%%%%%%%%%%%%%%%%%%%%%%%%%%%%%%%%%%%%%%%%%%%%%%%%
\subsection{Stimuli}
\label{chapitre4stimuli}
%%%%%%%%%%%%%%%%%%%%%%%%%%%%%%%%%%%%%%%%%%%%%%%%%%%%%%%%%%%%%%%%%%%%%%%%%%%%%%%

Tous les stimuli auditifs (Fig.~\ref{fig:figure1}) étaient composés d'un masqueur multi-tonalités et, dans $67$\% des essais, d'une cible \citep{dykstra2016neural, gutschalk2008neural, konigs2012functional, neff1987masking}. 
Les stimuli auditifs ont tous été construits à l'aide d'une conception commune basée sur un ensemble de paramètres acoustiques présentés dans la Table~\ref{tab:expe_param} \citep{akram2014investigating, elhilali2009interaction, elhilali2009temporal, giani2015detecting, gutschalk2008neural, wiegand2012correlates, wiegand2018cortical}. 
Les tonalités de la cible ont été présentées au même niveau que les tonalités individuelles du masqueur : le rapport entre le niveau de la cible et celui du masqueur était de $0$~dB \citep{dykstra2016neural}. 
Les tonalités du masqueur et de la cible présentaient tous deux des rampes d'entrée et de sortie en forme de cosinus de $10$~ms. 
Lorsqu'il était présent, le son cible qui se répétait régulièrement commençait toujours $600$~ms après le début du masqueur.

\begin{figure}[!t]
\includegraphics[width=\textwidth]{/home/link/Documents/thèse_onera/articles_alex/article_1/PloSOne/PLoS-One-R1/manuscript/figures/Fig1.pdf}
\caption[Représentations graphiques d'échantillon de stimuli auditifs]{Représentations graphiques d'échantillon de stimuli auditifs utilisés dans les expériences. Un flux de tonalités cible (en rouge) est présenté dans la région protégée (en vert) au sein d'un masqueur multi-tonalités. Dans les exemples, les tonalités cibles ont une fréquence de $1$~kHz, une durée de $100$~ms et un taux de répétition de $1$~Hz. Les tonalités du masqueur (en noir) vont de $239$ à $5000$~Hz et ont une durée de $100$~ms. L'intervalle inter-tonalité moyen (miti) est indiqué sur les lignes et les fréquences par octave (fpo) sont indiquées sur les colonnes.} 
\label{fig:figure1} 
\end{figure}

\begin{table}[!t]
\caption[Table des paramètres expérimentaux pour les trois Expériences de l'Étude 1]{Table des paramètres expérimentaux pour les trois Expériences de l'Étude 1. Les fréquences du masqueur variaient de $239$ à $5000$~Hz et les fréquences des tonalités cibles étaient tirées d'un ensemble de six fréquences allant de $489$ jusqu'à $2924$~Hz.}
\label{tab:expe_param}
\footnotesize
\centering
% \begin{tabular}{lllll}
\begin{tabular}{|l||*{5}{c|}}
\hline
& & \textbf{Exp.~I} & \textbf{Exp.~II} & \textbf{Exp.~III} \\
\hline
                    & Durée des tonalités (ms) & 20, 60, 100 & 20, 60, 100 & 20  \\
\textbf{Masqueur}   & Fréquences par octave (fpo) & 4, 16, 64 & 4, 16, 64  & 16, 32, 64 \\
                    & Intervalle inter-tonalités moyen (ms) & 800 & 800 & 200, 600, 1200 \\
\hline
\textbf{Cible}      & Durée des tonalités (ms) & 20, 60, 100 & 20 & 60 \\
                    & Taux de répétition (Hz) & 1  & 5, 10, 20  & 1, 2, 5  \\
\hline
\end{tabular}
\end{table}

Les cibles étaient composées d'une série régulière de tonalités définies par la durée de la tonalité et le taux de répétition des tonalités (\textit{i.e.} nombre par seconde) en fonction du début de la tonalité. 
Pour empêcher les sujets de porter leur attention de manière sélective sur une gamme de fréquences spécifique, les fréquences des tonalités cibles changeaient de manière aléatoire d'un essai à l'autre. 
La fréquence de la tonalité cible a été choisie au hasard parmi un ensemble de six fréquences ($489$, $699$, $1000$, $1430$, $2045$ et $2924$~Hz) \citep{dykstra2016neural, gutschalk2008neural}.

Les masqueurs étaient caractérisés par leur durée, le nombre de fréquences par octave (fpo) et les intervalles inter-tonalités moyens (miti). 
Pour les trois expériences, les intervalles entre les tonalités du masqueur ont été tirés au hasard selon une distribution uniforme avec un minimum constant ($100$~ms) et un maximum différent conduisant à des paramètres d'échelle différents. 
Les fréquences des tonalités du masqueur étaient également espacées sur une échelle logarithmique entre $239$ et $5000$~Hz \citep{dykstra2016neural, gutschalk2008neural}.

Afin d'assurer un masquage énergétique minimal, une région protégée entourant la cible était maintenue sans tonalités dans le masqueur. 
Pour chaque cible, une largeur de bande rectangulaire équivalente (ERB) \citep{glasberg1990derivation, moore1995frequency} a été calculée en utilisant : $ERB = 24,7\, (4,37\,F_t + 1) $ avec $F_t$ la fréquence de la cible en kHz. 
La région protégée était centrée sur $F_t$ et avait une extension totale de deux ERB (\textit{i.e.} une de chaque côté de $F_t$, voir Fig.~\ref{fig:figure1}).

La similarité temporelle entre le masqueur et la cible a été définie comme la valeur de la différence entre les durées des tonalités du masqueur et de la cible. 
Les valeurs négatives indiquent une durée de tonalité cible plus longue que la durée de tonalité du masqueur et vice versa. 
L'incertitude du masqueur a été quantifiée en utilisant l'entropie de la distribution des tonalités. 
Elle caractérise l'incertitude temporelle et fréquentielle en fonction de l'intervalle inter-tonalités moyen et du nombre de fréquences de tonalité par octave respectivement. 
Étant donné que les intervalles inter-tonalités du masqueur ont été tirés d'une distribution uniforme avec un intervalle $\Delta$ (en ms) entre les intervalles inter-tonalités minimum et maximum, l'entropie $H$ de cette distribution uniforme est : $H(x) = \log(\Delta)$. 
En considérant chaque fréquence par octave comme un processus indépendant, l'entropie de l'ensemble du masqueur $M$ pour $n$ fréquences par octave est : $H(M) = n \log(\Delta)$. 
Les valeurs d'entropie pour toutes les combinaisons des paramètres du masqueur sont données dans la Table~\ref{tab:entropy}.

\begin{table}[!t]
\caption[Table des valeurs d'incertitude du masqueur spécifiques aux trois expériences de l'Étude 1]{Table des valeurs d'incertitude du masqueur quantifiée par l'entropie (en nats) de la distribution des tonalités. miti : intervalle inter-tonalités moyen, fpo : fréquences par octave, $\Delta$ : gamme des intervalles inter-tonalités (max. - min.).}
\label{tab:entropy}
\footnotesize
\centering
% \begin{tabular}{rrrrrr}
\begin{tabular}{|l||*{6}{c|}}
\hline
 & & \multicolumn{1}{c|}{\textbf{Exp.~I - II}} & \multicolumn{3}{c|}{\textbf{Exp.~III}}\\
\hline
\textbf{miti (ms)}    & & 800  & 200 & 600  & 1200 \\ 
\textbf{$\Delta$ (ms)} & & 1400 & 200 & 1000 & 2200\\
\hline
                & 4  & 28.97 & ---  & ---  & ---\\
                & 16 & --- & 84.77 & 110.52 & 123.13\\
\textbf{fpo}    & 32 & 231.82& 169.55 & 221.05 & 246.28\\
                & 64 & 463.63& 339.09 & 442.09 & 492.56\\
\hline
\end{tabular}
\end{table}

Les stimuli auditifs ont été générés à l'aide du langage de programmation Python (version $3.5.2$) \citep{van2007python}.  
Ils ont été numérisés avec un taux d'échantillonnage de $44100$~Hz et une profondeur de $16$~bits audio. 
Tous les stimuli ont été délivrés avec un niveau d'intensité acoustique calibré à $70$~dB\,SPL.

%%%%%%%%%%%%%%%%%%%%%%%%%%%%%%%%%%%%%%%%%%%%%%%%%%%%%%%%%%%%%%%%%%%%%%%%%%%%%%%
\subsection{Procédure}
\label{chapitre4procedure}
%%%%%%%%%%%%%%%%%%%%%%%%%%%%%%%%%%%%%%%%%%%%%%%%%%%%%%%%%%%%%%%%%%%%%%%%%%%%%%%

Chaque expérience était composée de $243$~essais, répartis aléatoirement en $6$~blocs de $41$ ou $40$~essais. 
Chaque essai durait $12$~secondes et les essais étaient séparés par $4$~secondes de silence. 
La durée totale de l'expérience était de $65$~minutes. 
La tâche des participants consistait à appuyer sur la barre d'espace d'un clavier d'ordinateur dès qu'ils détectaient la cible. 
Les sujets avaient pour instruction de répondre le plus précisément et le plus rapidement possible dès qu'ils étaient certains de la présence de la cible. 
Les sujets n'avaient pas la possibilité de modifier leur réponse dans le cas où ils se rendaient compte qu'ils avaient fait une erreur (fausse alarme). 
Il n'y a pas eu d'enregistrement systématique d'une telle situation et il n'est donc pas possible d'en mesurer l'ampleur. 

Les sujets ont été informés que la cible ne serait pas nécessairement présente à chaque essai. 
Aucune information concernant la probabilité de la cible n'a été donnée. 
Les participants ont suivi une session d'entraînement, composée d'essais avec et sans cible, qui s'est poursuivie jusqu'à ce que le sujet détecte correctement $10$~essais avec cible. 
Les essais utilisés dans le bloc d'entraînement étaient composés de masqueurs avec une durée de tonalité de $20$ ou $60$~ms et un intervalle inter-tonalités moyen de $600$ ou $800$~ms. 
Les cibles étaient composées de tonalités d'une fréquence de $1$~kHz et d'une durée de $100$~ms. 
Le taux de répétition des cibles variait de façon aléatoire entre les essais et était de $1$ ou $2$~Hz. 

Les stimuli auditifs ont été produits par un ordinateur DELL PRECISION M4800 (processeur i7~$4900$~MQ, $16$~GB DDR3 RAM, NVidia Quadro K2100M sous Windows $7$ avec une carte son Intel Lynx Point PCH) et présentés de manière diotique aux participants avec un casque circumaural calibré (Sennheiser HDA $600$) dans une pièce insonorisée. 
Les stimuli ont été présentés aux participants à l'aide du logiciel Eprime (version $2.0$, Psychology Software Tools). 

%%%%%%%%%%%%%%%%%%%%%%%%%%%%%%%%%%%%%%%%%%%%%%%%%%%%%%%%%%%%%%%%%%%%%%%%%%%%%%%
\subsection{Expériences}
\label{chapitre4experiences}
%%%%%%%%%%%%%%%%%%%%%%%%%%%%%%%%%%%%%%%%%%%%%%%%%%%%%%%%%%%%%%%%%%%%%%%%%%%%%%%

Dans chaque expérience, la probabilité d'essais sans cible était de $33$~\%, ce qui a donné lieu à $81$~essais sans cible et $162$~essais avec cible, soit un total de $243$~essais par expérience. 
Chaque combinaison expérimentale de propriétés du masqueur et de la cible a été présentée une fois par fréquence de tonalité de la cible ($6$) à chaque sujet. 
Dans les trois expériences, les fréquences de tonalité du masqueur par octave variaient selon trois niveaux : $4$, $16$ et $64$~fréquences par octave pour les expériences~I et II et $16$, $32$, $64$~fréquences par octave pour l'expérience~III (voir Table~\ref{tab:expe_param}). 

Dans l'Expérience~I, les durées des tonalités du masqueur et de la cible ont été manipulées selon trois niveaux : $20$, $60$ et $100$~ms. 
Le taux de répétition de la cible était égal à $1$~Hz. 
Les intervalles inter-tonalités du masqueur ont été tirés d'une distribution uniforme (min. : $100$~ms, max. : $1500$~ms) conduisant à un intervalle inter-tonalités moyen de $800$~ms. 
La similarité cible-masqueur variait donc de $-80$ à $80$~ms et les valeurs de l'incertitude du masqueur sont données dans la Table~\ref{tab:entropy}.

Dans l'Expérience~II, le taux de répétition de la cible a varié selon trois niveaux : $5$, $10$ et $20$~Hz. 
La durée des tonalités cibles a été fixée à $20$~ms afin de s'assurer que les tonalités successives ne se chevauchent pas dans la condition de taux de répétition élevé. 
Les durées des tonalités du masqueur ont été ajustées aux mêmes valeurs que dans l'Expérience~I ($20$, $60$ ou $100$~ms). 
Les intervalles inter-tonalités du masqueur ont été tirés de la même distribution uniforme que dans l'Expérience~I, conduisant au même intervalle inter-tonalités moyen de $800$~ms et à la même incertitude du masqueur (Table~\ref{tab:entropy}). 
La similarité cible-masqueur variait donc de $0$ à $80$~ms.

Dans l'Expérience~III, la durée de la tonalité cible était fixée à $60$~ms et la durée de la tonalité du masqueur était fixée à $20$~ms, conduisant à une seule valeur de similarité cible-masqueur ($40$~ms). 
Le taux de répétition de la cible variait selon trois niveaux : $1$, $2$ et $5$~Hz. 
Les intervalles inter-tonalités du masqueur ont été tirés de distributions uniformes avec un minimum constant ($100$~ms) et des maxima variables ($300$, $1100$ ou $2300$~ms), donnant trois valeurs d'intervalle inter-tonalité moyen : $200$, $600$ et $1200$~ms conduisant à une augmentation de l'incertitude du masqueur avec la durée de l'intervalle inter-tonalités moyen. 
Les valeurs de l'incertitude du masqueur sont données dans la Table~\ref{tab:entropy}.  

%%%%%%%%%%%%%%%%%%%%%%%%%%%%%%%%%%%%%%%%%%%%%%%%%%%%%%%%%%%%%%%%%%%%%%%%%%%%%%%
\subsection{Analyses}
\label{chapitre4analyses}
%%%%%%%%%%%%%%%%%%%%%%%%%%%%%%%%%%%%%%%%%%%%%%%%%%%%%%%%%%%%%%%%%%%%%%%%%%%%%%%

Les temps de détection des essais étaient enregistrés chaque fois que le participant appuyait pour la première fois sur la barre d'espace. 
Pour chaque expérience, les données ont été lues à partir du fichier brut EPrime à l'aide de scripts Python \citep{van2007python} et les statistiques ont été réalisées avec le logiciel \textit{R} \citep{Rlanguage2017}. 

Chaque essai était catégorisé comme un succès, une omission, une fausse alarme ou un rejet correct en fonction de la présence de la cible et de la réponse du participant. 
Comme la détection d'une régularité nécessite d'entendre au moins deux répétitions de la tonalité cible, toute détection survenant plus rapidement était considérée comme une supposition et écartée des réponses valides. 
Pour chaque expérience, la coupure temporelle a été adaptée au taux de répétition le plus rapide des tonalités cible, ce qui a conduit à une coupure de $1600$~ms pour l'expérience~I, $700$~ms pour l'expérience~II et $1100$~ms pour l'expérience~III.

L'indice de performance de détection ($d^\prime$) a été calculé à partir du taux de réussite (HR) et du taux de fausses alarmes (FAR) après une transformation en z-score \citep{green1966signal, macmillan2004detection} : $d^\prime = z(\text{HR})-z(\text{FAR})$. 
Cet indice a été calculé pour chaque sujet et dans chaque condition expérimentale où le FAR peut être défini. 
Puisqu'une fausse alarme correspond à un essai où la cible est absente, FAR ne peut pas être défini pour les conditions expérimentales caractérisées par les propriétés de la cible et donc $d^\prime$ ne peut être calculé que pour les conditions expérimentales caractérisées par les propriétés du masqueur (\textit{i.e.} incertitude).

Les données de performance de détection ont été analysées à l'aide de modèles à effets mixtes avec la bibliothèque \texttt{nlme} \textit{R} \citep{pinheiro2012nlme}. 
Les modèles à effets mixtes ont été estimés en utilisant l'indice de performance ($d^\prime$) comme variable de réponse, l'incertitude du masqueur comme effet fixe et l'id. du sujet comme effet aléatoire pour le paramètre d'intercept.
Les essais avec une cible sont caractérisés par leur temps de détection qui peut être censuré (à droite) dans le cas d'une cible manquée (omission). 
Les temps de détection obtenus dans les trois expériences ont été analysés à l'aide de modèles de fragilité mis en œuvre dans la bibliothèque \texttt{survival} \textit{R} \citep{therneau2013r}. 
L'incertitude du masqueur, la similarité temporelle cible-masqueur et le taux de répétition de la cible représentent les effets fixes, tandis que les termes de fragilité avec l'id. du sujet correspondent aux effets aléatoires des modèles ajustés. 

Les résidus de Cox-Snell \citep{cox1968general, andersen1993statistical, letue2018statistical} sont utilisés dans le cas des modèles de régression à risques proportionnels de Cox. 
Pour chaque modèle ajusté, après avoir supprimé toutes les valeurs influentes présentant un comportement aberrant, par exemple un FAR élevé, des analyses de variance ont été réalisées afin d'évaluer la signification statistique globale des effets des paramètres expérimentaux et de leurs interactions.

Dans le cas d'un effet statistique global, les interactions complètes entre les paramètres expérimentaux ont été étudiées sur la base de toutes les comparaisons par paires en utilisant les moyennes marginales estimées mises en œuvre dans la bibliothèque \texttt{emmeans} \textit{R}. 
Les moyennes marginales estimées correspondent aux valeurs des paramètres du modèle dont la moyenne a été calculée pour la combinaison adéquate des modalités du facteur. 
Les résultats sont résumés dans des Tables qui regroupent les conditions expérimentales en utilisant le CLD (compact letter display) \citep{piepho2004an}. 
Les résultats sont aussi illustrés par des courbes de taux de risque estimées en fonction des conditions groupées données dans les Tables.

%%%%%%%%%%%%%%%%%%%%%%%%%%%%%%%%%%%%%%%%%%%%%%%%%%%%%%%%%%%%%%%%%%%%%%%%%%%%%%%
\section{Résultats expérimentaux}
\label{chapitre4resultats}
%%%%%%%%%%%%%%%%%%%%%%%%%%%%%%%%%%%%%%%%%%%%%%%%%%%%%%%%%%%%%%%%%%%%%%%%%%%%%%%

L'observation des performances de détection pour chaque bloc expérimental a montré que l'indice de performance du premier bloc était inférieur à celui des autres blocs (voir Fig.~\ref{fig:dprime_bloc}). 
Les données du premier bloc ont donc été écartées pour chaque expérience afin de prévenir la variabilité de l'apprentissage. 
Les résultats comportementaux des performances et des temps de détection sont reportés dans la Table~\ref{tab:detection_time}. 

\begin{figure}[!t]
\centering
\includegraphics[width=0.7\textwidth]{/home/link/Documents/thèse_onera/articles_alex/article_1/PloSOne/PLoS-One-R1/manuscript/figures/S1_Fig.pdf}
\caption[Performances de détection $d^\prime$ pour chaque bloc expérimental des trois expériences]{Performances de détection $d^\prime$ pour chaque bloc expérimental dans les Exp.~I, II et~III.} 
\label{fig:dprime_bloc} 
\end{figure}

\begin{table}[!t]
\caption[Table des résultats comportementaux pour les temps et les performances de la détection de l'Étude 1]{Table des résultats comportementaux pour les temps et les performances de la détection dans les trois expériences. Les résultats sont donnés comme moyenne $\pm$ l'écart-type.}
\label{tab:detection_time}
\footnotesize
\centering
% \begin{tabular}{lllll}
\begin{tabular}{|l||*{4}{c|}}
\hline
& \textbf{Exp.~I} & \textbf{Exp.~II} & \textbf{Exp.~III} \\
\hline
\textbf{Temps de détection moyen (ms)}  & $5430 \pm 2804$  & $3019 \pm 2051$ & $3104 \pm 2070$  \\
\textbf{Taux de Hit (\%)} & $0.46 \pm 0.19$ & $0.78 \pm 0.09 $ & $0.88 \pm 0.08$   \\
\textbf{Taux de Fausse Alarme (\%)} & $0.25 \pm 0.19 $ & $0.13 \pm 0.11$ & $0.08 \pm 0.05$  \\
\textbf{$d^\prime$} & $0.71 \pm 0.32 $ & $2.11 \pm 0.27$ & $2.73 \pm 0.55$  \\
\hline
\end{tabular} 
\end{table}

L'indice de performance de détection ($d^\prime$) est représenté en fonction de l'incertitude du masqueur pour les expériences~I, II et~III sur la Fig.~\ref{fig:dprime_mu}. 
Le taux instantané de détection de la cible est représenté par les fonctions de taux de risque estimées, données pour chaque groupe de conditions expérimentales pour l'expérience~I, II et III sur les Fig.~\ref{fig:hr_Exp-I}, \ref{fig:hr_Exp-II} et \ref{fig:hr_Exp-III} respectivement. 

\begin{figure}[!t]
\centering
\includegraphics[width=0.8\textwidth]{/home/link/Documents/thèse_onera/articles_alex/article_1/PloSOne/PLoS-One-R1/manuscript/figures/Fig2.pdf}
\caption[Indice de performance de détection $d^\prime$ en fonction de l'incertitude du masqueur]{Indice de performance de détection $d^\prime$ en fonction de l'incertitude du masqueur dans les expériences~I (a), II (b) et~III (c). L'incertitude du masqueur est mesurée par l'entropie en nats.} \label{fig:dprime_mu} 
\end{figure}

%%%%%%%%%%%%%%%%%%%%%%%%%%%%%%%%%%%%%%%%%%%%%%%%%%%%%%%%%%%%%%%%%%%%%%%%%%%%%%%
\subsection{Expérience~I}
\label{chapitre4resultatsexpI}
%%%%%%%%%%%%%%%%%%%%%%%%%%%%%%%%%%%%%%%%%%%%%%%%%%%%%%%%%%%%%%%%%%%%%%%%%%%%%%%

Aucun sujet n'a été écarté sur la base de l'inspection qualitative des distributions des indices de performance et des temps de réaction ($n=14$).

Aucun effet significatif de l'incertitude du masqueur sur l'indice de performance de détection ($d^\prime$) n'a été observé (${F(2,26)=1.16}, {p=0.33}$, voir Fig.~\ref{fig:dprime_mu}).

L'analyse du modèle de Cox avec terme de fragilité n'a montré aucun effet significatif du terme de fragilité (${\chi^2=0.27}$, ${df=1}$, ${p=0.61}$) alors que la similarité temporelle cible-masqueur (${\chi^2=286.461}$, ${df=4}$, ${p<0.001}$), l'incertitude du masqueur (${\chi^2=167.07}$, ${df=2}$, ${p<0.001}$) et leur interaction (${\chi^2=430.98}$, ${df=20.5}$, ${p<0.001}$) ont montré un effet significatif sur le taux de risque de la conscience perceptive de la cible.

Les résultats des comparaisons par paires pour l'interaction entre l'incertitude du masqueur et la similarité temporelle cible-masqueur sont donnés dans la Table~\ref{tab:cld_Exp-I}. 
Une illustration des courbes de taux de risque correspondantes pour chaque niveau de similarité cible-masqueur dans chaque cas d'incertitude du masqueur est donnée sur la figure~\ref{fig:hr_Exp-I}.
% (et l'illustration inverse est donnée sur~\nameref{sup:fig_hr_Exp-I}). 
Le taux de risque le plus élevé de conscience perceptive de la cible est observé lorsque la durée de la tonalité de la cible est supérieure de $80$~ms à celle de la tonalité du masqueur (similarité cible-masqueur : $S=-80$~ms) pour les deux plus faibles incertitudes du masqueur ($H=29$~nats et $H=115$~nats). 
Ce taux de risque diminue pour la plus grande incertitude du masqueur ($H=463$~nats). 
Les courbes de taux de risque observées lorsque la durée de la tonalité cible est égale ou inférieure à la durée de la tonalité du masqueur ($S=0$~ms, $S=40$~ms et $S=80$~ms) peuvent difficilement être différenciées pour chaque niveau d'incertitude du masqueur. 
Les courbes de taux de risque observées dans le cas où la durée de la tonalité cible est supérieure de $40$~ms à celle de la tonalité du masqueur ($S=-40$~ms) ont un comportement intermédiaire qui peut être différencié des conditions où la durée de la tonalité cible est égale ou inférieure à celle de la tonalité du masqueur ($S=0$~ms, $S=40$~ms et $S=80$~ms) pour l'incertitude la plus faible ($H=29$~nats) et tend vers les courbes observées dans ces conditions lorsque l'incertitude augmente.  

\begin{figure}[!t]
\includegraphics[width=\textwidth]{/home/link/Documents/thèse_onera/articles_alex/article_1/PloSOne/PLoS-One-R1/manuscript/figures/Fig3.pdf}
\caption[Fonctions de taux de risque $h(t)$ pour l'Expérience~I]{Fonctions de taux de risque $h(t)$ pour l'incertitude du masqueur et la similarité cible-masqueur dans l'Expérience~I. 
Chaque figure représente les estimations des fonctions de taux de risque pour chaque combinaison de modalités des paramètres expérimentaux.} 
\label{fig:hr_Exp-I} 
\end{figure}

\begin{table}[!t]
\caption[Table des moyennes marginales estimées pour les fonctions de taux de risque pour l'Expérience~I]{Table des moyennes marginales estimées pour les fonctions de taux de risque et affichage en lettres compactes pour les comparaisons par paires de l'interaction entre la similarité cible-masqueur et l'incertitude du masqueur dans l'Exp.~I.} 
\label{tab:cld_Exp-I}
\footnotesize
\centering
% \begin{tabular}{llrrrrrl}
\begin{tabular}{|l|*{8}{c|}}
\hline
\textbf{Uncertainty} & \textbf{Similarity} & \textbf{emmean} & \textbf{SE} & \textbf{df} & \textbf{asymp.LCL} & \textbf{asymp.UCL} & \textbf{.group} \\ 
\hline
463 & 80 & -3.5107 & 0.3383 & Inf & -4.1738 & -2.8476 & 12 \\ 
463 & 0 & -3.2871 & 0.2497 & Inf & -3.7764 & -2.7978 & 1 \\ 
463 & 40 & -3.2672 & 0.2709 & Inf & -3.7981 & -2.7363 & 12 \\ 
115 & 0 & -2.6058 & 0.2337 & Inf & -3.0637 & -2.1478 & 23 \\ 
463 & -40 & -2.5652 & 0.2382 & Inf & -3.0320 & -2.0984 & 23 \\ 
115 & 40 & -2.3467 & 0.2410 & Inf & -2.8191 & -1.8743 & 34 \\ 
115 & 80 & -2.0614 & 0.2578 & Inf & -2.5666 & -1.5562 & 345 \\ 
29 & 0 & -2.0002 & 0.2250 & Inf & -2.4412 & -1.5591 & 45 \\ 
115 & -40 & -1.8532 & 0.2310 & Inf & -2.3061 & -1.4004 & 456 \\ 
29 & 40 & -1.7631 & 0.2284 & Inf & -2.2108 & -1.3153 & 56 \\ 
29 & 80 & -1.3134 & 0.2455 & Inf & -1.7946 & -0.8322 & 67 \\ 
29 & -40 & -1.0384 & 0.2248 & Inf & -1.4790 & -0.5978 & 7 \\ 
463 & -80 & -0.8056 & 0.2628 & Inf & -1.3207 & -0.2906 & 78 \\ 
115 & -80 & -0.2971 & 0.2686 & Inf & -0.8236 & 0.2294 & 8 \\ 
29 & -80 & 0.0000 & 0.0000 & Inf & 0.0000 & 0.0000 & 8 \\ 
\hline
\end{tabular}
\smallskip
\begin{flushleft}
Les résultats sont donnés sur l'échelle log (et non de la réponse). 
Niveau de confiance utilisé : $0.95$. 
Ajustement de la valeur p : Méthode de Tukey pour la comparaison d'une famille de $15$ estimations. 
Niveau de significativité utilisé : $\alpha= 0.05$. 
Incertitude : incertitude du masqueur, Similarité : similarité cible-masqueur, emmean : moyenne marginale estimée, SE : erreur standard, df : degrés de liberté, asymp.LCL : limite inférieure asymptotique de contraste, asymp.UCL : limite supérieure asymptotique de contraste, .group : groupe de lettres compactes.
\end{flushleft}
\end{table}

%%%%%%%%%%%%%%%%%%%%%%%%%%%%%%%%%%%%%%%%%%%%%%%%%%%%%%%%%%%%%%%%%%%%%%%%%%%%%%%
\subsection{Expérience~II}
\label{chapitre4resultatsexpII}
%%%%%%%%%%%%%%%%%%%%%%%%%%%%%%%%%%%%%%%%%%%%%%%%%%%%%%%%%%%%%%%%%%%%%%%%%%%%%%%

Un sujet (S6) a été écarté des analyses en raison de la distribution atypique de ses temps de réaction. 

L'analyse ultérieure du modèle linéaire à effets mixtes pour l'indice de performance ($d^\prime$) et du modèle de Cox pour les temps de détection a conduit à la suppression de trois autres sujets (S1, S2, S8) en tant que valeurs aberrantes influentes avec un FAR nettement supérieur à celui des autres sujets. 
Les analyses de l'Expérience~II ont donc été limitées à $n=10$ sujets.

L'analyse de variance du modèle linéaire à effet mixte a montré un effet significatif de l'incertitude du masqueur sur l'indice de performance $d^\prime$ (${F(2,27)=74.2}, {p<.001}$). 
L'analyse des comparaisons par paires a montré une différence significative entre toutes les paires de conditions qui reflète une diminution significative de la performance de détection avec l'augmentation de l'incertitude du masqueur (voir Figure~\ref{fig:dprime_mu} B).

L'analyse du modèle de Cox avec terme de fragilité a montré des effets significatifs pour le terme de fragilité (${\chi^2=13.5}$, ${df=1}$, ${p<0.001}$), l'incertitude du masqueur (${\chi^2=428.5}$, ${df=2}$, ${p<0.001}$), la similarité cible-masqueur (${\chi^2=170.2}$, ${df=2}$, ${p<0.001}$) et le taux de répétition de la cible (${\chi^2=249.1}$, ${df=2}$, ${p<0.001}$). 
Toutes les interactions ont également eu un effet significatif sur le taux de risque de détection de la cible (incertitude du masqueur $\times$ similarité cible-masqueur : ${\chi^2=13}$, ${df=4}$, ${p=0.011}$ ; incertitude du masqueur $\times$ taux de répétition de la cible : ${\chi^2=12.1}$, ${df=4}$, ${p=0.017}$ ; taux de répétition de la cible $\times$ similarité cible-masqueur : ${\chi^2=19.8}$, ${df=4}$, ${p<0.001}$ ; et interaction triple : ${\chi^2=178.7}$, $df=16.4$, ${p<0.001}$). 

Les résultats des comparaisons par paires sont donnés dans la Table~\ref{tab:cld_Exp-II}. 
Une illustration des courbes de taux de risque correspondantes pour chaque niveau de taux de répétition de la cible dans chaque condition définie par l'interaction entre la similarité temporelle cible-masqueur et l'incertitude du masqueur est donnée sur la Figure~\ref{fig:hr_Exp-II}. 
% (et les deux autres représentations possibles sont données sur~\nameref{sup:fig_hr_Exp-IIS3} et \nameref{sup:fig_hr_Exp-IIS4}). 
On peut observer une diminution du taux de risque de la conscience perceptive de la cible avec l'augmentation de l'incertitude du masqueur et avec la diminution du taux de répétition de la cible, sauf dans le cas de la plus faible incertitude du masqueur ($H=29$~nats) et de la plus grande différence entre les durées de tonalité de la cible et du masqueur ($S=80$~ms). 
Dans ce cas, le taux de risque de la conscience perceptive de la cible est plus élevé pour un taux de répétition de la cible de $5$~Hz que pour un taux de répétition de la cible de $10$~Hz. 
En règle générale, le taux de risque le plus faible de conscience perceptive de la cible est observé lorsque la durée des tonalités cibles est égale à celle des tonalités du masqueur ($S=0$~ms), mais l'effet de la similarité temporelle cible-masqueur dépend fortement de l'interaction des deux autres paramètres. 
% (voir~\nameref{sup:fig_hr_Exp-IIS4}).

\begin{figure}[!t]
\includegraphics[width=\textwidth]{/home/link/Documents/thèse_onera/articles_alex/article_1/PloSOne/PLoS-One-R1/manuscript/figures/Fig4.pdf}
\caption[Fonctions de taux de risque $h(t)$ pour l'Expérience~II]{Fonctions de taux de risque $h(t)$ pour l'incertitude du masqueur, la similarité cible-masqueur et le taux de répétition de la cible dans l'Expérience~II. 
Chaque figure représente les estimations des fonctions de taux de risque pour chaque combinaison de modalités des paramètres expérimentaux.} 
\label{fig:hr_Exp-II}
\end{figure}

\begin{table}[!t]
\caption[Table des moyennes marginales estimées pour les fonctions de taux de risque pour l'Expérience~II]{Table des moyennes marginales estimées pour les fonctions de taux de risque et affichage en lettres compactes pour les comparaisons par paires de l'interaction entre la similarité cible-masqueur et l'incertitude du masqueur dans l'Exp II.} 
\label{tab:cld_Exp-II}
\footnotesize
% \begin{adjustwidth}{-.75in}{0in} % Comment out/remove adjustwidth environment if table fits in text column.
\centering
% \begin{tabular}{lllrrrrrl}
\begin{tabular}{|l|*{9}{c|}}
\hline
\textbf{Similarity} & \textbf{T.Rate} & \textbf{Uncertainty} & \textbf{emmean} & \textbf{SE} & \textbf{df} & \textbf{asymp.LCL} & \textbf{asymp.UCL} & \textbf{.group} \\ 
\hline
0 & 5 & 463 & -1.3599 & 0.3220 & Inf & -1.9910 & -0.7289 & 1 \\ 
0 & 10 & 463 & -1.2571 & 0.2986 & Inf & -1.8425 & -0.6718 & 1 \\ 
0 & 5 & 115 & -1.1938 & 0.2925 & Inf & -1.7670 & -0.6206 & 1 \\ 
80 & 5 & 463 & -1.0947 & 0.2818 & Inf & -1.6471 & -0.5424 & 1 \\ 
40 & 5 & 463 & -1.0778 & 0.2770 & Inf & -1.6208 & -0.5348 & 1 \\ 
0 & 20 & 463 & -0.6114 & 0.2551 & Inf & -1.1114 & -0.1114 & 12 \\ 
40 & 10 & 463 & -0.3128 & 0.2425 & Inf & -0.7881 & 0.1625 & 123 \\ 
80 & 10 & 463 & -0.2938 & 0.2403 & Inf & -0.7648 & 0.1772 & 123 \\ 
0 & 5 & 29 & 0.0000 & 0.0000 & Inf & 0.0000 & 0.0000 & 234 \\ 
0 & 10 & 115 & 0.0116 & 0.2313 & Inf & -0.4417 & 0.4649 & 2345 \\ 
40 & 5 & 115 & 0.3892 & 0.2253 & Inf & -0.0524 & 0.8309 & 3456 \\ 
40 & 20 & 463 & 0.4186 & 0.2255 & Inf & -0.0233 & 0.8605 & 34567 \\ 
80 & 20 & 463 & 0.6976 & 0.2236 & Inf & 0.2594 & 1.1357 & 45678 \\ 
80 & 5 & 115 & 0.8100 & 0.2164 & Inf & 0.3859 & 1.2341 & 5678 \\ 
40 & 10 & 115 & 1.0322 & 0.2200 & Inf & 0.6009 & 1.4635 & 6789 \\ 
0 & 20 & 115 & 1.1872 & 0.2205 & Inf & 0.7551 & 1.6193 & 7890 \\ 
40 & 5 & 29 & 1.2853 & 0.2185 & Inf & 0.8570 & 1.7135 & 890A \\ 
0 & 10 & 29 & 1.5819 & 0.2193 & Inf & 1.1520 & 2.0118 & 90AB \\ 
80 & 10 & 115 & 1.7138 & 0.2206 & Inf & 1.2814 & 2.1462 & 90AB \\ 
80 & 5 & 29 & 1.8252 & 0.2248 & Inf & 1.3847 & 2.2657 & 0AB \\ 
0 & 20 & 29 & 1.8364 & 0.2202 & Inf & 1.4048 & 2.2680 & 0AB \\ 
40 & 20 & 115 & 1.9370 & 0.2181 & Inf & 1.5095 & 2.3645 & 0AB \\ 
80 & 10 & 29 & 1.9539 & 0.2202 & Inf & 1.5224 & 2.3855 & AB \\ 
80 & 20 & 115 & 1.9883 & 0.2196 & Inf & 1.5579 & 2.4186 & AB \\ 
40 & 10 & 29 & 1.9949 & 0.2204 & Inf & 1.5629 & 2.4269 & AB \\ 
40 & 20 & 29 & 2.2040 & 0.2205 & Inf & 1.7718 & 2.6361 & B \\ 
80 & 20 & 29 & 2.2100 & 0.2203 & Inf & 1.7783 & 2.6418 & B \\ 
\hline
\end{tabular}
\smallskip
\begin{flushleft}
Les résultats sont donnés sur l'échelle log (et non de la réponse). 
Niveau de confiance utilisé : $0.95$. 
Ajustement de la valeur p : Méthode de Tukey pour la comparaison d'une famille de $27$ estimations. 
Niveau de significativité utilisé : $\alpha = 0.05$. 
Incertitude : incertitude du masqueur, Similarité : similarité cible-masqueur, T.Rate : taux de répétition de la cible. emmean : moyenne marginale estimée, SE : erreur standard, df : degrés de liberté, asymp.LCL : limite inférieure asymptotique du contraste, asymp.UCL : limite supérieure asymptotique du contraste, .group : groupe de lettres compactes.
\end{flushleft}
% \end{adjustwidth}
\end{table}

%%%%%%%%%%%%%%%%%%%%%%%%%%%%%%%%%%%%%%%%%%%%%%%%%%%%%%%%%%%%%%%%%%%%%%%%%%%%%%%
\subsection{Expérience~III}
\label{chapitre4resultatsexpIII}
%%%%%%%%%%%%%%%%%%%%%%%%%%%%%%%%%%%%%%%%%%%%%%%%%%%%%%%%%%%%%%%%%%%%%%%%%%%%%%%

\begin{figure}[!t]
\includegraphics[width=\textwidth]{/home/link/Documents/thèse_onera/articles_alex/article_1/PloSOne/PLoS-One-R1/manuscript/figures/Fig5.pdf}
\caption[Fonctions de taux de risque $h(t)$ pour l'Expérience~III]{Fonctions de taux de risque $h(t)$ pour l'incertitude du masqueur et le taux de répétition de la cible dans l'Expérience~III. 
Chaque figure représente les estimations des fonctions de taux de risque pour chaque combinaison de modalités des paramètres expérimentaux.} 
\label{fig:hr_Exp-III} 
\end{figure}

L'inspection qualitative des distributions de l'indice de performance de détection ($d^\prime$) a montré qu'un sujet (S8) a un FAR élevé. 
Ce sujet a donc été écarté de l'analyse ultérieure ($n=12$).

Un effet significatif a été trouvé pour l'incertitude du masqueur sur l'indice de performance de détection ($d^\prime$) (${F(8,88)=11.9}$, ${p<0.001}$). 
Les comparaisons par paires montrent que la performance de détection est significativement plus faible pour la condition d'incertitude du masqueur où $H=339$~nats (c'est-à-dire $64$~fpo et miti : $200$~ms) que pour toutes les autres conditions d'incertitude.

L'analyse du modèle de Cox avec terme de fragilité n'a montré aucun effet significatif du terme de fragilité (${\chi^2=1.25}$, ${df=1}$, ${p=0.26}$) alors que des effets significatifs ont été observés pour l'incertitude du masqueur (${\chi^2=458.32}$, ${df=8}$, ${p<0.001}$), pour le taux de répétition de la cible (${\chi^2=153.81}$, ${df=2}$, ${p<0.001}$) et pour leur interaction (${\chi^2=197.78}$, ${df=26.1}$, ${p<0.001}$).  

Les résultats des comparaisons par paires pour l'interaction entre l'incertitude du masqueur et le taux de répétition de la cible sont donnés dans la Table~\ref{tab:cld_Exp-III}. 
Une illustration des courbes de taux de risque correspondantes pour chaque niveau de taux de répétition de la cible dans chaque cas d'incertitude du masqueur est donnée sur la figure~\ref{fig:hr_Exp-III}. 
% (et la représentation inverse est donnée sur~\nameref{sup:fig_hr_Exp-III}). 
Le taux de risque de la conscience perceptive de la cible augmente significativement avec le taux de répétition de la cible mais il n'y a pas d'effet monotone de l'incertitude du masqueur sur le taux de risque de la conscience perceptive de la cible. 
On peut observer (Figure~\ref{fig:hr_Exp-III}) que les deux paramètres définissant l'incertitude du masqueur, c'est-à-dire le nombre de fréquences par octave et l'intervalle inter-tonalités moyen, ont des effets sous-jacents. 
Le taux de risque de la conscience perceptive de la cible augmente avec l'intervalle inter-tonalités moyen et diminue avec le nombre de fréquences par octave. 
Néanmoins, ces effets globaux des paramètres du masqueur sont modulés par l'effet du taux de répétition des tonalités cible.

\begin{table}[!t]
\caption[Table des moyennes marginales estimées pour les fonctions de taux de risque pour l'Expérience~III]{Table des moyennes marginales estimées pour les fonctions de taux de risque et affichage en lettres compactes pour les comparaisons par paires de l'interaction entre la similarité cible-masqueur et l'incertitude du masqueur dans l'Exp.~III.} 
\label{tab:cld_Exp-III}
\footnotesize
\centering
% \begin{tabular}{llrrrrrl}
\begin{tabular}{|l|*{8}{c|}}
\hline
\textbf{T.Rate} & \textbf{Uncertainty} & \textbf{emmean} & \textbf{SE} & \textbf{df} & \textbf{asymp.LCL} & \textbf{asymp.UCL} & \textbf{.group} \\ 
\hline
1 & 339 & -1.7181 & 0.2668 & Inf & -2.2411 & -1.1951 & 1 \\ 
2 & 339 & -1.0969 & 0.2231 & Inf & -1.5343 & -0.6596 & 12 \\ 
5 & 339 & -0.8521 & 0.2168 & Inf & -1.2771 & -0.4271 & 12 \\ 
1 & 442 & -0.3084 & 0.1990 & Inf & -0.6984 & 0.0816 & 23 \\ 
1 & 169 & -0.2980 & 0.1928 & Inf & -0.6759 & 0.0798 & 23 \\ 
1 & 84 & 0.0000 & 0.0000 & Inf & 0.0000 & 0.0000 & 34 \\ 
2 & 169 & 0.0723 & 0.1960 & Inf & -0.3118 & 0.4564 & 345 \\ 
1 & 492 & 0.1572 & 0.1885 & Inf & -0.2122 & 0.5267 & 3456 \\ 
1 & 221 & 0.2434 & 0.1895 & Inf & -0.1280 & 0.6149 & 3456 \\ 
2 & 442 & 0.2913 & 0.1974 & Inf & -0.0956 & 0.6783 & 3456 \\ 
5 & 169 & 0.3264 & 0.1903 & Inf & -0.0466 & 0.6993 & 3456 \\ 
5 & 442 & 0.4842 & 0.1943 & Inf & 0.1035 & 0.8649 & 4567 \\ 
2 & 84 & 0.5693 & 0.1866 & Inf & 0.2037 & 0.9349 & 45678 \\ 
1 & 246 & 0.7110 & 0.1927 & Inf & 0.3333 & 1.0888 & 456789 \\ 
2 & 221 & 0.7120 & 0.1910 & Inf & 0.3377 & 1.0862 & 56789 \\ 
1 & 110 & 0.7510 & 0.1878 & Inf & 0.3829 & 1.1191 & 56789 \\ 
2 & 110 & 0.7557 & 0.1933 & Inf & 0.3769 & 1.1346 & 56789 \\ 
2 & 492 & 0.7829 & 0.1912 & Inf & 0.4081 & 1.1577 & 56789 \\ 
1 & 123 & 0.8308 & 0.1895 & Inf & 0.4595 & 1.2022 & 6789 \\ 
5 & 221 & 1.0993 & 0.1891 & Inf & 0.7286 & 1.4700 & 7890 \\ 
5 & 110 & 1.1616 & 0.2030 & Inf & 0.7637 & 1.5596 & 7890 \\ 
2 & 123 & 1.2463 & 0.1920 & Inf & 0.8700 & 1.6226 & 890 \\ 
2 & 246 & 1.3122 & 0.1927 & Inf & 0.9345 & 1.6899 & 90 \\ 
5 & 84 & 1.3657 & 0.1915 & Inf & 0.9903 & 1.7411 & 90 \\ 
5 & 246 & 1.5825 & 0.1931 & Inf & 1.2040 & 1.9610 & 0 \\ 
5 & 492 & 1.7711 & 0.1907 & Inf & 1.3974 & 2.1449 & 0 \\ 
5 & 123 & 1.7801 & 0.1931 & Inf & 1.4017 & 2.1586 & 0 \\ 
\hline
\end{tabular}
\smallskip
\begin{flushleft}
Les résultats sont donnés sur l'échelle log (et non de la réponse). 
Niveau de confiance utilisé : $0.95$. 
Ajustement de la valeur p : Méthode de Tukey pour la comparaison d'une famille de $27$ estimations. 
Niveau de signification utilisé : $\alpha = 0.05$. 
Incertitude : incertitude du masqueur, T.Rate : taux de répétition de la cible, emmean : moyenne marginale estimée, SE : erreur standard, df : degrés de liberté, asymp.LCL : limite inférieure asymptotique du contraste, asymp.UCL : limite supérieure asymptotique du contraste, .group : groupe de lettres compactes.
\end{flushleft}
\end{table}

%%%%%%%%%%%%%%%%%%%%%%%%%%%%%%%%%%%%%%%%%%%%%%%%%%%%%%%%%%%%%%%%%%%%%%%%%%%%%%%
\subsection{Résumé}
\label{chapitre4resultatsresume}
%%%%%%%%%%%%%%%%%%%%%%%%%%%%%%%%%%%%%%%%%%%%%%%%%%%%%%%%%%%%%%%%%%%%%%%%%%%%%%%

L'indice de discrimination ($d^\prime$) permet d'étudier uniquement l'effet de l'incertitude du masqueur sur la performance de détection dans les trois expériences. 
Alors que l'incertitude du masqueur n'a aucun effet dans l'Expérience~I, l'Expérience~II montre une diminution de la performance de détection lorsque l'incertitude du masqueur augmente et l'Expérience~III montre qu'une combinaison spécifique de fréquences par octave et d'intervalle inter-tonalités moyen diminue la performance de détection.

L'Expérience~I montre que la conscience perceptive de la cible diminue lorsque l'incertitude du masqueur augmente et lorsque la similarité temporelle cible-masqueur augmente. 
Plus précisément, la perception de la cible diminue lorsque la durée des tonalités du masqueur est supérieure à celle des tonalités de la cible. 
L'Expérience~II montre aussi globalement que la conscience perceptive de la cible diminue lorsque l'incertitude du masqueur augmente et lorsque le taux de répétition de la cible diminue. 
De plus, l'effet de la similarité cible-masqueur dépend des valeurs des autres paramètres. 
L'Expérience~III montre que la conscience perceptive de la cible diminue lorsque le taux de répétition de la cible diminue. 
Dans le cas de l'incertitude du masqueur, l'incertitude temporelle et l'incertitude fréquentielle du masqueur ont des effets opposés : la conscience perceptive de la cible augmente lorsque l'incertitude temporelle augmente et lorsque l'incertitude fréquentielle diminue. 
Dans toutes les expériences, les interactions entre les facteurs expérimentaux sont significatives, ce qui permet de conclure que les effets des propriétés du masqueur et de la cible sur la dynamique de la conscience perceptive dépendent fortement du contexte.

%%%%%%%%%%%%%%%%%%%%%%%%%%%%%%%%%%%%%%%%%%%%%%%%%%%%%%%%%%%%%%%%%%%%%%%%%%%%%%%
\section{Discussion}
\label{chapitre4discussion}
%%%%%%%%%%%%%%%%%%%%%%%%%%%%%%%%%%%%%%%%%%%%%%%%%%%%%%%%%%%%%%%%%%%%%%%%%%%%%%%

Cette étude a examiné comment la similarité temporelle cible-masqueur (principalement Exp.~I), le taux de répétition de la cible (principalement Exp.~II) et l'incertitude du masqueur (principalement Exp.~III) ont affecté la performance et la dynamique de la conscience perceptive liée à la détection d'une cible régulière en utilisant un paradigme de MI. 
Les temps de réaction ont été analysés en utilisant l'analyse des données de survie avec effets mixtes pour prendre en compte à la fois les caractéristiques temporelles des données et la variabilité inter-individuelle de manière quantitative. 

Premièrement, nous avons trouvé un effet asymétrique dans la similarité temporelle entre le masqueur et la cible : la cible était détectée plus facilement lorsque ses tonalités étaient plus longues que celles du masqueur, mais la durée des tonalités du masqueur avait un effet faible, voire nul, si elle était plus longue que celle de la cible. 
Deuxièmement, la détection de la cible était améliorée pour un taux de répétition de la cible plus élevé. 
Troisièmement, l'incertitude du masqueur diminue les performances de détection de la cible, mais la détection de la cible est diminuée lorsque l'incertitude temporelle augmente ou lorsque l'incertitude fréquentielle diminue.

Le MI est connu pour produire une grande quantité de variabilité inter-individuelle, même plus grande que pour le masquage énergétique : \citep{durlach2003informational, durlach2003note, durlach2005informational, kidd2008informationalreview, leek1984learning, neff1995individual}. 
Dans les trois expériences, différents niveaux de variabilité inter-individuelle ont été observés. 
D'une part, dans l'Expérience~II, le modèle de survie avec termes de fragilité a montré un effet significatif de la fragilité qui révèle un niveau élevé d'hétérogénéité inter-individuelle dans la dynamique de la conscience perceptive. 
En revanche, dans les Expérience~I et~III, aucun effet significatif de la fragilité n'a été observé. 
Ce résultat caractérise une plus grande homogénéité entre les sujets dans ces conditions expérimentales. 

La variabilité inter-individuelle observée dans les études sur le MI a été expliquée par la façon dont les auditeurs traitent les paramètres du stimulus afin d'améliorer la ségrégation entre la cible et le masqueur et adoptent des stratégies idéales ou non-idéales \citep{kidd2008informationalreview}. 
Puisque des groupes indépendants et homogènes ont réalisé les trois expériences, les différences observées dans les stratégies de traitement inter-individuelles devraient être principalement attribuées à la combinaison spécifique des propriétés du masqueur et de la cible de chaque expérience. 
L'interaction entre l'incertitude du masqueur, la similarité temporelle cible-masqueur et le taux de répétition de la cible dans les plages utilisées dans l'Expérience~II conduit à des stratégies plus variables que celles des Expériences~I et~III. 
L'interaction entre les trois paramètres utilisés dans l'Expérience~II augmente la variabilité inter-individuelle liée à l'augmentation des situations expérimentales. 
Le faible nombre de stimuli par condition expérimentale présentés à chaque sujet pourrait aussi être un facteur qui tend à augmenter la variabilité intra- et inter-individuelle observée. 

Afin de former une représentation cohérente des objets auditifs, le système auditif doit exploiter les différences dans les statistiques de la structure temporelle des signaux pour séparer la figure du fond \citep{lutfi2013information}. 
Au moins deux phénomènes complémentaires contribuent à la levée du masquage qui conduit à la détection de la cible. 
Premièrement, les propriétés acoustiques du masqueur et de la cible et leurs différences statistiques contribuent à la difficulté de la tâche. 
Deuxièmement, la façon dont le système auditif intègre le contenu informationnel du stimulus et extrait les informations pertinentes pour former le percept de la cible conduit à la dynamique de la conscience perceptive. 
Les résultats obtenus ici permettent de confirmer des résultats antérieurs et de suggérer de nouvelles perspectives d'étude pour ces deux phénomènes dans le cadre de la ségrégation des flux auditifs.

Premièrement, cette étude a manipulé les paramètres acoustiques du stimulus, les relations entre ces paramètres permettant d'obtenir des caractéristiques de second ordre du stimulus telles que l'incertitude du masqueur et la similarité temporelle cible-masqueur. 
Un paramètre acoustique tel que le taux de répétition de la cible est connu pour être d'une importance critique dans le regroupement des indices physiques et perceptifs dans une scène acoustique complexe : \citep{miller2002spectrotemporal, moore2002factors}. 
Dans des cas extrêmes, le taux de répétition des cibles au-delà d'une valeur critique (environ $40$~Hz) peut provoquer la fusion des sons en un flux facilement détectable par les sujets. 
Dans les Expériences~II et~III, les effets observés du taux de répétition de la cible sur la dynamique de la conscience perceptive sont en accord avec la littérature \citep{xiang2010competing, akram2014investigating}.

Deuxièmement, dans les deux expériences où la similarité temporelle cible-masqueur a été manipulée, l'observation de son effet conduit à la conclusion que la conscience perceptive est généralement renforcée lorsque la similarité est faible. 
Cette observation complète les études précédentes où la similarité fréquencielle cible-masqueur \citep{kidd2002similarity} et un degré élevé de similarité temporelle \citep{durlach2003informational} diminuaient fortement les performances de détection. 
Notre étude montre également que l'effet de la similarité temporelle cible-masqueur est asymétrique. 
En effet, dans l'Expérience~II, l'effet était plus prononcé lorsque la durée de la cible était plus grande que la durée des tonalités du masqueur. 
Ce résultat suggère que la durée cible-masqueur est très importante et saillante pour l'organisation des scènes auditives, en particulier des flux auditifs, voire même aussi importante que l'effet du taux de répétition de la cible. 
Ces deux variables, la durée des tonalités de la cible et le taux de répétition de la cible, peuvent agir de concert pour définir la durée du silence entre les cibles successives. 
En résumé, de plus petites durées de silence entre les cibles, associées à un taux de répétition des cibles plus élevé ou à une durée des cibles plus longue (pour un taux de répétition des cibles fixe), favorisent l'accumulation de flux auditifs.

Troisièmement, l'incertitude du masqueur a été manipulée en utilisant le nombre de fréquences par octave et l'intervalle inter-tonalités moyen. 
Il a été montré ici que la conscience perceptive est facilitée lorsque le nombre de fréquences par octave est faible. 
Pour un nombre donné de fréquences par octave, la conscience perceptive est améliorée pour les longs intervalles inter-tonalités moyens. 
Ces résultats sont cohérents avec la littérature précédente où, par exemple, l'incertitude de la fréquence des composants individuels du masqueur était prédominante dans la production de l'effet de masquage \citep{neff1988effective, neff1995individual}. 
Néanmoins, l'incertitude du masqueur, quantifiée par l'entropie de la distribution des tonalités, n'est pas linéairement liée à la facilitation de la conscience perceptive. 
Cette propriété statistique de la composition en tonalités du masqueur n'explique donc pas directement les changements dans la dynamique de la conscience perceptive. 

Bien que les caractéristiques statistiques des stimuli, telles que l'incertitude, aient été utilisées pour expliquer les performances de détection \citep{chang2016detection, lutfi2013information}, nous observons ici que les caractéristiques acoustiques du masqueur, telles que la densité spectro-temporelle (définie comme un nombre de tonalités par seconde et par octave), peuvent expliquer les changements dans la dynamique de la conscience perceptive de manière plus directe (voir Fig. \ref{fig:HvsSTD} et Table \ref{tab:DST} ). 
Néanmoins, il a été démontré que d'autres propriétés statistiques du masqueur, comme l'entropie relative de la composante du masqueur, peuvent rendre compte des caractéristiques du MI \citep{lutfi1993model, oh1998nonmonotonicity}. 
Dans cette étude, toutes les tonalités ont été présentées au même niveau. 
Le niveau global du masqueur est alors plus élevé pour une densité spectro-temporelle plus importante, ce qui pourrait avoir influencé la détection : la cible est potentiellement moins saillante pour une densité spectro-temporelle du masqueur élevée.

\begin{figure}[!t]
\includegraphics[width=\textwidth]{/home/link/Documents/thèse_onera/articles_alex/article_1/PloSOne/PLoS-One-R1/manuscript/figures/S6_Fig.pdf}
\caption[Comparaison entre constantes de temps et entropie et densité spectro-temporelle]{
Comparaison de la relation entre la constante de temps de ségrégation auditive et l'entropie (H) avec celle de la constante de temps de ségrégation auditive et la densité spectro-temporelle (DST) du masqueur. 
La constante de temps de ségrégation auditive est définie comme le temps $\tau$ pour lequel la fonction de distribution cumulative associée au taux de risque de ségrégation auditive est égale à $0,63$ (voir Panel~A.). 
La densité spectro-temporelle du masqueur est définie comme $\mathrm{fpo}/\mathrm{miti}$ en $\mathrm{s}^{-1}\mathrm{oct}^{-1}$ où miti : intervalle inter-tonalités moyen, fpo : fréquences par octave. 
Le Panel~B. compare la relation entre $\tau$ et la DST ou l'entropie. 
Les données pour lesquelles $\tau>12$ ne sont pas représentées.}
\label{fig:HvsSTD}
\end{figure}

\begin{table}[!t]
\caption[Table des valeurs de densité spectro-temporelle du masqueur]{Table des valeurs de la densité spectro-temporelle du masqueur pour l'ensemble des paramètres du masqueur utilisés dans les trois expériences. miti: intervalle inter-tonalités moyen, fpo: fréquences par octave.} 
\label{tab:DST}
\footnotesize
\centering
% \begin{tabular}{rrrrrr}
\begin{tabular}{|l||*{6}{c|}}
\hline
 & & \multicolumn{1}{c|}{\textbf{Exp.~I - II}} & \multicolumn{3}{c|}{\textbf{Exp.~III}} \\
\hline
\textbf{miti (ms)}      & & 800  & 200 & 600 & 1200 \\ 
\textbf{$\Delta$ (ms)}  & & 1400 & 200 & 1000 & 2200\\
\hline
                & 4  & 5 & --- & --- & ---\\
                & 16 & --- & 80 & 26.7 & 13.3\\
\textbf{fpo}    & 32 & 40 & 160 & 53.3 & 26.7\\
                & 64 & 80 & 320 & 106.7 & 53.3\\
\hline
\end{tabular}
\end{table}

Les études précédentes sur le MI ont principalement analysé les résultats des procédures de choix forcé à deux alternatives en utilisant des concepts de la théorie de la détection du signal tels que les seuils de discrimination auditive \citep{alexander2004informational, neff1987masking, watson1976factors} ou la performance de détection. 
Dans la procédure de notre étude, nous n'avons pas été en mesure de calculer l'indice $d^\prime$ pour les conditions expérimentales définies par les propriétés de la cible. 
Ainsi, cela ne nous permet pas d'étudier l'effet des propriétés de la cible et de l'interaction entre les propriétés de la cible et du bruit sur le processus de détection. 
Néanmoins, sur la base de la dynamique de la conscience perceptive, nous avons vu que la relation entre les propriétés du masqueur et de la cible est essentielle, en particulier dans le type de paradigme de MI utilisé ici. 
L'effet d'interaction entre le nombre de fréquences par octave et l'intervalle inter-tonalités moyen peut être considéré comme un cas spécifique des effets significatifs de l'interaction entre les paramètres du stimulus observés sur la dynamique de la conscience perceptive. 
Une telle observation suggère que, du point de vue du traitement auditif, la scène acoustique n'est pas divisée en paramètres indépendants et leurs effets connexes. 
Cependant, la théorie de la détection du signal définit des paramètres statiques qui ne permettent pas de traiter l'intégration progressive des informations conduisant à la conscience perceptive. 
De plus, ces aspects dynamiques du phénomène de détection ne sont pas pris en compte par les méthodes classiques d'analyse du temps de réaction. 
Ainsi, l'analyse par des modèles de survie à effets mixtes apporte une contribution pertinente à l'étude de la dynamique de la conscience perceptive en prenant en compte les effets respectifs des paramètres du stimulus sur la probabilité de détection.

\begin{figure}[!t]
\includegraphics[width=\textwidth]{/home/link/Documents/thèse_onera/articles_alex/article_1/PloSOne/PLoS-One-R1/manuscript/figures/S7_Fig.pdf}
\caption[Fonctions de taux de risque obtenues à partir du modèle d'accumulation de preuves]{Fonctions de taux de risque obtenues à partir du modèle d'accumulation de preuves pour plusieurs valeurs des paramètres. 
Un modèle simple d'accumulation de preuves \citep{nguyen2020buildup} décrit une activité qui s'accumule et sature à un niveau cible $T$. 
L'activité $X_n$ est mise à jour séquentiellement selon : $X_{n+1} = X_n + r (T-X_n) + \varepsilon_{n+1}$ où $T<1$ et où $\varepsilon_{n+1}\sim\mathcal{N}(0,\sigma^2)$ sont des variables aléatoires indépendantes (bruit gaussien de moyenne nulle et d'écart type $\sigma$). 
Les incréments d'activité dépendent de l'état et sont proportionnels à la différence $T-X_n$, avec un taux constant $r$. 
En conséquence, l'activité $X$ dérive vers $T$ de manière stochastique si $0<r<1$. 
L'accumulation ralentit avec $X_n$ près de $T$ et l'activité ne peut franchir le seuil qu'en raison du bruit. 
Exemples de l'effet de changements dans les paramètres du modèle. 
Les paramètres constants sont : $T=0.9$, $X_0=0$. 
En l'absence de variation, $r=0.7$ et $\sigma=0.15$.}
\label{fig:eva}
\end{figure}

Enfin, il a été proposé que l'intégration auditive de l'information puisse être modélisée comme un processus d'accumulation de preuves \citep{barniv2015auditory, nguyen2020buildup}. 
Un tel modèle peut rendre compte de l'accumulation du flux auditif et donc de la dynamique de la conscience perceptive. 
Une comparaison qualitative entre le taux de risque obtenu en simulant un modèle simple d'accumulation de preuves \citep{nguyen2020buildup} et ceux obtenus dans la présente étude suggère que les différentes combinaisons de propriétés du masqueur et de la cible conduisent à des processus d'accumulation de preuves différents (voir Fig. \ref{fig:eva}). 
Bien que ces courbes ne soient que qualitativement similaires aux courbes expérimentales, elles montrent la possibilité d'utiliser l'analyse de survie pour l'analyse des résultats expérimentaux dans le contexte des modèles d'accumulation de preuves. 
En outre, une étude récente portant sur la base neuronale de la conscience perceptive a montré que les changements progressifs de la dynamique neuronale pendant l'accumulation de preuves peuvent être liés à la conscience perceptive et au monitoring perceptif chez l'homme \citep{pereira2021evidence}. 
Ainsi, l'étude de la conscience perceptive auditive pourrait bénéficier de la combinaison des modèles de survie et d'accumulation de preuves avec l'enregistrement de l'activité électrophysiologique à différentes échelles dans des tâches de MI pour déchiffrer la relation entre la perception auditive consciente et la dynamique neuronale.

%%%%%%%%%%%%%%%%%%%%%%%%%%%%%%%%%%%%%%%%%%%%%%%%%%%%%%%%%%%%%%%%%%%%%%%%%%%%%%%
\section{Conclusion}
\label{chapitre4conclusion}
%%%%%%%%%%%%%%%%%%%%%%%%%%%%%%%%%%%%%%%%%%%%%%%%%%%%%%%%%%%%%%%%%%%%%%%%%%%%%%%

En résumé, la similarité temporelle entre le masqueur et la cible, le taux de répétition de la cible et les paramètres d'incertitude du masqueur modulent la dynamique de la conscience perceptive dans le MI. 
La comparaison avec un modèle d'accumulation de preuves montre que ces effets peuvent être médiés aussi par des changements dans le paramètre d'accumulation. 
Notre étude suggère donc que l'utilisation de modèles de survie pour analyser la dynamique de l'accumulation de la perception auditive pourrait être adaptée au test expérimental du modèle d'accumulation de preuves. 
L'étude des effets de ces paramètres de stimulus à l'aide de modèles de survie permet d'étudier l'évolution dans le temps de la probabilité de détection et de mieux déterminer la prévisibilité de la détection du signal. 
Cette étude fournit des informations utiles et nouvelles sur l'évolution temporelle de la conscience perceptive liée à un ensemble de paramètres de stimulus pour concevoir une analyse plus approfondie de la dynamique de la conscience perceptive. 
En particulier, d'autres études sur les corrélats neuronaux possibles de la conscience perceptive peuvent être basées sur ces informations temporelles pour étudier comment la dynamique de la conscience perceptive est liée de manière causale aux transitions dynamiques de la transmission de l'information au niveau de la population neuronale.

%%%%%%%%%%%%%%%%%%%%%%%%%%%%%%%%%%%%%%%%%%%%%%%%%%%%%%%%%%%%%%%%%%%%%%%%%%%%%%%
\clearpage\null\newpage
%%%%%%%%%%%%%%%%%%%%%%%%%%%%%%%%%%%%%%%%%%%%%%%%%%%%%%%%%%%%%%%%%%%%%%%%%%%%%%%
