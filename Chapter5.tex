%%%%%%%%%%%%%%%%%%%%%%%%%%%%%%%%%%%%%%%%%%%%%%%%%%%%%%%%%%%%%%%%%%%%%%%%%%%%%%%
\chapter[Étude 2 : Caractérisation électrophysiologique de la perception consciente dans le MI]{Étude 2 : Caractérisation électrophysiologique de la dynamique de la perception auditive consciente dans le MI}
\label{chapitre5}
\noindent \hrulefill \\
%%%%%%%%%%%%%%%%%%%%%%%%%%%%%%%%%%%%%%%%%%%%%%%%%%%%%%%%%%%%%%%%%%%%%%%%%%%%%%%

Dans le chapitre précédent, nous avons caractérisé l'influence des paramètres déterminants du MI sur la dynamique de la prise de conscience d'un signal auditif. 
Cette première étude nous a permis de spécifier les conditions expérimentales adaptées à une analyse des corrélats neuronaux de la dynamique de la prise de conscience perceptive. 
En effet, caractériser la construction du percept auditif via l'utilisation de mesures spécifiques de caractérisation du signal associé au décours temporel de la perception auditive consciente n'est possible qu'avec une quantité suffisante de données neuronales. 
En ce sens, il est d'une part nécessaire de disposer d'une quantité suffisante de données d'enregistrement d'activité cérébrale pour les cibles détectées ainsi que pour celles manquées et donc d'obtenir suffisamment d'essais de détections correctes et d'essais d'omissions de cibles. 
Il est d'autre part nécessaire de pouvoir disposer d'un décours temporel suffisamment long avant que le sujet ne perçoive le signal cible, afin d'avoir un nombre suffisant d'échantillons du signal neuronal permettant d'étudier la construction progressive du percept conscient. 

Dans la première partie, nous avons montré que plusieurs indicateurs issus des signaux d'activité cérébrale ont été utilisés pour caractériser les états de conscience dont certains ont été associés à la perception auditive consciente. 
Parmi ces indicateurs, ceux basés sur une caractérisation des propriétés statistiques du signal, nécessitent un nombre suffisant d'échantillons temporels pour que leurs estimations soient fiables. 
Ces contraintes nécessitent donc d'obtenir pour notre deuxième étude, un temps moyen de la détection compris entre $3$ et $5$ secondes, une performance de la détection $d^\prime$ supérieure à $1.5$ impliquant des taux de détections correctes supérieurs à $70\%$ et des taux de fausses alarmes inférieurs à $15\%$. 
À partir de cela, nous serons en mesure d'avoir suffisamment d'essais de détections correctes, d'essais de fausses alarmes et des temps d'enregistrement de l'activité neuronale suffisants. 
Les résultats de la première étude étaient donc indispensables pour pouvoir régler le protocole de la deuxième étude de manière quantitative. 

Dans cette seconde partie, l'étude présentée a été motivée par l'émergence de travaux réalisés sur l'analyse comparative de signatures du signal électrophysiologique neural lors d'expériences visant à caractériser les états de conscience de sujets humains \citep{curley2018characterization, engemann2018robust, engemann2020combining, liang2015eeg, sitt2014large}. 
De telles signatures neurophysiologiques consistent fondamentalement en des propriétés intrinsèques de la dynamique neuronale associée aux états de conscience des sujets. 
Nous avons mentionné précédemment que la perception d'un son d’intérêt peut être limitée par des goulots d’étranglement au niveau du traitement de l’information dans le système auditif central plutôt qu'au niveau de la résolution de la périphérie auditive \citep{gutschalk2008neural}. 
En outre, nous avons également vu que la conscience perceptive pourrait provenir de traitements récursifs et intégratifs au sein de modules cérébraux associés aux aires frontales, pariétales et temporales \citep{eriksson2007similar, eriksson2017activity, giani2015detecting}. 
La conscience perceptive dans le MI apparaît ainsi pouvoir émerger à partir d'une cascade dynamique complexe de traitements de l'information à l'échelle neuronale qui s'accumulent sur plusieurs des tonalités du signal cibles \citep{gartner2021auditory, giani2015detecting}. 
En ce sens, les transitions dynamiques au niveau des populations neuronales à l'échelle EEG vont nous permettre de caractériser la construction d'un percept conscient associé à une cible auditive. 

De cette manière, nous étudions ici la dynamique de la construction d'un percept auditif conscient et notamment son lien avec l'évolution de l'activité des aires cérébrales frontales, pariétales et temporales en conjuguant deux approches. 
Ces deux approches ont été proposées dans le but de fournir des modèles de la conscience sur la base de nombreux travaux expérimentaux pour caractériser la dynamique cérébrale associée à l'accès conscient. 
Une première approche, pragmatique, consiste à caractériser la dynamique cérébrale à travers la caractérisation directe du signal au moyen d'outils de mesures des caractéristiques statistiques du signal électrophysiologique. 
Une seconde approche, théorique, consiste à caractériser la conscience sur la base de l'utilisation de mesures théoriques de l'état de conscience issues, dans notre cas, de la théorie de l'information intégrée. \\

Ces deux approches se basent sur une série de mesures organisées en quatre catégories :
\begin{itemize}
\item[1.] les potentiels liés aux événements (ERPs) ;
\item[2.] le contenu informationnel et le degré de complexité associés au signal neuronal (unidimensionnel) ; 
\item[3.] la dynamique locale par rapport à l'échange d'informations entre électrodes (bidimensionnel) ;
\item[4.] l'intégration de l’information à une structure large-échelle (multidimensionnel). \\
\end{itemize}

Dans ce chapitre, nous présentons les résultats de notre deuxième étude, conduite chez le sujet humain adulte sain lors d'une tâche de MI et construite sur la base des résultats de l'étude I.
Le but de cette étude est donc de confronter sur un même ensemble de données les différentes approches considérées préalablement : i) les ERPs ; ii) la caractérisation macroscopique des états de conscience ; et iii) les mesures fondées sur la TII de la conscience. \\

De cette manière, notre objectif est double : 
\begin{itemize}
\item[$\bullet$] comparer les résultats et explications fournis par ces différentes mesures sur la prise de conscience.
\item[$\bullet$] sélectionner des mesures propices à être utilisées pratiquement pour le suivi de la prise de conscience d'un stimulus en situation opérationnelle (par ex. la détection d'une alarme sonore). \\
\end{itemize} 

D'abord, nous avons cherché à reproduire les résultats principaux de la littérature sur les corrélats neuronaux de la perception auditive consciente \citep{dykstra2016neural, giani2015detecting, gutschalk2008neural, wiegand2012correlates} en se basant sur l'acquisition du signal neuronal par EEG dans un paradigme de MI. 
Ensuite, nous avons souhaité déterminer si les indicateurs basés sur des mesures d'information et de complexité du signal neuronal présentés dans la première partie pouvaient nous permettre de caractériser la dynamique de la perception auditive consciente. 
Nous avons donc voulu caractériser l'évolution du contenu informationnel et de la complexité de l'activité associée aux aires cérébrales frontales, temporales et pariétales du fait de leur association présumée avec les mécanismes de perception auditive consciente. 
Puis, nous avons étudié si un transfert d'information plus important s'exprimait entre ces différentes zones cérébrales frontales, temporales et pariétales, lors de la perception auditive consciente. 
Finalement, nous avons cherché à tester les mesures d’information intégrée issues de la TII de la conscience dans leur capacité à caractériser la perception auditive consciente dans un paradigme spécifique de MI. \\

%%%%%%%%%%%%%%%%%%%%%%%%%%%%%%%%%%%%%%%%%%%%%%%%%%%%%%%%%%%%%%%%%%%%%%%%%%%%%%%
\section{Matériel et Méthodes}
\label{etude2materielmethode}
%%%%%%%%%%%%%%%%%%%%%%%%%%%%%%%%%%%%%%%%%%%%%%%%%%%%%%%%%%%%%%%%%%%%%%%%%%%%%%%

Cette étude a été approuvée par le comité d'éthique sous la référence : IRB00011835-2020-06-09-253. 

%%%%%%%%%%%%%%%%%%%%%%%%%%%%%%%%%%%%%%%%%%%%%%%%%%%%%%%%%%%%%%%%%%%%%%%%%%%%%%%
\subsection{Participants}
\label{etude2participants}
%%%%%%%%%%%%%%%%%%%%%%%%%%%%%%%%%%%%%%%%%%%%%%%%%%%%%%%%%%%%%%%%%%%%%%%%%%%%%%%

Nous avons réalisé une analyse préliminaire pour obtenir une taille d'échantillon statistique appropriée en utilisant la bibliothèque \texttt{SIMR} du logiciel R \citep{Rlanguage2021}.
Nous avons prédéfini un seuil statistique $\alpha$ de 5\%, une puissance statistique $1-\beta$ de 84\% ainsi qu'une taille d'effet $d$ de $0.44$ (effet statistique de taille «moyen»). 
Nous avons ensuite prédéfini un nombre d'items statistiques de $20\times4$ correspondant à $20$ items statistiques par bloc expérimental (au nombre de $4$), ainsi qu'un nombre d'observations total supérieur à $1200$. 
Sur la base de ces critères prédéfinis, les simulations statistiques ont indiqué qu'une taille d'échantillon de $n=15$ sujets seraient à minima nécessaires pour atteindre les réquisitions statistiques. 

De fait, vingt participants ont été recrutés suite à un appel de volontariat pour effectuer l'expérience. 
Tous ont déclaré avoir une vision et une audition normales. 
Les participants ne présentaient aucun trouble neurologique ou psychiatrique et n'étaient pas sous traitement médical quelconque. 
Les participants ont reçu une carte cadeau de $30$ euros pour leur participation à l'étude. 

%%%%%%%%%%%%%%%%%%%%%%%%%%%%%%%%%%%%%%%%%%%%%%%%%%%%%%%%%%%%%%%%%%%%%%%%%%%%%%%
\subsection{Stimuli}
\label{etude2stimuli}
%%%%%%%%%%%%%%%%%%%%%%%%%%%%%%%%%%%%%%%%%%%%%%%%%%%%%%%%%%%%%%%%%%%%%%%%%%%%%%%

Dans cette seconde étude, la construction des stimuli était réalisée de la même manière que ceux de la première étude (voir Chapitre~\ref{chapitre4}). 
Tous les stimuli auditifs étaient composés d'un masqueur multi-tonalités et éventuellement d'un signal cible. 
Les signaux cibles étaient composés d'une série régulière de tonalités définie par les paramètres suivants : durée des tonalité (en ms), fréquence (en Hz) et taux de tonalités par seconde (en Hz). 
Dans les essais où il est présent, le signal cible débutait toujours avec un délai de $600$ ms après le commencement du masqueur. 
Afin d'éviter une habituation vis-à-vis des propriétés acoustiques de la cible, les fréquences des tonalités utilisées pour le signal cible étaient tirées aléatoirement d'un ensemble de cinq fréquences : $699$, $1000$, $1430$, $2045$ et $2924$ Hz. 
La durée des tonalités cibles était de $60$ ou $100$ ms.

Le masqueur était composé d'un bruit multi-tonalités caractérisé par la gamme de fréquences du masqueur, le nombre de fréquences par octave, la durée de tonalités et l'intervalle moyen entre les tonalités. 
Le nombre de fréquences par octave permettait de quantifier la densité spectrale du masqueur et les intervalles entre les tonalités permettaient de quantifier la densité temporelle du masqueur. 
Les fréquences des tonalités dans le masqueur étaient espacées de manière égale sur une échelle logarithmique entre $239$ et $5000$ Hz. 
L'intervalle entre les tonalités de masqueur a été tiré au hasard à partir d'une distribution uniforme avec les paramètres d'échelle suivants possibles : (min=$100$ms, max=$300,700,1100,1500$ms, $\mu=200,400,600,800$ms). 
Toutes les tonalités du masqueur présentaient une durée de $20$ ms et comprenaient des rampes de $10$ ms en forme de cosinus. 
Comme dans les expériences du Chapitre~\ref{chapitre4}, une région protégée entourant le signal cible a été maintenue exempte de toute tonalité dans le signal masqueur (\textit{i.e.}, de toute source d'énergie). 

%%%%%%%%%%%%%%%%%%%%%%%%%%%%%%%%%%%%%%%%%%%%%%%%%%%%%%%%%%%%%%%%%%%%%%%%%%%%%%%
\subsection{Procédure}
\label{etude2procedure}
%%%%%%%%%%%%%%%%%%%%%%%%%%%%%%%%%%%%%%%%%%%%%%%%%%%%%%%%%%%%%%%%%%%%%%%%%%%%%%%

Un bloc d'entraînement de $60$ essais a d'abord été présenté afin de sensibiliser les participants à la tâche expérimentale et d'évaluer leurs taux de détection concernant les différentes fréquences de la cible. 
Le bloc de session d'entraînement était composé d'essais avec et sans signal cible. 
Pendant ce bloc d'entraînement, nous nous sommes assurés que les participants puissent bien entendre les $5$ différentes fréquences par contrôle visuel en-ligne des résultats affichés sur le logiciel de visualisation des signaux EEG et par correspondance des triggers cible-masqueur et appui-bouton par le participant. 

La session expérimentale était composée de $240$ essais répartis aléatoirement en $4$ blocs de $60$ essais. 
Chaque essai durait $10$ secondes et les essais successifs étaient séparés par un intervalle de $3$ secondes de silence. 
Chaque essai était composé d'un signal masqueur et éventuellement d'un signal cible. 
La tâche des participants était d'appuyer sur la touche droite avec leur index droit en utilisant une boîte de réponse (Chronos Psychology Software Tools Inc., Pittsburg, USA) aussitôt qu'ils pensaient avoir détecté le signal cible. 
Les sujets étaient encouragés à répondre aussi précisément et aussi rapidement que possible. 
Les sujets étaient informés que le signal cible ne serait pas présent à chaque essai mais aucune information concernant la probabilité d'occurrence de la cible n'était donnée. 
Un tiers des essais ($80:240$) ne comportait aucune cible et les deux tiers restants ($160:240$) comportaient une cible. 

Les stimulations auditives étaient produites par un ordinateur DELL PRECISION M4800 (processeur i7 $4900$ MQ, $16$~GB DDR3 RAM, NVidia Quadro K2100M sous Windows $7$ avec une carte son Intel Lynx Point PCH) et présentées aux auditeurs via des écouteurs insert ER-3 (Etymotic Research) à un niveau d'écoute confortable. 
Une croix de fixation était affichée en blanc sur un fond noir à l'aide du logiciel E-prime $2.0$ (v.2.0.10.356, E-prime Psychology Software Tools Inc., Pittsburg, USA) sur un moniteur CRT $19$ pouces (avec une résolution de $1024 \times 768$ pixels et un taux de rafraîchissement de $100$~Hz) situé à $46$~cm du participant dans une pièce sombre et insonorisée. 
E-Prime a également été utilisé pour présenter les stimuli auditifs. 
Un trigger était enregistré par E-Prime et envoyé au système EEG chaque fois qu'un essai commençait. 

%%%%%%%%%%%%%%%%%%%%%%%%%%%%%%%%%%%%%%%%%%%%%%%%%%%%%%%%%%%%%%%%%%%%%%%%%%%%%%%
\subsection{Enregistrements EEG}
\label{etude2enregistrementsEEG}
%%%%%%%%%%%%%%%%%%%%%%%%%%%%%%%%%%%%%%%%%%%%%%%%%%%%%%%%%%%%%%%%%%%%%%%%%%%%%%%

\begin{figure*}[!t]
\centering
\includegraphics[width=0.54\linewidth]{Figures/illustrations/Exp_EEG/systemeeg.pdf}
\includegraphics[width=0.45\linewidth]{Figures/illustrations/Exp_EEG/clustering.pdf}
\caption[Système EEG, positionnement 10-20 et agrégation des électrodes]{
(Gauche) Référence du système de positionnement international 10-20 utilisé pour le placement des 64 électrodes. 
Deux électrodes d'électro-oculographie ont été placées sur la face externe des yeux gauche (lHEOG) et droit (rHEOG) afin de détecter les artefacts tels que les clignements et les mouvements oculaires. 
Deux électrodes actives, TP9 et TP10 (non figurées ici), ont été utilisées pour enregistrer le signal de la mastoïde gauche et droite. 
(Droite) Procédure d'agrégation des électrodes par moyennage arithmétique des valeurs d'intérêts utilisée dans l'étude II EEG. 
Cette procédure a été précédemment utilisée par \cite{grabner2012oscillatory} dans l'étude des corrélats neuronaux associés à l'apprentissage. 
Elle présente l'avantage de disposer de 8 aires corticales différentes ainsi que d'une aire sagittale et également d'avoir une latéralisation des aires cérébrales.}
\label{fig:figure5systemeegclustering}
\end{figure*}

L'électroencéphalogramme a été enregistré en continu à l'aide d'un bonnet élastique ActiCAP (Brain Products GmbH) adapté et équipé de 64 électrodes actives unipolaires Ag/AgCl selon le positionnement international des électrodes 10/20 étendu (Figure~\ref{fig:figure5systemeegclustering} Gauche). 
Le signal brut EEG a été enregistré à l'aide du logiciel Brain Vision Recording (version 1.20.0801) avec une impédance maintenue en dessous de $10$~k$\Omega$ pour toutes les électrodes. 
Le signal était amplifié à l'aide d'un système ActiCHampTM (Brain Products, Inc.), numérisé sur $24$~bits et échantillonné à $1000$~Hz, avec une résolution de $0,05$~$\mu$V.

Deux électrodes actives, TP9 et TP10, ont été utilisées pour enregistrer le signal des mastoïdes droite et gauche pour lequel l'activité moyenne a été utilisée comme référence. 
L'électrode de masse utilisée pour l'acquisition des données EEG a été positionnée sur le front (électrode Fpz). 
Deux électrodes d'électro-oculographie en argent placées sur les tempes gauche et droite étaient utilisées pour enregistrer l'électro-oculogramme (EOG) afin de détecter les artefacts comme les clignements et les mouvements oculaires. 
Enfin, les participants ont tous reçu pour instruction de limiter autant que possible les clignements et mouvements oculaires ainsi que les mouvements de corps principalement pendant la session expérimentale. 
Des pauses étaients disposées à cet effet entre les différents blocs expérimentaux. 

%%%%%%%%%%%%%%%%%%%%%%%%%%%%%%%%%%%%%%%%%%%%%%%%%%%%%%%%%%%%%%%%%%%%%%%%%%%%%%%
\subsection{Traitements et analyses des enregistrements EEG}
\label{etude2traitementetanalysesEEG}
%%%%%%%%%%%%%%%%%%%%%%%%%%%%%%%%%%%%%%%%%%%%%%%%%%%%%%%%%%%%%%%%%%%%%%%%%%%%%%%

\begin{figure*}[!t]
\centering
\includegraphics[width=0.49\linewidth]{Figures/illustrations/Exp_EEG/trial.pdf}
\includegraphics[width=0.49\linewidth]{Figures/illustrations/Exp_EEG/epochsdesign.pdf}
\caption[Illustration graphique d'un essai de stimulus et du design de segmentation des epochs]{(Gauche) Illustration graphique de la stimulation auditive. Un flux de tonalités cible (en rouge) était présenté en fonction des essais, intégré dans un masqueur multi-tonalités aléatoire en temps et en fréquence. Une largeur de bande rectangulaire (en vert) était disposé de part et d'autre de la fréquence de la cible de manière à éviter le masquage énergétique. La fréquence de la cible était fixée ici à $1000$~Hz avec une durée de $60$~ms et un taux de répétition de tonalités à $1$~Hz. Les tonalités aléatoires du masqueur s'étendant de $239$ à $5000$~Hz sont figurées en noir avec une durée de $20$~ms. Le nombre de séquences fréquentielles par octave du masqueur a été ici fixé à $32$~fpo avec une distribution ITI de $\mu = 800$ ms. (Droite) Illustration graphique de la procédure de segmentation des fenêtres temporelles (epochs). En fonction de la présence du flux de tonalités cible dans les essais et de la réponse du participant, quatre catégories d'epochs étaient obtenues par découpage en fenêtres de $3$~secondes avant la détection (ou non-détection) et $3$~secondes après la détection (ou non-détection) : "Epoch Before Hit", "Epoch After Hit", "Epoch Before Miss" et "Epoch After Miss".}
\label{fig:figure5segmentationepochs}
\end{figure*}

Toutes les analyses des données EEG ont été effectuées à l'aide de Python-MNE v0.20.5 \citep{gramfort2013meg}. 
Les données EEG brutes ont été reréférencées hors ligne à la moyenne des électrodes. 
Globalement, le signal EEG a ensuite été sous-échantillonné à $500$~Hz et des filtres non-causaux passe-bas ($80$~Hz) et passe-haut ($1$~Hz) ont été appliqués aux données. 
Pour les analyses des potentiels reliés à l'évènement, un filtrage différent a été appliqué : filtres non-causaux passe-bas ($20$~Hz) et passe-haut ($1$~Hz).
Les artefacts liés aux mouvements oculaires (saccades et clignements) ont été corrigés à l'aide d'une analyse en composantes indépendantes (ICA, $20$ composantes, $800$ itérations). 

De plus, afin d'écarter les essais présentant un bruit excessif, nous avons utilisé le module python \texttt{autoreject} v0.2.1 \citep{jas2016automated, jas2017autoreject}, permettant de supprimer les essais pour lesquels l'amplitude pic-à-pic dépassait un seuil de rejet. 
Les seuils ont été appris à partir des données en utilisant un algorithme non-supervisé qui minimise une erreur de validation croisée. 
Cette procédure a permis de supprimer les essais contenant des sauts transitoires dans des canaux EEG isolés, mais aussi des artefacts de clignement des yeux affectant des groupes de canaux. 
Elle a également permis de réparer les canaux EEG défectueux en utilisant une interpolation par spline sphérique afin d'obtenir le même ensemble de canaux pour chaque sujet. 
Ensuite, tous les segments restants contaminés par une activité musculaire et/ou des artefacts non-physiologiques ont été rejetés hors-ligne après une inspection visuelle.

Pour permettre une analyse comparative de la perception auditive consciente, nous avons catégorisé les essais. 
Pour distinguer les essais masqueur des essais masqueur-cible, les essais masqueur ont été étiquetés «M» et les essais masqueur-cible «MT». 
En fonction de la présence de la cible dans les essais et de la réponse du participant, le signal EEG de chaque essai présentant une cible a été classé soit dans la catégorie détection correcte «Hit» lorsque le participant appuyait sur le bouton lorsque la cible était présente ou soit dans la catégorie détection manquée «Miss» lorsque le participant n'appuyait pas sur le bouton alors que la cible était présente.
En raison de l'obligation de percevoir au moins deux tonalités cibles, seules les réponses après $1600$~ms ont été retenues et tout appui-bouton avant ce temps était considéré comme une cible devinée (voir Chapitre~\ref{chapitre4}).
Pour étudier la dynamique autour de la détection de manière comparative, nous avons utilisé les temps de détection des sujets pour les détections correctes et le temps de détection moyen des sujets pour les détections manquées (c'est-à-dire $3.4$~sec). 

Ensuite, le signal a été segmenté en fenêtres de $3$~sec avant et après la référence (appui-bouton par le sujet pour la détection ou temps moyen de détection par les sujets pour l'omission). 
Chaque fenêtre a été corrigé par la ligne de base en utilisant la fenêtre temporelle totale correspondante. 
Ces fenêtres ont été respectivement étiquetées "Epoch Before" et "Epoch After". 
Il en résulte donc quatre types/catégories de fenêtres : "Epoch Before Hit", "Epoch After Hit", "Epoch Before Miss" et "Epoch After Miss". 
De cette façon, chacune des fenêtres temporelles pour les hits ainsi que pour les miss a une durée totale de $6$~sec. 
Une représentation de la procédure de segmentation des fenêtres temporelles est visible sur la Figure~\ref{fig:figure5segmentationepochs}. 

Enfin, nous avons réalisé une procédure d'agrégation topographique des électrodes consistant à agréger entre elles les électrodes sur la base de leur localisation. 
Les électrodes ainsi agrégées forment des clusters représentant des zones cérébrales. 
Plusieurs méthodes d'agrégation topographique ont été proposées dans la littérature. 
Une procédure intéressante a été utilisée par \cite{grabner2012oscillatory} lors de l'enregistrement de l'EEG dans une étude d'entraînement à l'arithmétique chez l'être humain. 
Cette procédure d'agrégation topographique (Figure~\ref{fig:figure5systemeegclustering} Droite) présente l'avantage de disposer de $8$ aires corticales différentes ainsi que d'une aire sagittale et également de permettre une latéralisation des aires. 
Les noms de cluster et les positions des électrodes sont données à titre d'exemple ici pour l'hémisphère gauche : antérofrontale (AF ; Fp1, AF7, AF3), frontale (F ; F7, F5, F3, F1), frontocentrale (FC ; FC5, FC3, FC1), centrale (C ; C5, C3, C1), centropariétale (CP ; CP5, CP3, CP1), pariétale (P ; P7, P5, P3, P1) ; pariéto-occipitale (PO ; PO7, PO3, O1), et temporale (T ; FT7, T7, TP7). 
Une neuvième zone correspond à l'aire sagittale comprenant les électrodes AFz, Fz, FCz, Cz, CPz, Pz, POz et Oz. 

%%%%%%%%%%%%%%%%%%%%%%%%%%%%%%%%%%%%%%%%%%%%%%%%%%%%%%%%%%%%%%%%%%%%%%%%%%%%%%%
\section{Résultats comportementaux}
\label{etude2resultatscomportementaux}
%%%%%%%%%%%%%%%%%%%%%%%%%%%%%%%%%%%%%%%%%%%%%%%%%%%%%%%%%%%%%%%%%%%%%%%%%%%%%%%

La Table~\ref{fig:table5resultatscomportementauxindividuels} présente les résultats comportementaux de l'étude EEG. 
Pour chaque sujet, il est indiqué le nombre de cibles détectées (Hit), de cibles non-détectées (Miss), de fausses alarmes (FA), de rejets corrects (RC) ainsi que les taux de détections correctes (HIR), les taux de fausses alarmes (FAR), les indices de performance de la détection $d^\prime$ et les temps de détections moyens (TD) et leurs écart-types (TD $\sigma$). 
Le temps de détection moyen était de $3449 \pm 1581$ ms tandis que l'indice de performance de la détection $d^\prime$ moyen était de $2.15 \pm 0.84$. 
Les taux de détection correcte et les taux de fausse alarme étaient respectivement de $0.71\% \pm 0.14\%$ et $0.11\% \pm 0.14\%$. 

Les résultats comportementaux des indices de performance $d^\prime$ ainsi que les temps de détection pour chacun des sujets sont observables sur la Figure~\ref{fig:figure5resultatscomportementaux}.
On observe notamment une relative homogénéité de la performance de la détection, excepté pour les sujets $11$, $17$, $18$ et $20$, lesquels présentent un indice $d^\prime$ inférieur ou proche de $1$, 
En effet, ces sujets affichent des taux de fausses alarmes largement supérieurs à la moyenne de l'ensemble (\textit{i.e.}, $0.11$~\%). 
En ce qui concerne les temps de détection, on observe également un pattern homogène, si ce n'est que les sujets $10$, $12$, $17$ et $18$ présentent des temps de détection moyens relativement plus élevés que la moyenne d'ensemble (\textit{i.e.}, $3449$~ms). 

\begin{table}[!h]
\centering
\scriptsize
\caption[Table des résultats comportementaux pour les temps et les performances de la détection de l'Étude 2 EEG.]{Table des résultats comportementaux pour les temps et les performances de la détection de l'étude 2 EEG. Pour chaque sujet (Sujet Id.) est indiqué le nombre de cibles détectées (Hit), de cibles non-détectées (Miss), de fausses alarmes (FA), de rejets corrects (RC) ainsi que les taux de détections correctes (HIR), les taux de fausses alarmes (FAR), les indices de performance de la détection $d^\prime$ et les temps de détection moyen (TD) et leur écart-type standard (TD $\sigma$).}
\label{fig:table5resultatscomportementauxindividuels}
\begin{tabular}{|l||*{10}{c|}}\hline
\backslashbox{\textbf{Sujet Id.}}{\textbf{Variable}} & 
\makebox[2em]{\textbf{Hit}} & \makebox[2em]{\textbf{Miss}} & 
\makebox[2em]{\textbf{FA}} & \makebox[2em]{\textbf{RC}} & 
\makebox[3em]{\textbf{HIR (\%)}} & \makebox[3em]{\textbf{FAR (\%)}} & 
\makebox[2em]{\textbf{$d^\prime$}} & \makebox[3em]{\textbf{TD (ms)}} & 
\makebox[3em]{\textbf{TD $\sigma$ (ms)}} \\\hline\hline
1 & 125 & 31 & 2 & 78 & 79 & 3 & 2.70 & 3016.38 & 1266.45 \\\hline
2 & 93 & 40 & 1 & 79 & 69 & 1 & 2.60 & 2928.46 & 1521.20 \\\hline
3 & 96 & 35 & 8 & 72 & 73 & 9 & 1.93 & 3277.79 & 1452.31 \\\hline
4 & 105 & 44 & 4 & 76 & 70 & 5 & 2.12 & 2874.57 & 1461.29 \\\hline
5 & 120 & 34 & 2 & 78 & 77 & 3 & 2.63 & 2631.48 & 1093.50 \\\hline
6 & 97 & 22 & 7 & 73 & 81 & 8 & 2.28 & 2891.18 & 1623.26 \\\hline
7 & 63 & 81 & 1 & 79 & 43 & 1 & 1.92 & 4006.44 & 1926.99 \\\hline
8 & 132 & 28 & 0 & 80 & 82 & 0 & 3.42 & 4046.70 & 1691.18 \\\hline
9 & 69 & 31 & 7 & 73 & 68 & 6 & 1.96 & 3003.09 & 1393.35 \\\hline
10 & 150 & 10 & 0 & 80 & 93 & 0 & 4.01 & 4467.13 & 2077.63 \\\hline
11 & 113 & 44 & 39 & 41 & 71 & 48 & 0.60 & 3609.21 & 1458.15 \\\hline
12 & 82 & 78 & 0 & 80 & 51 & 0 & 2.53 & 4629.27 & 1806.98 \\\hline
13 & 91 & 61 & 2 & 78 & 59 & 3 & 2.11 & 2693.79 & 1186.18 \\\hline
14 & 136 & 1 & 25 & 55 & 98 & 31 & 2.77 & 2843.01 & 1106.80 \\\hline
15 & 120 & 31 & 3 & 77 & 79 & 4 & 2.53 & 2857.52 & 1380.11 \\\hline
16 & 123 & 37 & 7 & 73 & 76 & 9 & 2.05 & 3391.00 & 1559.34 \\\hline
17 & 68 & 92 & 14 & 66 & 42 & 17 & 0.73 & 4540.31 & 2304.32 \\\hline
18 & 120 & 40 & 37 & 43 & 74 & 46 & 0.76 & 4849.09 & 1932.22 \\\hline
19 & 117 & 37 & 3 & 77 & 75 & 4 & 2.41 & 3318.07 & 1704.78 \\\hline
20 & 78 & 38 & 26 & 54 & 67 & 26 & 1.07 & 3115.44 & 1684.26 \\\hline
Moyenne & 105 & 40 & 9 & 70 & 71 & 11 & 2.15 & 3449 & 1581 \\\hline
\end{tabular}
\end{table}

\begin{figure*}[!t]
\includegraphics[width=0.5\linewidth]{Figures/illustrations/Exp_EEG/dprime_all_subjects_exp_EEG.pdf}
\includegraphics[width=0.5\linewidth]{Figures/illustrations/Exp_EEG/rates_all_subjects_exp_EEG.pdf}
\begin{minipage}{\textwidth}
\centering
\includegraphics[width=0.5\linewidth]{Figures/illustrations/Exp_EEG/times_all_subjects_exp_EEG.pdf}
\end{minipage}
\caption[Résultats comportementaux de l'étude EEG.]{Résultats comportementaux des indices de performance $d^\prime$ et des temps de détection dans l'étude EEG. (Haut Gauche) : Indice de performance $d^\prime$ (carrés) de chaque sujet. (Haut Droite) : Taux de réussite (cercles) et taux de fausses alarmes (triangles) de chaque sujet. (Bas) : Temps de détection (DT) représentés sous forme de violon-plot de chaque sujet.}
\label{fig:figure5resultatscomportementaux}
\end{figure*}

\begin{figure*}[!t]
\includegraphics[width=0.5\linewidth]{Figures/illustrations/Exp_EEG/dprime_density_by_subject_exp_EEG.pdf}
\includegraphics[width=0.5\linewidth]{Figures/illustrations/Exp_EEG/dprime_uncertainty_by_subject_exp_EEG.pdf}
\includegraphics[width=0.48\linewidth]{Figures/illustrations/Exp_EEG/cdf_m_density_exp_EEG.pdf}
\includegraphics[width=0.48\linewidth]{Figures/illustrations/Exp_EEG/cdf_uncertainty_exp_EEG.pdf}
\caption[Indices de performance et fonctions de répartition de l'étude EEG.]{
Indices de performance $d^\prime$ et fonctions de répartition $F(t)$ pour la densité spectro-temporelle et l'incertitude du masqueur dans l'étude EEG. 
(Haut Gauche) : Indices de performance $d^\prime$ des sujets en fonction des différents indices de densité spectro-temporelle du masqueur. 
(Haut Droite) : Indices de performance $d^\prime$ des sujets en fonction des différents indices d'incertitude du masqueur.
(Bas Droite) : Fonction de distribution cumulée $F(t)$ en fonction des différents indices de densité spectro-temporelle du masqueur. 
(Bas Gauche) : Fonction de distribution cumulée $F(t)$ en fonction des différents indices d'incertitude du masqueur.
Les deux tables ci-dessous présentent les résultats statistiques des modèles linéaires mixtes réalisés sur les indices $d^\prime$ pour la densité spectro-temporelle (table de gauche) et pour l'incertitude (table de droite) du masqueur. 
Les p-values significatives sont accompagnées d'une étoile (*) dans la colonne correspondante.}
\label{fig:figure5resultatscomportementaux2}

\bigskip

\scriptsize
\begin{multicols}{2}
% \begin{tabular}{lllll}
\begin{tabular}{|l||*{5}{c|}}
\hline
& numDF & denDF & F & p \\ 
\hline
(Intercept) & 1 & 57.00 & 142.10 & 0.00 \\ 
Density & 3 & 57.00 & 4.46 & 0.01 *\\ 
\hline
& Estimate & Std. Error & z & p \\ 
\hline
20 - 11 & -0.39 & 0.12 & -3.23 & 0.01 *\\ 
28 - 11 & -0.34 & 0.12 & -2.80 & 0.03 *\\ 
36 - 11 & -0.11 & 0.12 & -0.91 & 0.80 \\ 
28 - 20 & 0.05 & 0.12 & 0.43 & 0.97 \\ 
36 - 20 & 0.28 & 0.12 & 2.32 & 0.09 \\ 
36 - 28 & 0.23 & 0.12 & 1.89 & 0.23 \\ 
\hline
\end{tabular}

% \begin{tabular}{lllll}
\begin{tabular}{|l||*{5}{c|}}
\hline
& numDF & denDF & F & p \\ 
\hline
(Intercept) & 1 & 38.00 & 122.21 & 0.00 \\ 
Uncertainty & 2 & 38.00 & 5.03 & 0.01 *\\ 
\hline
& Estimate & Std. Error & z & p \\ 
\hline
409 - 169 & -0.39 & 0.13 & -3.00 & 0.01 *\\ 
678 - 169 & -0.34 & 0.13 & -2.60 & 0.03 *\\ 
678 - 409 & 0.05 & 0.13 & 0.40 & 0.92 \\ 
\hline
\end{tabular}
\end{multicols}
\end{figure*}

Les indices de performance $d^\prime$ et fonctions de répartition $F(t)$ pour la densité spectro-temporelle et l'incertitude du masqueur de l'expérience sont visibles sur la Figure~\ref{fig:figure5resultatscomportementaux2}. 
Un modèle linéaire à effets mixtes a été ajusté pour étudier l'effet de l'incertitude du masqueur sur l'indice de performance $d^\prime$ en incluant l'Id. Sujet comme effet aléatoire. 
Le pouvoir explicatif total du modèle est substantiel ($R^2$ conditionnel $=0,69$) et la partie liée aux seuls effets fixes ($R^2$ marginal) est de $0,06$. 
Dans ce modèle, l'effet de l'incertitude est statistiquement significatif ($F(2,38)=5.03$, $p=0.01$). 
Les tests post-hoc ont montré que les valeurs de $d^\prime$ sont significativement plus faibles pour une incertitude de $409$ et $678$ nats comparativement à une incertitude de $169$ nats, signifiant que de plus hautes valeurs d'incertitude du masqueur ont diminué la performance de la détection. 

De la même manière, un modèle linéaire à effets mixtes a été ajusté pour étudier l'effet de la densité spectro-temporelle du masqueur sur l'indice de performance $d^\prime$ en incluant l'Id. Sujet comme effet aléatoire. 
Le pouvoir explicatif total du modèle est substantiel ($R^2$ conditionnel $=0,70$) et la partie liée aux seuls effets fixes ($R^2$ marginal) est de $0,05$. 
Dans ce modèle, l'effet de la densité du masqueur est statistiquement significatif ($F(2,57)=4.46$, $p=0.01$). 
Les tests post-hoc ont révélé que les valeurs de $d^\prime$ sont significativement plus faibles pour une densité de masqueur de $20$ et de $28$ comparativement à une densité de $11$.
Aucun autre effet significatif n'a été trouvé pour les autres comparaisons multiples. 

De manière importante, nous observons principalement ici que nos objectifs a priori établis précédemment ont pu être réalisés. 
Nous avons souhaité obtenir un temps moyen de la détection compris entre $3$ et $5$ secondes, une performance de la détection $d^\prime$ supérieure à $1.5$ avec des taux de détection correcte supérieurs à $70\%$ et des taux de fausses alarme inférieurs à $15\%$. 
Ainsi, pour cette seconde étude, le temps de détection moyen est supérieur à $3$~secondes, l'indice de performance $d^\prime$ moyen est supérieur à $2$ avec des taux de détection correcte et des taux de fausse alarme respectivement à $0.72\%$ et $0.11\%$. 
Cette possibilité de régler correctement le protocole expérimental de cette expérience est une conséquence majeure de notre première étude (Chapitre~\ref{chapitre4}). 
Malgré des conditions expérimentales de densité de masqueur relativement homogènes, nous avons observé un effet de la densité et de l'incertitude du masqueur sur l'indice de performance $d^\prime$. 
Nous ne nous attendions pas à avoir ces effets mais ils n'altèrent en rien la suite de l'analyse de l'activité EEG liée à la perception auditive consciente. 

%%%%%%%%%%%%%%%%%%%%%%%%%%%%%%%%%%%%%%%%%%%%%%%%%%%%%%%%%%%%%%%%%%%%%%%%%%%%%%%%%%%%%%%%%%%%%%%%%%%%%%%%%%%%%%%%%%%%%%%%%%%%%%%%%%%%%%%%%%%%%%%%%%%%%%%%%%%%%%%%%%%%%%%%%%%%%%%%%%%%%%%%%%%%%%%%%%%%%%%%%%%%%%%%%%%%%%%%%%%%%%%%%%%%%%%%%%%%%%%
%%%%%%%%%%%%%%%%%%%%%%%%%%%%%%%%%%%%%%%%%%%%%%%%%%%%%%%%%%%%%%%%%%%%%%%%%%%%%%%%%%%%%%%%%%%%%%%%%%%%%%%%%%%%%%%%%%%%%%%%%%%%%%%%%%%%%%%%%%%%%%%%%%%%%%%%%%%%%%%%%%%%%%%%%%%%%%%%%%%%%%%%%%%%%%%%%%%%%%%%%%%%%%%%%%%%%%%%%%%%%%%%%%%%%%%%%%%%%%%
%%%%%%%%%%%%%%%%%%%%%%%%%%%%%%%%%%%%%%%%%%%%%%%%%%%%%%%%%%%%%%%%%%%%%%%%%%%%%%%%%%%%%%%%%%%%%%%%%%%%%%%%%%%%%%%%%%%%%%%%%%%%%%%%%%%%%%%%%%%%%%%%%%%%%%%%%%%%%%%%%%%%%%%%%%%%%%%%%%%%%%%%%%%%%%%%%%%%%%%%%%%%%%%%%%%%%%%%%%%%%%%%%%%%%%%%%%%%%%%
%%%%%%%%%%%%%%%%%%%%%%%%%%%%%%%%%%%%%%%%%%%%%%%%%%%%%%%%%%%%%%%%%%%%%%%%%%%%%%%%%%%%%%%%%%%%%%%%%%%%%%%%%%%%%%%%%%%%%%%%%%%%%%%%%%%%%%%%%%%%%%%%%%%%%%%%%%%%%%%%%%%%%%%%%%%%%%%%%%%%%%%%%%%%%%%%%%%%%%%%%%%%%%%%%%%%%%%%%%%%%%%%%%%%%%%%%%%%%%%
%%%%%%%%%%%%%%%%%%%%%%%%%%%%%%%%%%%%%%%%%%%%%%%%%%%%%%%%%%%%%%%%%%%%%%%%%%%%%%%%%%%%%%%%%%%%%%%%%%%%%%%%%%%%%%%%%%%%%%%%%%%%%%%%%%%%%%%%%%%%%%%%%%%%%%%%%%%%%%%%%%%%%%%%%%%%%%%%%%%%%%%%%%%%%%%%%%%%%%%%%%%%%%%%%%%%%%%%%%%%%%%%%%%%%%%%%%%%%%%
%%%%%%%%%%%%%%%%%%%%%%%%%%%%%%%%%%%%%%%%%%%%%%%%%%%%%%%%%%%%%%%%%%%%%%%%%%%%%%%%%%%%%%%%%%%%%%%%%%%%%%%%%%%%%%%%%%%%%%%%%%%%%%%%%%%%%%%%%%%%%%%%%%%%%%%%%%%%%%%%%%%%%%%%%%%%%%%%%%%%%%%%%%%%%%%%%%%%%%%%%%%%%%%%%%%%%%%%%%%%%%%%%%%%%%%%%%%%%%%
%%%%%%%%%%%%%%%%%%%%%%%%%%%%%%%%%%%%%%%%%%%%%%%%%%%%%%%%%%%%%%%%%%%%%%%%%%%%%%%%%%%%%%%%%%%%%%%%%%%%%%%%%%%%%%%%%%%%%%%%%%%%%%%%%%%%%%%%%%%%%%%%%%%%%%%%%%%%%%%%%%%%%%%%%%%%%%%%%%%%%%%%%%%%%%%%%%%%%%%%%%%%%%%%%%%%%%%%%%%%%%%%%%%%%%%%%%%%%%%
%%%%%%%%%%%%%%%%%%%%%%%%%%%%%%%%%%%%%%%%%%%%%%%%%%%%%%%%%%%%%%%%%%%%%%%%%%%%%%%%%%%%%%%%%%%%%%%%%%%%%%%%%%%%%%%%%%%%%%%%%%%%%%%%%%%%%%%%%%%%%%%%%%%%%%%%%%%%%%%%%%%%%%%%%%%%%%%%%%%%%%%%%%%%%%%%%%%%%%%%%%%%%%%%%%%%%%%%%%%%%%%%%%%%%%%%%%%%%%%
%%%%%%%%%%%%%%%%%%%%%%%%%%%%%%%%%%%%%%%%%%%%%%%%%%%%%%%%%%%%%%%%%%%%%%%%%%%%%%%%%%%%%%%%%%%%%%%%%%%%%%%%%%%%%%%%%%%%%%%%%%%%%%%%%%%%%%%%%%%%%%%%%%%%%%%%%%%%%%%%%%%%%%%%%%%%%%%%%%%%%%%%%%%%%%%%%%%%%%%%%%%%%%%%%%%%%%%%%%%%%%%%%%%%%%%%%%%%%%%
%%%%%%%%%%%%%%%%%%%%%%%%%%%%%%%%%%%%%%%%%%%%%%%%%%%%%%%%%%%%%%%%%%%%%%%%%%%%%%%%%%%%%%%%%%%%%%%%%%%%%%%%%%%%%%%%%%%%%%%%%%%%%%%%%%%%%%%%%%%%%%%%%%%%%%%%%%%%%%%%%%%%%%%%%%%%%%%%%%%%%%%%%%%%%%%%%%%%%%%%%%%%%%%%%%%%%%%%%%%%%%%%%%%%%%%%%%%%%%%

%%%%%%%%%%%%%%%%%%%%%%%%%%%%%%%%%%%%%%%%%%%%%%%%%%%%%%%%%%%%%%%%%%%%%%%%%%%%%%%
\clearpage
\section{Corrélats neuronaux ERPs dans le MI}
\label{etude2analysesERP}
%%%%%%%%%%%%%%%%%%%%%%%%%%%%%%%%%%%%%%%%%%%%%%%%%%%%%%%%%%%%%%%%%%%%%%%%%%%%%%%

Dans un premier temps, notre étude sur les corrélats neuronaux associés à la perception auditive consciente s'est portée sur l'analyse des composantes ERPs du signal EEG sous MI. 
Nous avons cherché à reproduire les résultats principaux de la littérature \citep{dykstra2016neural, giani2015detecting, gutschalk2008neural, wiegand2012correlates} pour vérifier que l'expérience donnait des résultats comparables à ceux déjà publiés. 
Les deux composantes ERPs du signal EEG --- ARN et P300 --- ont été étudiées comme une première approche des mécanismes de traitement de l'information au niveau cérébral. 

Dans leur étude, \cite{gutschalk2008neural} ont montré que la composante ARN observée au niveau du cortex auditif lors de la perception de la cible auditive dans le MI se présente comme un possible corrélat neuronal de la perception auditive consciente chez l'être humain. 
Cette forme d'onde négative présentait une amplitude plus grande lorsque les cibles étaient détectées par les sujets et était observée deux tonalités avant le report comportemental. 
Dans l'article de \cite{wiegand2012correlates}, une ARN n'a été observée que pour les cibles détectées mais a été clairement identifiée dans les deux hémisphères et pour chaque tonalité de cible, excepté pour la première. 
Au contraire, dans l'étude de \cite{giani2015detecting}, l'ARN n'est apparue que pour la deuxième des deux tonalités (paire de tonalités), c'est-à-dire une tonalité avant le report. 
Dans notre étude, nous avons utilisé une cible auditive composée de 10 tonalités et nous avons ainsi cherché à observer la présence et le moment d'apparition de la composante ARN. 

D'un autre coté, il a été montré que la composante P300, précédemment considérée comme un indicateur de traitement de l'intégration globale du stimulus \citep{dehaene2006conscious, del2007brain, sergent2004neural, sergent2005timing}, était plus grande et avait une latence plus courte pour les tonalités détectées avec un niveau de confiance élevé par rapport à celles avec un niveau de confiance faible \citep{parasuraman1980brain, paul1972evoked, squires1973vertex}. 
Dans leur étude, \cite{giani2015detecting} ont montré que la P300 était la seule composante amplifiée de manière significative pour les deux tonalités de la paire, quand elles étaient détectées, suggérant un lien important avec la conscience perceptive auditive. 
\cite{dykstra2016neural} ont trouvé une composante P300 pour les cibles détectées, révélée avec des générateurs dans les cortex temporo-frontal et temporo-latéral. 
En combinant MEEG et reports perceptifs essai par essai avec des estimations de sources, \cite{dykstra2017electrophysiological} ont trouvé une réponse robuste de type P300 avec des sources distribuées pour la deuxième tonalité cible détectée (précédant immédiatement les reports des sujets). 
Cette réponse était fortement diminuée, voire absente pour les cibles précédentes et suivantes du report des sujets. 
Ces résultats mettent en évidence les aspects tardifs et distribués de l'activité neuronale associés au traitement post-perceptuel lié à la tâche \citep{pitts2014gamma}. 
Néanmoins, ils ne permettent pas de conclure que cette activité sous-tend la perception consciente, en soi \citep{tsuchiya2015no}. 
Comme nous avons cherché à caractériser et indiquer la perception auditive consciente au travers du signal EEG, nous avons observé la présence de cette composante P300 autour de la perception du flux de tonalités cible. 

Sur la base de ces différents travaux, nous avons cherché à mettre en évidence, dans notre protocole expérimental de MI, des formes d’ondes de potentiels évoqués ARN et P300 lors de la perception de tonalités du signal cible autour du report des sujets.  
Nous avons cherché à savoir si les cibles détectées suscitaient des valeurs plus élevées de ces ERPs pour les tonalités du signal cible, avant et après l'appui-bouton par le sujet. 
En se basant notamment sur les études de \cite{gutschalk2008neural}, de \cite{wiegand2012correlates} et de \cite{giani2015detecting}, nous faisons l'hypothèse que les deux tonalités du signal cible situées avant le report des sujets seraient susceptibles de générer une plus grande composante négative ARN dans les zones auditives temporales ainsi qu'une plus grande composante positive P300 au niveau de la zone centro-sagittale (vertex). 
Enfin, nous avons également cherché à savoir si de telles composantes sont suscitées par les deux tonalités suivant le report perceptif du sujet. 

%%%%%%%%%%%%%%%%%%%%%%%%%%%%%%%%%%%%%%%%%%%%%%%%%%%%%%%%%%%%%%%%%%%%%%%%%%%%%%%
\subsection{Analyses}
\label{etude2analysesERPanalyses}
%%%%%%%%%%%%%%%%%%%%%%%%%%%%%%%%%%%%%%%%%%%%%%%%%%%%%%%%%%%%%%%%%%%%%%%%%%%%%%%

Nous avons considéré quatre tonalités autour d'une référence temporelle correspondant à l'appui-bouton par le sujet pour les cibles détectées et au temps moyen de la détection des sujets (\textit{i.e.}, 3,4 sec) pour les cibles non-détectées. 
Les quatre tonalités ont été notées "B2", "B1", "A1", "A2" en fonction de leur place dans le signal : "B2" pour la 2ème tonalité avant la référence (Before Second), "B1" pour la 1ère tonalité avant la référence (Before First), "A1" pour la 1ère tonalité après la référence (After First) et "A2" pour la 2ème tonalité après la référence (After Second). 

Les potentiels évoqués ont été récupérés sur la base de cette référence temporelle dans la fenêtre temporelle classique de $-250$ à $+500$~ms. 
Puis, nous avons obtenu les formes d'ondes grandes moyennes (\textit{i.e.}, les formes d'ondes moyennées sur tous les sujets) pour chaque électrode et pour chacune des tonalités. 
Nous les avons représentées sur des topographies d'activités moyennes pour chaque électrode et chaque tonalité (Figure~\ref{fig:figure5compareevokedsB2}, ~\ref{fig:figure5compareevokedsB1}, ~\ref{fig:figure5compareevokedsA1} et ~\ref{fig:figure5compareevokedsA2}). 
Ces quatre figures permettent ainsi de comparer les formes d'ondes directement pour les cibles détectées et celles non-détectées en fonction des différentes tonalités étudiées. 
Nous avons également représenté ces potentiels évoqués comme «image plot» entre $-200$ et $+500$ ms pour les cibles détectées et les cibles non-détectées les B2, B1, A1 et A2 (Figure~\ref{fig:figure5imageplot}). 

Toutes les statistiques ont été réalisées avec le logiciel R \citep{Rlanguage2021}. 
Les valeurs des pics d'amplitude en fonction de la composante étudiée ont été analysées à l'aide de modèles linéaire à effets mixtes avec la bibliothèque \texttt{lme} \citep{bates2007lme4}. 
Trois facteurs ont été utilisés : Détection (Hits/Miss), Tonalité (B2,B1,A1,A2) et Sujet. 
Les facteurs expérimentaux (Détection et Tonalité) ont été traités comme des variables à effet fixe. 
Tous les modèles incluaient le facteur Sujets Id. comme effet aléatoire pour le paramètre d'intercept du modèle. 
Nous avons donc cherché à étudier les effets des facteurs détection et tonalité sur les valeurs des pics d'amplitude (Amplitude $\sim$ Détection * Tonalité + $1|$Sujet) des composantes ARN et P300.  

Les modèles linéaires à effets mixtes ont été ajustés et estimés en utilisant les optimiseurs par maximisation de la vraisemblance. 
Les résultats de l'ANOVA réalisée sur le modèle sont présentés ainsi que les comparaisons mutliples le cas échéant. 
Dans le cas d'effets statistiques de facteurs et de leurs interactions, nous avons étudié les comparaisons appariées en utilisant les moyennes marginales estimées implémentées dans la bibliothèque R \texttt{emmeans}.
Les moyennes marginales estimées (moyennes des moindres carrés) sont les valeurs des paramètres du modèle moyennées pour les combinaisons adéquates des modalités de facteurs (voir Section~\ref{chapitre4analyses}). 

\begin{landscape}
\begin{figure*}[!t]
\centering
\includegraphics[width=0.8\linewidth]{Figures/illustrations/Exp_EEG/ERP/plots/compare_evokeds_B2.jpg}
\caption[Grandes moyennes comparées des ERPs (-200:+500) ms pour B2]{Grandes moyennes comparées des potentiels évoqués entre $-200$ et $+500$ ms pour B2 pour les cibles détectées (Hit, en rouge) et les cibles non-détectées (Miss, en bleu).}
\label{fig:figure5compareevokedsB2}
\end{figure*}
\end{landscape}

\begin{landscape}
\begin{figure*}[!t]
\centering
\includegraphics[width=0.8\linewidth]{Figures/illustrations/Exp_EEG/ERP/plots/compare_evokeds_B1.jpg}
\caption[Grandes moyennes comparées des ERPs (-200:+500) ms pour B1]{Grandes moyennes comparées des potentiels évoqués entre $-200$ et $+500$ ms pour B1 pour les cibles détectées (Hit, en rouge) et les cibles non-détectées (Miss, en bleu).}
\label{fig:figure5compareevokedsB1}
\end{figure*}
\end{landscape}

\begin{landscape}
\begin{figure*}[!t]
\centering
\includegraphics[width=0.8\linewidth]{Figures/illustrations/Exp_EEG/ERP/plots/compare_evokeds_A1.jpg}
\caption[Grandes moyennes comparées des ERP (-200:+500) ms pour A1]{Grandes moyennes comparées des potentiels évoqués entre $-200$ et $+500$ ms pour A1 pour les cibles détectées (Hit, en rouge) et les cibles non-détectées (Miss, en bleu).}
\label{fig:figure5compareevokedsA1}
\end{figure*}
\end{landscape}

\begin{landscape}
\begin{figure*}[!t]
\centering
\includegraphics[width=0.8\linewidth]{Figures/illustrations/Exp_EEG/ERP/plots/compare_evokeds_A2.jpg}
\caption[Grandes moyennes comparées des ERP (-200:+500) ms pour A2]{Grandes moyennes comparées des potentiels évoqués entre $-200$ et $+500$ ms pour A2 pour les cibles détectées (Hit, en rouge) et les cibles non-détectées (Miss, en bleu).}
\label{fig:figure5compareevokedsA2}
\end{figure*}
\end{landscape}

\begin{landscape}
\begin{figure*}[!t]
\centering
\begin{tabular}{cccc}
\footnotesize{\textbf{B2}} & \footnotesize{\textbf{B1}} & \footnotesize{\textbf{A1}} & \footnotesize{\textbf{A2}} \\
\includegraphics[width=14em, height=15em]{Figures/illustrations/Exp_EEG/ERP/plots/plot_image_hits_B2.jpg} & \includegraphics[width=14em, height=15em]{Figures/illustrations/Exp_EEG/ERP/plots/plot_image_hits_B1.jpg} & \includegraphics[width=14em, height=15em]{Figures/illustrations/Exp_EEG/ERP/plots/plot_image_hits_A1.jpg} & \includegraphics[width=14em, height=15em]{Figures/illustrations/Exp_EEG/ERP/plots/plot_image_hits_A2.jpg} \\ \\
\includegraphics[width=14em, height=15em]{Figures/illustrations/Exp_EEG/ERP/plots/plot_image_miss_B2.jpg} & \includegraphics[width=14em, height=15em]{Figures/illustrations/Exp_EEG/ERP/plots/plot_image_miss_B1.jpg} & \includegraphics[width=14em, height=15em]{Figures/illustrations/Exp_EEG/ERP/plots/plot_image_miss_A1.jpg} & \includegraphics[width=14em, height=15em]{Figures/illustrations/Exp_EEG/ERP/plots/plot_image_miss_A2.jpg}  \\
% \textbf{MISS~~~~~~~~~~~~~~~~~~~~~~~~~~~~~~~~~~~~~~~~~~~~~~~~~~~~~~~~~~~~~~~~~~~~~~~~~HITS}\par\medskip
% \includegraphics[width=14em, height=15em]{Figures/illustrations/Exp_EEG/ERP/plots/plot_image_hits_B2.jpg}
% \includegraphics[width=14em, height=15em]{Figures/illustrations/Exp_EEG/ERP/plots/plot_image_hits_B1.jpg}
% \includegraphics[width=14em, height=15em]{Figures/illustrations/Exp_EEG/ERP/plots/plot_image_hits_A1.jpg}
% \includegraphics[width=14em, height=15em]{Figures/illustrations/Exp_EEG/ERP/plots/plot_image_hits_A2.jpg} \\
% \includegraphics[width=14em, height=15em]{Figures/illustrations/Exp_EEG/ERP/plots/plot_image_miss_B2.jpg}
% \includegraphics[width=14em, height=15em]{Figures/illustrations/Exp_EEG/ERP/plots/plot_image_miss_B1.jpg}
% \includegraphics[width=14em, height=15em]{Figures/illustrations/Exp_EEG/ERP/plots/plot_image_miss_A1.jpg}
% \includegraphics[width=14em, height=15em]{Figures/illustrations/Exp_EEG/ERP/plots/plot_image_miss_A2.jpg} 
\end{tabular}
\caption[Image plot des ERP hits/miss entre $-200$:$+500$~ms pour B2, B1, A1 et A2]{Potentiels évoqués entre $-200$ et $+500$~ms (abscisse) sur toutes les électrodes (ordonnée) pour les cibles détectées (panel du haut) et les cibles non-détectées (panel du bas) pour les quatre tonalités (de gauche à droite : B2, B1, A1 et A2). L'échelle d'amplitude des potentiels évoqués est située entre $-1.5$ (bleu) et $+1.5$ (rouge) $\mu Volt$.}
\label{fig:figure5imageplot}
\end{figure*}
\end{landscape}

%%%%%%%%%%%%%%%%%%%%%%%%%%%%%%%%%%%%%%%%%%%%%%%%%%%%%%%%%%%%%%%%%%%%%%%%%%%%%%%
\subsection{Résultats}
\label{etude2analysesERPresultats}
%%%%%%%%%%%%%%%%%%%%%%%%%%%%%%%%%%%%%%%%%%%%%%%%%%%%%%%%%%%%%%%%%%%%%%%%%%%%%%%

On observe sur les topographies d'activités moyennes (Figure~\ref{fig:figure5compareevokedsB1}) que la première tonalité avant l'appui-bouton (B1) présente de nombreuses électrodes qui affichent de fortes variations du potentiel pour les cibles détectées (en rouge) comparativement à celles non-détectées (en bleu). 
On observe également sur la Figure~\ref{fig:figure5imageplot}, pour les tonalités B1 et A1, que ces variations se caractérisent par la présence de composantes positives et négatives entre $100$ et $400$~ms. 

La composante ARN a principalement été localisée dans le cortex auditif (primaire et secondaire).
Nous avons donc cherché sa présence au niveau des électrodes du cortex auditif. 
Sur la procédure d'agrégation topographique des électrodes (Figure~\ref{fig:figure5systemeegclustering} Droite), l'aire temporale gauche est localisée au niveau des électrodes FT7, T7 et TP7 et l'aire temporale droite au niveau des électrodes FT8, T8 et TP8. 

Contrairement à ce qui est attendu, nous observons sur les Figures~\ref{fig:figure5compareevokedsB2} et ~\ref{fig:figure5compareevokedsB1} qu'aucune de ces électrodes ne présente de large déflexion négative pour les cibles détectées sur ces deux tonalités B2 et B1. 
On voit que l'électrode FT8 montre seulement une faible diminution d'amplitude pour les cibles détectées aux environs de 250 ms. 
Cependant, des formes d'ondes négatives sont observées sur les électrodes F5, F6, F7, F8, C5 et CP4.  
Comme F6, F7 et C5 sont localisées de manière adjacentes aux électrodes de la zone temporale, nous les avons inclu dans les analyses des composantes ARN.
De cette manière, pour la composante ARN, notre recherche a inclut les électrodes C5, F6, F7, FT7, FT8, T7, T8, TP7 et TP8. 

Pour la composante P300, nous avons recherché les formes d'ondes localisées au niveau du vertex, c'est-à-dire principalement les électrodes sagitto-centrales (FCz, Cz, CPz et Pz). 
On observe sur la Figure~\ref{fig:figure5compareevokedsB1} (tonalité B1) que les électrodes FCz, Cz et Pz présentent une signature P300 avec une transition forte lorsque les cibles ont été détectées. 
Cette forme d'onde n'est cependant pas observée pour les cibles détectées sur les autres tonalités B2, A1 et A2 (Figures~\ref{fig:figure5compareevokedsB2}, ~\ref{fig:figure5compareevokedsA1} et ~\ref{fig:figure5compareevokedsA2}). 

Le pic d'amplitude de chaque composante ERP a été identifié visuellement dans les grandes moyennes ERPs pour les différentes tonalités des cibles détectées et non-détectées. 
Nous avons analysé statistiquement les ERPs correspondants (\textit{i.e.}, ARN et P300) pour chacune des électrodes d'intérêt issues de la littérature. 
Pour chaque composante ERP, nous avons considéré une fenêtre temporelle spécifique dans laquelle nous avons sélectionné le pic d'amplitude maximal/minimal correspondant à l'ERP. 

%%%%%%%%%%%%%%%%%%%%%%%%%%%%%%%%%%%%%%%%%%%%%%%%%%%%%%%%%%%%%%%%%%%%%%%%%%%%%%%
\newpage
\subsubsection{Composante ARN}
%%%%%%%%%%%%%%%%%%%%%%%%%%%%%%%%%%%%%%%%%%%%%%%%%%%%%%%%%%%%%%%%%%%%%%%%%%%%%%%

Les valeurs d'amplitude de la composante ARN ont été sélectionnées pour les électrodes d'intérêts (FT7, FT8, T7, T8, TP7, TP8, C5, F6 et F7) dans la fenêtre temporelle $50$-$350$~ms pour les quatre tonalités (B2, B1, A1, A2). 
Les valeurs d'amplitudes négatives minimales ont été trouvées pour les électrodes C5, F6 et F7 pour la première tonalité avant l'appui-bouton (B1) pour les cibles détectées. 
La Figure~\ref{fig:figure5ARNwaveforms} montre les formes d'ondes évoquées dans le temps pour la composante ARN pour les électrodes C5, F6 et F7 pour les quatre tonalités cibles (B2, B1, A1, A2). 
On observe une différence d'amplitude pour la première tonalité avant la détection (B1) pour les cibles détectées (hits, en rouge) et celles non-détectées (miss, en bleu) dans la fenêtre $50$-$350$~ms, pour les électrodes C5, F6 et F7. 
On observe également une large déflexion négative pour cette même tonalité (B1) lors de la détection de la cible à quasi $100$~ms. 
La déflexion négative de l'ARN a précédemment été localisée aux alentours de $250$~ms \citep{gutschalk2008neural}. 
Par conséquent, les pics d'amplitude ont été obtenus par sélection dans la fenêtre temporelle spécifique de $250$ à $350$ ms et l'analyse statistique s'est portée sur les électrodes C5, F6 et F7 dans cette fenêtre. 

Ainsi, pour chaque électrode, un modèle linéaire à effets mixtes a été ajusté pour étudier l'effet des facteurs détection et tonalité sur les valeurs des pics d'amplitude (Amplitude $\sim$ Détection * Tonalité + ~$1|$Sujet). 
La Table~\ref{tab:table5statsARN} reporte les résultats des analyses statistiques du modèle linéaire mixte réalisé sur les valeurs des pics d'amplitude de l'ARN pour les électrodes C5 (table de gauche), F6 (table du milieu) et F7 (table de droite) pour les deux tonalités du signal cible avant et après l'appui-bouton pour les cibles détectées et pour celles avant et après le temps moyen de la détection pour les cibles non-détectées. \\

\underline{Pour l'électrode C5}, aucun effet significatif n'a été montré, ni pour la détection ($F(1,81)=0.69$, $p=0.41$), ni pour la tonalité ($F(3,81)=1.76$, $p=0.16$).
Un effet significatif a été trouvé pour l'interaction entre la tonalité et la détection ($F(3,81)=4.54$, $p=.01$). 
Les tests post-hoc ont montré que les valeurs d'amplitude sont significativement inférieures pour la tonalité B1 lorsque les cibles ont été détectées comparativement aux cibles manquées (Table~\ref{tab:table5statsARN}). 
La comparaison post-hoc a également indiqué que les valeurs d'amplitude pour la tonalité B1 sont significativement inférieures à celles de la tonalité B2 pour les cibles détectées. 

\underline{Pour l'électrode F6}, aucun effet significatif n'a été montré pour la détection ($F(1,95)=3.47$, $p=0.07$). 
Un effet significatif a été trouvé pour la tonalité ($F(3,95)=3.81$, $p=0.01$) et pour l'interaction entre la tonalité et la détection ($F(3,95)=8.12$, $p<.001$). 
Les tests post-hoc ont montré que les valeurs d'amplitude sont significativement inférieures pour la tonalité B1 lorsque les cibles ont été détectées comparativement aux cibles manquées (Table~\ref{tab:table5statsARN}). 
Les tests post-hoc ont aussi indiqué que les valeurs d'amplitude pour la tonalité B1 sont significativement inférieures à celles de la tonalité B2 pour les cibles détectées. 

\underline{Pour l'électrode F7}, aucun effet significatif n'a été montré pour la détection ($F(1,90)=2.74$, $p=0.1$). 
Un effet significatif a été trouvé pour la tonalité ($F(3,90)=2.99$, $p=0.04$) et pour l'interaction entre la tonalité et la détection ($F(3,90)=5.05$, $p<.001$). 
Les tests post-hoc ont montré que les valeurs d'amplitude sont significativement moindres pour la tonalité B1 lorsque les cibles ont été manquées comparativement aux cibles détectées (Table~\ref{tab:table5statsARN}). 
Les tests ont également indiqué que les valeurs d'amplitude sont significativement inférieures pour les cibles détectées pour la tonalité B1 comparativement aux tonalité B2 et A1.   

\begin{landscape}
\begin{table}
\centering
\tiny
\caption[Table des résultats des analyses statistiques de l'ARN pour les électrodes C5, F6 et F7]{Table des résultats des analyses statistiques sur les valeurs des pics d'amplitude de l'ARN pour les électrodes C5, F6 et F7 pour les quatre tonalités (B2, B1, A1 et A2). E correspond aux estimations des paramètres et SE à leurs erreurs standards.}
\label{tab:table5statsARN}
\begin{multicols}{3}
% \begin{tabular}{lllll}
\begin{tabular}{|l||*{5}{c|}}
\hline
\textbf{ARN - C5} & & & & \\
\hline
\textcolor{blue}{ANOVA Modèle} & numDF & denDF & $F$ & $p$ \\ 
\hline
(Intercept) & 1 & 81 & 65.06 & 0.00 * \\ 
Détection & 1 & 81 & 0.69 & 0.41 \\ 
Tonalité & 3 & 81 & 1.76 & 0.16 \\ 
\textit{Détection:Tonalité} & 3 & 81 & 4.54 & 0.01 *\\ 
\hline
\textcolor{blue}{Détection} & E & SE & $z$ & \\ 
\hline
Miss - Hit & 0.16 & 0.20 & 0.80 & 0.42 \\ 
\hline
\textcolor{blue}{Tonalité} & & & & \\ 
\hline
B1 - B2 & -0.58 & 0.28 & -2.05 & 0.24 \\ 
A1 - B2 & -0.16 & 0.28 & -0.58 & 1.00 \\ 
A2 - B2 & -0.07 & 0.29 & -0.23 & 1.00 \\ 
A1 - B1 & 0.42 & 0.27 & 1.55 & 0.73 \\ 
A2 - B1 & 0.51 & 0.29 & 1.78 & 0.45 \\ 
A2 - A1 & 0.09 & 0.28 & 0.33 & 1.00 \\ 
\hline
\textcolor{blue}{Détection*Tonalité} & & & & \\ 
\hline
Miss.B2 - Hit.B2 & -0.58 & 0.38 & -1.54 & 1.00 \\ 
\textit{Hit.B1 - Hit.B2} & -1.43 & 0.36 & -3.97 & 0.00 * \\ 
Miss.B1 - Hit.B2 & -0.12 & 0.39 & -0.32 & 1.00 \\ 
Hit.A1 - Hit.B2 & -0.45 & 0.36 & -1.27 & 1.00 \\ 
Miss.A1 - Hit.B2 & -0.49 & 0.37 & -1.31 & 1.00 \\ 
Hit.A2 - Hit.B2 & -0.24 & 0.41 & -0.59 & 1.00 \\ 
Miss.A2 - Hit.B2 & -0.47 & 0.37 & -1.27 & 1.00 \\ 
Hit.B1 - Miss.B2 & -0.85 & 0.35 & -2.41 & 0.45 \\ 
Miss.B1 - Miss.B2 & 0.46 & 0.39 & 1.18 & 1.00 \\ 
Hit.A1 - Miss.B2 & 0.13 & 0.35 & 0.37 & 1.00 \\ 
Miss.A1 - Miss.B2 & 0.09 & 0.37 & 0.25 & 1.00 \\ 
Hit.A2 - Miss.B2 & 0.34 & 0.40 & 0.86 & 1.00 \\ 
Miss.A2 - Miss.B2 & 0.11 & 0.36 & 0.32 & 1.00 \\ 
\textit{Miss.B1 - Hit.B1} & 1.31 & 0.37 & 3.51 & 0.01 * \\ 
Hit.A1 - Hit.B1 & 0.98 & 0.33 & 2.94 & 0.09 . \\ 
Miss.A1 - Hit.B1 & 0.94 & 0.35 & 2.68 & 0.21 \\ 
Hit.A2 - Hit.B1 & 1.19 & 0.39 & 3.10 & 0.05 . \\ 
Miss.A2 - Hit.B1 & 0.97 & 0.35 & 2.79 & 0.15 \\ 
Hit.A1 - Miss.B1 & -0.33 & 0.37 & -0.89 & 1.00 \\ 
Miss.A1 - Miss.B1 & -0.37 & 0.39 & -0.94 & 1.00 \\ 
Hit.A2 - Miss.B1 & -0.12 & 0.42 & -0.28 & 1.00 \\ 
Miss.A2 - Miss.B1 & -0.34 & 0.38 & -0.90 & 1.00 \\ 
Miss.A1 - Hit.A1 & -0.04 & 0.35 & -0.11 & 1.00 \\ 
Hit.A2 - Hit.A1 & 0.21 & 0.38 & 0.56 & 1.00 \\ 
Miss.A2 - Hit.A1 & -0.01 & 0.34 & -0.04 & 1.00 \\ 
Hit.A2 - Miss.A1 & 0.25 & 0.40 & 0.63 & 1.00 \\ 
Miss.A2 - Miss.A1 & 0.02 & 0.36 & 0.07 & 1.00 \\ 
Miss.A2 - Hit.A2 & -0.22 & 0.39 & -0.58 & 1.00 \\ 
\hline
\end{tabular}

% \begin{tabular}{lllll}
\begin{tabular}{|l||*{5}{c|}}
\hline
\textbf{ARN - F6} & & & & \\
\hline
\textcolor{blue}{ANOVA Modèle} & numDF & denDF & $F$ & $p$ \\ 
\hline
(Intercept) & 1 & 95 & 70.55 & 0.00 \\ 
Detection & 1 & 95 & 3.47 & 0.07 .\\ 
\textit{Tonalité} & 3 & 95 & 3.81 & 0.01 * \\ 
\textit{Detection:Tonalité} & 3 & 95 & 8.12 & 0.00 * \\ 
\hline
\textcolor{blue}{Détection} & E & SE & z & \\ 
\hline
Miss - Hit & 0.31 & 0.19 & 1.63 & 0.10 \\ 
\hline
\textcolor{blue}{Tonalité} & & & & \\ 
\hline
\textit{B1 - B2} & -0.72 & 0.27 & -2.69 & 0.04 * \\ 
A1 - B2 & -0.13 & 0.26 & -0.48 & 1.00 \\ 
A2 - B2 & -0.03 & 0.27 & -0.10 & 1.00 \\ 
A1 - B1 & 0.59 & 0.26 & 2.28 & 0.14 \\ 
A2 - B1 & 0.69 & 0.26 & 2.62 & 0.05 \\ 
A2 - A1 & 0.10 & 0.26 & 0.38 & 1.00 \\ 
\hline
\textcolor{blue}{Détection*Tonalité} & & & & \\ 
\hline
Miss.B2 - Hit.B2 & -0.02 & 0.34 & -0.07 & 1.00 \\ 
\textit{Hit.B1 - Hit.B2} & -1.37 & 0.32 & -4.20 & 0.00 * \\ 
Miss.B1 - Hit.B2 & 0.04 & 0.34 & 0.12 & 1.00 \\ 
Hit.A1 - Hit.B2 & -0.43 & 0.32 & -1.31 & 1.00 \\ 
Miss.A1 - Hit.B2 & 0.20 & 0.33 & 0.61 & 1.00 \\ 
Hit.A2 - Hit.B2 & 0.40 & 0.33 & 1.20 & 1.00 \\ 
Miss.A2 - Hit.B2 & -0.47 & 0.34 & -1.41 & 1.00 \\ 
\textit{Hit.B1 - Miss.B2} & -1.34 & 0.32 & -4.13 & 0.00 * \\ 
Miss.B1 - Miss.B2 & 0.06 & 0.34 & 0.19 & 1.00 \\ 
Hit.A1 - Miss.B2 & -0.40 & 0.32 & -1.24 & 1.00 \\ 
Miss.A1 - Miss.B2 & 0.22 & 0.33 & 0.68 & 1.00 \\ 
Hit.A2 - Miss.B2 & 0.42 & 0.33 & 1.27 & 1.00 \\ 
Miss.A2 - Miss.B2 & -0.45 & 0.34 & -1.33 & 1.00 \\ 
\textit{Miss.B1 - Hit.B1} & 1.41 & 0.32 & 4.33 & 0.00 * \\ 
Hit.A1 - Hit.B1 & 0.94 & 0.31 & 3.00 & 0.07 . \\ 
\textit{Miss.A1 - Hit.B1} & 1.57 & 0.32 & 4.92 & 0.00 * \\ 
\textit{Hit.A2 - Hit.B1} & 1.76 & 0.32 & 5.53 & 0.00 * \\ 
Miss.A2 - Hit.B1 & 0.89 & 0.33 & 2.74 & 0.17 \\ 
Hit.A1 - Miss.B1 & -0.47 & 0.32 & -1.43 & 1.00 \\ 
Miss.A1 - Miss.B1 & 0.16 & 0.33 & 0.49 & 1.00 \\ 
Hit.A2 - Miss.B1 & 0.36 & 0.33 & 1.08 & 1.00 \\ 
Miss.A2 - Miss.B1 & -0.51 & 0.34 & -1.53 & 1.00 \\ 
Miss.A1 - Hit.A1 & 0.63 & 0.32 & 1.97 & 1.00 \\ 
Hit.A2 - Hit.A1 & 0.82 & 0.32 & 2.58 & 0.28 \\ 
Miss.A2 - Hit.A1 & -0.05 & 0.33 & -0.15 & 1.00 \\ 
Hit.A2 - Miss.A1 & 0.20 & 0.32 & 0.60 & 1.00 \\ 
Miss.A2 - Miss.A1 & -0.67 & 0.33 & -2.04 & 1.00 \\ 
Miss.A2 - Hit.A2 & -0.87 & 0.33 & -2.63 & 0.24 \\ 
\hline
\end{tabular}

% \begin{tabular}{lllll}
\begin{tabular}{|l||*{5}{c|}}
\hline
\textbf{ARN - F7} & & & & \\
\hline
\textcolor{blue}{ANOVA Modèle} & numDF & denDF & $F$ & $p$ \\ 
\hline
(Intercept) & 1 & 90 & 161.45 & 0.00 * \\ 
Détection & 1 & 90 & 2.74 & 0.10 \\ 
\textit{Tonalité} & 3 & 90 & 2.99 & 0.04 * \\
\textit{Détection:Tonalité} & 3 & 90 & 5.05 & 0.00 * \\
\hline
\textcolor{blue}{Détection} & E & SE & $z$ & \\
\hline
Miss - Hit & 0.32 & 0.20 & 1.54 & 0.12 \\
\hline
\textcolor{blue}{Tonalité} & & & & \\
\hline
B1 - B2 & -0.74 & 0.28 & -2.61 & 0.05 . \\
A1 - B2 & -0.33 & 0.28 & -1.19 & 1.00 \\
A2 - B2 & -0.03 & 0.29 & -0.12 & 1.00 \\
A1 - B1 & 0.41 & 0.27 & 1.50 & 0.80 \\
A2 - B1 & 0.71 & 0.28 & 2.48 & 0.08 . \\
A2 - A1 & 0.29 & 0.28 & 1.06 & 1.00 \\
\hline
\textcolor{blue}{Détection*Tonalité} & & & & \\
\hline
Miss.B2 - Hit.B2 & -0.24 & 0.38 & -0.65 & 1.00 \\ 
\textit{Hit.B1 - Hit.B2} & -1.58 & 0.37 & -4.29 & 0.00 * \\ 
Miss.B1 - Hit.B2 & -0.03 & 0.38 & -0.09 & 1.00 \\ 
Hit.A1 - Hit.B2 & -0.48 & 0.36 & -1.33 & 1.00 \\ 
Miss.A1 - Hit.B2 & -0.43 & 0.37 & -1.16 & 1.00 \\ 
Hit.A2 - Hit.B2 & -0.01 & 0.39 & -0.02 & 1.00 \\ 
Miss.A2 - Hit.B2 & -0.28 & 0.37 & -0.76 & 1.00 \\ 
\textit{Hit.B1 - Miss.B2} & -1.34 & 0.36 & -3.71 & 0.01 * \\ 
Miss.B1 - Miss.B2 & 0.21 & 0.38 & 0.56 & 1.00 \\ 
Hit.A1 - Miss.B2 & -0.24 & 0.36 & -0.67 & 1.00 \\ 
Miss.A1 - Miss.B2 & -0.18 & 0.36 & -0.50 & 1.00 \\ 
Hit.A2 - Miss.B2 & 0.24 & 0.38 & 0.61 & 1.00 \\ 
Miss.A2 - Miss.B2 & -0.03 & 0.36 & -0.10 & 1.00 \\ 
\textit{Miss.B1 - Hit.B1} & 1.55 & 0.37 & 4.20 & 0.00 * \\ 
\textit{Hit.A1 - Hit.B1} & 1.10 & 0.35 & 3.16 & 0.04 * \\ 
\textit{Miss.A1 - Hit.B1} & 1.16 & 0.35 & 3.27 & 0.03 * \\ 
\textit{Hit.A2 - Hit.B1} & 1.58 & 0.38 & 4.17 & 0.00 * \\ 
\textit{Miss.A2 - Hit.B1} & 1.31 & 0.35 & 3.68 & 0.01 * \\ 
Hit.A1 - Miss.B1 & -0.45 & 0.36 & -1.23 & 1.00 \\ 
Miss.A1 - Miss.B1 & -0.39 & 0.37 & -1.06 & 1.00 \\ 
Hit.A2 - Miss.B1 & 0.03 & 0.39 & 0.06 & 1.00 \\ 
Miss.A2 - Miss.B1 & -0.24 & 0.37 & -0.66 & 1.00 \\ 
Miss.A1 - Hit.A1 & 0.06 & 0.35 & 0.16 & 1.00 \\ 
Hit.A2 - Hit.A1 & 0.47 & 0.37 & 1.27 & 1.00 \\ 
Miss.A2 - Hit.A1 & 0.20 & 0.35 & 0.58 & 1.00 \\ 
Hit.A2 - Miss.A1 & 0.42 & 0.38 & 1.11 & 1.00 \\ 
Miss.A2 - Miss.A1 & 0.15 & 0.35 & 0.42 & 1.00 \\ 
Miss.A2 - Hit.A2 & -0.27 & 0.38 & -0.72 & 1.00 \\ 
\hline
\end{tabular}
\end{multicols}
\end{table}
\end{landscape}

\begin{figure*}[!t]
\centering
\includegraphics[width=0.9\linewidth, height=0.23\textheight]{Figures/illustrations/Exp_EEG/ERP/ARN_waveform_true_tones_C5.jpeg}
\includegraphics[width=0.9\linewidth, height=0.23\textheight]{Figures/illustrations/Exp_EEG/ERP/ARN_waveform_true_tones_F6.jpeg}
\includegraphics[width=0.9\linewidth, height=0.23\textheight]{Figures/illustrations/Exp_EEG/ERP/ARN_waveform_true_tones_F7.jpeg}
\caption[Formes d'ondes évoquées de l'ARN sur C5, F6 et F7 entre $50$ et $350$ ms]{Formes d'ondes évoquées de l'ARN sur les électrodes C5 (panel du haut), F6 (panel du milieu) et F7 (panel du bas) pour la première et la seconde tonalité avant (B1 et B2, respectivement) et après (A1 et A2, respectivement) la détection (appui-bouton) et non-détection (temps moyen de détection) de la cible entre $50$ et $350$ ms.}
\label{fig:figure5ARNwaveforms}
\end{figure*}


\begin{figure*}[!t]
\centering
\includegraphics[width=0.9\linewidth, height=0.22\textheight]{Figures/illustrations/Exp_EEG/ERP/P300_waveform_true_tones_FCz.jpeg}
\includegraphics[width=0.9\linewidth, height=0.22\textheight]{Figures/illustrations/Exp_EEG/ERP/P300_waveform_true_tones_Cz.jpeg}
\includegraphics[width=0.9\linewidth, height=0.22\textheight]{Figures/illustrations/Exp_EEG/ERP/P300_waveform_true_tones_CPz.jpeg}
\includegraphics[width=0.9\linewidth, height=0.22\textheight]{Figures/illustrations/Exp_EEG/ERP/P300_waveform_true_tones_Pz.jpeg}
\caption[Formes d'ondes évoquées de la P300 sur FCz, Cz, CPz et Pz entre $250$ et $500$ ms]{Formes d'ondes évoquées de la P300 sur les électrodes FCz (premier panel), Cz (deuxième panel), CPz (troisième panel) et Pz (quatrième panel) pour les quatre tonalités (B2, B1, A1 et A2) entre $250$ et $500$ ms.}
\label{fig:figure5amplitudeerpelectrodesP300waveform}
\end{figure*}

\begin{landscape}
\begin{table}
\centering
\tiny
\caption[Table des résultats des analyses statistiques de la P300 pour les électrodes FCz, Cz, CPz et Pz]{Table des résultats des analyses statistiques sur les valeurs des pics d'amplitude de la P300 pour les électrodes FCz, Cz, CPz et Pz pour les quatre tonalités (B2, B1, A1 et A2). E correspond aux estimations des paramètres et SE à leurs erreurs standards.}
\label{tab:table5statsP300}
\begin{multicols}{4}
% \begin{tabular}{lllll}
\begin{tabular}{|l|*{5}{c|}}
\hline
\textbf{P300 - FCz} & & & & \\
\hline
\textcolor{blue}{ANOVA Modèle} & numDF & denDF & $F$ & $p$ \\ 
\hline
(Intercept) & 1 & 101.00 & 248.71 & 0.00 * \\ 
Detection & 1 & 101.00 & 0.14 & 0.71 \\ 
\textit{Tonalité} & 3 & 101.00 & 10.30 & 0.00 * \\ 
\textit{Detection:Tonalité} & 3 & 101.00 & 7.14 & 0.00 * \\ 
\hline
\textcolor{blue}{Détection} & E & SE & $z$ & $p$ \\ 
\hline
Miss - Hit & -0.03 & 0.10 & -0.32 & 0.75 \\ 
\hline
\textcolor{blue}{Tonalité} & & & & \\ 
\hline
\textit{B1 - B2} & 0.46 & 0.13 & 3.55 & 0.00 * \\ 
A1 - B2 & 0.03 & 0.13 & 0.25 & 1.00 \\ 
A2 - B2 & -0.19 & 0.13 & -1.55 & 0.72 \\ 
\textit{A1 - B1} & -0.43 & 0.13 & -3.34 & 0.01 * \\ 
\textit{A2 - B1} & -0.65 & 0.13 & -5.11 & 0.00 * \\ 
A2 - A1 & -0.23 & 0.12 & -1.81 & 0.42 \\ 
\hline
\textcolor{blue}{Détection*Tonalité} & & & & \\ 
\hline
Miss.B2 - Hit.B2 & 0.13 & 0.16 & 0.80 & 1.00 \\ 
\textit{Hit.B1 - Hit.B2} & 0.82 & 0.16 & 5.04 & 0.00 * \\ 
Miss.B1 - Hit.B2 & 0.16 & 0.17 & 0.91 & 1.00 \\ 
Hit.A1 - Hit.B2 & 0.09 & 0.16 & 0.58 & 1.00 \\ 
Miss.A1 - Hit.B2 & 0.10 & 0.16 & 0.63 & 1.00 \\ 
Hit.A2 - Hit.B2 & -0.34 & 0.16 & -2.07 & 1.00 \\ 
Miss.A2 - Hit.B2 & 0.08 & 0.16 & 0.51 & 1.00 \\ 
\textit{Hit.B1 - Miss.B2} & 0.69 & 0.16 & 4.32 & 0.00 * \\ 
Miss.B1 - Miss.B2 & 0.03 & 0.17 & 0.15 & 1.00 \\ 
Hit.A1 - Miss.B2 & -0.04 & 0.16 & -0.22 & 1.00 \\ 
Miss.A1 - Miss.B2 & -0.03 & 0.16 & -0.17 & 1.00 \\ 
Hit.A2 - Miss.B2 & -0.47 & 0.16 & -2.92 & 0.10 \\ 
Miss.A2 - Miss.B2 & -0.05 & 0.16 & -0.30 & 1.00 \\ 
\textit{Miss.B1 - Hit.B1} & -0.66 & 0.17 & -3.92 & 0.00 * \\ 
\textit{Hit.A1 - Hit.B1} & -0.72 & 0.16 & -4.54 & 0.00 * \\ 
\textit{Miss.A1 - Hit.B1} & -0.72 & 0.16 & -4.49 & 0.00 * \\ 
\textit{Hit.A2 - Hit.B1} & -1.16 & 0.16 & -7.24 & 0.00 * \\ 
\textit{Miss.A2 - Hit.B1} & -0.74 & 0.16 & -4.62 & 0.00 * \\ 
Hit.A1 - Miss.B1 & -0.06 & 0.17 & -0.36 & 1.00 \\ 
Miss.A1 - Miss.B1 & -0.05 & 0.17 & -0.32 & 1.00 \\ 
Hit.A2 - Miss.B1 & -0.49 & 0.17 & -2.91 & 0.10 \\ 
Miss.A2 - Miss.B1 & -0.07 & 0.17 & -0.43 & 1.00 \\ 
Miss.A1 - Hit.A1 & 0.01 & 0.16 & 0.05 & 1.00 \\ 
Hit.A2 - Hit.A1 & -0.43 & 0.16 & -2.70 & 0.19 \\ 
Miss.A2 - Hit.A1 & -0.01 & 0.16 & -0.07 & 1.00 \\ 
Hit.A2 - Miss.A1 & -0.44 & 0.16 & -2.75 & 0.17 \\ 
Miss.A2 - Miss.A1 & -0.02 & 0.16 & -0.12 & 1.00 \\ 
Miss.A2 - Hit.A2 & 0.42 & 0.16 & 2.63 & 0.24 \\ 
\hline
\end{tabular}

% \begin{tabular}{lllll}
\begin{tabular}{|l|*{5}{c|}}
\hline
\textbf{P300 - Cz} & & & & \\
\hline
\textcolor{blue}{ANOVA Modèle} & numDF & denDF & $F$ & $p$ \\ 
\hline
(Intercept) & 1 & 102.00 & 214.68 & 0.00 * \\ 
Detection & 1 & 102.00 & 0.35 & 0.55 \\ 
\textit{Tonalité} & 3 & 102.00 & 8.02 & 0.00 * \\ 
\textit{Detection:Tonalité} & 3 & 102.00 & 8.06 & 0.00 * \\ 
\hline
\textcolor{blue}{Détection} & E & SE & $z$ & $p$ \\
\hline
Miss - Hit & -0.06 & 0.11 & -0.54 & 0.59 \\ 
\hline
\textcolor{blue}{Tonalité} & & & & \\
\hline
\textit{B1 - B2} & 0.53 & 0.15 & 3.68 & 0.00 * \\ 
A1 - B2 & 0.22 & 0.14 & 1.53 & 0.76 \\ 
A2 - B2 & -0.07 & 0.15 & -0.50 & 1.00 \\ 
A1 - B1 & -0.31 & 0.15 & -2.16 & 0.18 \\ 
\textit{A2 - B1} & -0.61 & 0.15 & -4.11 & 0.00 * \\ 
A2 - A1 & -0.29 & 0.15 & -2.00 & 0.27 \\ 
\hline
\textcolor{blue}{Détection*Tonalité} & & & & \\
\hline
Miss.B2 - Hit.B2 & 0.06 & 0.18 & 0.33 & 1.00 \\ 
\textit{Hit.B1 - Hit.B2} & 0.96 & 0.18 & 5.26 & 0.00 * \\ 
Miss.B1 - Hit.B2 & 0.14 & 0.19 & 0.76 & 1.00 \\ 
Hit.A1 - Hit.B2 & 0.18 & 0.18 & 0.99 & 1.00 \\ 
Miss.A1 - Hit.B2 & 0.32 & 0.18 & 1.75 & 1.00 \\ 
Hit.A2 - Hit.B2 & -0.29 & 0.19 & -1.54 & 1.00 \\ 
Miss.A2 - Hit.B2 & 0.17 & 0.18 & 0.92 & 1.00 \\ 
\textit{Hit.B1 - Miss.B2} & 0.90 & 0.18 & 4.93 & 0.00 * \\ 
Miss.B1 - Miss.B2 & 0.08 & 0.19 & 0.43 & 1.00 \\ 
Hit.A1 - Miss.B2 & 0.12 & 0.18 & 0.66 & 1.00 \\ 
Miss.A1 - Miss.B2 & 0.26 & 0.18 & 1.42 & 1.00 \\ 
Hit.A2 - Miss.B2 & -0.35 & 0.19 & -1.86 & 1.00 \\ 
Miss.A2 - Miss.B2 & 0.11 & 0.18 & 0.59 & 1.00 \\ 
\textit{Miss.B1 - Hit.B1} & -0.82 & 0.19 & -4.41 & 0.00 * \\ 
\textit{Hit.A1 - Hit.B1} & -0.78 & 0.18 & -4.27 & 0.00 * \\ 
\textit{Miss.A1 - Hit.B1} & -0.64 & 0.18 & -3.51 & 0.01 * \\ 
\textit{Hit.A2 - Hit.B1}& -1.25 & 0.19 & -6.61 & 0.00 * \\ 
\textit{Miss.A2 - Hit.B1} & -0.79 & 0.18 & -4.33 & 0.00 * \\ 
Hit.A1 - Miss.B1 & 0.04 & 0.19 & 0.21 & 1.00 \\ 
Miss.A1 - Miss.B1 & 0.18 & 0.19 & 0.96 & 1.00 \\ 
Hit.A2 - Miss.B1 & -0.43 & 0.19 & -2.25 & 0.69 \\ 
Miss.A2 - Miss.B1 & 0.03 & 0.19 & 0.15 & 1.00 \\ 
Miss.A1 - Hit.A1 & 0.14 & 0.18 & 0.76 & 1.00 \\ 
Hit.A2 - Hit.A1 & -0.47 & 0.19 & -2.50 & 0.35 \\ 
Miss.A2 - Hit.A1 & -0.01 & 0.18 & -0.06 & 1.00 \\ 
\textit{Hit.A2 - Miss.A1} & -0.61 & 0.19 & -3.23 & 0.04 * \\ 
Miss.A2 - Miss.A1 & -0.15 & 0.18 & -0.82 & 1.00 \\ 
Miss.A2 - Hit.A2 & 0.46 & 0.19 & 2.43 & 0.42 \\ 
\hline
\end{tabular}
% \end{multicols}

% \begin{multicols}{2}
% \begin{tabular}{lllll}
\begin{tabular}{|l|*{5}{c|}}
\hline
\textbf{P300 - CPz} & & & & \\
\hline
\textcolor{blue}{ANOVA Modèle} & numDF & denDF & $F$ & $p$ \\ 
\hline
(Intercept) & 1 & 99.00 & 129.74 & 0.00 * \\ 
Detection & 1 & 99.00 & 0.13 & 0.71 \\ 
Tonalité & 3 & 99.00 & 0.92 & 0.43 \\ 
\textit{Detection:Tonalité} & 3 & 99.00 & 11.21 & 0.00 * \\ 
\hline
\textcolor{blue}{Détection} & E & SE & $z$ & $p$ \\ 
\hline
Miss - Hit & 0.04 & 0.11 & 0.31 & 0.75 \\ 
\hline
\textcolor{blue}{Tonalité} & & & & \\ 
\hline
B1 - B2 & 0.18 & 0.16 & 1.11 & 1.00 \\ 
A1 - B2 & 0.17 & 0.16 & 1.07 & 1.00 \\ 
A2 - B2 & 0.03 & 0.16 & 0.17 & 1.00 \\ 
A1 - B1 & -0.01 & 0.16 & -0.06 & 1.00 \\ 
A2 - B1 & -0.15 & 0.16 & -0.95 & 1.00 \\ 
A2 - A1 & -0.14 & 0.16 & -0.90 & 1.00 \\ 
\hline
\textcolor{blue}{Détection*Tonalité} & & & & \\ 
\hline
Miss.B2 - Hit.B2 & 0.57 & 0.19 & 2.94 & 0.09 . \\ 
\textit{Hit.B1 - Hit.B2} & 0.97 & 0.20 & 4.92 & 0.00 * \\ 
Miss.B1 - Hit.B2 & -0.01 & 0.19 & -0.05 & 1.00 \\ 
Hit.A1 - Hit.B2 & 0.35 & 0.19 & 1.86 & 1.00 \\ 
Miss.A1 - Hit.B2 & 0.54 & 0.19 & 2.81 & 0.14 \\ 
Hit.A2 - Hit.B2 & 0.14 & 0.19 & 0.72 & 1.00 \\ 
Miss.A2 - Hit.B2 & 0.44 & 0.19 & 2.34 & 0.54 \\ 
Hit.B1 - Miss.B2 & 0.40 & 0.20 & 2.00 & 1.00 \\ 
Miss.B1 - Miss.B2 & -0.58 & 0.20 & -2.94 & 0.09 . \\ 
Hit.A1 - Miss.B2 & -0.21 & 0.19 & -1.11 & 1.00 \\ 
Miss.A1 - Miss.B2 & -0.03 & 0.20 & -0.13 & 1.00 \\ 
Hit.A2 - Miss.B2 & -0.43 & 0.20 & -2.18 & 0.83 \\ 
Miss.A2 - Miss.B2 & -0.12 & 0.19 & -0.64 & 1.00 \\ 
\textit{Miss.B1 - Hit.B1} & -0.98 & 0.20 & -4.91 & 0.00 * \\ 
Hit.A1 - Hit.B1 & -0.62 & 0.20 & -3.13 & 0.05 . \\ 
Miss.A1 - Hit.B1 & -0.43 & 0.20 & -2.13 & 0.93 \\ 
\textit{Hit.A2 - Hit.B1} & -0.83 & 0.20 & -4.14 & 0.00 * \\ 
Miss.A2 - Hit.B1 & -0.52 & 0.20 & -2.66 & 0.22 \\ 
Hit.A1 - Miss.B1 & 0.36 & 0.19 & 1.88 & 1.00 \\ 
Miss.A1 - Miss.B1 & 0.55 & 0.20 & 2.81 & 0.14 \\ 
Hit.A2 - Miss.B1 & 0.15 & 0.20 & 0.76 & 1.00 \\ 
Miss.A2 - Miss.B1 & 0.45 & 0.19 & 2.35 & 0.52 \\ 
Miss.A1 - Hit.A1 & 0.19 & 0.19 & 0.98 & 1.00 \\ 
Hit.A2 - Hit.A1 & -0.21 & 0.19 & -1.10 & 1.00 \\ 
Miss.A2 - Hit.A1 & 0.09 & 0.19 & 0.48 & 1.00 \\ 
Hit.A2 - Miss.A1 & -0.40 & 0.20 & -2.05 & 1.00 \\ 
Miss.A2 - Miss.A1 & -0.10 & 0.19 & -0.51 & 1.00 \\ 
Miss.A2 - Hit.A2 & 0.30 & 0.19 & 1.58 & 1.00 \\ 
\hline
\end{tabular}

% \begin{tabular}{lllll}
\begin{tabular}{|l|*{5}{c|}}
\hline
\textbf{P300 - Pz} & & & & \\
\hline
\textcolor{blue}{ANOVA Modèle} & numDF & denDF & $F$ & $p$ \\ 
\hline
(Intercept) & 1 & 101.00 & 115.72 & 0.00 * \\ 
Detection & 1 & 101.00 & 0.95 & 0.33 \\ 
\textit{Tonalité} & 3 & 101.00 & 6.75 & 0.00 * \\ 
\textit{Detection:Tonalité} & 3 & 101.00 & 9.54 & 0.00 * \\ 
\hline
\textcolor{blue}{Détection} & E & SE & $z$ & $p$ \\
\hline
Miss - Hit & 0.11 & 0.13 & 0.82 & 0.41 \\ 
\hline
\textcolor{blue}{Tonalité} & & & & \\
\hline
\textit{B1 - B2} & 0.70 & 0.18 & 3.98 & 0.00 * \\ 
A1 - B2 & 0.30 & 0.17 & 1.73 & 0.51 \\ 
A2 - B2 & 0.22 & 0.18 & 1.22 & 1.00 \\ 
A1 - B1 & -0.40 & 0.18 & -2.27 & 0.14 \\ 
A2 - B1 & -0.48 & 0.18 & -2.67 & 0.05 . \\ 
A2 - A1 & -0.08 & 0.18 & -0.46 & 1.00 \\ 
\hline
\textcolor{blue}{Détection*Tonalité} & & & & \\
\hline
Miss.B2 - Hit.B2 & 0.43 & 0.22 & 1.97 & 1.00 \\ 
\textit{Hit.B1 - Hit.B2} & 1.34 & 0.22 & 6.15 & 0.00 * \\ 
Miss.B1 - Hit.B2 & 0.47 & 0.22 & 2.11 & 0.98 \\ 
Hit.A1 - Hit.B2 & 0.42 & 0.22 & 1.94 & 1.00 \\ 
Miss.A1 - Hit.B2 & 0.61 & 0.22 & 2.80 & 0.14 \\ 
Hit.A2 - Hit.B2 & -0.01 & 0.23 & -0.03 & 1.00 \\ 
\textit{Miss.A2 - Hit.B2} & 0.78 & 0.22 & 3.60 & 0.01 * \\ 
\textit{Hit.B1 - Miss.B2} & 0.91 & 0.22 & 4.19 & 0.00 * \\ 
Miss.B1 - Miss.B2 & 0.04 & 0.22 & 0.17 & 1.00 \\ 
Hit.A1 - Miss.B2 & -0.01 & 0.22 & -0.03 & 1.00 \\ 
Miss.A1 - Miss.B2 & 0.18 & 0.22 & 0.84 & 1.00 \\ 
Hit.A2 - Miss.B2 & -0.43 & 0.23 & -1.88 & 1.00 \\ 
Miss.A2 - Miss.B2 & 0.35 & 0.22 & 1.63 & 1.00 \\ 
\textit{Miss.B1 - Hit.B1} & -0.87 & 0.22 & -3.93 & 0.00 * \\ 
\textit{Hit.A1 - Hit.B1} & -0.91 & 0.22 & -4.22 & 0.00 * \\ 
\textit{Miss.A1 - Hit.B1} & -0.73 & 0.22 & -3.35 & 0.02 * \\ 
\textit{Hit.A2 - Hit.B1} & -1.34 & 0.23 & -5.82 & 0.00 * \\ 
Miss.A2 - Hit.B1 & -0.55 & 0.22 & -2.55 & 0.30 \\ 
Hit.A1 - Miss.B1 & -0.05 & 0.22 & -0.20 & 1.00 \\ 
Miss.A1 - Miss.B1 & 0.14 & 0.22 & 0.65 & 1.00 \\ 
Hit.A2 - Miss.B1 & -0.47 & 0.23 & -2.02 & 1.00 \\ 
Miss.A2 - Miss.B1 & 0.32 & 0.22 & 1.43 & 1.00 \\ 
Miss.A1 - Hit.A1 & 0.19 & 0.22 & 0.87 & 1.00 \\ 
Hit.A2 - Hit.A1 & -0.43 & 0.23 & -1.85 & 1.00 \\ 
Miss.A2 - Hit.A1 & 0.36 & 0.22 & 1.66 & 1.00 \\ 
Hit.A2 - Miss.A1 & -0.62 & 0.23 & -2.67 & 0.21 \\ 
Miss.A2 - Miss.A1 & 0.17 & 0.22 & 0.80 & 1.00 \\ 
\textit{Miss.A2 - Hit.A2} & 0.79 & 0.23 & 3.42 & 0.02 * \\ 
\hline
\end{tabular}
\end{multicols}
\end{table}
\end{landscape}

%%%%%%%%%%%%%%%%%%%%%%%%%%%%%%%%%%%%%%%%%%%%%%%%%%%%%%%%%%%%%%%%%%%%%%%%%%%%%%%
\subsubsection{Composante P300}
%%%%%%%%%%%%%%%%%%%%%%%%%%%%%%%%%%%%%%%%%%%%%%%%%%%%%%%%%%%%%%%%%%%%%%%%%%%%%%%

Pour la composante P300, les valeurs d'amplitude ont été sélectionnées pour les électrodes d'intérêts (FCz, Cz, CPz et Pz) dans la fenêtre temporelle $250$-$500$~ms pour les quatre tonalités (B2, B1,, A1, A2). 
La Figure~\ref{fig:figure5amplitudeerpelectrodesP300waveform} montre les formes d'ondes évoquées dans le temps associées à la P300 entre $250$ et $500$ ms pour ces mêmes électrodes et tonalités. 
Les valeurs d'amplitudes positives les plus grandes sont trouvées visuellement pour les quatre électrodes pour la première tonalité avant l'appui-bouton (B1) pour les cibles détectées comparativement aux cibles non-détectées. 
Ces graphiques indiquent visuellement que la composante P300 présente une plus grande amplitude pour la première tonalité avant la détection (B1) principalement entre $250$ et $350$ ms, pour les quatre électrodes. 
Par conséquent, nous reportons les résultats des analyses statistiques sur les valeurs d'amplitudes de la P300 sur ces quatre électrodes pour les quatre tonalités cibles dans la fenêtre $250$-$350$~ms. 

Un modèle linéaire à effets mixtes a été ajusté pour chaque électrode pour étudier l'effet des facteurs détection et tonalité sur les valeurs des pics d'amplitude (formule R : Amplitude $\sim$ Détection * Tonalité + ~$1|$Sujet). 
La Table~\ref{tab:table5statsP300} présente les résultats des analyses statistiques de ces modèles linéaires à effets mixtes réalisés sur les valeurs des pics d'amplitude de la P300 pour les quatre électrodes et pour les deux tonalités du signal cible avant et après l'appui-bouton pour les cibles détectées et avant et après le temps moyen de la détection pour les cibles non-détectées. \\

\underline{Pour l'électrode FCz}, aucun effet significatif n'a été montré pour la détection ($F(1,101)=0.14$, $p=0.71$). 
Un effet significatif a été trouvé à la fois pour la tonalité ($F(3,101)=10.3$, $p<.001$) et pour l'interaction entre la tonalité et la détection ($F(3,101)=7.14$, $p<.001$). 
Les tests post-hoc ont révélé que les valeurs d'amplitude sont significativement plus élevées pour la tonalité B1 lorsque les cibles ont été détectées comparativement aux cibles manquées (Table~\ref{tab:table5statsP300}). 
En outre, les tests post-hoc ont également montré que les valeurs d'amplitude sont significativement plus élevées pour les cibles détectées pour la tonalité B1 comparativement aux tonalités B2 et A1. 

\underline{Pour l'électrode Cz}, aucun effet significatif n'a été montré pour la détection ($F(1,102)=0.35$, $p=0.55$). 
Un effet significatif a été trouvé à la fois pour la tonalité ($F(3,102)=8.02$, $p<.001$) et pour l'interaction entre la tonalité et la détection ($F(3,102)=8.06$, $p<.001$). 
Les tests post-hoc ont montré que les valeurs d'amplitude sont significativement plus élevées pour la tonalité B1 lorsque les cibles ont été détectées comparativement aux cibles manquées (Table~\ref{tab:table5statsP300}). 
De plus, les tests post-hoc ont aussi révelé que les valeurs d'amplitude sont significativement plus élevées pour les cibles détectées pour la tonalité B1 comparativement aux tonalités B2, A1 et A2. 

\underline{Pour l'électrode CPz}, aucun effet significatif n'a été montré ni pour la détection ($F(1,99)=0.13$, $p=0.71$) ni pour la tonalité ($F(3,99)=0.92$, $p=0.43$). 
Un effet significatif a été trouvé seulement pour l'interaction entre la tonalité et la détection ($F(3,99)=11.21$, $p<.001$). 
Les tests post-hoc ont montré que les valeurs d'amplitude sont significativement plus élevées pour la tonalité B1 lorsque les cibles ont été détectées comparativement aux cibles manquées (Table~\ref{tab:table5statsP300}). 
Ils ont aussi montré que les valeurs d'amplitude sont significativement plus élevées pour les cibles détectées pour la tonalité B1 comparativement à la tonalité B2. 

\underline{Pour l'électrode Pz}, aucun effet significatif n'a été montré pour la détection ($F(1,101)=0.95$, $p=0.33$). 
Un effet significatif a été trouvé à la fois pour la tonalité ($F(3,101)=6.75$, $p<.001$) et pour l'interaction entre la tonalité et la détection ($F(3,101)=9.54$, $p<.001$). 
Les tests post-hoc ont montré que les valeurs d'amplitude sont significativement plus élevées pour la tonalité B1 lorsque les cibles ont été détectées comparativement aux cibles manquées (Table~\ref{tab:table5statsP300}). 
Ils ont aussi révelé que les valeurs d'amplitude sont significativement plus élevées pour les cibles détectées pour la tonalité B1 comparativement aux tonalités B2, A1 et A2. 

%%%%%%%%%%%%%%%%%%%%%%%%%%%%%%%%%%%%%%%%%%%%%%%%%%%%%%%%%%%%%%%%%%%%%%%%%%%%%%%
\subsection{Synthèse et discussion sur les analyses des ERPs}
\label{etude2analysesERPsyntheseresultats}
%%%%%%%%%%%%%%%%%%%%%%%%%%%%%%%%%%%%%%%%%%%%%%%%%%%%%%%%%%%%%%%%%%%%%%%%%%%%%%%

Dans cette section, nous avons cherché à reproduire les principaux résultats de la littérature sur les corrélats neuronaux de la perception auditive consciente \citep{dykstra2016neural, giani2015detecting, gutschalk2008neural, wiegand2012correlates} à partir de données EEG. 
Nous avons souhaité observer dans un protocole expérimental de MI multi-tonalités des formes d’ondes de potentiels évoqués ARN et P300 lors de la perception de tonalités du signal cible. 
Nous avons supposé que les cibles détectées susciteraient des valeurs d'amplitude plus élevées de ces ERPs avant et après l'appui-bouton par le sujet. 

La composante ARN est un corrélat neuronal présumé de la perception auditive consciente, observé au niveau du cortex auditif lors de la perception des tonalités du signal cible. 
De plus grandes valeurs d'amplitude sont associées à cette forme d'onde négative observée environ $250$~ms après que les tonalités aient été détectées par les sujets. 
Notre paradigme de MI multi-tonalités était assez similaire à celui employé par \cite{gutschalk2008neural} et \cite{dykstra2016neural} puisque la cible auditive comprenait $10$ tonalités ($12$ et $8$ respectivement dans les études antérieures), contrairement aux paradigmes de \cite{giani2015detecting} où la cible consistait en une paire de tonalités et de \cite{wiegand2012correlates} où la cible auditive comprenait quatre tonalités. 

Dans l'étude de \cite{gutschalk2008neural}, la composante ARN a été observée deux tonalités avant le report comportemental, ce qui correspondrait à la tonalité B2 dans notre protocole. 
Dans l'étude de \cite{wiegand2012correlates}, la composante ARN a été suscitée pour toutes les tonalités de la cible excepté pour la première tonalité (\textit{i.e.}, la deuxième, la troisième et la quatrième). 
Dans l'étude de \cite{giani2015detecting}, la composante ARN n'est apparue que pour la deuxième des deux tonalités, c'est-à-dire juste avant le report perceptif par le sujet, et correspondrait ainsi à la tonalité B1 dans notre étude. 
Dans notre cas, aucune forme d'onde similaire à la composante ARN n'a été observée pour la deuxième tonalité (B2) avant le report lorsque les cibles ont été détectées au sein des cortex auditifs. 
Cependant, une onde négative (Figure~\ref{fig:figure5ARNwaveforms}) a été observée sur les électrodes C5, F6 et F7 pour la première tonalité (B1) avant le report lorsque la cible a été détectée. 
Cette négativité localisée au niveau des lobes temporaux droit et gauche était située dans un intervalle de $250$ à $350$~ms et présentait des caractéristiques semblables à celles d'une composante ARN \citep{giani2015detecting}. 

Les résultats ont montré que la perception consciente de la cible auditive par le sujet a significativement diminué les valeurs d'amplitude de la composante ARN pour la première tonalité B1 dans la fenêtre $250-350$~ms. 
Les valeurs d'amplitude suscitées par cette même tonalité B1 se sont montrées significativement inférieures à celles suscitées par la tonalité B2. 
Cela est cohérent avec le fait que la consigne donnée aux participants était de n'appuyer sur le presse-bouton qu'une fois qu'ils étaient sûr d'avoir détecté une cible tonale régulière, et donc un pattern d'au moins deux tonalités cibles. 
Ainsi, la perception consciente auditive de la deuxième tonalité est associée à une diminution des valeurs d'amplitude de la forme d'onde négative observée au niveau des lobes temporaux droit et gauche, représentée par la composante ARN. 

Nous avons trouvé que la composante ARN était présente au niveau des électrodes C5, F6 et F7. 
Lorsque l'on observe la Figure~\ref{fig:chap2gyrusheschl} et la Figure~\ref{fig:figure5systemeegclustering} (Droite) sur lesquelles apparaissent respectivement, le gyrus temporal supérieur au sein du lobe temporal et la procédure de regroupement des électrodes, on voit que les électrodes frontales F6 et F7 peuvent correspondre à des zones cérébrales antéro-latérales du lobe temporal. 
De la même manière, l'électrode centrale latérale gauche C5 peut appartenir au lobe temporal gauche et il serait vraisemblable qu'elle corresponde à une zone cérébrale proche du gyrus temporal supérieur, dans lequel on peut retrouver le gyrus de Heschl (voir Figure~\ref{fig:chap2gyrusheschl}), centre fonctionnel du cortex auditif primaire dans le traitement et l'intégration des informations auditives \citep{beauchamp2004integration, brugge2009coding}. 
Par conséquent, il semble raisonnable de penser que l'on peut faire correspondre les électrodes C5 et F7 au cortex auditif situé dans le lobe temporal gauche et l'électrode F6 au cortex auditif situé dans le lobe temporal droit. 

Dans le cortex auditif humain, les tonalités détectées sous MI évoquent une réponse négative tardive \citep{gutschalk2008neural}, qui n'est pas évoquée par les cibles manquées et les potentiels évoqués plus tôt dans le cortex auditif ne dissocient pas les cibles détectées des cibles manquées \citep{gutschalk2008neural, konigs2012functional, prilop2017auditory}.
La réponse négative ARN évoquée par les cibles peut-être observée dans le cortex auditif gauche et droit, mais l'équilibre inter-hémisphérique montre une dépendance constante de l'oreille de présentation ou de la latéralisation par les différences de temps interaurales \citep{konigs2012functional, prilop2017auditory}. 

Dans notre cas, les stimuli auditifs ont été présentés de manière diotique aux deux oreilles.
En effet, nous n'avons pas cherché à étudier spécifiquement la latéralisation hémisphérique de la composante ARN. 
Il est donc cohérent de voir apparaître de telles formes d'ondes au niveau des deux lobes temporaux. 
Cependant la question expérimentale de la latéralisation hémisphérique de l'ARN devrait être plus précisément étudiée au niveau spatial et temporel, comme on peut l'observer sur la Figure~\ref{fig:figure5ARNwaveforms} au niveau de l'électrode C5 pour la tonalité B1, plusieurs composantes négatives sont visibles. 
En fait, on voit que ces formes d'ondes de potentiel négatif se concentrent à un niveau antérieur de l'aire temporale et on observe aucun autre pattern similaire sur les électrodes situées plus en arrière. 
Dans notre cas, cela pourrait suggérer que la perception consciente de la cible auditive, reflétée par le report après la perception de la deuxième tonalité cible perçue (B1), soit associée à des variations de l'activité cérébrale au niveau de la zone fronto-centro-temporale, vraisemblablement incluse dans le lobe temporal et plus précisément au niveau du cortex auditif. 

Les résultats de cette analyse montrent donc que de larges formes d'ondes négatives peuvent être observées dans des aires cérébrales situées au niveau des cortex auditifs sur la première tonalité avant le report perceptif du sujet. 
Nous apportons ici un nouvel argument expérimental permettant de considérer la composante ARN comme un corrélat neuronal EEG de la perception consciente d'un signal auditif. 
Il est par ailleurs possible de considérer la composante ARN comme un indice de l'intégration locale des caractéristiques du stimulus à la conscience du sujet. 
Cette intégration à une échelle locale pourrait dès lors être la conséquence de boucles de traitements récurrents ayant pour fonction de réduire l'incertitude vis-à-vis des caractéristiques statistiques des informations auditives présentées. 

L'une des premières étapes critiques des processus associés à la perception des sons dans l'environnement acoustique est l'encodage de l'information en un code neural robuste qui permet un traitement subséquent efficient dans le système auditif \citep{lewicki2002efficient}. 
Cet encodage réalisé tout au long de la structure de traitement de l'information auditive pourrait aboutir à une émergence des formes d'ondes ARN à l'échelle cérébrale macroscopique. 
Il est intéressant de noter également que le planum temporal, situé sur la face supérieure du gyrus temporal supérieur et postérieur au gyrus de Heschl, usuellement impliqué dans le traitement auditif \citep{nakada2001planum} et dans le traitement lexical du language \citep{bookheimer2002functional} ait montré une relation entre l'activité neuronale et les demandes énergétiques liées à une augmentation de l'entropie des signaux auditifs présentées chez l'humain \citep{overath2007information}. 

\cite{overath2007information} ont montré qu'une telle relation se produit dans un mécanisme d'encodage efficace utilisant moins de ressources computationnelles lorsque moins d'informations sont présentes dans le signal et que le planum temporal est un centre neuronal fonctionnel demandant moins de ressources computationnelles pour encoder des signaux redondants par rapport à ceux qui ont une entropie élevée.
De cette manière, si l'ARN est le reflet d'une telle structure de traitements récurrents liée à la perception auditive consciente, alors notre analyse serait un argument en faveur des modèles théoriques de la conscience basés sur le traitement récurrent de l'information au sein de zones cérébrales locales \citep{lamme2000distinct, lamme2003visual, lamme2006towards}. 
La composante ARN serait alors un indicateur caractéristique des différents traitements récurrents de l'information au sein des --- ou en relation avec les --- zones usuellement impliquées dans l'intégration des propriétés acoustiques des stimuli auditifs. 

En outre, l'analyse par modélisation causale dynamique de \cite{giani2015detecting} suggère que l'ARN caractériserait la ségrégation des flux auditifs en étant associée à des changements dans la connectivité intrinsèque des cortex auditifs.
Une détection réussie du signal cible pourrait reposer sur un traitement récurrent entre les zones corticales auditives et pariétales d'ordre supérieur et ainsi porter la réponse ARN à un niveau de signature caractéristique de la conscience perceptive auditive. 
Dans notre analyse, on reporte que les formes d'ondes ARN sont localisées sur des parties relativement latérales des cortex auditifs, chevauchant les zones cérébrales frontales et centrales. 
Il est nécessaire de compléter ces recherches sur la façon dont la conscience perceptive dans le MI émerge dans une cascade complexe de traitement neuronal qui s'accumule sur plusieurs tonalités cibles au sein de diverses zones cérébrales et notamment au sein du réseau fronto-temporo-pariétal et de ses différentes connexions \citep{demertzi2013consciousness, dykstra2017roadmap, eklund2019electrophysiological, eriksson2007similar, eriksson2017activity, giani2015detecting, wiegand2018cortical}. 

La composante P300 a été traditionnellement considéré comme un corrélat neuronal présumé de la perception consciente \citep{dehaene2006conscious, del2007brain, sergent2004neural, sergent2005timing}. 
Elle est considérée comme un indice du traitement intégratif global à un haut niveau des caractéristiques du stimulus et donc reliant la conscience du stimulus à la perception du sujet. 
Il a été montré que la P300 était plus grande et avait une latence plus courte pour les tonalités détectées. 
Dans leur étude, \cite{giani2015detecting} ont trouvé que la perception consciente augmentait les amplitudes de la P300 de manière significative pour les deux tonalités de la cible. 
Les auteurs ont alors suggéré qu'avec la P300 comme marqueur de la perception consciente, percevoir la paire de tonalités cibles devrait reposer sur un traitement récurrent entre les zones corticales auditives et pariétales. 
Dans leurs analyses, \cite{dykstra2016neural} ont trouvé une large composante P300 pour les cibles détectées, associée à des générateurs dans les cortex temporo-frontal et temporo-latéral. 

Dans notre étude, les résultats ont montré un effet de la perception auditive consciente en fonction de la tonalité sur les valeurs d'amplitude de la P300 pour les quatre électrodes d'intérêts (FCz, Fz, CPz et Pz) dans une fenêtre temporelle de $250-350$~ms. 
Pour ces quatre électrodes, la perception consciente de la cible a significativement augmenté les valeurs d'amplitude de la P300 de la première tonalité avant la détection (B1) dans cet intervalle $250-350$~ms. 
Cela pourrait suggérer que la composante P300 serait liée à l'intégration des caractéristiques de la cible, ce qui s'observe donc par une augmentation de son amplitude sur le vertex à l'approche du report perceptif par le sujet. 
Certaines études récentes viennent cependant contredire ce fait en considérant la P300 plutôt comme un processus post-perceptuel plutôt que comme un processus d'intégration consciente \citep{cohen2020distinguishing, fishman2021learning, pitts2014gamma, pitts2014isolating, tsuchiya2015no}. 

Finalement, lors de la présentation d'une cible auditive composée de 10 tonalités intégrée dans un masqueur multi-tonalités, seule la première tonalité (B1) avant la détection pour les cibles détectées a suscité une composante ARN dans la fenêtre temporelle $250-350$~ms sur trois électrodes (C5, F6 et F7) localisées dans les cortex auditifs droit et gauche ainsi qu'une composante P300 sur quatre électrodes localisées dans une aire sagittale (FCz, Fz, CPz et Pz) dans la fenêtre temporelle $250-350$~ms chez le sujet adulte. 
Par conséquent, à partir des résultats de cette analyse, la perception auditive consciente peut être associée à l'amplification des potentiels des composantes ARN et P300, principalement pour la première tonalité avant le report perceptif du sujet, ce qui correspondrait donc à la deuxième tonalité qu'il ait perçu. 
Sur la base de ces premiers éléments, la composante ARN serait plus à même que la composante P300 d'être un indicateur pertinent du signal EEG pour indiquer la perception auditive consciente. 

%%%%%%%%%%%%%%%%%%%%%%%%%%%%%%%%%%%%%%%%%%%%%%%%%%%%%%%%%%%%%%%%%%%%%%%%%%%%%%%%%%%%%%%%%%%%%%%%%%%%%%%%%%%%%%%%%%%%%%%%%%%%%%%%%%%%%%%%%%%%%%%%%%%%%%%%%%%%%%%%%%%%%%%%%%%%%%%%%%%%%%%%%%%%%%%%%%%%%%%%%%%%%%%%%%%%%%%%%%%%%%%%%%%%%%%%%%%%%%%
%%%%%%%%%%%%%%%%%%%%%%%%%%%%%%%%%%%%%%%%%%%%%%%%%%%%%%%%%%%%%%%%%%%%%%%%%%%%%%%%%%%%%%%%%%%%%%%%%%%%%%%%%%%%%%%%%%%%%%%%%%%%%%%%%%%%%%%%%%%%%%%%%%%%%%%%%%%%%%%%%%%%%%%%%%%%%%%%%%%%%%%%%%%%%%%%%%%%%%%%%%%%%%%%%%%%%%%%%%%%%%%%%%%%%%%%%%%%%%%
%%%%%%%%%%%%%%%%%%%%%%%%%%%%%%%%%%%%%%%%%%%%%%%%%%%%%%%%%%%%%%%%%%%%%%%%%%%%%%%%%%%%%%%%%%%%%%%%%%%%%%%%%%%%%%%%%%%%%%%%%%%%%%%%%%%%%%%%%%%%%%%%%%%%%%%%%%%%%%%%%%%%%%%%%%%%%%%%%%%%%%%%%%%%%%%%%%%%%%%%%%%%%%%%%%%%%%%%%%%%%%%%%%%%%%%%%%%%%%%
%%%%%%%%%%%%%%%%%%%%%%%%%%%%%%%%%%%%%%%%%%%%%%%%%%%%%%%%%%%%%%%%%%%%%%%%%%%%%%%%%%%%%%%%%%%%%%%%%%%%%%%%%%%%%%%%%%%%%%%%%%%%%%%%%%%%%%%%%%%%%%%%%%%%%%%%%%%%%%%%%%%%%%%%%%%%%%%%%%%%%%%%%%%%%%%%%%%%%%%%%%%%%%%%%%%%%%%%%%%%%%%%%%%%%%%%%%%%%%%
%%%%%%%%%%%%%%%%%%%%%%%%%%%%%%%%%%%%%%%%%%%%%%%%%%%%%%%%%%%%%%%%%%%%%%%%%%%%%%%%%%%%%%%%%%%%%%%%%%%%%%%%%%%%%%%%%%%%%%%%%%%%%%%%%%%%%%%%%%%%%%%%%%%%%%%%%%%%%%%%%%%%%%%%%%%%%%%%%%%%%%%%%%%%%%%%%%%%%%%%%%%%%%%%%%%%%%%%%%%%%%%%%%%%%%%%%%%%%%%
%%%%%%%%%%%%%%%%%%%%%%%%%%%%%%%%%%%%%%%%%%%%%%%%%%%%%%%%%%%%%%%%%%%%%%%%%%%%%%%%%%%%%%%%%%%%%%%%%%%%%%%%%%%%%%%%%%%%%%%%%%%%%%%%%%%%%%%%%%%%%%%%%%%%%%%%%%%%%%%%%%%%%%%%%%%%%%%%%%%%%%%%%%%%%%%%%%%%%%%%%%%%%%%%%%%%%%%%%%%%%%%%%%%%%%%%%%%%%%%
%%%%%%%%%%%%%%%%%%%%%%%%%%%%%%%%%%%%%%%%%%%%%%%%%%%%%%%%%%%%%%%%%%%%%%%%%%%%%%%%%%%%%%%%%%%%%%%%%%%%%%%%%%%%%%%%%%%%%%%%%%%%%%%%%%%%%%%%%%%%%%%%%%%%%%%%%%%%%%%%%%%%%%%%%%%%%%%%%%%%%%%%%%%%%%%%%%%%%%%%%%%%%%%%%%%%%%%%%%%%%%%%%%%%%%%%%%%%%%%
%%%%%%%%%%%%%%%%%%%%%%%%%%%%%%%%%%%%%%%%%%%%%%%%%%%%%%%%%%%%%%%%%%%%%%%%%%%%%%%%%%%%%%%%%%%%%%%%%%%%%%%%%%%%%%%%%%%%%%%%%%%%%%%%%%%%%%%%%%%%%%%%%%%%%%%%%%%%%%%%%%%%%%%%%%%%%%%%%%%%%%%%%%%%%%%%%%%%%%%%%%%%%%%%%%%%%%%%%%%%%%%%%%%%%%%%%%%%%%%
%%%%%%%%%%%%%%%%%%%%%%%%%%%%%%%%%%%%%%%%%%%%%%%%%%%%%%%%%%%%%%%%%%%%%%%%%%%%%%%%%%%%%%%%%%%%%%%%%%%%%%%%%%%%%%%%%%%%%%%%%%%%%%%%%%%%%%%%%%%%%%%%%%%%%%%%%%%%%%%%%%%%%%%%%%%%%%%%%%%%%%%%%%%%%%%%%%%%%%%%%%%%%%%%%%%%%%%%%%%%%%%%%%%%%%%%%%%%%%%
%%%%%%%%%%%%%%%%%%%%%%%%%%%%%%%%%%%%%%%%%%%%%%%%%%%%%%%%%%%%%%%%%%%%%%%%%%%%%%%%%%%%%%%%%%%%%%%%%%%%%%%%%%%%%%%%%%%%%%%%%%%%%%%%%%%%%%%%%%%%%%%%%%%%%%%%%%%%%%%%%%%%%%%%%%%%%%%%%%%%%%%%%%%%%%%%%%%%%%%%%%%%%%%%%%%%%%%%%%%%%%%%%%%%%%%%%%%%%%%

%%%%%%%%%%%%%%%%%%%%%%%%%%%%%%%%%%%%%%%%%%%%%%%%%%%%%%%%%%%%%%%%%%%%%%%%%%%%%%%
\clearpage
\section{Contenu informationnel et complexité de l'EEG dans le MI}
\label{contenuinformationnel}
%%%%%%%%%%%%%%%%%%%%%%%%%%%%%%%%%%%%%%%%%%%%%%%%%%%%%%%%%%%%%%%%%%%%%%%%%%%%%%%

Dans un second temps, nous avons cherché à caractériser le contenu informationnel et la complexité du signal neuronal en lien avec la perception auditive consciente d'un flux de tonalités cible sous MI. 
Nous avons montré dans la première partie que certains algorithmes de mesures d’entropie et de complexité des signaux neurophysiologiques ont permis d’obtenir une caractérisation des états de conscience \citep{curley2018characterization, engemann2018robust, engemann2020combining, king2014characterizingthesis, liang2015eeg, sitt2014large}. 
En outre, plusieurs de ces algorithmes sont désormais utilisables dans le cadre de la caractérisation des contenus de conscience. 
Ces signatures neuronales permettent de : i) fournir de l’information sur la neurodynamique cérébrale et sur les contenus neuronaux associés aux percepts conscients ; et ii) caractériser le décours temporel des processus et des mécanismes à l’œuvre lors de la perception consciente.

Dans le cadre d'un protocole de MI, ce décours temporel associé à la conscience de la cible correspond à la construction progressive d'un percept qui peut être envisagé comme une cascade de traitements de l'information à plusieurs échelles spatio-temporelles de l'activité cérébrale. 
Puisque les aires cérébrales frontales, temporales et pariétales ont des degrés d'implication différents au sein des réseaux cérébraux associés aux processus conscients \citep{dehaene2011experimental, eriksson2007similar, wiegand2018cortical}, il est nécessaire de mieux comprendre comment les caractéristiques statistiques des signaux qui en sont issus s'en trouvent modifiées lors de la perception consciente. 
À partir de ces observations, nous avons cherché à comprendre dans quelle mesure l'évolution du contenu informationnel et de la complexité de l'activité neuronale associée aux signaux EEG issus des aires frontales, temporales et pariétales peut nous informer sur la perception consciente de tonalités cibles au sein d'un masqueur multi-tonalités. 

En devenant perceptible pour le sujet, la cible auditive provoquerait vraisemblablement une modification du contenu informationnel et de la complexité intrinsèque du signal EEG, comme on s'attend à ce que le signal neurophysiologique enregistré à la surface du scalp présente une variation de ses propriétés statistiques qui soit révélatrice de changements du transfert d'information dans les populations neuronales sous-jacentes. 
En présentant une telle différenciation lors de la perception auditive consciente, cela pourrait représenter un indice utile pour la caractérisation de l'accès conscient. 

Au delà de cette différenciation, la structure des patterns dynamiques associés à la variation des contenus d'information et de complexité peut nous renseigner sur l'indication de la conscience perceptive. 
Comme nous l'avons vu précédemment, de tels indices se montreraient pertinents pour l'implémentation de technologies de monitoring neuroadaptatives permettant la présentation de neurofeedbacks. 
Ce type de variations n'a pas été montré jusqu'à présent dans des situations de MI.
Ainsi, nous avons étudié l'évolution de différentes caractéristiques de contenu informationnel et de complexité des signaux EEG en lien avec la dynamique autour de la perception consciente de la cible auditive sous MI. 

Une perception réussie de la cible pourrait être supportée par une augmentation de l'information contenue dans les signaux neuronaux associés à un réseau cérébral fronto-temporo-pariétal fonctionnel \citep{demertzi2013consciousness, eklund2019electrophysiological, eriksson2007similar, eriksson2017activity, giani2015detecting, wiegand2018cortical}. 
La conscience perceptive serait alors liée à des fluctuations du contenu en information du signal neuronal localement disponible et observable à l'échelle EEG macroscopique. 
De la même manière, on pourrait supposer que la structure de complexité associée au signal neuronal lors de la perception consciente serait différente de celle observée lorsque la cible auditive n'a pas atteint la conscience. 
Ainsi, une cible détectée serait susceptible d'augmenter le degré de complexité au sein des signaux EEG issus de ce réseau fronto-temporo-pariétal. 
De plus, dans le cadre de la caractérisation de la dynamique de la construction du percept conscient, on pourrait s'attendre à observer une augmentation progressive du contenu informationnel et du degré de complexité associés aux signaux neuronaux issus de ce réseau lors de la construction du percept auditif. 
Pour tester ces hypothèses, nous avons utilisé i) des algorithmes permettant le calcul de mesures d'entropie et de complexité du signal neuronal et ii) une approche par agrégation des électrodes EEG (voir Section \ref{etude2enregistrementsEEG}). 

%%%%%%%%%%%%%%%%%%%%%%%%%%%%%%%%%%%%%%%%%%%%%%%%%%%%%%%%%%%%%%%%%%%%%%%%%%%%%%%
\subsection{Prétraitement et agrégation des électrodes}
\label{pretraitementaggregationelectrodes}
%%%%%%%%%%%%%%%%%%%%%%%%%%%%%%%%%%%%%%%%%%%%%%%%%%%%%%%%%%%%%%%%%%%%%%%%%%%%%%%

Les mêmes données EEG issues de la tâche de MI (voir Section \ref{etude2materielmethode}) ont été utilisées dans le cadre de cette analyse. 
Afin de caractériser la dynamique des caractéristiques du signal EEG sous-jacente à la construction du percept auditif pendant la tâche de perception auditive, un sur-échantillonnage des valeurs du signal EEG a été réalisé. 
En effet, bien que nous ayons réussi à cibler des combinaisons de paramètres adéquates pour favoriser un décours temporel spécifique, certaines mesures d'information et de complexité sont largement dépendantes du nombre d'échantillons temporels entrant dans le calcul de leur estimateur. 
Dans cette analyse, nous avons donc du procéder à un tel sur-échantillonnage des valeurs du signal afin de permettre l'obtention des valeurs de mesures pour le décours temporel étudié. 

Le signal de base échantillonné par le système à $1000$~Hz a ensuite été sur-échantillonné à $4000$~Hz afin d'obtenir $24000$ échantillons temporels ($6$~secondes à $4000$~Hz).
Nous avons ainsi obtenu un fenêtrage temporel des signaux EEG composé de $12$ fenêtres temporelles de $1000$ points de données chacune ($24000$ points de données au total : $12000$ avant la détection et $12000$ après la détection). 
De cette manière, les mesures d'entropie et de complexité ont pu être calculées sur chacune des fenêtres de $1000$ échantillons temporels et une valeur de la mesure a ainsi été obtenue pour chaque fenêtre. 
Un nombre de $1000$ échantillons temporels est une quantité suffisante usuellement utilisée dans la littérature pour les mesures de complexité \citep{bruhn2000approximate, pincus1991approximate}. 
En fait, puisque nous avons cherché à étudier la dynamique de l'activité cérébrale autour de la perception consciente et de son absence, nous avons souhaité obtenir suffisamment de fenêtres temporelles pour le calcul des mesures. 
En sur-échantillonnant le signal, cela nous a permis d'obtenir $12$ fenêtres de $1000$ points avant la détection/non-détection et $12$ fenêtres de $1000$ points après la détection/non-détection. 
Ainsi, nous avons ensuite pu obtenir les estimations fiables des mesures d'entropie et de complexité pour chacune de ces fenêtres temporelles, ce qui n'aurait pas été possible avec un échantillonnage plus bas du signal. 

Les mêmes procédures de traitement du signal EEG qu'à la Section \ref{etude2materielmethode} ont été appliquées ici (filtres non-causaux passe-bas ($80$~Hz) et passe-haut ($1$~Hz), rejet des artefacts, ICA, autoreject et inspection visuelle).
Comme indiqué dans la Section \ref{etude2traitementetanalysesEEG}, afin de permettre l'étude de la dynamique autour de la détection par une approche comparative, nous avons utilisé comme référence temporelle les temps de détection des sujets pour les détections correctes et le temps de détection moyen des sujets pour les détections manquées (\textit{i.e.}, $3.4$~s). 
Le signal prétraité était ensuite segmenté en deux fenêtres de $3$ secondes : l'une avant («Before») la référence $[-3s:0s]$, et l'autre après («After») la référence $[0s:+3s]$. 
Nous avons donc obtenu quatre époques EEG qui ont été étiquetées : «Epoch Before Hit», «Epoch After Hit», «Epoch Before Miss» et «Epoch After Miss» (voir Figure~\ref{fig:figure5segmentationepochs} Droite). 

Les valeurs des mesures d'entropie et de complexité ont été agrégées topographiquement pour les analyses statistiques en utilisant la moyenne arithmétique de l'ensemble des électrodes d'un cluster pour obtenir huit zones corticales par hémisphère \citep{grabner2012oscillatory} : AF pour Antéro-Frontale, F pour Frontale, FC pour Fronto-Centrale, C pour Centrale, CP pour Centro-Pariétale, P pour Pariétale, PO pour Pariéto-Occipitale et T pour Temporale (Figure~\ref{fig:figure5systemeegclustering} Droite). 
La neuvième zone corticale correspondait à une zone sagittale (électrodes AFz, Fz, FCz, Cz, CPz, Pz, POz et Oz). 

%%%%%%%%%%%%%%%%%%%%%%%%%%%%%%%%%%%%%%%%%%%%%%%%%%%%%%%%%%%%%%%%%%%%%%%%%%%%%%%
\newpage
\subsection{Analyses statistiques}
\label{analysesstatmesuresentropiecomplexite}
%%%%%%%%%%%%%%%%%%%%%%%%%%%%%%%%%%%%%%%%%%%%%%%%%%%%%%%%%%%%%%%%%%%%%%%%%%%%%%%

L'ensemble des analyses a été réalisé avec le logiciel R \citep{Rlanguage2021}. 
Les mesures d'entropie et de complexité ont été analysées à l'aide de modèles à effets mixtes avec la bibliothèque \texttt{lme} \citep{bates2007lme4}. 
Six facteurs ont été employés : Détection (cible détectée/manquée), Condition (avant/après la détection/non-détection), Cluster (AF~/~F~/~FC~/~C~/~CP~/~P~/~PO~/~T~/~S), Électrode (d'un Cluster spécifique), Fenêtre ($22$, $11$~avant, $11$~après), Sujet ($20$). 
Le facteur Électrode et le facteur Fenêtre sont des facteurs emboités respectivement dans les facteurs Cluster et Condition. 
Dans toutes les analyses, les facteurs expérimentaux (\textit{i.e.}, la détection, la condition, le cluster, l'électrode, la fenêtre) ont été traités comme des variables à effet fixe et l'Id. Sujet a été traité comme un effet aléatoire pour le paramètre d'intercept. 

D'abord, nous avons réalisé une première analyse par modèle linéaire à effets mixtes de sorte à étudier les effets des facteurs détection, condition et cluster sur les valeurs de mesure (formule R : Mesure $\sim$ Détection * Condition * Cluster + $1|$Sujet). 
L'objectif était d'étudier l'effet de la perception consciente de la cible auditive sur les valeurs des différentes mesures avant et après la référence et au sein des différentes zones cérébrales d'intérêts sur la base des valeurs agrégées. 
Cette première analyse nous a permis de sélectionner des zones cibles spécifiques pour étudier l'effet de la perception consciente plus précisément à l'échelle spatiale des clusters. 

Ensuite, nous avons réalisé une deuxième analyse par modèle linéaire à effets mixtes afin d'étudier les effets associés au facteur électrode de ces zones cérébrales cibles. 
L'objectif était, cette fois, d'étudier l'effet de la perception consciente de la cible auditive sur les valeurs des différentes mesures avant et après la référence pour les électrodes appartenant à un cluster cérébral spécifique (formule R : Mesure $\sim$ Détection * Condition * Électrode du Cluster + $1|$Sujet).

Enfin, nous avons réalisé une troisième analyse par modèle linéaire à effets mixtes pour étudier la dynamique de la construction du percept dans les zones cérébrales cibles. 
Sur la base des valeurs de mesures issues du découpage par fenêtres temporelles autour de la référence, nous avons cherché à étudier l'effet de la perception consciente de la cible auditive sur la dynamique temporelle des mesures au sein des clusters d'intérêts (formule R : Mesure $\sim$ Detection * Fenetre + $1|$Sujet). 

À chaque fois, des analyses de la variance ont été effectuées pour tester le modèle nul par rapport au modèle à effet mixte correspondant et les différents modèles ont tous été vérifiés sur la base de leurs résidus. 
Ensuite, une ANOVA a été réalisée sur le modèle à effets mixtes afin d'évaluer les effets et les interactions de chaque facteur expérimental sur la valeur de la mesure donnée. 
Dans le cas d'effets statistiques de facteurs et de leurs interactions, nous avons étudié les comparaisons appariées en utilisant les moyennes marginales estimées implémentées dans la bibliothèque R \texttt{emmeans} (voir Section~\ref{chapitre4analyses}).

%%%%%%%%%%%%%%%%%%%%%%%%%%%%%%%%%%%%%%%%%%%%%%%%%%%%%%%%%%%%%%%%%%%%%%%%%%%%%%%
\newpage
\subsection{Algorithmes de calcul des mesures d'entropie et de complexité}
\label{algorithmescalculmesures}
%%%%%%%%%%%%%%%%%%%%%%%%%%%%%%%%%%%%%%%%%%%%%%%%%%%%%%%%%%%%%%%%%%%%%%%%%%%%%%%

Dans la Section \ref{theorieinformationmesuresassociees}, nous avons présenté les différentes caractéristiques de contenu informationnel et de complexité ainsi que certaines des études les ayant utilisées pour caractériser les états de conscience. 
Nous présentons ici les algorithmes permettant l'estimation des différentes caractéristiques employées. 
Toutes les mesures d'entropie et de complexité ont été estimées à l'aide de procédures écrites en Python \citep{van2007python}, basées sur les modules antropy, pyEntropy et pyEEG \citep{bao2011pyeeg}.

%%%%%%%%%%%%%%%%%%%%%%%%%%%%%%%%%%%%%%%%%%%%%%%%%%%%%%%%%%%%%%%%%%%%%%%%%%%%%%%
\subsubsection{Mesures d'entropie}
%%%%%%%%%%%%%%%%%%%%%%%%%%%%%%%%%%%%%%%%%%%%%%%%%%%%%%%%%%%%%%%%%%%%%%%%%%%%%%%

%%%%%%%%%%%%%%%%%%%%%%%%%%%%%%%%%%%%%%%%%%%%%%%%%%%%%%%%%%%%%%%%%%%%%%%%%%%%%%%
\subsubsection*{Entropie spectrale}
%%%%%%%%%%%%%%%%%%%%%%%%%%%%%%%%%%%%%%%%%%%%%%%%%%%%%%%%%%%%%%%%%%%%%%%%%%%%%%%

L'entropie spectrale \textbf{SpEn} est un marqueur de complexité caractérisant l'irrégularité d'un signal temporel $x_t$ \citep{inouye1991quantification}. 
Elle est définie comme l'entropie de Shannon de la densité de puissance spectrale des données et peut être définie comme suit : \\

\begin{itemize}
\item[1.] Calculer le spectre $X(w_i)$ du signal.
\item[2.] Calculer la densité de puissance spectrale du signal via le carré de son amplitude et normaliser par le nombre de bins $N$ défini.
\begin{equation} 
P(\omega_i) = \frac{1}{N} |X(\omega_i)|^2
\end{equation}
\item[3.] Normaliser la densité de puissance spectrale calculée de façon à ce qu'elle puisse être considérée comme une fonction de masse de probabilité :
\begin{equation}
p_i = \frac{P(\omega_i)}{\sum_i P(\omega_i)}
\end{equation}
\item[4.] \textbf{SpEn} peut ensuite être calculée en utilisant la formule standard de l'entropie de Shannon.
\begin{equation}
\textbf{SpEn} = - \sum_{i=1}^n~p_i~\log p_i
\end{equation}
\end{itemize}

%%%%%%%%%%%%%%%%%%%%%%%%%%%%%%%%%%%%%%%%%%%%%%%%%%%%%%%%%%%%%%%%%%%%%%%%%%%%%%%
\subsubsection*{Entropie approximée}
%%%%%%%%%%%%%%%%%%%%%%%%%%%%%%%%%%%%%%%%%%%%%%%%%%%%%%%%%%%%%%%%%%%%%%%%%%%%%%%

L'entropie approximée \textbf{ApEn} est un marqueur de complexité caractérisant la régularité dans les fluctuations des données d'un signal temporel $x_t$ \citep{pincus1991regularity, pincus1991approximate}. 
Elle dépend de plusieurs paramètres dont les principaux sont le seuil de tolérance $r$, sous lequel une récurrence est trouvée (également appelé niveau de filtrage), la longueur du vecteur de données considérées $m$ (également appelée dimension d'intégration, voir \cite{kantz2004nonlinear}) et la durée $T$ de fenêtre temporelle.

L'entropie approximée \textbf{ApEn} est définie de la manière suivante : la série temporelle est intégrée dans un espace de phase\footnote{Dans la théorie des systèmes dynamiques, un espace des phases est un espace mathématique dans lequel tous les états possibles d'un système sont représentés ; chaque état possible correspondant à un point unique dans l'espace des phases. Un espace de phase est un espace multidimensionnel, dont chaque axe représente un degré de liberté du système. Chaque état du système ou combinaison de valeurs des variables d'état du système, est représenté par un point dans l'espace multidimensionnel. L'évolution dans le temps de l'état du système trace un chemin (une trajectoire) dans cet espace multidimensionnel.} de vecteurs $X_i$ de coordonnées retardées (appelées phases) :

\begin{equation}
X_i = [x_i,~x_{i-1},~\ldots~,~x_{i-m+1}]
\end{equation}
où $x_i$ est le ième échantillon de la série temporelle étudiée. 
L'intégrale de corrélation $C_i^m(r)$ indique la probabilité que le vecteur intégré $X_i$  soit similaire à d'autres vecteurs à l'intérieur d'un seuil $r$ : 

\begin{equation}
C_i^m(r) = \frac{N_i^r}{N-m},~~~~~~~~~~~~~~~~i=1,~\ldots~,N-m+1
 \end{equation}
où $N$ est le nombre d'échantillons de données et $N_i^r$ est le nombre de vecteurs dont la distance à $X_i$ est inférieure à $r$. 
La norme $L_\infty$ est choisie comme définition de la distance (\textit{i.e.}, la distance maximale entre des paires d'éléments de l'ensemble des vecteurs). 

Ainsi, la définition de l'intégrale de corrélation $C_i^m(r)$ nécessite de compter le nombre de récurrences $N_i^r$ de la trajectoire vers des points proches de $X_i$ et de le diviser par le nombre de paires possibles en estimant ainsi le pourcentage de points voisins de $X_i$ (\textit{i.e.}, la probabilité que la trajectoire ait des récurrences\footnote{L'étude des récurrences (également appelées points voisins ou correspondances) est très importante dans l'analyse non-linéaire des séries temporelles et conduit à la définition de nombreux indices de complexité et de non-linéarité.} proches d'elle). 
Finalement, l'entropie approximée \textbf{ApEn} est définie à partir du degré moyen de similarité $\phi^m(r)$ calculé sur la base de l'intégrale de corrélation pour deux dimensions d'intégration $m$ et $m+1$: \\

\begin{equation}
\textbf{ApEn} = \phi^m(r) - \phi^{m+1}(r)
\end{equation}
où 
\begin{equation}
\phi^m(r) = \frac{1}{(N-m+1)} ~ \sum_{i=1}^{N-m+1}~\log~C_i^m(r)
\end{equation}

Le nombre de récurrences est plus élevé dans la dimension inférieure. 
En effet, en augmentant la dimension de $m$ à $m+1$, un élément est ajouté aux vecteurs. 
Cela signifie que les récurrences en dimension $m+1$ sont également des récurrences en dimension $m$. 
Il peut arriver que deux vecteurs proches en dimension $m$ ne soient pas voisins en dimension $m+1$, signifiant que les derniers éléments ajoutés aux deux vecteurs sont plus éloignés que le seuil de tolérance $r$. 

L'entropie approximée \textbf{ApEn} est donc plus élevée lorsque la probabilité que les trajectoires divergent est plus grande. 
En effet, le logarithme dans la définition de $\phi^m(r)$ est monotone, de sorte que l'entropie augmente si le nombre de récurrences diminue lorsque la dimension d'intégration est augmentée (de $m$ à $m+1$).
L'entropie approximée est ainsi largement influencée par la longueur des données $N$, le seuil de tolérance $r$ et la dimension d'intégration $m$ et les valeurs suivantes ont été recommandées: $N=1000$, $r$ allant de $0.1$ à $0.25$~\% de l'écart-type du signal et $m=2-3$ \citep{bruhn2000approximate, pincus1991approximate}.

%%%%%%%%%%%%%%%%%%%%%%%%%%%%%%%%%%%%%%%%%%%%%%%%%%%%%%%%%%%%%%%%%%%%%%%%%%%%%%%
\subsubsection*{Entropie échantillonnée}
%%%%%%%%%%%%%%%%%%%%%%%%%%%%%%%%%%%%%%%%%%%%%%%%%%%%%%%%%%%%%%%%%%%%%%%%%%%%%%%

L'entropie échantillonnée \textbf{SaEn} est un autre marqueur de complexité caractérisant, tout comme l'entropie approximée, les fluctuations des données d'un signal temporel $x_t$ \citep{richman2000physiological}. 
L'entropie approximée \textbf{ApEn} présente le biais d'inclure de potentiels modèles d'auto-similarité dans les données ainsi que de dépendre de la taille du jeu de données. 
L'entropie échantillonnée \textbf{SaEn} essaie de compenser ces biais en utilisant une procédure de calcul légèrement différente de l'entropie approximée \textbf{ApEn}.
L'entropie échantillonnée \textbf{SaEn} est définie de la manière suivante : \\

\begin{itemize}
\item[1.] Soit $[x_1,~\ldots~,~x_N]$, une série temporelle de longueur $N$ ; 
\item[2.] Soit  $X_i$ la série intégrée pour $1\leq i \leq N-m+1$, des vecteurs de longueur $m$ :
\begin{equation}
X_i = [x_i,~x_{i+1},~\ldots~,~x_{i+m-1}]
\end{equation}
\item[3.] Soit $n_i^m(r)$, le nombre de vecteurs $x_j$ à une distance $r$ des vecteurs $x_i$, où $j \neq i$ et $j=1,\ldots,N-m+1$ afin d'exclure les modèles auto-similaires ;
\item[4.] Soit $C_i^m(r)$, qui est $(N-m)^{-1}$ fois le nombre de $n_i^m(r)$, est défini comme la probabilité que tout $x_j$ se trouve à une distance $r$ de $x_i$ ;
\item[5.] Soit $\phi^m(r)$ le degré moyen de similarité pouvant être calculé comme :
\begin{equation}
\phi^m(r) = \frac{\sum_{i=1}^{N-m+1} \log C_i^m(r)}{N-m+1}
\end{equation}
\item[6.] De même, $\phi^{m+1}(r)$ peut être calculé pour la dimension intégrée de $m+1$ : 
\begin{equation}
\textbf{SaEn} = - \log \frac{\phi^{m+1}(r)}{\phi^m(r)}
\end{equation}
\end{itemize}
où $\phi^m(r)$ représente la probabilité que deux séquences correspondent en dimension $m$, et $\phi^{m+1}(r)$ correspond à la probabilité que deux séquences correspondent en dimension $m+1$. 

De cette manière, l'entropie échantillonnée \textbf{SaEn} n'inclut pas les modèles auto-similaires et ne dépend pas de la taille des données. 
Des valeurs de paramètres similaires à celles de l'entropie approximée \textbf{ApEn} ont été recommandées dans la littérature pour l'entropie échantillonnée \textbf{SaEn} \citep{wang2018real, zhang2015performance, zurek2018bootstrapping}. 

%%%%%%%%%%%%%%%%%%%%%%%%%%%%%%%%%%%%%%%%%%%%%%%%%%%%%%%%%%%%%%%%%%%%%%%%%%%%%%%
\subsubsection*{Entropie de permutation}
%%%%%%%%%%%%%%%%%%%%%%%%%%%%%%%%%%%%%%%%%%%%%%%%%%%%%%%%%%%%%%%%%%%%%%%%%%%%%%%

L'entropie de permutation \textbf{PeEn} est un marqueur de complexité capturant les relations d'ordre entre les valeurs d'un signal temporel $x_t$ associé à un système dynamique \citep{bandt2002permutation}. 
Le signal temporel est transformé en une séquence de symboles discrets et l'entropie du signal est quantifiée à partir des densités de probabilité de ces symboles. 
La transformation est effectuée en extrayant des sous-vecteurs du signal, chacun comprenant $m$ mesures séparées par un délai temporel fixe $\tau$. 
Comme \textbf{ApEn} et \textbf{SaEn}, l'entropie de permutation \textbf{PeEn} repose sur trois paramètres : la dimension d'intégration $m$, le délai d'intégration $\tau$ et la longueur du signal $N$.
L'entropie de permutation \textbf{PeEn} est calculée telle que : \\

\begin{itemize}
\item[1.] Soit une série temporelle d'entrée $[x_0,~x_1,~\ldots~,~x_{N-1}]$, et une dimension d'intégration $m>1$ ;
\item[2.] Pour chaque sous-séquence extraite au temps $t$, $[x_{t-(m-1)},~x_{t-(m-2)},~\ldots~,~x_{t-1},~x_t]$, un modèle de rangs $\pi$ relatif à $t$ est obtenu sous la forme $\pi=[r_0,~r_1,~\ldots~,~r_{m-1}]$ ; 
\item[3.] Ce modèle de rangs est défini par un pattern d'ordre : $x_{t-r_{m-1}}~\leq~x_{t-r_{m-2}}~\leq~\ldots~\leq~x_{t-r_1}~\leq~x_{t-r_0}$ ; 
\item[4.] Pour toutes les $m!$ permutations possibles, chaque probabilité $p(\pi)$ est estimée comme la fréquence relative de chaque motif $\pi$ différent trouvé ; 
\item[5.] Une fois que toutes ces probabilités ont été obtenues, la valeur finale de l'entropie de permutation \textbf{PeEn} est donnée par :
\begin{equation}
\textbf{PeEn} = -\sum_{j=0}^{m!-1}~p(\pi_j)~\log~p(\pi_j)
\end{equation}
\end{itemize}

Les valeurs de paramètres qui ont été recommandées dans la littérature pour l'entropie de permutation \textbf{PeEn} sont : $3 \leq m \leq 7$, $\tau = 1$ et $N>>m!$ \citep{bandt2002permutation, cuesta2019embedded}. 

%%%%%%%%%%%%%%%%%%%%%%%%%%%%%%%%%%%%%%%%%%%%%%%%%%%%%%%%%%%%%%%%%%%%%%%%%%%%%%%
\subsubsection*{Entropie de décomposition en valeurs singulières}
%%%%%%%%%%%%%%%%%%%%%%%%%%%%%%%%%%%%%%%%%%%%%%%%%%%%%%%%%%%%%%%%%%%%%%%%%%%%%%%

L'entropie de décomposition en valeurs singulières \textbf{SvEn} est un marqueur de complexité caractérisant la dimension des données d'un signal temporel $x_t$ \citep{varshavsky2006novel, banerjee2014feature} et représente un outil qui peut compléter les méthodes d'analyse non-linéaire existantes pour tester la complexité des séries temporelles \citep{alvarez2021singular}.
Elle indique le nombre de vecteurs propres nécessaires à une explication adéquate des données associées au signal. 
Globalement, l'entropie de décomposition en valeurs singulières \textbf{SvEn} est calculée à partir de la distribution des valeurs singulières d'une matrice $M$ comprenant tous les vecteurs construits selon une procédure de délais. 
D'abord, on construit un vecteur de délais $y_i$ sur la base d'un signal $[x_1, x_2, \ldots~, x_n]$ tel que : 

\begin{equation}
y_i = [x_i,~x_{i+\tau},~\ldots~,~x_{i+(m-1)*\tau}]
\end{equation}
où $\tau$ correspond au délai considéré et $m$ à la dimension d'intégration. 

On construit ensuite une matrice d'intégration $Y$ telle que : 

\begin{equation}
Y = [y_i,~y_2,~\ldots~,~x_{N-(m-1)*\tau}]^T
\end{equation}

On réalise ensuite une décomposition en valeurs singulières sur la matrice $Y$ afin de produire $M$ valeurs $\sigma_1, \sigma_2, \ldots~, \sigma_M$, qui vont représenter un spectre singulier de valeurs. 

L'entropie de décomposition en valeurs singulières \textbf{SvEn} peut alors être définie comme : 

\begin{equation}
\textbf{SvEn} = -\sum_{i=1}^M~\sigma_i~\log~\sigma_i
\end{equation}

%%%%%%%%%%%%%%%%%%%%%%%%%%%%%%%%%%%%%%%%%%%%%%%%%%%%%%%%%%%%%%%%%%%%%%%%%%%%%%%
\subsubsection{Mesures de complexité}
%%%%%%%%%%%%%%%%%%%%%%%%%%%%%%%%%%%%%%%%%%%%%%%%%%%%%%%%%%%%%%%%%%%%%%%%%%%%%%%

%%%%%%%%%%%%%%%%%%%%%%%%%%%%%%%%%%%%%%%%%%%%%%%%%%%%%%%%%%%%%%%%%%%%%%%%%%%%%%%
\subsubsection*{Exposant de Hurst}
%%%%%%%%%%%%%%%%%%%%%%%%%%%%%%%%%%%%%%%%%%%%%%%%%%%%%%%%%%%%%%%%%%%%%%%%%%%%%%%

L'exposant de Hurst $\textbf{H}$ est un marqueur de complexité caractérisant le degré de dépendance à long-terme d'un signal temporel $x_t$ \citep{hurst1951long}.  
De nombreux estimateurs de la dépendance long-terme ont été proposés précédemment. 
Le plus connu parmi eux est obtenu par analyse de l'étendue de mise à l'échelle (ang: «rescaled range», noté RS) \citep{mandelbrot1968noah, mandelbrot1969robustness}, laquelle est basée sur les travaux de \cite{hurst1951long}. 
D'autres estimateurs peuvent être obtenus comme par exemple grâce à l'analyse de fluctuation à tendance retirée (ang: «detrended fluctuation analysis», notée DFA). 

Pour obtenir l'exposant de Hurst, il est nécessaire d'estimer la dépendance de l'étendue de mise à l'échelle sur une durée d'observation $n$. 
Une série temporelle $X_i : [x_1,~x_2,~\ldots~,~x_n]$ de longueur totale $N$ est divisée en un nombre de vecteurs plus courts de longueur $n=N, \frac{N}{2}, \frac{N}{4},\ldots$. 

L'étendue de mise à l'échelle moyenne est ensuite calculée pour chaque valeur de $n$ comme suit : 

\begin{itemize}
\item[1.] Calcul de la moyenne :
\begin{equation}
m = \frac{1}{n} \sum_{i=1}^n x_i
\end{equation}
\item[2.] Nouvelle série ajustée à la moyenne :
\begin{equation}
y_t = x_t - m
\end{equation}
\item[3.] Calcul de la série déviante cumulative $z$ :
\begin{equation}
z_t = \sum_{i=1}^t y_i
\end{equation}
\item[4.] Calcul de l'étendue $R$ :
\begin{equation}
R(n) = \max (z_1, \ldots~, z_n) - \min (z_1, \ldots~, z_n)
\end{equation}
\item[5.] Calcul de la déviation standard :
\begin{equation}
S(n) = \sqrt{\frac{1}{n} \sum\nolimits_{i=1}^n (x_i - m)^2}
\end{equation}
\item[6.] Calcul de l'étendue mise à l'échelle $\frac{R(n)}{S(n)}$ et moyennage sur tous les vecteurs de longueur $n$. 
$R(n)$ est l'étendue des premiers $n$ écarts cumulatifs par rapport à la moyenne, $S(n)$ leur écart-type, $E[x]$ leur espérance et $n$ la durée d'observation, c'est-à-dire le nombre de points de la série temporelle. 
L'exposant de Hurst est ensuite estimé en modélisant une loi de puissance $E[\frac{R(n)}{S(n)}]=Cn^\textbf{H}$. 
Cela peut être fait en traçant $\log[\frac{R(n)}{S(n)}]$ en fonction de $\log(n)$ et en calculant l'ajustement linéaire. 
La pente de la droite donne ainsi l'exposant de Hurst $\textbf{H}$. 
\end{itemize}

%%%%%%%%%%%%%%%%%%%%%%%%%%%%%%%%%%%%%%%%%%%%%%%%%%%%%%%%%%%%%%%%%%%%%%%%%%%%%%%
\subsubsection*{Exposant fractal}
%%%%%%%%%%%%%%%%%%%%%%%%%%%%%%%%%%%%%%%%%%%%%%%%%%%%%%%%%%%%%%%%%%%%%%%%%%%%%%%

L'analyse de fluctuation à tendance retirée (DFA) permet de caractériser le degré d'auto-similarité statistique d'un signal $x(t)$ en mesurant son exposant fractal $\alpha$ \citep{peng1993long}. 
D'abord, la série $x(t)$ de taille $N$ est intégrée, en calculant pour chaque temps $t$ l'écart cumulé par rapport à la moyenne :
\begin{equation}
X_t = \sum_{i=1}^t (x_i - \overline{x})
\end{equation}
où $X_t$ est appelée somme cumulative ou "profil". 
Ensuite, la série intégrée $X_t$ est segmentée en fenêtres temporelles non-chevauchantes de longueur de $n$ échantillons chacune. 
Un ajustement linéaire local des moindres carrés (\textit{i.e.}, tendance locale) est calculé dans chaque fenêtre temporelle. 
La série $X_t$ est localement «détrendée» en soustrayant les valeurs théoriques données par la régression, c'est-à-dire que la tendance linéaire calculée est soustraite à la série pour une série «détrendée». 
$Y_t$ indique la séquence résultante des ajustements de la droite.
Pour une fenêtre temporelle de longueur $n$ donnée, la taille caractéristique de la fluctuation pour cette série intégrée et détrendée est calculée par :
\begin{equation}
F(n) = \sqrt{\frac{1}{N} \sum_{t=1}^N (X_t - Y_t)^2}
\end{equation}
Ce processus est répété sur plusieurs tailles de fenêtre différentes $n$ (en pratique, la longueur la plus courte est d'environ 10, et la plus grande $\frac{N}{2}$, donnant deux fenêtres adjacentes). 
Enfin, un graphique log-log de $F(n)$ est construit. 
L'exposant fractal $\alpha$ est estimé comme la pente de la droite ajustée au graphique log-log, laquelle indique l'auto-similarité statistique. 
Une loi d'échelle de la forme suivante est attendue : 
\begin{equation}
F(n) \propto n^\alpha
\end{equation}

%%%%%%%%%%%%%%%%%%%%%%%%%%%%%%%%%%%%%%%%%%%%%%%%%%%%%%%%%%%%%%%%%%%%%%%%%%%%%%%
\subsubsection*{Dimension fractale}
%%%%%%%%%%%%%%%%%%%%%%%%%%%%%%%%%%%%%%%%%%%%%%%%%%%%%%%%%%%%%%%%%%%%%%%%%%%%%%%

La dimension fractale $D$ est un marqueur de complexité qui caractérise le degré «fractal» d'un signal en le quantifiant en tant que rapport du changement de détail par rapport au changement d'échelle \citep{goh2005comparison, klonowski2002complexity}. 
Lorsqu'on pense aux dimensions d'un objet, on les associe à la dimension topologique de cet objet, par exemple, une ligne a une dimension topologique de 1, un carré de 2 et un cube de 3. 
Cela peut être exprimé en termes de longueur de l'objet en tant que normes de dimension : $L^1$, $L^2$ et $L^3$ pour une ligne, un carré et un cube respectivement, où l'exposant est la dimension. 
Cependant, la dimension topologique ne convient pas pour mesurer des dimensions d'objets fractals, nombres réels non-entiers. 
Au lieu de cela, la dimension de Hausdorff, qui peut être une valeur réelle non-entière, est usuellement utilisée \citep{hausdorff1918dimension}. 
En géométrie fractale, où les dimensions ne sont plus des nombres entiers, des relations de mise à l'échelle peuvent être définies par une règle générale de mise à l'échelle :

\begin{equation}
N = \varepsilon^{-D}
\end{equation}
\begin{equation}
\log_\varepsilon N = -D = \frac{\log N}{\log \varepsilon}
\end{equation}
où $N$ représente le nombre de points, $\varepsilon$ le facteur d'échelle et $D$ la dimension fractale.

Dans le cadre du signal EEG, $D$ peut être obtenue en divisant le signal en plus petites sections similaires de longueur $m$. 
Il est possible d'obtenir un nombre de patterns auto-similaires $S$ formant le signal d'origine en agrandissant chaque sous-section d'un facteur $\varepsilon$ et en l’élevant à la puissance de la dimension $D$. 
Cela est régi par une loi de puissance exprimée comme :

\begin{equation}
S \propto \varepsilon^D
\end{equation}
La dimension fractale classique de Hausdorff peut alors être obtenue comme :
\begin{equation}
D_{Hausdorff} = \frac{\log S}{ \log \varepsilon}
\end{equation}
$D$ mesure ainsi le taux d’addition de détails avec une échelle (\textit{i.e.}, une résolution) accrue. 
Usuellement, la dimension fractale de l'EEG oscille entre 1 et 2 car elle caractérise la complexité d'un signal considéré sur un plan bidimensionnel (voltage et temps). 

\cite{higuchi1988approach} a proposé un algorithme pour estimer la dimension fractale directement dans le domaine temporel. 
Cette méthode donne une estimation raisonnable de $D$ dans le cas de segments de signal courts et présente un temps de calcul rapide. 
L'algorithme de Higuchi est tel que : \\

\begin{itemize}
\item[1.] Soit un signal donné : $x=\{x_1,~x_2,~\ldots~,~x_N\}$ pour lequel $k$ nouvelles courbes $x_m^k$ sont construites comme suit :
\begin{equation}
x_m^k=\{x_m,~x_{m+k},~\ldots~,~x_{m+(\frac{N-m}{k})~\cdot~k}\}
\end{equation}
avec $m=1~,~2,~\ldots~,~k$ où $m$ et $k$ sont des entiers indiquant le temps initial et le temps d'intervalle, respectivement. 
\item[2.] La longeur, $L_m(k)$ de chaque courbe $s_m^k$ est calculée comme : 
\begin{equation}
L_m(k)=\frac{1}{k}~\left[~\left(~\sum_{i=1}^{\frac{N-m}{k}}~|~x_{m+ik}- x_{m+(i-1)k}~|~\right)~\cdot~\frac{N-1}{\frac{N-m}{k}}~\right]
\end{equation}
\item[3.] La longueur de la courbe pour l'intervalle $k$, $L(k)$, est calculée comme la moyenne des $m$ courbes $L_m(k)$ for $m=1,~\ldots~,~k$. 
Si $L(k)~\propto~k^{-D}$, alors le signal est fractal de dimension fractale $D$ et si $L(k)$ est représenté graphiquement contre $1/k$, où $k~=~1,~\ldots~,~k_{max}$, dans une échelle log-log, les points de données tombent sur la ligne droite de pente $D$. 
\item[4.] Enfin, un ajustement linéaire par moindres carrés est appliqué sur les paires $\{\log 1/k, \log L(k)\}$, et la pente de la ligne obtenue donne l'estimation de la dimension fractale $D$. 
\end{itemize}

\cite{katz1988fractals} a proposé un algorithme dans lequel la dimension fractale peut également être obtenue directement à partir de la série temporelle. 
L'algorithme de Katz est basé sur le traitement suivant : \\
\begin{equation}
D = \frac{\log_{10} (L) }{\log_{10} (d)}
\end{equation}
où $L$ est la longueur totale de la série temporelle EEG définie comme le nombre de point total et $d$ est la distance (euclidienne) entre le premier point de la série et le point qui fournit la distance la plus éloignée vis-à-vis du premier point. 
Afin d'outrepasser la dépendance aux unités de mesures utilisées pour le calcul, \cite{katz1988fractals} a proposé une normalisation exprimée comme :
\begin{equation}
D_{Katz} = \frac{\log_{10} (\frac{L}{a}) }{\log_{10} (\frac{d}{a})}
\end{equation}
où $a$ est le nombre moyen d'étapes ou distance moyenne entre les points successifs de la série. Ainsi :
\begin{equation}
D_{Katz} = \frac{\log_{10} (n) }{\log_{10} (\frac{d}{L}) + \log_{10}(n)}
\end{equation}

Enfin, \cite{petrosian1995kolmogorov} a proposé un algorithme pouvant également être utilisé pour fournir un calcul rapide de $D$ d'un signal en traduisant la série en une séquence binaire. 
Il existe plusieurs variantes de l'algorithme différant principalement par la façon dont la séquence binaire est créée. 
Dans un algorithme nommé Petrosian C, des échantillons consécutifs de la série temporelle sont soustraits et la séquence binaire est créée en fonction du résultat de la soustraction. 
Un "+1" ou "-1" est attribué à chaque résultat positif ou négatif respectivement. 
Dans un algorithme nommé Petrosian D, la séquence binaire est formée en attribuant un 1 pour chaque différence entre des échantillons consécutifs dans la série temporelle qui dépasse une amplitude d'une fois l'écart type et un 0 est attribué autrement. 
$D$ est alors calculé comme :
\begin{equation}
D_{Petrosian} = \frac{\log_{10} n}{\log_{10} n + 
\log_{10} \left( \frac{n}{n+0.4N_\Delta} \right)}
\end{equation}
où $n$ est la longueur de la séquence et $N_\Delta$ est le nombre de changements de signe dans la séquence binaire. 

%%%%%%%%%%%%%%%%%%%%%%%%%%%%%%%%%%%%%%%%%%%%%%%%%%%%%%%%%%%%%%%%%%%%%%%%%%%%%%%
\subsubsection*{Paramètres de Hjorth}
%%%%%%%%%%%%%%%%%%%%%%%%%%%%%%%%%%%%%%%%%%%%%%%%%%%%%%%%%%%%%%%%%%%%%%%%%%%%%%%

Les paramètres de Hjorth sont des marqueurs caractérisant les propriétés statistiques d'un signal temporel $x_t$ \citep{hjorth1970eeg}. 
L'activité de Hjorth représente la variance de la fonction temporelle du signal et est simplement définie comme :
\begin{equation}
A_{Hjorth} = Var(x_t)
\end{equation}
La mobilité de Hjorth représente la fréquence moyenne du signal et est définie comme la racine carrée du ratio entre la variance de la dérivée première du signal $x_t$ et la variance du signal $x_t$ :
\begin{equation}
M_{Hjorth} = \sqrt{ \frac{Var(x'_t)}{Var(x_t)}}
\end{equation}
La complexité de Hjorth représente la mesure des variations fréquentielles au cours d'une période temporelle spécifique et indique comment la forme du signal est similaire à celle d'une onde sinusoïdale pure. 
La valeur converge vers 1 avec la similarité entre le signal et la sinusoïde : 
\begin{equation}
C_{Hjorth} = \frac{M_{Hjorth}(x'_t)}{M_{Hjorth}(x_t)}
\end{equation}

%%%%%%%%%%%%%%%%%%%%%%%%%%%%%%%%%%%%%%%%%%%%%%%%%%%%%%%%%%%%%%%%%%%%%%%%%%%%%%%
\subsection{Résultats des mesures d'entropies}
\label{resultatsmesuresentropies}
%%%%%%%%%%%%%%%%%%%%%%%%%%%%%%%%%%%%%%%%%%%%%%%%%%%%%%%%%%%%%%%%%%%%%%%%%%%%%%%

Nous présentons tout d'abord les résultats obtenus pour les mesures d'entropie calculées. 
Ces mesures sont au nombre de cinq : Entropie spectrale \textbf{SpEn}, Entropie approximée \textbf{ApEn}, Entropie échantillonnée \textbf{SaEn}, Entropie de permutation \textbf{PeEn} et Entropie de décomposition en valeurs singulières \textbf{SvEn}.
La Figure~\ref{fig:figure5valeursmesuresentropie} présente les valeurs moyennes et leurs erreurs standards pour ces cinq mesures d'entropie calculées pour chacun des clusters d'électrodes. 
Ces valeurs ont été obtenues pour des fenêtres de $3$~sec avant et des fenêtres de $3$~sec après la référence temporelle. 
Les mesures ont été calculées à partir du signal issu de chaque électrode. 
Ensuite, ces mesures ont été agrégées de manière topographique en utilisant la moyenne arithmétique de l'ensemble des électrodes d'un cluster. 
Ainsi, cela rendait une valeur de mesure par électrode et une valeur de mesure par cluster. 

Nous avons représenté sur la Figure~\ref{fig:figure5valeursmesuresentropie} les différentes valeurs des mesures par cluster de manière ordonnée par amplitude. 
On observe que le cluster fronto-central exprime les valeurs les plus élevées pour l'ensemble des cinq mesures d'entropie lorsque la cible est détectée par le sujet. 
Le cluster temporal et frontal présentent ensuite les valeurs les plus élevées. 
Au contraire, les valeurs les plus faibles sont observées pour les clusters pariétal et sagittal principalement. 
La variabilité au sein du cluster fronto-central est la plus haute à la fois pour les conditions avant et après pour les cinq mesures. 

Ainsi, un premier modèle linéaire à effets mixtes a été ajusté pour étudier l'effet des facteurs détection, condition et cluster sur les valeurs des mesures d'entropie (formule R : Mesure d'entropie $\sim$ Détection * Condition * Cluster + $1|$Sujet). 
La Table~\ref{tab:table5statsmesuresentropie} reporte les résultats des analyses statistiques des modèles linéaire à effets mixtes pour le modèle simple réalisées sur les mesures d'entropie avant et après la référence. \\

\begin{figure*}[!t]
\begin{multicols}{2}
\includegraphics[width=0.5\textwidth]{/home/link/Documents/thèse_onera/diapos_phd_thesis/images/EEG/info_content/entropy/spen_cluster_by_detection_eb.jpeg}
\includegraphics[width=0.5\textwidth]{/home/link/Documents/thèse_onera/diapos_phd_thesis/images/EEG/info_content/entropy/apen_cluster_by_detection_eb.jpeg}
\includegraphics[width=0.5\textwidth]{/home/link/Documents/thèse_onera/diapos_phd_thesis/images/EEG/info_content/entropy/saen_cluster_by_detection_eb.jpeg}
\includegraphics[width=0.5\textwidth]{/home/link/Documents/thèse_onera/diapos_phd_thesis/images/EEG/info_content/entropy/peen_cluster_by_detection_eb.jpeg}
\end{multicols}
\centering \includegraphics[width=0.5\textwidth]{/home/link/Documents/thèse_onera/diapos_phd_thesis/images/EEG/info_content/entropy/sven_cluster_by_detection_eb.jpeg}
\caption[Valeurs des mesures d'entropie calculées pour chacun des clusters avant et après la détection/non-détection]{Valeurs moyennes et barres d'erreur standard des mesures d'entropie calculées pour chacun des clusters (AF : Antéro-Frontal, F : Frontal, FC : Fronto-Central, C : Central, CP : Centro-Pariétal, P : Pariétal, PO : Pariéto-Occipital, T : Temporal et S : Sagittal) avant et après la détection (hit, en rouge clair) ou la non-détection (miss, en bleu clair) de la cible. Les mesures ont été calculées à partir de chaque électrode du cluster puis obtenues par agrégation topographique des valeurs. Globalement, le cluster fronto-central exprime les valeurs les plus élevées pour les cinq mesures d'entropie lorsque la cible est détectée (hit) par le sujet.}
\label{fig:figure5valeursmesuresentropie}
\end{figure*}

\begin{table}[!t]
\centering
\scriptsize
\caption[Table des résultats des analyses statistiques pour les mesures d'entropie]{Table des résultats des analyses statistiques du modèle linéaire à effets mixtes Détection * Condition * Cluster + $1~|~$Sujet réalisées sur les mesures d'entropie avant et après la référence temporelle. Seuls les résultats des analyses des facteurs principaux et de leurs interactions sont reportés. Les résultats des comparaisons mutliples pertinentes pour nous sont reportées directement dans le texte.}
\label{tab:table5statsmesuresentropie}

\textbf{Mesure d'entropie $\sim$ Détection * Condition * Cluster + $1~|~$Sujet}

% \begin{tabular}{lllllllll}
\begin{tabular}{|l|*{9}{c|}}
\hline
& \textbf{Measure} & \textbf{Sum Sq} & \textbf{Mean Sq} & \textbf{NumDF} & \textbf{DenDF} & \textbf{F value} & \textbf{Pr($>$F)} & \textbf{Sign.} \\ 
\hline
\textcolor{blue}{Détection} & & & & & & & & \\ 
\hline
& \textit{SpEn} & 0.004 & 0.004 & 1 & 647 & 6.644 & 0.010 & ** \\ 
& \textit{ApEn} & 0.016 & 0.016 & 1 & 647 & 7.937 & 0.005 & ** \\ 
& \textit{SaEn} & 0.014 & 0.014 & 1 & 647 & 5.540 & 0.019 & * \\ 
& \textit{PeEn} & 0.012 & 0.012 & 1 & 647 & 91.162 & $<$.0001 & *** \\ 
& SvEn & 0.001 & 0.001 & 1 & 647 & 1.653 & 0.199 & \\ 
\hline
\textcolor{blue}{Condition} & & & & & & & & \\ 
\hline
& SpEn & 0.001 & 0.001 & 1 & 646 & 1.091 & 0.297 & \\ 
& ApEn & 0.002 & 0.002 & 1 & 647 & 0.846 & 0.358 & \\ 
& SaEn & 0.002 & 0.002 & 1 & 647 & 0.685 & 0.408 & \\ 
& PeEn & 0.000 & 0.000 & 1 & 647 & 0.124 & 0.725 & \\ 
& SvEn & 0.001 & 0.001 & 1 & 646 & 1.637 & 0.201 & \\ 
\hline
\textcolor{blue}{Cluster} & & & & & & & & \\ 
\hline
& \textit{SpEn} & 0.266 & 0.033 & 8 & 646 & 51.210 & $<$.0001 & *** \\ 
& \textit{ApEn} & 0.793 & 0.099 & 8 & 647 & 49.651 & $<$.0001 & *** \\ 
& \textit{SaEn} & 0.907 & 0.113 & 8 & 647 & 46.062 & $<$.0001 & *** \\ 
& \textit{PeEn} & 0.152 & 0.019 & 8 & 647 & 147.722 & $<$.0001 & ***\\ 
& \textit{SvEn} & 0.285 & 0.036 & 8 & 646 & 46.043 & $<$.0001 & *** \\ 
\hline
\textcolor{blue}{Détection~*~Condition} & & & & & & & & \\ 
\hline
& SpEn & 0.000 & 0.000 & 1 & 646 & 0.421 & 0.517 & \\ 
& ApEn & 0.000 & 0.000 & 1 & 647 & 0.080 & 0.778 & \\ 
& SaEn & 0.000 & 0.000 & 1 & 647 & 0.003 & 0.957 & \\ 
& \textit{PeEn} & 0.001 & 0.001 & 1 & 647 & 4.539 & 0.034 & * \\ 
& SvEn & 0.001 & 0.001 & 1 & 646 & 1.488 & 0.223 & \\ 
\hline
\textcolor{blue}{Détection~*~Cluster} & & & & & & & & \\ 
\hline
& \textit{SpEn} & 0.015 & 0.002 & 8 & 646 & 2.817 & 0.004 & ** \\ 
& \textit{ApEn} & 0.154 & 0.019 & 8 & 647 & 9.611 & $<$.0001 & *** \\ 
& \textit{SaEn} & 0.155 & 0.019 & 8 & 647 & 7.883 & $<$.0001 & *** \\ 
& \textit{PeEn} & 0.079 & 0.010 & 8 & 647 & 76.564 & $<$.0001 & *** \\ 
& \textit{SvEn} & 0.017 & 0.002 & 8 & 646 & 2.696 & 0.006 & ** \\ 
\hline
\textcolor{blue}{Condition~*~Cluster} & & & & & & & & \\ 
\hline
& SpEn & 0.001 & 0.000 & 8 & 646 & 0.253 & 0.980 & \\ 
& ApEn & 0.005 & 0.001 & 8 & 647 & 0.291 & 0.969 & \\ 
& SaEn & 0.006 & 0.001 & 8 & 647 & 0.294 & 0.968 & \\ 
& PeEn & 0.000 & 0.000 & 8 & 647 & 0.131 & 0.998 & \\ 
& SvEn & 0.001 & 0.000 & 8 & 646 & 0.186 & 0.993 & \\ 
\hline
\textcolor{blue}{Détection~*~Condition~*~Cluster} & & & & & & & & \\ 
\hline
& SpEn & 0.001 & 0.000 & 8 & 646 & 0.269 & 0.976 & \\ 
& ApEn & 0.007 & 0.001 & 8 & 647 & 0.421 & 0.909 & \\ 
& SaEn & 0.009 & 0.001 & 8 & 647 & 0.464 & 0.882 & \\ 
& PeEn & 0.000 & 0.000 & 8 & 647 & 0.062 & 1.000 & \\ 
& SvEn & 0.001 & 0.000 & 8 & 646 & 0.240 & 0.983 & \\ 
\hline
\end{tabular}
\end{table}

Les effets significatifs mis en évidence par l'analyse de variance sont : 
\begin{itemize}
\item[$\bullet$] \underline{pour l'entropie spectrale \textbf{SpEn}} : 
\begin{itemize} 
\item l'effet principal de la détection est significatif et faible ($F(1)=6.64$, $p=0.01$, $\eta^2=0.01$) ; 
\item l'effet principal du cluster est significatif et important ($F(8)=51.21$, $p<.001$, $\eta^2=0.39$) ; 
\item l'interaction entre la détection et le cluster est significative et faible ($F(8)=2.82$, $p=0.004$, $\eta^2=0.03$) ;
\end{itemize}
\item[$\bullet$] \underline{pour l'entropie approximée \textbf{ApEn}} :
\begin{itemize} 
\item l'effet principal de la détection est significatif et faible ($F(1)=7.94$, $p=0.005$, $\eta^2=0.01$) ; 
\item l'effet principal du cluster est significatif et important ($F(8)=49.65$, $p<.001$, $\eta^2=0.38$) ; 
\item l'interaction entre la détection et le cluster est significative et moyenne ($F(8)=9.61$, $p<.001$, $\eta^2=0.11$) ; 
\end{itemize}
\item[$\bullet$] \underline{pour l'entropie échantillonnée \textbf{SaEn}} :
\begin{itemize} 
\item l'effet principal de la détection est significatif et très faible ($F(1)=5.54$, $p=0.019$, $\eta^2=8.49\times10^{-3}$) ; 
\item l'effet principal du cluster est significatif et important ($F(8)=46.06$, $p<.001$, $\eta^2=0.36$) ; 
\item l'interaction entre la détection et le cluster est significative et moyenne ($F(8)=7.88$, $p<.001$, $\eta^2=0.09$) ; 
\end{itemize}
\item[$\bullet$] \underline{pour l'entropie de permutation \textbf{PeEn}} :
\begin{itemize}
\item l'effet principal de la détection est significatif et moyen ($F(1)=91.16$, $p<.001$, $\eta^2=0.12$) ; 
\item l'effet principal du cluster est significatif et important ($F(8)=147.72$, $p<.001$, $\eta^2=0.65$) ; 
\item l'interaction entre la détection et la condition est significative et très faible ($F(1)=4.54$, $p=0.034$, $\eta^2=6.97\times10^{-3}$) ; 
\item l'interaction entre la détection et le cluster est significative et importante ($F(8)=76.56$, $p<.001$, $\eta^2=0.49$) ; 
\end{itemize}
\item[$\bullet$] \underline{pour l'entropie de décomposition en valeurs singulières \textbf{SvEn}} :
\begin{itemize} 
\item l'effet principal du cluster est significatif et important ($F(8)=46.04$, $p<.001$, $\eta^2=0.36$) ; 
\item l'interaction entre la détection et le cluster est significative et faible ($F(8)=2.70$, $p=0.006$, $\eta^2=0.03$). \\
\end{itemize}
\end{itemize}

En résumé, dans le modèle Détection * Condition * Cluster + $1|$Sujet, le facteur cluster et l'interaction détection * cluster ont tous les deux montré des effets significatifs sur toutes les mesures d'entropie mais l'interaction condition * détection * cluster n'a pas d'effet significatif sur les mesures d'entropie. 
Du fait de l'effet significatif de l'interaction détection * cluster sur les mesures d'entropie, nous avons cherché une éventuelle différence entre ces clusters. 
Les résultats des comparaisons mutliples, montrent une augmentation des valeurs de mesure d'entropie dans le cluster fronto-central pour les cibles détectées par rapport aux cibles manquées pour : 

\begin{itemize}
\begin{multicols}{2}
\item[$\bullet$] \textbf{SpEn} ($t=4.71$, $p<.001$) ; 
\item[$\bullet$] \textbf{ApEn} ($t=8.70$, $p<.001$) ; 
\item[$\bullet$] \textbf{SaEn} ($t=7.88$, $p<.001$) ; 
\item[$\bullet$] \textbf{PeEn} ($t=24.72$, $p<.001$) et 
\item[$\bullet$] \textbf{SvEn} ($t=4.54$, $p<.001$). 
\end{multicols}
\end{itemize}

À partir de ces résultats, nous avons cherché à obtenir plus de précisions sur la localisation topographique des variations d'entropie du cluster fronto-central. 
Nous avons ajusté un second modèle linéaire à effets mixtes pour étudier l'effet des facteurs détection, condition et électrodes sur les valeurs des mesures d'entropie (formule R : Mesure d'entropie $\sim$ Détection * Condition * Électrode Cluster FC + $1|$Sujet). 
La Figure~\ref{fig:figure5valeursmesuresentropieelectrode} présente les valeurs moyennes et barres d'erreur standard des mesures d'entropie calculées pour chacune des électrodes du cluster fronto-central avant et après le temps de référence. 
On observe que les valeurs des mesures d'entropie au sein du cluster fronto-central augmentent progressivement avec la latéralité pour les cibles détectées. 
Les électrodes fronto-centrales FC5 et FC6 expriment les valeurs les plus élevées tandis que les électrodes plus centrales, FC1 et FC2, montrent les valeurs les plus faibles. 
On voit également que la variabilité est la plus élevée pour les deux électrodes FC5 et FC6 pour les cibles détectées à la fois avant et après la détection. 
La Table~\ref{tab:table5statsmesuresentropieelectrode} reporte les résultats des analyses statistiques par modélisation linéaire à effets mixtes pour ce deuxième modèle. \\

\begin{figure*}[!t]
\begin{multicols}{2}
\includegraphics[width=0.5\textwidth]{/home/link/Documents/thèse_onera/diapos_phd_thesis/images/EEG/info_content/entropy/spen_cluster_FC_electrodes_by_detection_eb.jpeg}
\includegraphics[width=0.5\textwidth]{/home/link/Documents/thèse_onera/diapos_phd_thesis/images/EEG/info_content/entropy/apen_cluster_FC_electrodes_by_detection_eb.jpeg}
\includegraphics[width=0.5\textwidth]{/home/link/Documents/thèse_onera/diapos_phd_thesis/images/EEG/info_content/entropy/saen_cluster_FC_electrodes_by_detection_eb.jpeg}
\includegraphics[width=0.5\textwidth]{/home/link/Documents/thèse_onera/diapos_phd_thesis/images/EEG/info_content/entropy/peen_cluster_FC_electrodes_by_detection_eb.jpeg}
\end{multicols}
\centering \includegraphics[width=0.5\textwidth]{/home/link/Documents/thèse_onera/diapos_phd_thesis/images/EEG/info_content/entropy/sven_cluster_FC_electrodes_by_detection_eb.jpeg}
\caption[Valeurs des mesures d'entropie calculées pour les électrodes du cluster FC]{Valeurs moyennes et barres d'erreur standard des mesures d'entropie calculées pour chacune des électrodes du cluster fronto-central (FC1, FC2, FC3, FC4, FC5 et FC6) avant et après la référence temporelle. Les électrodes fronto-centrales FC5 et FC6 montrent les valeurs les plus élevées lorsque la cible a été détectée par le sujet.}
\label{fig:figure5valeursmesuresentropieelectrode}
\end{figure*}

\begin{table}[!t]
\centering
\scriptsize
\caption[Table des résultats des analyses statistiques pour les mesures d'entropie des électrodes du cluster FC]{Tables des résultats des analyses statistiques du modèle linéaire à effets mixtes Détection * Condition * Électrodes Cluster FC + $1~|~$Sujet réalisées sur les mesures d'entropie avant et après la référence.}
\label{tab:table5statsmesuresentropieelectrode}

\textbf{Mesure d'entropie $\sim$  Détection * Condition * Électrodes Cluster FC + $1~|~$Sujet}

% \begin{tabular}{lllllllll}
\begin{tabular}{|l|*{9}{c|}}
\hline
& \textbf{Measure} & \textbf{Sum Sq} & \textbf{Mean Sq} & \textbf{NumDF} & \textbf{DenDF} & \textbf{F value} & \textbf{Pr($>$F)} & \textbf{Sign.} \\ 
\hline
\textcolor{blue}{Détection} & & & & & & & & \\ 
\hline
& \textit{SpEn} & 0.116 & 0.116 & 1 & 425 & 54.726 & $<$.0001 & ***\\ 
& \textit{ApEn} & 0.973 & 0.973 & 1 & 426 & 97.662 & $<$.0001 & *** \\ 
& \textit{SaEn} & 0.974 & 0.974 & 1 & 426 & 75.928 & $<$.0001 & *** \\ 
& \textit{PeEn} & 0.533 & 0.533 & 1 & 425 & 2903.578 & $<$.0001 & *** \\ 
& \textit{SvEn} & 0.104 & 0.104 & 1 & 425 & 39.382 & $<$.0001 & *** \\ 
\hline
\textcolor{blue}{Condition} & & & & & & & & \\ 
\hline
& SpEn & 0.001 & 0.001 & 1 & 425 & 0.533 & 0.466 & \\ 
& ApEn & 0.010 & 0.010 & 1 & 425 & 1.042 & 0.308 & \\ 
& SaEn & 0.014 & 0.014 & 1 & 425 & 1.076 & 0.300 & \\ 
& PeEn & 0.000 & 0.000 & 1 & 425 & 0.015 & 0.903 & \\ 
& SvEn & 0.001 & 0.001 & 1 & 425 & 0.282 & 0.596 & \\ 
\hline
\textcolor{blue}{Electrode} & & & & & & & & \\ 
\hline
& \textit{SpEn} & 0.700 & 0.140 & 5 & 425 & 66.306 & $<$.0001 & *** \\ 
& \textit{ApEn} & 5.134 & 1.027 & 5 & 425 & 103.068 & $<$.0001 & *** \\ 
& \textit{SaEn} & 5.452 & 1.090 & 5 & 425 & 84.962 & $<$.0001 & *** \\ 
& \textit{PeEn} & 1.475 & 0.295 & 5 & 425 & 1605.774 & $<$.0001 & *** \\ 
& \textit{SvEn} & 1.102 & 0.220 & 5 & 425 & 83.303 & $<$.0001 & *** \\ 
\hline
\textcolor{blue}{Détection~*~Condition} & & & & & & & & \\ 
\hline
& SpEn & 0.002 & 0.002 & 1 & 425 & 1.064 & 0.303 & \\ 
& ApEn & 0.015 & 0.015 & 1 & 425 & 1.527 & 0.217 & \\ 
& SaEn & 0.022 & 0.022 & 1 & 425 & 1.738 & 0.188 & \\ 
& PeEn & 0.000 & 0.000 & 1 & 425 & 1.073 & 0.301 & \\ 
& SvEn & 0.000 & 0.000 & 1 & 425 & 0.136 & 0.712 & \\ 
\hline
\textcolor{blue}{Détection~*~Electrode} & & & & & & & & \\ 
\hline
& \textit{SpEn} & 0.166 & 0.033 & 5 & 425 & 15.755 & $<$.0001 & *** \\ 
& \textit{ApEn} & 1.994 & 0.399 & 5 & 425 & 40.026 & $<$.0001 & *** \\ 
& \textit{SaEn} & 2.021 & 0.404 & 5 & 425 & 31.495 & $<$.0001 & *** \\ 
& \textit{PeEn} & 1.027 & 0.205 & 5 & 425 & 1118.773 & $<$.0001 & *** \\ 
& \textit{SvEn} & 0.209 & 0.042 & 5 & 425 & 15.793 & $<$.0001 & *** \\ 
\hline
\textcolor{blue}{Condition~*~Electrode} & & & & & & & & \\ 
\hline
& SpEn & 0.009 & 0.002 & 5 & 425 & 0.808 & 0.544 & \\ 
& ApEn & 0.055 & 0.011 & 5 & 425 & 1.114 & 0.352 & \\ 
& SaEn & 0.071 & 0.014 & 5 & 425 & 1.100 & 0.360 & \\ 
& PeEn & 0.000 & 0.000 & 5 & 425 & 0.094 & 0.993 & \\ 
& SvEn & 0.013 & 0.003 & 5 & 425 & 0.956 & 0.445 & \\ 
\hline
\textcolor{blue}{Détection~*~Condition~*~Electrode} & & & & & & & & \\ 
\hline
& SpEn & 0.008 & 0.002 & 5 & 425 & 0.761 & 0.578 & \\ 
& ApEn & 0.055 & 0.011 & 5 & 425 & 1.100 & 0.360 & \\ 
& SaEn & 0.070 & 0.014 & 5 & 425 & 1.098 & 0.361 & \\ 
& PeEn & 0.001 & 0.000 & 5 & 425 & 0.641 & 0.668 & \\ 
& SvEn & 0.012 & 0.002 & 5 & 425 & 0.898 & 0.483 & \\ 
\hline
\end{tabular}

\end{table}

L'analyse de variance du modèle montre que l'interaction entre la détection et l'électrode est significative et importante pour : 
\begin{itemize}
\item[$\bullet$] \textbf{SpEn} ($F(5)=15.75$, $p<.001$, $\eta^2=0.16$) ; 
\item[$\bullet$] \textbf{ApEn} ($F(5)=40.03$, $p<.001$, $\eta^2=0.32$) ; 
\item[$\bullet$] \textbf{SaEn} ($F(5)=31.50$, $p<.001$, $\eta^2=0.27$) ; 
\item[$\bullet$] \textbf{PeEn} ($F(5)=1118.77$, $p<.001$, $\eta^2=0.93$) et 
\item[$\bullet$] \textbf{SvEn} ($F(5)=15.79$, $p<.001$, $\eta^2=0.16$). \\
\end{itemize}

Les analyses par comparaisons multiples ont permis d'observer une augmentation significative des valeurs de mesure pour les cibles détectées par rapport aux cibles manquées sur l'électrode FC5 pour :
\begin{itemize}
\begin{multicols}{2}
\item[$\bullet$] \textbf{SpEn} ($t=5.15$, $p<.0001$) ; 
\item[$\bullet$] \textbf{ApEn} ($t=8.17$, $p<.0001$) ;
\item[$\bullet$] \textbf{SaEn} ($t=7.22$, $p<.0001$) ;
\item[$\bullet$] \textbf{PeEn} ($t=46.73$, $p<.0001$) ;
\item[$\bullet$] \textbf{SvEn} ($t=4.81$, $p<.0001$) ;
\end{multicols}
\end{itemize}
et sur l'électrode FC6 pour : 
\begin{itemize}
\begin{multicols}{2}
\item[$\bullet$] \textbf{SpEn} ($t=6.03$, $p<.0001$) ;
\item[$\bullet$] \textbf{ApEn} ($t=9.67$, $p<.0001$) ;
\item[$\bullet$] \textbf{SaEn} ($t=8.60$, $p<.0001$) ;
\item[$\bullet$] \textbf{PeEn} ($t=47.86$, $p<.0001$) ;
\item[$\bullet$] \textbf{SvEn} ($t=6.34$, $p<.0001$). \\
\end{multicols}
\end{itemize}

Au contraire, une réduction significative des valeurs de mesure a été principalement observée pour les cibles détectées par rapport aux cibles manquées sur l'électrode FC1 pour :
\begin{itemize}
\begin{multicols}{2}
\item[$\bullet$] \textbf{SpEn} ($t=-2.88$, $p=.02$) ;
\item[$\bullet$] \textbf{ApEn} ($t=-4.65$, $p<.0001$) ;
\item[$\bullet$] \textbf{SaEn} ($t=-4.12$, $p<.001$) ;
\item[$\bullet$] \textbf{PeEn} ($t=-24.22$, $p<.0001$) ;
\item[$\bullet$] \textbf{SvEn} ($t=-3.08$, $p=.01$) ; 
\end{multicols}
\end{itemize}
sur l'électrode FC2 pour : 
\begin{itemize}
\begin{multicols}{2}
\item[$\bullet$] \textbf{SpEn} ($t=-3.05$, $p=.01$) ;
\item[$\bullet$] \textbf{ApEn} ($t=-4.56$, $p<.0001$) ;
\item[$\bullet$] \textbf{SaEn} ($t=-4.03$, $p<.001$) ;
\item[$\bullet$] \textbf{PeEn} ($t=-23.41$, $p<.0001$) ; 
\item[$\bullet$] \textbf{SvEn} ($t=-2.56$, $p=.02$) ; 
\end{multicols}
\end{itemize}
sur l'électrode FC3 pour :
\begin{itemize}
\begin{multicols}{2}
\item[$\bullet$] \textbf{ApEn} ($t=-4.41$, $p=.0001$) ;
\item[$\bullet$] \textbf{SaEn} ($t=-3.87$, $<.001$) ; 
\item[$\bullet$] \textbf{PeEn} ($t=-23.85$, $p<.0001$) ; 
\end{multicols}
\end{itemize}
et sur l'électrode FC4 pour : 
\begin{itemize}
\begin{multicols}{2}
\item[$\bullet$] \textbf{SpEn} ($t=-2.93$, $p=.02$) ;
\item[$\bullet$] \textbf{ApEn} ($t=-4.21$, $p<.001$) ;
\item[$\bullet$] \textbf{SaEn} ($t=-3.79$, $p<.001$) ;
\item[$\bullet$] \textbf{PeEn} ($t=-23.11$, $p<.0001$). \\ 
\end{multicols}
\end{itemize}

En résumé, dans le modèle Détection * Condition * Électrodes Cluster FC + $1|$Sujet, le facteur détection, le facteur électrode et leur interaction détection * électrode ont montré un effet significatif sur l'ensemble des mesures d'entropie, mais ni l'interaction condition * détection * électrode, ni les interactions détection * condition et condition * électrode ont montré d'effet significatif sur les mesures d'entropie. 
Les cinq mesures ont présenté des valeurs significativement supérieures pour les électrodes FC5 et FC6 lorsque les cibles auditives ont été détectées. 
Au contraire, les électrodes les plus centrales FC1 et FC2 ont montré des valeurs statistiquement inférieures pour les cibles détectées. \\

Finalement, nous avons étudié la dynamique d'évolution des mesures d'entropie autour de la détection (appui-bouton) et de l'omission (temps moyen de détection) des cibles. 
Nous avons cherché à savoir quelle mesure était à même de présenter une variation caractéristique susceptible de discriminer la construction du percept conscient. 
Nous avons donc réalisé une analyse par modèle linéaire à effets mixtes sur le cluster fronto-central pour étudier l'effet des facteurs détection et fenêtres sur les valeurs des différentes mesures d'entropie (formule R : Mesure d'entropie $\sim$ Détection * Fenetre + $1|$Sujet). 

\begin{figure*}[!t]
\centering \textbf{Mesures d'entropie en fonction du temps autour de la référence temporelle}\par\medskip
% \begin{multicols}{2}
% \includegraphics[width=0.5\textwidth]{/home/link/Documents/thèse_onera/diapos_phd_thesis/images/EEG/info_content/entropy/spen_cluster_FC_window_by_detection_point_stats.jpeg}
% \includegraphics[width=0.5\textwidth]{/home/link/Documents/thèse_onera/diapos_phd_thesis/images/EEG/info_content/entropy/apen_cluster_FC_window_by_detection_point_stats.jpeg}
% \includegraphics[width=0.5\textwidth]{/home/link/Documents/thèse_onera/diapos_phd_thesis/images/EEG/info_content/entropy/saen_cluster_FC_window_by_detection_point_stats.jpeg}
% \includegraphics[width=0.5\textwidth]{/home/link/Documents/thèse_onera/diapos_phd_thesis/images/EEG/info_content/entropy/peen_cluster_FC_window_by_detection_point_stats.jpeg}
% \end{multicols}
% \centering \includegraphics[width=0.5\textwidth]{/home/link/Documents/thèse_onera/diapos_phd_thesis/images/EEG/info_content/entropy/sven_cluster_FC_window_by_detection_point_stats.jpeg}
\includegraphics[width=\textwidth]{Figures/illustrations/Exp_EEG/entropy_measures_dynamics.jpg}
\caption[Valeurs des mesures d'entropie calculées pour les fenêtres temporelles dans le cluster FC]{Valeurs moyennes et barres d'erreur standard des mesures d'entropie calculées pour chacune des fenêtres de temps de $-3$ à $+3$ secondes, respectivement avant et après la référence temporelle (ligne rouge) pour le cluster fronto-central. Les mesures ont été estimées à partir de fenêtres de $1000$ points. Les points noirs montrent les fenêtres temporelles où une différence significative est observée entre la détection et la non-détection.}
\label{fig:figure5valeursmesuresentropiefenetre}
\end{figure*}

La Figure~\ref{fig:figure5valeursmesuresentropiefenetre} présente, pour chaque mesure d'entropie, l'évolution au cours du temps des valeurs calculées sur les fenêtres temporelles de $1000$ points de données pour le cluster fronto-central entre $-3$ et $+3$ secondes autour de la référence. 
La ligne verticale rouge indique la référence temporelle considérée. 
Pour les cinq mesures d'entropie, on observe avant la référence pour les cibles détectées des valeurs d'entropie supérieures comparativement aux cibles non-détectées. 
Après la référence, au contraire, le décours temporel apparaît plus variable : à certains moments, les valeurs d'entropie sont plus élevées pour les cibles détectées que pour les cibles non-détectées, tandis qu'à d'autres, c'est l'inverse. 
Un pattern caractéristique apparaît néanmoins pour toutes les mesures juste au moment de la référence : une diminution des valeurs jusqu'à plusieurs centaines de ms à partir de la référence. 
Cette chute est notamment la plus prononcée pour la mesure d'entropie de décomposition en valeurs singulières, qui par ailleurs, est la seule à montrer une tendance à la hausse aussi importante juste après. 

\begin{table}[!t]
\centering
\scriptsize
\caption[Table des résultats des analyses statistiques pour les mesures d'entropie fenêtrées du cluster FC]{Tables des résultats des analyses statistiques du modèle linéaire à effets mixtes Détection * Fenêtre + $1~|~$Sujet réalisées sur les mesures d'entropie avant et après la référence temporelle.}
\label{tab:table5statsmesuresentropiefenetre}

\textbf{Mesure d'entropie $\sim$ Détection * Fenêtre + $1~|~$Sujet}

% \begin{tabular}{lllllllll}
\begin{tabular}{|l|*{9}{c|}}
\hline
& \textbf{Measure} & \textbf{Sum Sq} & \textbf{Mean Sq} & \textbf{NumDF} & \textbf{DenDF} & \textbf{F value} & \textbf{Pr($>$F)} & \textbf{Sign.} \\ 
\hline
\textcolor{blue}{Détection} & & & & & & & & \\ 
\hline
& SpEn & 0.000 & 0.000 & 1 & 795 & 3.438 & 0.064 & .\\ 
& ApEn & 0.000 & 0.000 & 1 & 795 & 2.588 & 0.108 & \\ 
& SaEn & 0.000 & 0.000 & 1 & 795 & 1.332 & 0.249 & \\ 
& PeEn & 0.000 & 0.000 & 1 & 795 & 0.604 & 0.437 & \\ 
& SvEn & 0.000 & 0.000 & 1 & 795 & 2.794 & 0.095 & \\ 
\hline
\textcolor{blue}{Fenêtre} & & & & & & & & \\ 
\hline
& SpEn & 0.001 & 0.000 & 21 & 795 & 1.392 & 0.113 & \\ 
& \textit{ApEn} & 0.002 & 0.000 & 21 & 795 & 1.663 & 0.031 & * \\
& \textit{SaEn} & 0.001 & 0.000 & 21 & 795 & 1.595 & 0.044 & * \\ 
& PeEn & 0.000 & 0.000 & 21 & 795 & 1.427 & 0.097 & \\ 
& \textit{SvEn} & 0.001 & 0.000 & 21 & 795 & 1.625 & 0.038 & * \\ 
\hline
\textcolor{blue}{Détection~*~Fenêtre} & & & & & & & & \\ 
\hline
& \textit{SpEn} & 0.002 & 0.000 & 21 & 795 & 5.222 & $<$.0001 & ***\\ 
& \textit{ApEn} & 0.002 & 0.000 & 21 & 795 & 1.898 & 0.009 & ** \\ 
& \textit{SaEn} & 0.001 & 0.000 & 21 & 795 & 1.682 & 0.028 & * \\ 
& \textit{PeEn} & 0.000 & 0.000 & 21 & 795 & 2.762 & $<$.0001 & *** \\ 
& SvEn & 0.000 & 0.000 & 21 & 795 & 0.917 & 0.569 & \\ 
\hline
\end{tabular}

\end{table}

La Table~\ref{tab:table5statsmesuresentropiefenetre} reporte les résultats des analyses statistiques du modèle Mesure d'entropie $\sim$ Détection * Fenêtre + $1|$Sujet. 
Dans ce modèle, le facteur détection n'a pas d'effet significatif sur les mesures d'entropie et seule \textbf{SpEn} a montré une tendance statistique ($F(1)=3.44$, $p=0.06$, $\eta^2=4.31\times10^{-3}$).
Un effet significatif pour la fenêtre a été trouvé pour \textbf{ApEn} ($F(21)=1.66$, $p=0.031$, $\eta^2=0.04$), \textbf{SaEn} ($F(21)=1.60$, $p=0.044$, $\eta^2=0.04$) et \textbf{SvEn} ($F(21)=1.63$, $p=0.038$, $\eta^2=0.04$). 
Un effet significatif est reporté pour l'interaction entre la détection et la fenêtre pour \textbf{SpEn} ($F(21)=5.22$, $p<.001$, $\eta^2=0.12$), pour \textbf{ApEn} ($F(21)=1.90$, $p=0.009$, $\eta^2=0.05$), pour \textbf{SaEn} ($F(21)=1.68$, $p=0.028$, $\eta^2=0.04$) et pour \textbf{PeEn} ($F(21)=2.76$, $p<.001$, $\eta^2=0.07$). 
Cette interaction n'était cependant pas significative pour \textbf{SvEn} ($F(21)=0.92$, $p=0.569$, $\eta^2=0.02$). 

Enfin, les analyses par comparaisons multiples entre les différentes fenêtres et les cibles détectées ou non-détectées ont permis de déterminer les fenêtres temporelles pour lesquelles une différence significative a été trouvée entre les cibles détectées et celles non-détectées.
Ces fenêtres significatives sont représentées sur la Figure~\ref{fig:figure5valeursmesuresentropiefenetre} comme points noirs. 
Les mesures \textbf{SpEn}, \textbf{ApEn}, \textbf{SaEn} et \textbf{PeEn} présentent des fenêtres significatives autour de la référence temporelle. 
L'entropie spectrale \textbf{SpEn} est la mesure qui présente le plus de fenêtres significatives avant la référence de sorte que les valeurs trouvées pour les cibles détectées sont supérieures à celles trouvées pour les cibles non-détectées. 
Après la référence, l'entropie de permutation \textbf{PeEn} montre plusieurs fenêtres significatives. 
En outre, le pic visible pour l'entropie \textbf{SvEn} n'est pas significatif puisque aucun effet significatif n'était reporté pour l'interaction entre la détection et la fenêtre sur cette mesure. 

%%%%%%%%%%%%%%%%%%%%%%%%%%%%%%%%%%%%%%%%%%%%%%%%%%%%%%%%%%%%%%%%%%%%%%%%%%%%%%%
\subsection{Résultats des mesures de complexité}
\label{resultatsmesurescomplexite}
%%%%%%%%%%%%%%%%%%%%%%%%%%%%%%%%%%%%%%%%%%%%%%%%%%%%%%%%%%%%%%%%%%%%%%%%%%%%%%%

Nous présentons maintenant les résultats trouvés pour les mesures de complexité. 
Ces mesures de complexité sont au nombre de sept : l'exposant fractal $\alpha$, l'exposant de Hurst \textbf{H}, la dimension fractale de Higuchi \textbf{HFD}, la dimension fractale de Katz \textbf{KFD}, la dimension fractale de Petrosian \textbf{PFD}, le paramètre de \textbf{mobilité Hjorth} et le paramètre de \textbf{complexité Hjorth}. 
Les différentes étapes de l'analyse statistique ont été effectuées de manière similaire à celles des mesures d'entropie.

La Figure~\ref{fig:figure5valeursmesurescomplexite} présente les valeurs moyennes et erreurs standards des sept mesures de complexité calculées pour chaque cluster $3$~sec avant et $3$~sec après la détection (hit, en rouge clair) ou la non-détection (miss, en bleu clair) de la cible. 
On observe que le cluster fronto-central exprime les valeurs les plus élevées lorsque la cible est détectée pour toutes les mesures, excepté pour l'exposant $\alpha$. 
Pour cette mesure, le cluster fronto-central a des valeurs parmi les plus basses (avec le cluster temporal) et c'est le cluster sagittal qui présente les plus hautes valeurs. 
On voit que l'exposant de Hurst et la dimension fractale de Petrosian ont une variabilité très inférieure pour tous les clusters excepté pour le cluster fronto-central. 
Un premier modèle linéaire à effets mixtes a donc été ajusté pour étudier les effets des facteurs détection, condition et cluster sur les valeurs des mesures de complexité (formule R : Mesure de complexité $\sim$ Détection * Condition * Cluster + $1|$Sujet). 
La Table~\ref{tab:table5statsmesurescomplexite} reporte les résultats des analyses des modèles linéaire à effets mixtes sur les mesures de complexité avant et après la référence. \\

\begin{figure*}[!t]
\begin{multicols}{2}
\includegraphics[width=0.5\textwidth]{/home/link/Documents/thèse_onera/diapos_phd_thesis/images/EEG/info_content/complexity/alpha_cluster_by_detection_eb.jpeg}
\includegraphics[width=0.5\textwidth]{/home/link/Documents/thèse_onera/diapos_phd_thesis/images/EEG/info_content/complexity/hfd_cluster_by_detection_eb.jpeg}
\includegraphics[width=0.5\textwidth]{/home/link/Documents/thèse_onera/diapos_phd_thesis/images/EEG/info_content/complexity/pfd_cluster_by_detection_eb.jpeg}
\includegraphics[width=0.5\textwidth]{/home/link/Documents/thèse_onera/diapos_phd_thesis/images/EEG/info_content/complexity/hurst_cluster_by_detection_eb.jpeg}
\includegraphics[width=0.5\textwidth]{/home/link/Documents/thèse_onera/diapos_phd_thesis/images/EEG/info_content/complexity/kfd_cluster_by_detection_eb.jpeg}
\includegraphics[width=0.5\textwidth]{/home/link/Documents/thèse_onera/diapos_phd_thesis/images/EEG/info_content/complexity/hjorthm_cluster_by_detection_eb.jpeg}
\end{multicols}
\centering \includegraphics[width=0.5\textwidth]{/home/link/Documents/thèse_onera/diapos_phd_thesis/images/EEG/info_content/complexity/hjorthc_cluster_by_detection_eb.jpeg}
\caption[Valeurs des mesures de complexité sur chaque cluster avant et après le report/non-report]{Valeurs moyennes et barres d'erreur standard des mesures de complexité calculées pour chacun des clusters avant et après la détection (hit, en rouge clair) ou la non-détection (miss, en bleu clair) de la cible. Les mesures ont été calculées à partir de chaque électrode du cluster puis obtenues par agrégation topographique des valeurs.}
\label{fig:figure5valeursmesurescomplexite}
\end{figure*}

\begin{table}[!t]
\centering
\scriptsize
\caption[Table des résultats des analyses statistiques pour les mesures de complexité]{Tables des résultats des analyses statistiques du modèle linéaire à effets mixtes pour le modèle Détection * Condition * Cluster + $1~|~$Sujet réalisées sur les mesures de complexité avant et après la référence temporelle.}
\label{tab:table5statsmesurescomplexite}

\textbf{Mesure de complexité $\sim$ Détection * Condition * Cluster + $1~|~$Sujet}

% \begin{tabular}{lllllllll}
\begin{tabular}{|l|*{9}{c|}}
\hline
& \textbf{Measure} & \textbf{Sum Sq} & \textbf{Mean Sq} & \textbf{NumDF} & \textbf{DenDF} & \textbf{F value} & \textbf{Pr($>$F)} & \textbf{Sign.} \\ 
\hline
\textcolor{blue}{Détection} & & & & & & & & \\ 
\hline
& Alpha & 0.00 & 0.00 & 1 & 647 & 1.65 & 0.2000 & \\ 
& \textit{Hurst} & 0.08 & 0.08 & 1 & 647 & 118.19 & $<$.0001 & *** \\ 
& \textit{HFD} & 0.01 & 0.01 & 1 & 647 & 8.14 & 0.0045 & ** \\ 
& KFD & 0.01 & 0.01 & 1 & 647 & 0.83 & 0.3617 & \\ 
& \textit{PFD} & 0.00 & 0.00 & 1 & 647 & 265.17 & $<$.0001 & *** \\ 
& \textit{Complexité Hjorth} & 3711.52 & 3711.52 & 1 & 647 & 28.39 & $<$.0001 & *** \\ 
& Mobilité Hjorth & 0.00 & 0.00 & 1 & 647 & 2.85 & 0.0917 & \\ 
\hline
\textcolor{blue}{Condition} & & & & & & & & \\ 
\hline
& Alpha & 0.00 & 0.00 & 1 & 646 & 1.54 & 0.2158 & \\ 
& Hurst & 0.00 & 0.00 & 1 & 647 & 0.00 & 0.9826 & \\ 
& HFD & 0.00 & 0.00 & 1 & 647 & 2.33 & 0.1274 & \\ 
& KFD & 0.00 & 0.00 & 1 & 647 & 0.08 & 0.7782 & \\ 
& PFD & 0.00 & 0.00 & 1 & 647 & 0.15 & 0.6943 & \\ 
& Complexité Hjorth & 74.29 & 74.29 & 1 & 646 & 0.57 & 0.4513 & \\ 
& Mobilité Hjorth & 0.00 & 0.00 & 1 & 647 & 1.11 & 0.2931 & \\ 
\hline
\textcolor{blue}{Cluster} & & & & & & & & \\ 
\hline
& \textit{Alpha} & 0.73 & 0.09 & 8 & 646 & 44.79 & $<$.0001 & *** \\ 
& \textit{Hurst} & 0.51 & 0.06 & 8 & 647 & 95.88 & $<$.0001 & *** \\ 
& \textit{HFD} & 0.44 & 0.05 & 8 & 647 & 59.41 & $<$.0001 & *** \\ 
& \textit{KFD} & 2.21 & 0.28 & 8 & 647 & 37.48 & $<$.0001 & *** \\ 
& \textit{PFD} & 0.00 & 0.00 & 8 & 647 & 333.48 & $<$.0001 & *** \\ 
& \textit{Complexité Hjorth} & 22071.98 & 2759.00 & 8 & 646 & 21.10 & $<$.0001 & *** \\ 
& \textit{Mobilité Hjorth} & 0.00 & 0.00 & 8 & 647 & 43.44 & $<$.0001 & *** \\ 
\hline
\textcolor{blue}{Détection~*~Condition} & & & & & & & & \\ 
\hline
& Alpha & 0.01 & 0.01 & 1 & 646 & 2.67 & 0.1029 & \\ 
& \textit{Hurst} & 0.02 & 0.02 & 1 & 647 & 29.27 & $<$.0001 & *** \\ 
& HFD & 0.00 & 0.00 & 1 & 647 & 3.69 & 0.0550 & . \\ 
& KFD & 0.01 & 0.01 & 1 & 647 & 1.07 & 0.3013 & \\ 
& \textit{PFD} & 0.00 & 0.00 & 1 & 647 & 4.52 & 0.0339 & *\\ 
& Complexité Hjorth & 41.70 & 41.70 & 1 & 646 & 0.32 & 0.5725 & \\ 
& Mobilité Hjorth & 0.00 & 0.00 & 1 & 647 & 0.65 & 0.4218 & \\ 
\hline
\textcolor{blue}{Détection~*~Cluster} & & & & & & & & \\ 
\hline
& Alpha & 0.01 & 0.00 & 8 & 646 & 0.64 & 0.7465 & \\ 
& \textit{Hurst} & 0.15 & 0.02 & 8 & 647 & 27.78 & $<$.0001 & *** \\ 
& \textit{HFD} & 0.02 & 0.00 & 8 & 647 & 2.54 & 0.0101 & * \\ 
& \textit{KFD} & 0.14 & 0.02 & 8 & 647 & 2.39 & 0.0152 & * \\ 
& \textit{PFD} & 0.00 & 0.00 & 8 & 647 & 241.75 & $<$.0001 & *** \\ 
& \textit{Complexité Hjorth} & 19554.26 & 2444.28 & 8 & 646 & 18.69 & $<$.0001 & *** \\ 
& \textit{Mobilité Hjorth} & 0.00 & 0.00 & 8 & 647 & 3.46 & 0.0006 & *** \\ 
\hline
\textcolor{blue}{Condition~*~Cluster} & & & & & & & & \\ 
\hline
& Alpha & 0.00 & 0.00 & 8 & 646 & 0.06 & 0.9999 & \\ 
& Hurst & 0.00 & 0.00 & 8 & 647 & 0.12 & 0.9983 & \\
& HFD & 0.00 & 0.00 & 8 & 647 & 0.07 & 0.9998 & \\ 
& KFD & 0.01 & 0.00 & 8 & 647 & 0.12 & 0.9982 & \\ 
& PFD & 0.00 & 0.00 & 8 & 647 & 0.12 & 0.9984 & \\ 
& Complexité Hjorth & 388.95 & 48.62 & 8 & 646 & 0.37 & 0.9355 & \\ 
& Mobilité Hjorth & 0.00 & 0.00 & 8 & 647 & 0.16 & 0.9961 & \\ 
\hline
\textcolor{blue}{Détection~*~Condition~*~Cluster} & & & & & & & & \\ 
\hline
& Alpha & 0.00 & 0.00 & 8 & 646 & 0.08 & 0.9997 & \\ 
& Hurst & 0.00 & 0.00 & 8 & 647 & 0.24 & 0.9837 & \\ 
& HFD & 0.00 & 0.00 & 8 & 647 & 0.07 & 0.9998 & \\ 
& KFD & 0.01 & 0.00 & 8 & 647 & 0.22 & 0.9866 & \\ 
& PFD & 0.00 & 0.00 & 8 & 647 & 0.06 & 0.9999 & \\ 
& Complexité Hjorth & 322.68 & 40.33 & 8 & 646 & 0.31 & 0.9629 & \\ 
& Mobilité Hjorth & 0.00 & 0.00 & 8 & 647 & 0.19 & 0.9929 & \\ 
\hline
\end{tabular}

\end{table}

Les effets significatifs mis en évidence par l'analyse de variance sont :

\begin{itemize}
\item[$\bullet$] \underline{pour l'exposant \textbf{Alpha}} : 
\begin{itemize}
\item l'effet principal du cluster est significatif et important ($F(8)=44.79$, $p<.001$, $\eta^2=0.36$) ;
\end{itemize}
\item[$\bullet$] \underline{pour l'exposant de Hurst \textbf{H}} : 
\begin{itemize} 
\item l'effet principal de la détection est significatif et important ($F(1)=118.19$, $p<.001$, $\eta^2=0.15$) ; 
\item l'effet principal du cluster est significatif et important ($F(8)=95.88$, $p<.001$, $\eta^2=0.54$) ; 
\item l'interaction entre la détection et la condition est significative et faible ($F(1)=29.27$, $p<.001$, $\eta^2=0.04$) ; 
\item l'interaction entre la détection et le cluster est significative et importante ($F(8)=27.78$, $p<.001$, $\eta^2=0.26$) ; 
\end{itemize}
\item[$\bullet$] \underline{pour la dimension fractale de Higuchi \textbf{HFD}} : 
\begin{itemize}
\item l'effet principal de la détection est significatif et faible ($F(1)=8.14$, $p=0.004$, $\eta^2=0.01$) ; 
\item l'effet principal du cluster est significatif et important ($F(8)=59.41$, $p<.001$, $\eta^2=0.42$) ; 
\item l'interaction entre détection et cluster est significative et faible ($F(8)=2.54$, $p=0.01$, $\eta^2=0.03$) ; 
\end{itemize}
\item[$\bullet$] \underline{pour la dimension fractale de Katz \textbf{KFD}} : 
\begin{itemize} 
\item l'effet principal du cluster est significatif et important ($F(8)=37.48$, $p<.001$, $\eta^2=0.32$) ; 
\item l'interaction entre la détection et la condition est non significative et très faible ($F(1)=1.07$, $p=0.301$, $\eta^2=1.65\times10^{-3}$) ; 
\item l'interaction entre détection et cluster est significative et faible ($F(8)=2.39$, $p=0.015$, $\eta^2=0.03$) ; 
\end{itemize}
\item[$\bullet$] \underline{pour la dimension fractale de Petrosian \textbf{PFD}} : 
\begin{itemize} 
\item l'effet principal de la détection est significatif et important ($F(1)=265.17$, $p<.001$, $\eta^2=0.29$) ; 
\item l'effet principal du cluster est significatif et important ($F(8)=333.48$, $p<.001$, $\eta^2=0.80$) ; 
\item l'interaction entre la détection et la condition est significative et très faible ($F(1)=4.52$, $p=0.034$, $\eta^2=6.94\times10^{-3}$) ; 
\item l'interaction entre la détection et le cluster est significative et importante ($F(8)=241.75$, $p<.001$, $\eta^2=0.75$) ;
\end{itemize}
\item[$\bullet$] \underline{pour le paramètre de \textbf{mobilité Hjorth}} : 
\begin{itemize} 
\item l'effet principal du cluster est significatif et important ($F(8)=43.44$, $p<.001$, $\eta^2=0.35$) ; 
\item l'interaction entre la détection et le cluster est significative et faible ($F(8)=3.46$, $p<.001$, $\eta^2=0.04$) ; 
\end{itemize}
\item[$\bullet$] \underline{pour le paramètre de \textbf{complexité Hjorth}} : 
\begin{itemize} 
\item l'effet principal de la détection est significatif et faible ($F(1)=28.39$, $p<.001$, $\eta^2=0.04$) ; 
\item l'effet principal du cluster est significatif et important ($F(8)=21.10$, $p<.001$, $\eta^2=0.21$) ; 
\item l'interaction entre détection et cluster est significative et importante ($F(8)=18.69$, $p<.001$, $\eta^2=0.19$). \\
\end{itemize}
\end{itemize}

En résumé, dans le modèle Mesure de complexité $\sim$ Détection * Condition * Cluster + $1|$Sujet, le facteur cluster et l'interaction entre la détection et le cluster ont montré un effet significatif pour toutes les mesures de complexité excepté pour l'exposant $\alpha$. 
Le facteur détection a présenté un effet significatif pour l'exposant de Hurst \textbf{H}, la dimension fractale de Higuchi \textbf{HFD}, la dimension fractale de Petrosian \textbf{PFD} et le paramètre de \textbf{complexité Hjorth}. 
L'interaction entre la détection et la condition a montré un effet significatif pour l'exposant de Hurst \textbf{H}, la dimension fractale de Katz \textbf{KFD}, la dimension fractale de Petrosian \textbf{PFD}, 
Néanmoins, l'interaction condition * détection * cluster n'a pas montré d'effet significatif sur les mesures de complexité.  
Comme l'interaction détection * cluster était significative, nous avons regardé les clusters significatifs. 
Les comparaisons mutliples entre les différents clusters nous ont permis d'observer une augmentation significative des valeurs de mesure dans le cluster fronto-central pour les cibles détectées par rapport aux cibles manquées pour : 

\begin{itemize}
\begin{multicols}{2}
\item[$\bullet$] \textbf{Hurst} ($t=14.77$, $p<.0001$) ;
\item[$\bullet$] \textbf{HFD} ($t=4.41$, $p<.001$) ;
\item[$\bullet$] \textbf{KFD} ($t=4.31$, $p<.001$) ; 
\item[$\bullet$] \textbf{PFD} ($t=43.96$, $p<.0001$) ; 
\item[$\bullet$] \textbf{complexité Hjorth} ($t=12.15$, $p<.0001$) ;
\item[$\bullet$] \textbf{mobilité Hjorth} ($t=5.19$, $p<.0001$). \\
\end{multicols}
\end{itemize}

Ensuite, nous avons cherché à obtenir plus de précisions sur la localisation topographique des variations de complexité du signal à l'échelle du cluster fronto-central. 
Nous avons ajusté un second modèle linéaire à effets mixtes pour étudier les effets des facteurs de détection, de condition et d'électrodes du cluster sur les valeurs de chaque mesure de complexité (formule R: Mesure de complexité $\sim$ Détection * Condition * Electrode Cluster FC + $1|$Sujet). 
La Figure~\ref{fig:figure5valeursmesurescomplexiteelectrode} présente les valeurs moyennes et barres d'erreur standard des mesures de complexité calculées pour chacune des électrodes du cluster fronto-central avant et après la référence. 
La Table~\ref{tab:table5statsmesurescomplexiteelectrode} reporte les résultats des analyses statistiques par modélisation linéaire à effets mixtes pour ce deuxième modèle. \\

\begin{figure*}[!t]
\begin{multicols}{2}
\includegraphics[width=0.5\textwidth]{/home/link/Documents/thèse_onera/diapos_phd_thesis/images/EEG/info_content/complexity/alpha_cluster_FC_electrodes_by_detection_eb.jpeg}
\includegraphics[width=0.5\textwidth]{/home/link/Documents/thèse_onera/diapos_phd_thesis/images/EEG/info_content/complexity/hfd_cluster_FC_electrodes_by_detection_eb.jpeg}
\includegraphics[width=0.5\textwidth]{/home/link/Documents/thèse_onera/diapos_phd_thesis/images/EEG/info_content/complexity/pfd_cluster_FC_electrodes_by_detection_eb.jpeg}
\includegraphics[width=0.5\textwidth]{/home/link/Documents/thèse_onera/diapos_phd_thesis/images/EEG/info_content/complexity/hurst_cluster_FC_electrodes_by_detection_eb.jpeg}
\includegraphics[width=0.5\textwidth]{/home/link/Documents/thèse_onera/diapos_phd_thesis/images/EEG/info_content/complexity/kfd_cluster_FC_electrodes_by_detection_eb.jpeg}
\includegraphics[width=0.5\textwidth]{/home/link/Documents/thèse_onera/diapos_phd_thesis/images/EEG/info_content/complexity/hjorthm_cluster_FC_electrodes_by_detection_eb.jpeg}
\end{multicols}
\centering \includegraphics[width=0.5\textwidth]{/home/link/Documents/thèse_onera/diapos_phd_thesis/images/EEG/info_content/complexity/hjorthc_cluster_FC_electrodes_by_detection_eb.jpeg}
\caption[Valeurs des mesures de complexité calculées pour les électrodes du cluster FC]{Valeurs moyennes et barres d'erreur standard des mesures de complexité calculées pour chacune des électrodes du cluster fronto-central (FC1, FC2, FC3, FC4, FC5 et FC6) avant et après la détection (hit) ou la non-détection (miss). On retrouve ici des valeurs plus élevées pour les électrodes FC5 et FC6 pour toutes les mesures de complexité sauf l'exposant $\alpha$.}
\label{fig:figure5valeursmesurescomplexiteelectrode}
\end{figure*}

\begin{table}[!t]
\centering
\scriptsize
\caption[Table des résultats des analyses statistiques pour les mesures de complexité des électrodes du cluster FC]{Tables des résultats des analyses statistiques du modèle linéaire mixte Détection * Condition * Électrodes Cluster FC + $1~|~$Sujet réalisées sur les mesures de complexité avant et après la référence temporelle.}
\label{tab:table5statsmesurescomplexiteelectrode}

\textbf{Mesure de complexité $\sim$  Détection * Condition * Électrodes Cluster FC + $1~|~$Sujet}

% \begin{tabular}{lllllllll}
\begin{tabular}{|l|*{9}{c|}}
\hline
& \textbf{Measure} & \textbf{Sum Sq} & \textbf{Mean Sq} & \textbf{NumDF} & \textbf{DenDF} & \textbf{F value} & \textbf{Pr($>$F)} & \textbf{Sign.} \\ 
\hline
\textcolor{blue}{Détection} & & & & & & & & \\ 
\hline
& \textit{Alpha} & 0.09 & 0.09 & 1 & 425 & 15.96 & 0.0001 & *** \\ 
& \textit{Hurst} & 1.19 & 1.19 & 1 & 426 & 904.57 & $<$.0001 & *** \\ 
& \textit{HFD} & 0.16 & 0.16 & 1 & 425 & 71.76 & $<$.0001 & *** \\ 
& \textit{KFD} & 0.86 & 0.86 & 1 & 426 & 29.88 & $<$.0001 & *** \\ 
& \textit{PFD} & 0.00 & 0.00 & 1 & 425 & 5227.18 & $<$.0001 & *** \\ 
& \textit{Complexité Hjorth} & 135396.22 & 135396.22 & 1 & 427 & 133.95 & $<$.0001 & *** \\ 
& \textit{Mobilité Hjorth} & 0.00 & 0.00 & 1 & 426 & 35.63 & $<$.0001 & *** \\ 
\hline
\textcolor{blue}{Condition} & & & & & & & & \\ 
\hline
& Alpha & 0.00 & 0.00 & 1 & 425 & 0.06 & 0.8063 & \\ 
& Hurst & 0.00 & 0.00 & 1 & 425 & 0.35 & 0.5535 & \\ 
& HFD & 0.00 & 0.00 & 1 & 425 & 0.33 & 0.5634 & \\ 
& KFD & 0.01 & 0.01 & 1 & 425 & 0.41 & 0.5210 & \\ 
& PFD & 0.00 & 0.00 & 1 & 425 & 0.00 & 0.9952 & \\ 
& Complexité Hjorth & 1050.71 & 1050.71 & 1 & 424 & 1.04 & 0.3085 & \\ 
& Mobilité Hjorth & 0.00 & 0.00 & 1 & 425 & 0.28 & 0.5994 & \\ 
\hline
\textcolor{blue}{Electrode} & & & & & & & & \\ 
\hline
& \textit{Alpha} & 1.92 & 0.38 & 5 & 425 & 71.40 & $<$.0001 & *** \\ 
& \textit{Hurst} & 2.22 & 0.44 & 5 & 425 & 338.86 & $<$.0001 & *** \\ 
& \textit{HFD} & 1.37 & 0.27 & 5 & 425 & 125.17 & $<$.0001 & *** \\ 
& \textit{KFD} & 8.98 & 1.80 & 5 & 425 & 62.30 & $<$.0001 & *** \\ 
& \textit{PFD} & 0.01 & 0.00 & 5 & 425 & 2574.69 & $<$.0001 & *** \\ 
& \textit{Complexité Hjorth} & 206682.54 & 41336.51 & 5 & 424 & 40.89 & $<$.0001 & *** \\ 
& \textit{Mobilité Hjorth} & 0.00 & 0.00 & 5 & 425 & 63.01 & $<$.0001 & *** \\ 
\hline
\textcolor{blue}{Détection~*~Condition} & & & & & & & & \\ 
\hline
& Alpha & 0.00 & 0.00 & 1 & 425 & 0.51 & 0.4736 & \\ 
& \textit{Hurst} & 0.01 & 0.01 & 1 & 425 & 8.02 & 0.0049 & ** \\ 
& HFD & 0.00 & 0.00 & 1 & 425 & 1.43 & 0.2321 & \\ 
& KFD & 0.00 & 0.00 & 1 & 425 & 0.00 & 0.9935 & \\ 
& PFD & 0.00 & 0.00 & 1 & 425 & 0.73 & 0.3933 & \\ 
& Complexité Hjorth & 1543.11 & 1543.11 & 1 & 424 & 1.53 & 0.2173 & \\ 
& Mobilité Hjorth & 0.00 & 0.00 & 1 & 425 & 0.22 & 0.6392 & \\ 
\hline
\textcolor{blue}{Détection~*~Electrode} & & & & & & & & \\ 
\hline
& \textit{Alpha} & 0.13 & 0.03 & 5 & 425 & 4.96 & 0.0002 & *** \\ 
& \textit{Hurst} & 1.81 & 0.36 & 5 & 425 & 275.44 & $<$.0001 & *** \\ 
& \textit{HFD} & 0.22 & 0.04 & 5 & 425 & 20.48 & $<$.0001 & *** \\ 
& \textit{KFD} & 1.81 & 0.36 & 5 & 425 & 12.52 & $<$.0001 & *** \\ 
& \textit{PFD} & 0.01 & 0.00 & 5 & 425 & 2080.60 & $<$.0001 & *** \\ 
& \textit{Complexité Hjorth} & 272930.74 & 54586.15 & 5 & 424 & 54.00 & $<$.0001 & *** \\ 
& \textit{Mobilité Hjorth} & 0.00 & 0.00 & 5 & 425 & 15.77 & $<$.0001 & *** \\ 
\hline
\textcolor{blue}{Condition~*~Electrode} & & & & & & & & \\ 
\hline
& Alpha & 0.01 & 0.00 & 5 & 425 & 0.44 & 0.8194 & \\ 
& Hurst & 0.00 & 0.00 & 5 & 425 & 0.18 & 0.9706 & \\ 
& HFD & 0.00 & 0.00 & 5 & 425 & 0.29 & 0.9208 & \\ 
& KFD & 0.07 & 0.01 & 5 & 425 & 0.48 & 0.7941 & \\ 
& PFD & 0.00 & 0.00 & 5 & 425 & 0.06 & 0.9976 & \\ 
& Complexité Hjorth & 5808.51 & 1161.70 & 5 & 424 & 1.15 & 0.3336 & \\ 
& Mobilité Hjorth & 0.00 & 0.00 & 5 & 425 & 0.51 & 0.7694 & \\ 
\hline
\textcolor{blue}{Détection~*~Condition~*~Electrode} & & & & & & & & \\ 
\hline
& Alpha & 0.01 & 0.00 & 5 & 425 & 0.30 & 0.9152 & \\ 
& Hurst & 0.01 & 0.00 & 5 & 425 & 0.89 & 0.4899 & \\ 
& HFD & 0.00 & 0.00 & 5 & 425 & 0.28 & 0.9246 & \\ 
& KFD & 0.08 & 0.02 & 5 & 425 & 0.58 & 0.7121 & \\ 
& PFD & 0.00 & 0.00 & 5 & 425 & 0.45 & 0.8142 & \\ 
& Complexité Hjorth & 4955.83 & 991.17 & 5 & 424 & 0.98 & 0.4292 & \\ 
& Mobilité Hjorth & 0.00 & 0.00 & 5 & 425 & 0.50 & 0.7781 & \\ 
\hline
\end{tabular}

\end{table}

L'interaction entre la détection et la condition était statistiquement significative et faible pour l'exposant de \textbf{Hurst} ($F(1)=8.02$, $p=0.005$, $\eta^2=0,02$). 
L'interaction entre la détection et l'électrode était statistiquement significative et importante pour :
\begin{itemize}
\item[$\bullet$] \textbf{Alpha} ($F(5)=4.96$, $p<.001$, $\eta^2=0.06$) ;
\item[$\bullet$] \textbf{Hurst} ($F(5)=275.44$, $p<.0001$, $\eta^2=0.76$) ; 
\item[$\bullet$] \textbf{HFD} ($F(5)=20.48$, $p<.0001$, $\eta^2=0.19$) ;
\item[$\bullet$] \textbf{KFD} ($F(5)=12.52$, $p<.0001$, $\eta^2=0.13$) ;
\item[$\bullet$] \textbf{PFD} ($F(5)=2080.60$, $p<.0001$, $\eta^2=0.96$) ; 
\item[$\bullet$] \textbf{mobilité Hjorth} ($F(5)=15.77$, $p<.0001$, $\eta^2=0.16$) ;
\item[$\bullet$] \textbf{complexité Hjorth} ($F(5)=54.00$, $p<.0001$, $\eta^2=0.39$). \\
\end{itemize}

Les analyses statistiques des comparaisons multiples ont révélé une augmentation significative des valeurs de la mesure pour les cibles détectées par rapport aux cibles manquées sur l'électrode FC5 pour :
\begin{itemize}
\begin{multicols}{2}
\item[$\bullet$] \textbf{Hurst} ($t=22.28$, $p<.0001$) ;
\item[$\bullet$] \textbf{HFD} ($t=5.34$, $p<.0001$) ;
\item[$\bullet$] \textbf{KFD} ($t=4.43$, $p<.001$) ;
\item[$\bullet$] \textbf{PFD} ($t=62.47$, $p<.0001$) ;
\item[$\bullet$] \textbf{complexité Hjorth} ($t=11.43$, $p<.0001$) ;
\item[$\bullet$] \textbf{mobilité Hjorth} ($t=4.92$, $p<.0001$) ;
\end{multicols}
\end{itemize}
et sur l'électrode FC6 pour : 
\begin{itemize}
\begin{multicols}{2}
\item[$\bullet$] \textbf{Hurst} ($t=24.60$, $p<.0001$) ;
\item[$\bullet$] \textbf{HFD} ($t=7.33$, $p<.0001$) ;
\item[$\bullet$] \textbf{KFD} ($t=5.52$, $p<.001$) ;
\item[$\bullet$] \textbf{PFD} ($t=66.49$, $p<.0001$) ;
\item[$\bullet$] \textbf{complexité Hjorth} ($t=9.22$, $p<.0001$) ;
\item[$\bullet$] \textbf{mobilité Hjorth} ($t=6.23$, $p<.0001$). \\
\end{multicols}
\end{itemize} 

Inversement, une réduction significative des valeurs de la mesure a été principalement observée pour les cibles détectées par rapport aux cibles manquées sur l'électrode FC1 pour : 
\begin{itemize}
\begin{multicols}{2}
\item[$\bullet$] \textbf{Hurst} ($t=-12.20$, $p<.0001$) ;
\item[$\bullet$] \textbf{HFD} ($t=-3.53$, $p<.01$) ;
\item[$\bullet$] \textbf{PFD} ($t=-32.68$, $p<.0001$) ;
\item[$\bullet$] \textbf{complexité Hjorth} ($t=-4.99$, $p<.0001$) ; 
\item[$\bullet$] \textbf{mobilité Hjorth} ($t=-2.88$, $p<.05$) ;
\end{multicols}
\end{itemize}
sur l'électrode FC2 pour :
\begin{itemize}
\begin{multicols}{2}
\item[$\bullet$] \textbf{Hurst} ($t=-11.07$, $p<.0001$) ;
\item[$\bullet$] \textbf{HFD} ($t=-3.37$, $p<.01$) ;
\item[$\bullet$] \textbf{PFD} ($t=-32.09$, $p<.0001$) ;
\item[$\bullet$] \textbf{complexité Hjorth} ($t=-5.01$, $p<.0001$) ; 
\item[$\bullet$] \textbf{mobilité Hjorth} ($t=-3.06$, $p<.05$) ; 
\end{multicols}
\end{itemize}
sur l'électrode FC3 pour :
\begin{itemize}
\begin{multicols}{2}
\item[$\bullet$] \textbf{Hurst} ($t=-12.45$, $p<.0001$) ;
\item[$\bullet$] \textbf{HFD} ($t=-2.85$, $p<.05$) ;
\item[$\bullet$] \textbf{PFD} ($t=-32.39$, $p<.0001$) ; 
\item[$\bullet$] \textbf{complexité Hjorth} ($t=-5.11$, $p<.0001$) ;
\end{multicols}
\end{itemize}
et enfin sur l'électrode FC4 pour 
\begin{itemize}
\begin{multicols}{2}
\item[$\bullet$] \textbf{Hurst} ($t=-11.14$, $p<.0001$) ;
\item[$\bullet$] \textbf{HFD} ($t=-2.92$, $p<.05$) ;
\item[$\bullet$] \textbf{KFD} ($t=-2.70$, $p<.05$) ; 
\item[$\bullet$] \textbf{PFD} ($t=-31.79$, $p<.0001$) ;
\item[$\bullet$] \textbf{complexité Hjorth} ($t=-5.48$, $p<.0001$) ; 
\item[$\bullet$] \textbf{mobilité Hjorth} ($t=-2.98$, $p<.05$). \\
\end{multicols}
\end{itemize}

En résumé, dans le modèle Mesure de complexité $\sim$ Détection * Condition * Électrodes Cluster FC + $1|$Sujet, les facteurs détection et électrodes ainsi que leur interaction détection * électrode ont présenté un effet significatif sur l'ensemble des mesures de complexité, cependant l'interaction condition * détection * électrode n'a pas d'effet significatif sur les mesures de complexité. 
De plus, l'interaction détection * condition a montré un effet significatif pour l'exposant de Hurst seulement.  
Toutes les mesures ont montré des valeurs significativement supérieures pour les électrodes FC5 et FC6 lorsque les cibles auditives ont été détectées, excepté l'exposant $\alpha$. \\

Finalement, nous avons étudié la dynamique d'évolution des mesures de complexité autour de la référence temporelle au moyen d'une analyse statistique sur le cluster fronto-central en prenant en compte la détection et le fenêtrage temporel. 
La Table~\ref{tab:table5statsmesurescomplexitefenetre} reporte les résultats des analyses pour ce modèle Mesure de complexité $\sim$ Détection * Fenetre + $1|$Sujet. 
La Figure~\ref{fig:figure5valeursmesurescomplexitefenetre} montre l'évolution au cours du temps des mesures de complexité calculées sur les fenêtres de $1000$ points pour le cluster fronto-central entre $-3$ et $+3$ secondes autour de la référence temporelle (indiquée par la ligne verticale verte). 

Comparativement aux mesures d'entropie, les patterns sont plus hétérogènes.
Typiquement, les exposants \textbf{Alpha} et de \textbf{Hurst} affichent des valeurs inférieures avant la référence pour les cibles manquées comparativement aux cibles détectées. 
Au contraire, pour les mesures \textbf{HFD}, \textbf{KFD}, \textbf{PFD} et de \textbf{mobilité Hjorth}, des valeurs supérieures sont observées avant la référence pour les cibles détectées. 
On observe des valeurs assez similaires pour les cibles détectées et manquées pour le paramètre \textbf{complexité Hjorth} avant la référence. 
Après la référence, cependant, les valeurs semblent se compenser par de progressifs retours à la ligne de base. 
On voit notamment aussi pour l'exposant de \textbf{Hurst} et pour le paramètre de \textbf{complexité Hjorth} une augmentation importante vers $300$-$400$~ms après la référence.
À l'inverse, pour les dimensions fractales \textbf{HFD}, \textbf{KFD}, \textbf{PFD} et pour le paramètre de \textbf{mobilité Hjorth}, on observe plutôt une chute des valeurs de mesure sur plusieurs centaines de ms après la référence. \\

\begin{figure*}[!t]
\centering \textbf{Mesures de complexité en fonction du temps autour de la référence temporelle}\par\medskip
% \begin{multicols}{2}
% \includegraphics[width=0.5\textwidth]{/home/link/Documents/thèse_onera/diapos_phd_thesis/images/EEG/info_content/complexity/alpha_cluster_FC_window_by_detection_point_stats.jpeg}
% \includegraphics[width=0.5\textwidth]{/home/link/Documents/thèse_onera/diapos_phd_thesis/images/EEG/info_content/complexity/hfd_cluster_FC_window_by_detection_point_stats.jpeg}
% \includegraphics[width=0.5\textwidth]{/home/link/Documents/thèse_onera/diapos_phd_thesis/images/EEG/info_content/complexity/pfd_cluster_FC_window_by_detection_point_stats.jpeg}
% \includegraphics[width=0.5\textwidth]{/home/link/Documents/thèse_onera/diapos_phd_thesis/images/EEG/info_content/complexity/hurst_cluster_FC_window_by_detection_point_stats.jpeg}
% \includegraphics[width=0.5\textwidth]{/home/link/Documents/thèse_onera/diapos_phd_thesis/images/EEG/info_content/complexity/kfd_cluster_FC_window_by_detection_point_stats.jpeg}
% \includegraphics[width=0.5\textwidth]{/home/link/Documents/thèse_onera/diapos_phd_thesis/images/EEG/info_content/complexity/hjorthm_cluster_FC_window_by_detection_point_stats.jpeg}
% \end{multicols}
% \centering \includegraphics[width=0.5\textwidth]{/home/link/Documents/thèse_onera/diapos_phd_thesis/images/EEG/info_content/complexity/hjorthc_cluster_FC_window_by_detection_point_stats.jpeg}
\includegraphics[width=\textwidth]{Figures/illustrations/Exp_EEG/complexity_measures_dynamics.jpg}
\caption[Valeurs des mesures de complexité calculées sur le fenêtrage temporel pour le cluster FC]{Valeurs moyennes et barres SE des mesures de complexité calculées pour chacune des fenêtres de temps de $-3$ à $+3$ secondes, respectivement avant et après la détection (hit, en rouge) ou la non-détection (miss, en bleu) de la cible pour le cluster fronto-central. Les mesures ont été calculées à partir d'une fenêtre temporelle avec $1000$ points de données. Les points noirs montrent les fenêtres temporelles où une différence statistiquement significative a été trouvée dans les analyses entre la détection et la non-détection.}
\label{fig:figure5valeursmesurescomplexitefenetre}
\end{figure*}

\begin{table}[!t]
\centering
\scriptsize
\caption[Table des résultats des analyses statistiques pour les mesures de complexité fenêtrées du cluster FC]{Tables des résultats des analyses statistiques du modèle linéaire à effets mixtes Détection * Fenêtre + $1|$Sujet réalisées sur les mesures de complexité avant et après la référence temporelle.}
\label{tab:table5statsmesurescomplexitefenetre}

\textbf{Mesure de complexité $\sim$ Détection * Fenêtre + $1|$Sujet}

% \begin{tabular}{lllllllll}
\begin{tabular}{|l|*{9}{c|}}
\hline
& \textbf{Measure} & \textbf{Sum Sq} & \textbf{Mean Sq} & \textbf{NumDF} & \textbf{DenDF} & \textbf{F value} & \textbf{Pr($>$F)} & \textbf{Sign.} \\ 
\hline
\textcolor{blue}{Détection} & & & & & & & & \\ 
\hline
& \textit{Alpha} & 0.00 & 0.00 & 1 & 795 & 9.66 & 0.0019 & ** \\ 
& \textit{Hurst} & 0.00 & 0.00 & 1 & 795 & 4.46 & 0.0351 & * \\ 
& \textit{HFD} & 0.00 & 0.00 & 1 & 795 & 8.50 & 0.0036 & ** \\ 
& KFD & 0.00 & 0.00 & 1 & 795 & 0.02 & 0.8802 & \\ 
& PFD & 0.00 & 0.00 & 1 & 795 & 0.68 & 0.4099 & \\ 
& \textit{Complexité Hjorth} & 8580.97 & 8580.97 & 1 & 795 & 19.69 & $<$.0001 & *** \\ 
& Mobilité Hjorth & 0.00 & 0.00 & 1 & 795 & 0.01 & 0.9242 & \\ 
\hline
\textcolor{blue}{Fenêtre} & & & & & & & & \\ 
\hline
& \textit{Alpha} & 0.00 & 0.00 & 21 & 795 & 1.62 & 0.0390 & * \\
& \textit{Hurst} & 0.00 & 0.00 & 21 & 795 & 3.13 & $<$.0001 & *** \\ 
& HFD & 0.00 & 0.00 & 21 & 795 & 1.38 & 0.1185 & \\ 
& KFD & 0.00 & 0.00 & 21 & 795 & 1.35 & 0.1373 & \\ 
& \textit{PFD} & 0.00 & 0.00 & 21 & 795 & 1.61 & 0.0413 & * \\ 
& \textit{Complexité Hjorth} & 16819.59 & 800.93 & 21 & 795 & 1.84 & 0.0124 & * \\ 
& Mobilité Hjorth & 0.00 & 0.00 & 21 & 795 & 0.94 & 0.5437 & \\ 
\hline
\textcolor{blue}{Détection~*~Fenêtre} & & & & & & & & \\ 
\hline
& \textit{Alpha} & 0.00 & 0.00 & 21 & 795 & 4.64 & $<$.0001 & *** \\ 
& \textit{Hurst} & 0.00 & 0.00 & 21 & 795 & 2.30 & 0.0008 & *** \\ 
& \textit{HFD} & 0.00 & 0.00 & 21 & 795 & 4.86 & $<$.0001 & *** \\ 
& KFD & 0.00 & 0.00 & 21 & 795 & 1.47 & 0.0803 & \\ 
& \textit{PFD} & 0.00 & 0.00 & 21 & 795 & 3.31 & $<$.0001 & *** \\ 
& Complexité Hjorth & 9301.48 & 442.93 & 21 & 795 & 1.02 & 0.4400 & \\ 
& Mobilité Hjorth & 0.00 & 0.00 & 21 & 795 & 1.19 & 0.2490 & \\ 
\hline
\end{tabular}

\end{table}

Les effets significatifs mis en évidence par l'analyse de variance sont : 

\begin{itemize}
\item[$\bullet$] \underline{pour l'exposant \textbf{Alpha}} : 
\begin{itemize} 
\item l'effet principal de la détection est significatif et faible ($F(1)=9.66$, $p=0.002$, $\eta^2=0.01$) ; 
\item l'effet principal de la fenêtre est significatif et faible ($F(21)=1.62$, $p=0.039$, $\eta^2=0.04$) ;
\item l'interaction entre la détection et la fenêtre est significative et moyenne ($F(21)=4.64$, $p<0.001$, $\eta^2=0.11$) ; 
\end{itemize}
\item[$\bullet$] \underline{pour l'exposant de \textbf{Hurst}} : 
\begin{itemize} 
\item l'effet principal de détection est significatif et très faible ($F(1)=4.46$, $p=0.035$, $\eta^2=5.57\times10^{-3}$) ; 
\item l'effet principal de la fenêtre est significatif et moyen ($F(21)=3.13$, $p<0.001$, $\eta^2=0.08$) ;
\item l'interaction entre la détection et la fenêtre est significative et faible ($F(21)=2.30$, $p<0.001$, $\eta^2=0.06$) ; 
\end{itemize}
\item[$\bullet$] \underline{pour la dimension fractale de Higuchi \textbf{HFD}} : 
\begin{itemize} 
\item l'effet principal de la détection est significatif et faible ($F(1)=8.50$, $p=0.004$, $\eta^2=0.01$) ; 
\item l'interaction entre la détection et la fenêtre est significative et moyenne ($F(21)=4.86$, $p<0.001$, $\eta^2=0.11$) ;
\end{itemize}
\item[$\bullet$] \underline{pour la dimension fractale de Petrosian \textbf{PFD}} : 
\begin{itemize} 
\item l'effet principal de la fenêtre est significatif et faible ($F(21)=1.61$, $p=0.041$, $\eta^2=0.04$) ; 
\item l'interaction entre la détection et la fenêtre est significative et moyenne ($F(21)=3.31$, $p<0.001$, $\eta^2=0.08$) ;
\end{itemize}
\item[$\bullet$] \underline{pour le paramètre de \textbf{complexité Hjorth}} : 
\begin{itemize} 
\item l'effet principal de la détection est significatif et faible ($F(1)=19.69$, $p<.001$, $\eta^2=0.02$) ; 
\item l'effet principal de la fenêtre est significatif et faible ($F(21)=1.84$, $p=0.012$, $\eta^2=0.05$). \\
\end{itemize}
\end{itemize}

Enfin, l'interaction entre la détection et la fenêtre a montré un effet significatif pour quatre mesures : l'exposant \textbf{Alpha}, l'exposant de \textbf{Hurst} et les dimensions fractales \textbf{HFD} et \textbf{PFD}.
Les analyses par comparaisons multiples pour les fenêtres entre les cibles détectées et les cibles non-détectées ont permis de déterminer les fenêtres temporelles pour lesquelles la détection était significative (représentées sur la Figure~\ref{fig:figure5valeursmesurescomplexitefenetre} par des points noirs). 
Avant la référence, \textbf{HFD} et \textbf{PFD} présentent plusieurs fenêtres pour lesquelles des valeurs significativement supérieures sont observées pour les cibles détectées. 
Après la référence, \textbf{HFD}, \textbf{PFD} et l'exposant de \textbf{Hurst} montrent des fenêtres significatives. 
De manière importante, seul l'exposant de \textbf{Hurst} montre un pic de valeurs significatif juste après la référence pour les cibles détectées comparativement aux cibles manquées.

%%%%%%%%%%%%%%%%%%%%%%%%%%%%%%%%%%%%%%%%%%%%%%%%%%%%%%%%%%%%%%%%%%%%%%%%%%%%%%%
\subsection{Synthèse et discussion pour les mesures d'entropie et de complexité}
\label{synthesediscussionmesuresentropiecomplexite}
%%%%%%%%%%%%%%%%%%%%%%%%%%%%%%%%%%%%%%%%%%%%%%%%%%%%%%%%%%%%%%%%%%%%%%%%%%%%%%%

Dans cette section, nous avons étudié l’évolution des mesures d'entropie et de complexité des signaux EEG au moment de la perception consciente de la cible auditive dans le MI. 
Pour y parvenir, nous avons étudié la performance de cinq mesures d'entropie et sept mesures de complexité pour évaluer l'effet de la perception auditive consciente sur la dynamique de l'activité cérébrale macroscopique EEG. 
Afin d'évaluer l'effet de la conscience perceptive auditive, nous avons comparé la différence entre les cibles perçues et non-perçues, avant et après la perception, pour les mesures d'entropie, d'une part, et pour les mesures de complexité, d'autre part. 
Pour cela, nous avons employé une procédure d'agrégation topographique des électrodes en clusters définis comme la moyenne arithmétique des valeurs de mesure sur l'ensemble des électrodes d'un cluster afin de pouvoir disposer de mesures moyennes associées à des zones cérébrales localement définies \citep{grabner2012oscillatory}. 
Nous avons ainsi cherché à trouver à la fois des clusters, des électrodes et des fenêtres temporelles permettant de discriminer statistiquement la perception auditive consciente à l'échelle de l'activité cérébrale macroscopique.

Les résultats ont montré qu'au sein du cluster fronto-central, les cibles détectées correspondaient à des valeurs de mesures d'entropie et de complexité significativement plus élevées que les cibles non-détectées, excepté pour la mesure $\alpha$.
Dans les mesures d'entropie, les valeurs de l'entropie de permutation (PeEn) était statistiquement moins élevées pour les cibles détectées dans les clusters antéro-frontal, frontal, centro-pariétal, pariétal, pariéto-occipital et sagittal, comparativement aux cibles qui n'ont pas été détectées.  
Dans les mesures de complexité, seule la dimension fractale de Petrosian (PFD) présentait des valeurs statistiquement moins élevées pour les cibles détectées au sein de tous les autres clusters (autres que le cluster FC), comparativement aux cibles non détectées. 
De manière importante, l'entropie de permutation (PeEn) et la dimension fractale de Petrosian (PFD) étaient les seules mesures à présenter des valeurs statistiquement plus élevées dans le cluster FC et moins élevées dans la plupart des autres clusters lors de la perception auditive consciente. 
De cette manière, les signaux issus de l'aire cérébrale fronto-centrale peuvent être associés à une augmentation de leur contenu en information et en complexité lorsque une cible auditive devient perçue par le sujet. 

De plus, les résultats ont montré que toutes les mesures d'entropie et de complexité, excepté l'exposant $\alpha$, présentent des valeurs significativement supérieures pour les électrodes FC5 et FC6 lors de la perception consciente. 
Les électrodes FC5 et FC6 sont des électrodes adjacentes au cluster temporal et pourraient être vraisemblablement associées à des localisations du lobe temporal comme par exemple le gyrus temporal supérieur. 
En outre, l'entropie de permutation (PeEn) discrimine entre les cibles détectées et non-détectées pour quasiment tous les clusters (sauf les clusters central et temporal), tandis que la dimension fractale de Petrosian (PFD) discrimine entre les cibles détectées et non-détectées pour tous les clusters.
À partir de ces résultats, seul l'exposant fractal $\alpha$ ne peut pas être utilisé pour discriminer la perception auditive consciente de son absence dans le cluster fronto-central, tandis que seule la dimension fractale de Petrosian (PFD) peut être utilisée pour la discriminer dans tous les clusters. 

Ces résultats peuvent être mis en relation avec l'étude par IRMf de \cite{overath2007information} dans laquelle les auteurs ont cherché les zones cérébrales pour lesquelles la variation d'entropie contenue dans des séquences de tonalités auditives augmentait l'activité neuronale et les demandes énergétiques. 
Dans cette étude, le planum temporal correspondait à un centre neuronal fonctionnel consommant une quantité moindre de ressources computationnelles pour encoder les signaux présentant de la redondance par rapport à ceux présentant un contenu en information élevé. 
Dans notre analyse, les deux électrodes FC5 et FC6 pourraient être associées au recueil d'une activité générée au niveau de ce centre neuronal, en raison de leur localisation relativement proche du gyrus temporal supérieur. 
Dans ce cas, les valeurs de mesures d'entropie supérieures lors de la perception auditive consciente de la cible pourraient être le reflet de l'activation du planum temporal en réponse à la demande de traitement des flux issus de la cible et du masqueur. 
Si le planum temporal correspond à un centre neural efficient pour l'encodage des propriétés statistiques des signaux acoustiques, alors les valeurs augmentées lors de la perception auditive consciente des mesures d'entropie pourraient en être une signature neuronale caractéristique. 

L'évolution des mesures d'entropie et de complexité sur le décours temporel de l'essai a ensuite permis d'observer que les mesures d'entropie présentaient des valeurs plus élevées avant la référence pour les cibles détectées. 
Les valeurs apparaissaient ensuite revenir au niveau de celles des cibles non-détectées après la référence. 
Ce résultat peut suggérer un contenu en information plus élevé lors de la construction du percept conscient à l'échelle du signal cérébral. 
Si les flux auditifs en provenance de la cible et du masqueur sont traités de manière récurrente à différents niveaux dans la chaîne de traitement de l'information, on peut suggérer qu'un certain degré de redondance de l'information soit disponible à l'échelle du signal neuronal. 
Cette redondance pourrait alors être reflétée dans des transitions du contenu en information dans le signal neuronal lors de la dynamique de la construction du percept auditif conscient. 
Lorsque la cible a été perçue et que l'information qui lui est associée a été traitée, on pourrait s'attendre à observer une diminution des valeurs de mesures d'entropie puisque le contenu en information au niveau du signal neuronal serait progressivement diminué. 

La dynamique temporelle était plus nuancée pour les mesures de complexité. 
Quatre mesures (exposant $\alpha$, exposant de Hurst, HFD et PFD) ont montré une interaction significative entre la détection et le décours temporel. 
Avant la référence, seules les dimensions fractales HFD et PFD ont montré des fenêtres temporelles significatives avec des valeurs supérieures lors de la perception auditive consciente, tandis qu'après la référence, HFD, PFD et également l'exposant de Hurst ont montré de telles fenêtres significatives. 
De plus, seul l'exposant de Hurst a exprimé une variation d'amplitude significative dans la fenêtre temporelle juste après la référence pour les cibles détectées comparativement à celles manquées.
Bien qu'une forte variation d'amplitude ait été observée de manière similaire juste après le report pour le paramètre de complexité de Hjorth, l'interaction entre la détection et la fenêtre n'était pas significative. 

Dans le cluster fronto-central, une variation prononcée de l'entropie de décomposition en valeurs singulières (SvEn), de l'exposant de Hurst et du paramètre de complexité de Hjorth a été observée dans la fenêtre temporelle suivant la référence lors de la perception auditive consciente. 
Néanmoins, l'exposant de Hurst était la seule mesure a montrer un effet significatif de la détection pour cette fenêtre temporelle suivant la référence d'environ $400$~ms, et les valeurs de l'exposant y étaient significativement supérieures pour les cibles détectées comparativement à celles manquées. 
L'exposant de Hurst a usuellement été utilisé comme une mesure statistique pour caractériser les corrélations à long-terme d'un signal temporel \citep{hurst1951long}. 
Concrètement, une hausse de l'exposant de Hurst dans le signal neuronal fronto-central, associée à la perception consciente d'une cible composée de tonalités auditives, correspond à une élèvation du niveau de corrélations long-termes au sein de ce signal. 
Cela signifie que la perception auditive consciente a augmenté la structure de corrélations statistiques dans l'aire cérébrale fronto-centrale et ce, de manière visible juste après le report perceptif du sujet. 

Malheureusement, nous n'avons pas réalisé une analyse plus détaillée de la dynamique de la construction pour chacune des électrodes du cluster fronto-central. 
Cependant, on pourrait s'attendre à ce que les électrodes les plus latérales FC5 et FC6 présentent de telles hausses, ce qui signifierait que la conscience perceptive de la cible auditive pourrait reposer sur une augmentation de l'activation au sein du planum temporal vis-à-vis de l'augmentation du traitement des informations statistiques et redondantes associées au flux de la cible et du masqueur. 
De plus, cette augmentation de l'exposant de Hurst suivant le report perceptif du sujet peut se comprendre comme un rapprochement de la valeur maximale de l'exposant (\textit{i.e.} 1).
De cette manière, la conscience perceptive de la cible amenerait le signal neuronal à devenir un signal de plus en plus persistant (puisque $0,5<H<1$) et à présenter une auto-corrélation positive long-terme de plus en plus forte, suggérant ainsi une possible amplification des mécanismes de traitement de l'information au sein de cette zone cérébrale. 

De cette manière, la perception auditive consciente a suscité des variations significatives des valeurs de mesure dans le cluster cérébral fronto-central. 
Chaque mesure présentée dans cette analyse quantifie un aspect spécifique du signal neuronal et, même si elles partagent le terme «entropie», les mesures d'entropie se basent sur des propriétés différentes du signal. 
Les différentes mesures d'entropie et de complexité utilisées ici quantifient donc des propriétés linéaires ou non-linéaires en fonction des caractéristiques statistiques du signal neuronal. 
Certaines mesures sont basées sur des calculs dans le domaine fréquentiel tandis que d'autres réalisent leurs estimations dans le domaine temporel. 

Un exemple important est l'entropie spectrale, qui quantifie la variation de forme du spectre de puissance dans le domaine fréquentiel, et ne peut donc pas être comparée à des techniques analytiques dans le domaine temporel. 
Un autre exemple est la différence qu'il existe entre l'entropie de permutation (PeEn) et l'entropie approximée (ApEn). 
L'entropie de permutation (PeEn) évalue la distribution de probabilité des patterns de rang d'amplitude dans le signal, tandis que l'entropie approximée (ApEn) évalue la probabilité que des patterns d'amplitude absolue similaires détectés dans le signal restent similaires s'ils sont étendus d'une valeur d'amplitude supplémentaire. 
Par conséquent, elles ont des forces différentes dans leur capacité à caractériser le signal neuronal. 
L'entropie approximée (ApEn) a montré notamment un avantage dans la discrimination de différents niveaux d'inconscience, tandis que l'entropie de permutation (PeEn) a montré un avantage pour séparer les états de conscience des états d'inconscience \citep{schneider2014monitoring}. 
Dans notre analyse, nous montrons que les différentes mesures d'entropie sont toutes à même de discriminer la perception auditive consciente dans le cluster fronto-central et que les cinq mesures ont présenté des valeurs significativement supérieures pour les électrodes les plus latérales du cluster (FC5 et FC6) lorsque les cibles auditives ont été détectées. 
De cette manière, la conscience perceptive d'une cible auditive peut être caractérisée par des variations des propriétés à l'échelle macroscopique du signal neuronal, montrant la possibilité d'utiliser les différents algorithmes dans le cadre de la discrimination des états perceptifs conscients et donc des contenus de conscience associés aux représentations cérébrales.

Finalement, nous apportons ici un argument experimental selon lequel les mesures d'entropies et de complexité, sur la base des caractéristiques statistiques du signal neuronal qu'elles représentent, permettent de discriminer l'effet de la perception auditive consciente dans le cadre d'un protocole de MI chez le sujet sain. 
Parmi elles, au moins une mesure de complexité, l'exposant de Hurst, présente un pic de valeurs important suivant le report perceptif du sujet. 
Bien que ce pic de valeurs observé pour l'exposant de Hurst puisse ne pas être strictement associé à la conscience perceptive, mais plutôt à des processus post-perceptuels comme dans le cadre des discussions récentes sur la composante P300 \citep{cohen2020distinguishing, fishman2021learning, pitts2014gamma, pitts2014isolating, tsuchiya2015no}, il permet dans cette analyse de discriminer la perception de l'absence de perception par le sujet, et représente ainsi un indicateur pertinent pour la problématique pratique associée à ce travail. 

%%%%%%%%%%%%%%%%%%%%%%%%%%%%%%%%%%%%%%%%%%%%%%%%%%%%%%%%%%%%%%%%%%%%%%%%%%%%%%%%%%%%%%%%%%%%%%%%%%%%%%%%%%%%%%%%%%%%%%%%%%%%%%%%%%%%%%%%%%%%%%%%%%%%%%%%%%%%%%%%%%%%%%%%%%%%%%%%%%%%%%%%%%%%%%%%%%%%%%%%%%%%%%%%%%%%%%%%%%%%%%%%%%%%%%%%%%%%%%%
%%%%%%%%%%%%%%%%%%%%%%%%%%%%%%%%%%%%%%%%%%%%%%%%%%%%%%%%%%%%%%%%%%%%%%%%%%%%%%%%%%%%%%%%%%%%%%%%%%%%%%%%%%%%%%%%%%%%%%%%%%%%%%%%%%%%%%%%%%%%%%%%%%%%%%%%%%%%%%%%%%%%%%%%%%%%%%%%%%%%%%%%%%%%%%%%%%%%%%%%%%%%%%%%%%%%%%%%%%%%%%%%%%%%%%%%%%%%%%%
%%%%%%%%%%%%%%%%%%%%%%%%%%%%%%%%%%%%%%%%%%%%%%%%%%%%%%%%%%%%%%%%%%%%%%%%%%%%%%%%%%%%%%%%%%%%%%%%%%%%%%%%%%%%%%%%%%%%%%%%%%%%%%%%%%%%%%%%%%%%%%%%%%%%%%%%%%%%%%%%%%%%%%%%%%%%%%%%%%%%%%%%%%%%%%%%%%%%%%%%%%%%%%%%%%%%%%%%%%%%%%%%%%%%%%%%%%%%%%%
%%%%%%%%%%%%%%%%%%%%%%%%%%%%%%%%%%%%%%%%%%%%%%%%%%%%%%%%%%%%%%%%%%%%%%%%%%%%%%%%%%%%%%%%%%%%%%%%%%%%%%%%%%%%%%%%%%%%%%%%%%%%%%%%%%%%%%%%%%%%%%%%%%%%%%%%%%%%%%%%%%%%%%%%%%%%%%%%%%%%%%%%%%%%%%%%%%%%%%%%%%%%%%%%%%%%%%%%%%%%%%%%%%%%%%%%%%%%%%%
%%%%%%%%%%%%%%%%%%%%%%%%%%%%%%%%%%%%%%%%%%%%%%%%%%%%%%%%%%%%%%%%%%%%%%%%%%%%%%%%%%%%%%%%%%%%%%%%%%%%%%%%%%%%%%%%%%%%%%%%%%%%%%%%%%%%%%%%%%%%%%%%%%%%%%%%%%%%%%%%%%%%%%%%%%%%%%%%%%%%%%%%%%%%%%%%%%%%%%%%%%%%%%%%%%%%%%%%%%%%%%%%%%%%%%%%%%%%%%%
%%%%%%%%%%%%%%%%%%%%%%%%%%%%%%%%%%%%%%%%%%%%%%%%%%%%%%%%%%%%%%%%%%%%%%%%%%%%%%%%%%%%%%%%%%%%%%%%%%%%%%%%%%%%%%%%%%%%%%%%%%%%%%%%%%%%%%%%%%%%%%%%%%%%%%%%%%%%%%%%%%%%%%%%%%%%%%%%%%%%%%%%%%%%%%%%%%%%%%%%%%%%%%%%%%%%%%%%%%%%%%%%%%%%%%%%%%%%%%%
%%%%%%%%%%%%%%%%%%%%%%%%%%%%%%%%%%%%%%%%%%%%%%%%%%%%%%%%%%%%%%%%%%%%%%%%%%%%%%%%%%%%%%%%%%%%%%%%%%%%%%%%%%%%%%%%%%%%%%%%%%%%%%%%%%%%%%%%%%%%%%%%%%%%%%%%%%%%%%%%%%%%%%%%%%%%%%%%%%%%%%%%%%%%%%%%%%%%%%%%%%%%%%%%%%%%%%%%%%%%%%%%%%%%%%%%%%%%%%%
%%%%%%%%%%%%%%%%%%%%%%%%%%%%%%%%%%%%%%%%%%%%%%%%%%%%%%%%%%%%%%%%%%%%%%%%%%%%%%%%%%%%%%%%%%%%%%%%%%%%%%%%%%%%%%%%%%%%%%%%%%%%%%%%%%%%%%%%%%%%%%%%%%%%%%%%%%%%%%%%%%%%%%%%%%%%%%%%%%%%%%%%%%%%%%%%%%%%%%%%%%%%%%%%%%%%%%%%%%%%%%%%%%%%%%%%%%%%%%%
%%%%%%%%%%%%%%%%%%%%%%%%%%%%%%%%%%%%%%%%%%%%%%%%%%%%%%%%%%%%%%%%%%%%%%%%%%%%%%%%%%%%%%%%%%%%%%%%%%%%%%%%%%%%%%%%%%%%%%%%%%%%%%%%%%%%%%%%%%%%%%%%%%%%%%%%%%%%%%%%%%%%%%%%%%%%%%%%%%%%%%%%%%%%%%%%%%%%%%%%%%%%%%%%%%%%%%%%%%%%%%%%%%%%%%%%%%%%%%%
%%%%%%%%%%%%%%%%%%%%%%%%%%%%%%%%%%%%%%%%%%%%%%%%%%%%%%%%%%%%%%%%%%%%%%%%%%%%%%%%%%%%%%%%%%%%%%%%%%%%%%%%%%%%%%%%%%%%%%%%%%%%%%%%%%%%%%%%%%%%%%%%%%%%%%%%%%%%%%%%%%%%%%%%%%%%%%%%%%%%%%%%%%%%%%%%%%%%%%%%%%%%%%%%%%%%%%%%%%%%%%%%%%%%%%%%%%%%%%%

%%%%%%%%%%%%%%%%%%%%%%%%%%%%%%%%%%%%%%%%%%%%%%%%%%%%%%%%%%%%%%%%%%%%%%%%%%%%%%%
\clearpage
\section{Transmission de l'information à l'échelle cérébrale dans le MI}
\label{transmissiondinfo}
%%%%%%%%%%%%%%%%%%%%%%%%%%%%%%%%%%%%%%%%%%%%%%%%%%%%%%%%%%%%%%%%%%%%%%%%%%%%%%%

Le troisième ensemble de mesures que nous avons utilisé permet de caractériser la transmission de l'information à l'échelle cérébrale lors de la perception du flux de tonalités cible dans le MI. 
Nous avons cherché à déterminer les zones cérébrales où le transfert d'information serait modifié suite à l'accès conscient de la cible auditive. 
Dans le cadre du MI, la transmission de l'information entre les différentes zones cérébrales lors de la perception consciente de la cible auditive n'a pas été spécifiquement étudiée. 
En effet, les études de neuroimagerie sur la connectivité fonctionnelle et le transfert d'information au sein du réseau fronto-temporo-pariétal dans une situation de masquage sonore font largement défaut. 

Les mesures classiques de connectivité fonctionnelle principalement linéaires ne permettent pas de prendre en compte la non-linéarité des signaux cérébraux. 
Nous avons vu dans la Section \ref{theorieinformationmesuresassociees} que plusieurs mesures issues de la Théorie de l'information sont plus sensibles à l'égard de ce type de propriétés des signaux. 
Certaines de ces mesures nous permettent d'étudier la transmission d'information, dirigée ou non, à l'échelle du cerveau, en s'appuyant sur les signaux neuronaux issus de l'EEG. 
Parmi ces mesures, l'information mutuelle (IM) est une mesure de connectivité fonctionnelle non-dirigée (symétrique), et l'entropie de transfert (TE) est une mesure de connectivité fonctionnelle dirigée (non-symétrique). 
Pour cette analyse, nous avons donc utilisé des mesures de dépendance linéaire classiques (corrélation et covariance), ainsi que des mesures de transfert d'information plus spécifiques de dépendance non-linéaire (IM et TE) pour étudier la transmission de l’information à l’échelle cérébrale dans le MI.

Ces mesures sont usuellement obtenues sur la base d'un appariement des signaux issus des différentes électrodes, donnant lieu à une valeur de mesure pour chaque paire d'électrodes. 
L'utilisation de graphes, comme les graphes de corrélation, peuvent être intéressants pour représenter des liens statistiques entre les noeuds considérés. 
De fait, nous étudions ici l'information issue de graphes de corrélation, de covariance, d'information mutuelle et d'entropie de transfert entre les signaux temporels des différentes électrodes.

Nous avons mentionné précédemment que l'activité cérébrale associée à un réseau fronto-temporo-pariétal a été considérée comme un substrat neuronal essentiel dans les mécanismes et les processus associés à la perception consciente \citep{demertzi2013consciousness, eriksson2007similar, giani2015detecting}. 
Des changements significatifs du signal neuronal dans le cortex auditif primaire ont précédemment été associés à la perception de la cible et par conséquent à la ségrégation des flux auditifs \citep{kondo2009involvement}. 
De plus, l'activité du cortex pariétal semble essentielle à l'organisation perceptive et plus particulièrement à la liaison des différentes caractéristiques des objets auditifs \citep{cusack2005intraparietal}. 
Il a aussi été montré que l'activité neuronale au niveau du lobe pariétal peut être modulée par des demandes d'intégration liées au stimulus auditif dans le MI, notamment lorsque le sujet perçoit la cible \citep{eriksson2017activity}. 
Ainsi, l'activité du cortex pariétal est augmentée lorsque l'organisation perceptive donne lieu à deux flux auditifs plutôt qu'un seul. 
Récemment, \cite{pereira2021evidence} ont montré que la perception réussie de stimuli implique une accumulation d'évidence vers un centre informationnel et fonctionnel orchestré par le cortex pariétal postérieur. 
Ils ont suggéré que cette zone cérébrale, en tant que borne de décision, puisse servir de déclencheur à l'«embrasement» neuronal sous-jacent à la conscience perceptive du stimulus. 

Sur la base de ces différents résultats, il apparaît que des processus d'intégration et de transmission de l'information associés à un traitement cérébral distribué peuvent être à même d'expliquer une partie de l'activité neuronale associée aux expériences conscientes lors de la perception auditive de tonalités cibles. 
Nous avons donc supposé qu'une transmission de l'information plus importante serait observée lors de la perception consciente de la cible auditive au sein du réseau fronto-temporo-pariétal. 
Nous avons également supposé qu'un transfert d'information plus important serait dirigé vers les cortex pariétaux. 
De cette manière, la perception auditive consciente dans le MI pourrait être caractérisée par des mécanismes de transfert d'information plus importants entre les cortex frontal, temporal et pariétal, ce qui se refléterait dans les mesures d'IM et de TE pour les cibles détectées comparativement aux cibles non-détectées. 

%%%%%%%%%%%%%%%%%%%%%%%%%%%%%%%%%%%%%%%%%%%%%%%%%%%%%%%%%%%%%%%%%%%%%%%%%%%%%%%
\subsection{Procédure}
\label{proceduretransmissiondinfo}
%%%%%%%%%%%%%%%%%%%%%%%%%%%%%%%%%%%%%%%%%%%%%%%%%%%%%%%%%%%%%%%%%%%%%%%%%%%%%%%

%%%%%%%%%%%%%%%%%%%%%%%%%%%%%%%%%%%%%%%%%%%%%%%%%%%%%%%%%%%%%%%%%%%%%%%%%%%%%%%
\subsubsection{Prétraitement}
\label{pretraitementtransmissiondinfo}
%%%%%%%%%%%%%%%%%%%%%%%%%%%%%%%%%%%%%%%%%%%%%%%%%%%%%%%%%%%%%%%%%%%%%%%%%%%%%%%

Dans le cadre de cette analyse, les mêmes données EEG issues de la tâche de MI précédemment définie ont été utilisées (Section \ref{etude2materielmethode}). 
Les données ont été reréférencées à la moyenne des électrodes, puis le signal EEG a été sous-échantillonné à $500$~Hz et des filtres non-causaux passe-bas ($80$~Hz) et passe-haut ($1$~Hz) ont été appliqués aux données. 
Ensuite, les procédures de prétraitements ont été appliquées de la même manière que décrites précédemment (\textit{i.e.}, rejet des artefacts, ICA, autoreject, inspection visuelle). 
Nous avons considéré la différence entre les conditions «après» et «avant» la référence temporelle comme précédemment (\textit{i.e.}, temps de détection des sujets pour les détections correctes et moyenne des temps de détection des sujets pour les détections manquées, voir Section \ref{etude2traitementetanalysesEEG}). 

%%%%%%%%%%%%%%%%%%%%%%%%%%%%%%%%%%%%%%%%%%%%%%%%%%%%%%%%%%%%%%%%%%%%%%%%%%%%%%%
\subsubsection{Tests statistiques par permutation de clusters non-paramétriques}
\label{testspermutclusters}
%%%%%%%%%%%%%%%%%%%%%%%%%%%%%%%%%%%%%%%%%%%%%%%%%%%%%%%%%%%%%%%%%%%%%%%%%%%%%%%

Afin d'avoir une méthodologie statistique appropriée à cette analyse, nous avons employé une approche de tests statistiques non-paramétriques par permutation basés sur des groupes ou clusters d'électrodes. 
Ces tests permettent à partir de statistiques basées sur taille et poids de clusters, de trouver des clusters et des liens qui diffèrent de manière significative pour les facteurs de détection et d'électrodes étudiés. 
De cette manière, les mesures obtenues à partir des signaux issus de paires d'électrodes, et analysées à l'aide d'une approche statistique basée sur les clusters, vont nous permettre de mieux caractériser la transmission d'information entre zones cérébrales lors de la perception de la cible auditive dans le MI.

Les différentes mesures --- corrélation, covariance, information mutuelle et entropie de transfert --- ont ainsi été analysées en utilisant des statistiques basées sur les tests non-paramétriques de permutation de clusters \citep{maris2007nonparametric, maris2007nonparametricstats, nichols2002nonparametric}. 
Pour chaque paire de signaux et chaque sujet, la valeur de la mesure est calculée pour chaque groupe de détection (cibles détectées et cibles manquées) et pour chaque groupe de condition (avant et après la référence). 
Pour chaque mesure, on prend la différence $\delta$ entre la valeur de la mesure après la référence et la valeur de la mesure avant la référence (par exemple, pour l'IM, on a donc : $\delta$ = IM avant - IM après). 
On obtient ainsi deux ensembles de données, l'un comportant les indices $\delta$ pour chaque paire d'électrodes pour les cibles détectées, et l'autre comportant les indices $\delta$ pour chaque paire d'électrodes pour les cibles manquées. 
Ensuite, pour chaque mesure, le test est composé des étapes suivantes :

1. La différence entre les valeurs de mesure de chaque ensemble de données, notée $\Delta$, est calculée et correspond à la statistique expérimentale. 
Un premier test de permutation est ensuite utilisé pour évaluer la significativité de cette statistique expérimentale. 
L'hypothèse nulle est définie comme la situation où il n'y a pas de différence entre les cibles détectées et les cibles manquées. 
La distribution des $\Delta$ sous l'hypothèse nulle est alors obtenue en assignant aléatoirement les valeurs de mesure à l'un des deux groupes et en calculant $\Delta$ pour chaque permutation. 
L'ensemble des valeurs de $\Delta$ obtenues pour un grand nombre de permutations donne ainsi une estimation de la distribution de $\Delta$ sous l'hypothèse nulle. 
La valeur expérimentale de la statistique est considérée comme significative si elle s'écarte de la distribution de permutation pour une valeur $p<\alpha=0.01$. 
% \textit{Nous avons sélectionné une valeur $\alpha$ aussi restrictive car nos premières analyses réalisées avec une valeur $p<\alpha=0.05$ ont rendu beaucoup trop de liens et de clusters significatifs, rendant fortement compliquée l'interprétation des résultats.} \\

2. Sur la base de l'étape précédente, une statistique basée sur les clusters est ensuite déterminée. 
Un cluster est alors défini comme l'ensemble des paires de signaux ayant un signal de référence en commun \citep{maris2007nonparametric}. 
Chaque cluster se voit attribuer une taille (c'est-à-dire un nombre de paires pour lesquelles l'hypothèse nulle a été rejetée à l'étape 1) et un poids statistique (c'est-à-dire une différence entre les statistiques de seuil et les statistiques expérimentales observées à l'étape 1).

3. Un deuxième test de permutation est ensuite effectué sur les statistiques basées sur les clusters (taille et poids). 
Ce test permet de déterminer les distributions sous l'hypothèse nulle de l'absence de différence entre les groupes pour l'indice de taille et de poids des clusters. 
Seuls les clusters pour lesquels la statistique basée sur les clusters est supérieure à celle de la distribution de permutation (avec un risque $\alpha=0.01$) sont sélectionnés comme significatifs. 

Les tests de permutation ont été effectués à l'aide d'une procédure mise en œuvre par \cite{lenne2013decrease} dans le langage de programmation Python \citep{van2007python}. 
Le calcul des statistiques des clusters a été effectué en utilisant 100000 permutations et un seuil statistique $\alpha=0.01$. 
Les mesures électrophysiologiques ont été sélectionnées lorsqu'elles permettent de différencier significativement les cibles détectées des cibles manquées sur la base des statistiques de clusters. 

%%%%%%%%%%%%%%%%%%%%%%%%%%%%%%%%%%%%%%%%%%%%%%%%%%%%%%%%%%%%%%%%%%%%%%%%%%%%%%%
\subsection{Algorithmes de calcul}
\label{algocalculindicetransmissiondinfo}
%%%%%%%%%%%%%%%%%%%%%%%%%%%%%%%%%%%%%%%%%%%%%%%%%%%%%%%%%%%%%%%%%%%%%%%%%%%%%%%

%%%%%%%%%%%%%%%%%%%%%%%%%%%%%%%%%%%%%%%%%%%%%%%%%%%%%%%%%%%%%%%%%%%%%%%%%%%%%%%
\subsubsection*{Covariance et corrélation linéaire}
\label{algocalculindicecovcor}
%%%%%%%%%%%%%%%%%%%%%%%%%%%%%%%%%%%%%%%%%%%%%%%%%%%%%%%%%%%%%%%%%%%%%%%%%%%%%%%

La covariance caractérise les variations simultanées des deux VA : elle est positive lorsque les écarts entre les variables et leurs moyennes ont tendance à être de même signe, négative dans le cas contraire. 

Soient deux VAD $X$ et $Y$ de moyennes respectives $\mu_X$ et $\mu_Y$ et présentant le même nombre d'échantillons ($x_i$, $y_i$) pour $i=1,\ldots~,n$. 
La covariance peut s'écrire : 

\begin{equation}
Cov(X,Y) = \sigma_{X,Y}^2 = E(X - \mu_X)(Y - \mu_Y ) = \frac{1}{n} \sum_{i=1}^n(x_i - \mu_X)(y_i - \mu_Y )
\end{equation}

La variance est considérée comme un cas particulier de covariance où $X$ et $Y$ sont les mêmes variables : 

\begin{equation}
Var (X) = Cov(X,X) = \sigma_X^2 = E(X - \mu_X)(X-\mu_X) = E[(X-\mu_X)^2]
\end{equation}

Le cœfficient de corrélation de Pearson est une estimation linéaire paramétrique de la corrélation entre deux VA. 
Il représente la covariance divisée par le produit de leur écart type et peut ainsi être considéré comme une covariance normalisée, permettant de s'échelonner de $-1$ à $1$. 

Pour deux VAD $X$ et $Y$ avec un écart-type respectivement $\sigma_x$ et $\sigma_y$, le cœfficient de corrélation de Pearson $\rho_{X,Y}$ est :

\begin{equation}
\rho_{X,Y} = \frac{\sigma_{X,Y}^2}{\sigma_X \sigma_Y}
\end{equation}

%%%%%%%%%%%%%%%%%%%%%%%%%%%%%%%%%%%%%%%%%%%%%%%%%%%%%%%%%%%%%%%%%%%%%%%%%%%%%%%
\subsubsection*{Information mutuelle et entropie de transfert}
\label{algocalculindiceimte}
%%%%%%%%%%%%%%%%%%%%%%%%%%%%%%%%%%%%%%%%%%%%%%%%%%%%%%%%%%%%%%%%%%%%%%%%%%%%%%%

L'information mutuelle a été définie (voir Section \ref{infomut}) comme la quantité par laquelle une mesure de $Y$ réduit l'incertitude de $X$ \citep{shannon1948, cover2006}, c'est-à-dire qu'elle reflète la réduction de l'incertitude sur une variable aléatoire $X$ lorsque la mesure de $Y$ est connue. 
De cette façon, l'IM entre deux VA $X$ et $Y$ mesure la transmission d'information (en bits d'information si le logarithme est en base binaire) entre les variables $X$ et $Y$ \citep{shannon1948, lenne2013decrease}. 
En tant que mesure de dépendance statistique non-linéaire, l'IM augmente avec la force de la dépendance et présente une valeur maximale lorsque les deux VA sont totalement dépendantes. 
Inversement, lors de l'indépendance totale des deux variables ($p(x,y)=p(x)p(y)$), l'IM s'annule et devient égale à zéro. 
Nous avons vu que l'IM entre deux VAD peut être écrite sous plusieurs formes telles que : 

\begin{equation}
IM(X;Y) = \sum_{x \in \chi} \sum_{y \in \Upsilon} p(x,y) \log \frac{p(x,y)}{p(x)p(y)}
\end{equation}

\begin{equation}
\begin{split}
IM(X;Y) &= IM(Y;X) \\
&= H(X) - H(X|Y) \\ 
&= H(Y) - H(Y|X) \\ 
&= H(X) + H(Y) - H(X,Y) \\ 
% &= \sum_{x \in \chi} \sum_{y \in \Upsilon} p(x,y) \log \frac{p(x,y)}{p(x)p(y)}
\end{split}
\end{equation}

La principale difficulté du calcul de l'IM à partir de données expérimentales réside dans l'estimation de la distribution de probabilités jointes $p(x,y)$ à partir d'histogrammes. 
Généralement, les outils de calculs sont des variations d'algorithmes de «binning» afin de discrétiser $X$ et $Y$ pour permettre le calcul d'une IM discrète. 
La sélection du nombre $k$ de bins d'échantillonnage a une grande influence sur la précision de l'IM de par l'existence d'un compromis biais-variance \citep{hlavavckova2007causality, ince2017statistical, jeong2001mutual, sun2018mutual}. 

En dépit de diverses techniques de correction de biais \citep{paninski2003estimation, schurmann1996entropy} et du fait que l'IM par discrétisation approxime l'IM théorique, il est bien établi que cette discrétisation est une méthode sous-optimale pour la manipulation de distributions empiriques de données à valeurs réelles continues \citep{budden2016information, ross2014mutual, roulston1999estimating, seok2015mutual}. 
Néanmoins, pour estimer l'information mutuelle entre deux VA à $n$ échantillons, \cite{hlavavckova2007causality} ont établi que $\sqrt{n/5} \leq k \leq \sqrt{n}$ représente un bon compromis biais-variance pour la discrétisation. 
De ce fait, nous avons choisi d'utiliser cette procédure de discrétisation en prenant $k=40$ bins sur la base de données temporelles issues de $6$ secondes d'enregistrements échantillonnés à $500$ Hz. 

L'IM étant une mesure symétrique, toutes les matrices de mesures sont carrées et symétriques. 
Nous avons donc obtenu 1740 indices d'IM pour toutes les paires de signaux sur la base de 60 électrodes. 
Les indices d'IM ont été calculés entre les 60 paires de canaux unipolaires en utilisant la procédure suivante :\\
1. Le signal de chaque électrode a été transformé en un processus uniformément distribué utilisant $k=40$ bins afin que chaque symbole ait la même probabilité d'occurrence et que l'entropie soit maximale \citep{hlavavckova2007causality}. 
L'intervalle entre la valeur minimale et maximale de $X$ est divisé en $k=40$ sous-intervalles (indexés de $0$ à $k-1$) contenant chacun $1/40$ valeurs de $X$. 
Le signal est ainsi transformé en une séquence symbolique.\\
2. Pour chaque paire de représentation symbolique des signaux $X$ et $Y$, l'information mutuelle $IM(X;Y)$ a été calculée en utilisant une approche classique \citep{studholme1998normalized} dont l'expression prend la forme spécifique :

\begin{equation}
IM(X;Y) = H(X) + H(Y) - H(X,Y)
\end{equation}

L'entropie de transfert, présentée dans la Section \ref{transfen}, a été définie comme une extension de l'IM qui mesure le transfert d'information dirigé entre les séries temporelles d'une VA source et d'une VA cible \citep{schreiber2000,paluvs2001synchronization}. 
L'entropie de transfert permet de quantifier le degré d'influence causale parmi les interactions complexes en considérant l'histoire commune partagée entre les deux VA via un conditionnement de l'IM. 
Plus précisément, l'entropie de transfert conditionne le passé de $X_t$ pour éliminer toute information redondante ou partagée entre $X_t$ et son passé, et supprime toute information de $Y$ concernant $X$ au temps $t$ qui se trouve dans le passé de $X$ \citep{williams2011generalized, ikegwu2020pyif}. 
L'entropie de transfert de $Y$ vers $X$, notée $TE_{Y \rightarrow X}(t)$ peut être définie \citep{kraskov2004} comme :

\begin{equation}
\begin{split}
TE_{Y \rightarrow X}(t) &= IM(X_t~;~Y_{t-\tau}|X_{t-\tau}) \\
&= IM(X_t|X_{t-\tau},Y_{t-\tau}) - IM(X_t|X_{t-\tau})
\end{split}
\end{equation}

Toutes les routines ont été implémentées dans des scripts Python \citep{van2007python}.
Le module python \textit{PyIF} \citep{ikegwu2020pyif} a été utilisé pour estimer les entropies de transfert bivariés. 
Dans ce module, les entropies de transfert ont été estimées grâce à l'estimateur de Kraskov \citep{kraskov2004}. 

%%%%%%%%%%%%%%%%%%%%%%%%%%%%%%%%%%%%%%%%%%%%%%%%%%%%%%%%%%%%%%%%%%%%%%%%%%%%%%%
\subsection{Résultats des analyses sur les mesures de transfert d'information}
\label{resultatstransmissiondinfo}
%%%%%%%%%%%%%%%%%%%%%%%%%%%%%%%%%%%%%%%%%%%%%%%%%%%%%%%%%%%%%%%%%%%%%%%%%%%%%%%

Les tests par permutation de clusters nous ont permis d'obtenir des clusters et des liens qui diffèrent significativement entre les cibles détectées et les cibles non-détectées. 
Nous avons représenté les résultats statistiquement significatifs de cette analyse sous formes de graphes. 
La Figure~\ref{fig:chap5clusterpermtestresultatscovcor} représente les liens et les clusters significativement différents entre les cibles détectées et les cibles non-détectées pour les mesures de covariance et de corrélation. 
La Figure~\ref{fig:chap5clusterpermtestresultatsimte} représente les liens et les clusters significativement différents entre les cibles détectées et les cibles non-détectées pour les mesures d'information mutuelle et d'entropie de transfert. 
Pour chaque mesure, les poids statistiques sont présentés dans la Table~\ref{tab:table5statstestspermutationcluster} et nous reportons l'ensemble des clusters (ou «hubs» informationnels) de plus de deux liens significatifs. 

Au total, nous avons obtenu 28 liens significatifs pour la covariance, 40 pour la corrélation, 7 pour l'information mutuelle et 32 pour l'entropie de transfert. 
Premièrement, dans les mesures de dépendance linéaire, 11 clusters significatifs ont été observés pour la mesure de covariance (AF7, F5, F7, F8, FC4, FC6, C2, C5, FT7, FT8 et T7) tandis que 17 clusters significatifs ont été observés pour la mesure de corrélation (AF4, AFz, C5, CP5, F1, F2, F3, F5, F7, Fz, F8, FC4, FC5, 01, T7, FT7 et FT8). 
Deuxièmement, dans les mesures de dépendance non-linéaire, aucun cluster significatif n'a été observé pour l'information mutuelle alors que 9 clusters significatifs ont été observés pour l'entropie de transfert (Pz, P4, P6, P8, AF4, FC3, CPz, PO3 et 02). 

Les résultats associés à la mesure de covariance indiquent que la détection de la cible auditive provoque des variations simultanées entre les signaux issus des parties antéro-latérales principalement des hémisphères droit (F8, FT8) et gauche (F7, FT7 et T7). 
On observe également que l'électrode antéro-frontale gauche la plus latérale (AF7) présente ces variations avec plusieurs électrodes centrales et centro-pariétales. 
Cela montre donc une forte covariation entre les signaux cérébraux associés aux lobes fronto-temporal droit et gauche, et par conséquent une covariation de l'activité cérébrale des cortex auditifs droit et gauche. 
Les résultats associés à la mesure de corrélation permettent de retrouver un pattern assez similaire sur les électrodes les plus latérales (coté droit : F8 et FT8 ; coté gauche : F7, FT7 et T7). 
On observe également deux clusters significatifs à l'arrière du cerveau: un cluster au niveau occipital (O1) et un autre au niveau pariéto-occipital (O4).
Ainsi, une covariation linéaire importante est observable entre les signaux issus des électrodes localisées au sein des lobes temporaux droit et gauche. 

Les résultats associés aux mesures d'information mutuelle ne laissent apparaître que très peu de liens significatifs et aucun cluster significatif. 
Cependant, ceux associés aux mesures d'entropie de transfert montrent que les valeurs sont significativement plus élevées lorsque les cibles auditives étaient détectées que lorsqu'elles ne l'étaient pas. 
De plus, nous pouvons observer un cluster significatif de liens informationnels associé à l'aire pariétale droite. 
On observe notamment que les électrodes pariétales P6 et P8 sont des centres vers lesquels se dirigent l'information et représentent ainsi des clusters entrants. 
Au contraire, l'information dirigée vers ces clusters apparaît provenir d'un ensemble d'aires cérébrales différentes, qui représentent des clusters sortants. 
On voit également que l'électrode AF4 est un centre antérieur vers lequel se dirige de l'information. 

\begin{landscape}
\begin{figure}[!t]
\includegraphics[width=0.49\linewidth, height=0.45\textheight]{/home/link/Documents/thèse_onera/experimentation/eeg/manipeeg/results/EEG/cluster_perm_stats/Cov_topograph_sig_links_500Hz_100000boot_.png}
\includegraphics[width=0.49\linewidth, height=0.45\textheight]{/home/link/Documents/thèse_onera/experimentation/eeg/manipeeg/results/EEG/cluster_perm_stats/Cor_topograph_sig_links_500Hz_100000boot_.png}
\caption[Topographie des clusters et des liens significatifs pour Cov et Cor]{Schémas topographiques indiquant à la fois les clusters et liens dont les critères de taille et de poids sont remplis en utilisant la procédure de test par permutation de clusters non-paramétrique pour les mesures de covariance (Gauche) et de corrélation (Droite) pour 100000 permutations et un seuil statistique de $0.01$.}
\label{fig:chap5clusterpermtestresultatscovcor}
\end{figure}
\end{landscape}

\begin{landscape}
\begin{figure}[!t]
\includegraphics[width=0.47\linewidth, height=0.43\textheight]{/home/link/Documents/thèse_onera/experimentation/eeg/manipeeg/results/EEG/cluster_perm_stats/IM_topograph_sig_links_500Hz_100000boot_.png}
\includegraphics[width=0.49\linewidth, height=0.45\textheight]{/home/link/Documents/thèse_onera/experimentation/eeg/manipeeg/results/EEG/cluster_perm_stats/TE_topograph_sig_links_500Hz_100000boot_.png}
\caption[Topographie des clusters et des liens significatifs pour IM et TE]{Schémas topographiques indiquant à la fois les clusters et liens dont les critères de taille et de poids sont remplis en utilisant la procédure de test par permutation de clusters non-paramétrique pour les mesures d'information mutuelle (Gauche) et d'entropie de transfert (Droite) pour 100000 permutations et un seuil statistique de $0.01$.}
\label{fig:chap5clusterpermtestresultatsimte}
\end{figure}
\end{landscape}

\begin{table}[!t]
\scriptsize
\caption[Table des résultats des analyses statistiques des tests non-paramétriques par permutation de cluster]{Table reportant les poids statistiques des liens entre les électrodes sources et les électrodes cibles pour lesquelles à la fois la taille et le poids sont significativement différents entre les cibles détectées et non-détectées pour les mesures de corrélation linéaire (Haut Gauche), de covariance (Haut Droite), d'information mutuelle (Bas Gauche) et d'entropie de transfert (Bas Droite).}
\label{tab:table5statstestspermutationcluster}

\begin{multicols}{2}
% \begin{tabular}{llllll}
\begin{tabular}{|l|*{6}{c|}}
\hline
\textcolor{blue}{\textbf{Cov}} & & & & & \\
\hline
\textbf{Source} & \textbf{Target} & \textbf{Weights} & \textbf{Source} & \textbf{Target} & \textbf{Weights} \\
\hline
AF3 & C2 & $2,5.10^{-13}$ & C5 & FT8 & $1,4.10^{-12}$ \\
AF7 & C2 & $1,0.10^{-12}$ & F5 & FC6 & $4,7.10^{-13}$ \\
AF7 & C3 & $8,2.10^{-13}$ & F5 & FT8 & $6,3.10^{-13}$ \\
AF7 & C6 & $1,6.10^{-12}$ & F7 & F8 & $5,9.10^{-13}$ \\
AF7 & CP1 & $7,7.10^{-13}$ & F7 & FC6 & $1,8.10^{-12}$ \\
AF7 & CP5 & $9,0.10^{-14}$ & F7 & FT8 & $1,1.10^{-12}$ \\
AF7 & Cz & $2,6.10^{-12}$ & F8 & FT7 & $2,1.10^{-12}$ \\
AF7 & FC1 & $1,9.10^{-14}$ & F8 & T7 & $2,3.10^{-13}$ \\
AF7 & FC2 & $1,3.10^{-12}$ & FC3 & FT8 & $1,1.10^{-13}$ \\
AF7 & FC4 & $1,2.10^{-13}$ & FC4 & FT7 & $1,4.10^{-12}$ \\
AF7 & FCz & $2,1.10^{-12}$ & FC6 & FT7 & $9,5.10^{-14}$ \\
C5 & F8 & $6,3.10^{-13}$ & FC6 & T7 & $1,1.10^{-12}$ \\
C5 & FC4 & $9,1.10^{-13}$ & FT7 & FT8 & $2,2.10^{-12}$ \\
C5 & FC6 & $2,3.10^{-12}$ & FT8 & T7 & $1,4.10^{-13}$ \\
\hline
\end{tabular}

% \begin{tabular}{llllll}
\begin{tabular}{|l|*{6}{c|}}
\hline
\textcolor{blue}{\textbf{Cor}} & & & & & \\
\hline
\textbf{Source} & \textbf{Target} & \textbf{Weights} & \textbf{Source} & \textbf{Target} & \textbf{Weights} \\
\hline
AF4 & O1 & 0.0009 & F2 & O1 & 0.005 \\
AF4 & O2 & 0.006 & F2 & P5 & 0.002 \\
AF4 & PO3 & 0.002 & F3 & FC2 & 0.006 \\
AF4 & PO4 & 0.004 & F3 & FT8 & 0.007 \\
AF4 & PO8 & 0.001 & F3 & Fz & 0.009 \\
AF4 & POz & 0.005 & F5 & FC4 & 0.004 \\
AFz & FC5 & 0.0005 & F5 & FT8 & 0.005 \\
AFz & O1 & 0.0004 & F6 & T7 & 0.007 \\
AFz & T7 & $<.0001$ & F7 & F8 & 0.0003 \\
C1 & FT8 & 0.006 & F7 & FC4 & 0.005 \\
C3 & FT8 & 0.005 & F7 & FT8 & 0.001 \\
C5 & FC4 & 0.007 & F8 & FC5 & 0.002 \\
C5 & FC6 & 0.005 & F8 & FT7 & 0.001 \\
C5 & FT8 & 0.008 & F8 & T7 & 0.002 \\
CP5 & F8 & 0.009 & FC1 & FT8 & 0.0001 \\
CP5 & FC4 & 0.003 & FC3 & FT8 & $<.0001$ \\
CP5 & FT8 & 0.002 & FC4 & FT7 & 0.0007 \\
F1 & FT8 & 0.002 & FC4 & O1 & 0.001 \\
F1 & Fz & 0.002 & FC4 & P1 & 0.002 \\
F2 & F5 & 0.003 & F2 & F7 & 0.006 \\
\hline
\end{tabular}
\end{multicols}

\begin{multicols}{2}
\scriptsize
% \begin{tabular}{lll}
\begin{tabular}{|l|*{3}{c|}}
\hline
\textcolor{blue}{\textbf{IM}} & & \\
\hline
\textbf{Source} & \textbf{Target} & \textbf{Weights} \\
\hline
CP6 & CPz & 0.0007 \\
F1 & FC1 & 0.0005 \\
FC1 & Fz & 0.0006 \\
FT7 & P2 & 0.0005 \\
O2 & P8 & 0.002 \\
P2 & T7 & 0.0001 \\
P8 & PO4 & 0.003 \\
\hline
\end{tabular}

% \begin{tabular}{llllll}
\begin{tabular}{|l|*{6}{c|}}
\hline
\textcolor{blue}{\textbf{TE}} & & & & & \\
\hline
\textbf{Source} & \textbf{Target} & \textbf{Weights} & \textbf{Source} & \textbf{Target} & \textbf{Weights} \\
\hline
AF8 & P4 & 0.0002 & FT8 & P6 & 0.0001 \\
C2 & P4 & 0.001 & Fp1 & P8 & 0.001 \\
C5 & P6 & 0.0008 & O2 & Pz & 0.001 \\
C6 & AF4 & 0.0001 & Oz & P8 & 0.001 \\
CP2 & P8 & 0.003 & P1 & P6 & 0.0008 \\
CP6 & P6 & 0.004 & P4 & P8 & 0.001 \\
CPz & CP1 & 0.0006 & P5 & P6 & 0.0005 \\
CPz & P8 & 0.0003 & P8 & P4 & 0.001 \\
Cz & P8 & 0.002 & P8 & P6 & 0.0004 \\
F2 & AF4 & 0.0006 & PO3 & P6 & 0.0003 \\
F3 & Pz & 0.0004 & PO3 & P8 & 0.001 \\
F5 & O2 & 0.0008 & PO4 & O2 & 0.0008 \\
F8 & AF4 & 0.0007 & POz & Fp2 & 0.0008 \\
FC3 & AF4 & 0.001 & Pz & TP7 & 0.0004 \\
FC3 & P6 & 0.0007 & T7 & P8 & 0.0008 \\
FC4 & FT7 & $<.0001$ & TP7 & F1 & 0.0003 \\
\hline
\end{tabular}
\end{multicols}
\end{table}

%%%%%%%%%%%%%%%%%%%%%%%%%%%%%%%%%%%%%%%%%%%%%%%%%%%%%%%%%%%%%%%%%%%%%%%%%%%%%%%
\subsection{Synthèse et discussion pour les mesures de transfert d'information}
\label{synthesediscussiontransmissiondinfo}
%%%%%%%%%%%%%%%%%%%%%%%%%%%%%%%%%%%%%%%%%%%%%%%%%%%%%%%%%%%%%%%%%%%%%%%%%%%%%%%

Dans cette section, nous avons étudié l'effet de la perception auditive consciente sur le transfert d'information entre les différentes zones cérébrales dans le MI. 
Nous avons cherché à savoir si les aires frontales, temporales et pariétales étaient susceptibles de présenter une transmission de l'information plus importante dans le cas où la cible auditive serait perçue. 
Plus précisément, nous avons regardé si un échange d'information serait observé en provenance des aires cérébrales antérieures vers les aires pariétales. 
Pour cela, nous avons utilisé deux mesures classiques linéaires, la corrélation et la covariance, ainsi que deux mesures non-linéaires, l'information mutuelle et l'entropie de transfert. 

Les mesures de dépendance linéaire (covariance et corrélation) nous ont permis de montrer des relations linéaires entre les électrodes latérales des hémisphères cérébraux. 
Ces électrodes étaient principalement localisées au niveau des aires frontales, fronto-temporales et temporales. 
Cela semble suggérer que les signaux issus des électrodes situées au niveau des cortex frontal et temporal des deux hémisphères covarient linéairement lorsque les sujets ont détecté les cibles auditives et non lorsque les cibles n'étaient pas détectées.
Nous avons également trouvé que l'activité de l'aire antéro-frontale gauche covariait de manière significative avec des zones centrales et centro-pariétales lors de la perception de la cible auditive. 

Cependant, les mesures de dépendance non-linéaire, information mutuelle et entropie de transfert ont montré des résultats nettement différents.
La mesure d'entropie de transfert, quantifiant le transfert d'information dirigé, présentait des valeurs significativement supérieures lors de la perception de la cible pour plusieurs clusters cérébraux. 
Notamment, plusieurs électrodes pariétales de l'hémisphère droit sont apparues être de véritables cibles du transfert d'information (P6 et P8). 
L'ensemble de liens informationnels significatifs associé à l'aire pariétale droite et les électrodes le composant correspondaient à des centres cérébraux vers lesquels se dirigaient l’information. 
L'information arrivant au niveau de ce cluster pariétal provenait de plusieurs zones cérébrales comme les aires temporale, fronto-centrale, centrale ou encore pariétale gauches. 
De plus, l'électrode localisée au niveau du cortex antéro-frontal droit AF4 semblait également correspondre à un centre cérébral antérieur vers lequel était dirigé de l'information en provenance de différentes électrodes des deux hémisphères. 

De cette manière, cette analyse nous a permis d'observer un échange important d'information selon un axe antéro-postérieur basé sur des centres attracteurs dépendants de leur localisation dans les hémisphères cérébraux. 
Ces résultats suggèrent un transfert de l'information inter-hémisphérique augmenté lorsque les cibles ont été détectées, en provenance de plusieurs zones de l'hémisphère gauche et en direction de l'aire pariétale droite. 
Ces résultats permettent de renforcer l'hypothèse que l'activité au sein d'un réseau fronto-temporo-pariétal est associée à la perception consciente de stimuli. 

Dans le cadre de la ségrégation des flux auditifs et de l'organisation perceptive cohérente des objets auditifs, les mécanismes de transfert de l'information neuronale apparaîssent largement dépendants de l'activation du cortex pariétal et de sa capacité à accumuler l'information entrante \citep{overath2007information, pereira2021evidence}. 
L'augmentation de l'information vers ce centre fonctionnel pariétal pourrait signifier un rôle d'attracteur informationnel visant à donner un sens à la représentation perceptive des flux auditifs et mener à une ségrégation aboutie des flux dissociés de la cible et du masqueur. 
De plus, la localisation des électrodes pariétales postérieures P6 et P8 correspond notamment au sillon temporal supérieur situé en arrière du lobe temporal au niveau de la jonction temporo-pariétale. 
L'activité au sein des sillons intra-pariétaux inféro-postérieur du lobe pariétal et de la jonction temporo-pariétale a été montrée être dépendante des demandes d'intégration dans le MI \citep{eriksson2017activity}. 
Les résultats de notre analyse pourraient venir appuyer le rôle des aires pariétales droites dans leur capacité à associer les représentations sensorielles et perceptives liées à l'intégration des caractéristiques des stimuli chez l'humain. 
% Selon des recherches antérieures, ces deux régions sont fortement impliquées dans les processus attentionnels \citep{buschman2015behavior, cabeza2012cognitive}. 

Puisque la perception bistable correspond à la fluctuation dans le temps de la perception du sujet lorsque les entrées sensorielles permettent des interprétations perceptives multiples et concurrentes (comme dans le MI), la perception du sujet peut osciller entre deux percepts mutuellement exclusifs malgré la constance de l'entrée sensorielle. 
Les études d'imagerie avaient révélé des réponses transitoires au sein du cortex fronto-pariétal latéral droit au moment des transitions perceptives entre les interprétations, reflétant potentiellement l'initiation neuronale des transitions \citep{kleinschmidt2002human, lumer1998neural, sterzer2007neural}. 
En outre, une activation plus importante du cortex fronto-pariétal droit a également été montrée lors des transitions qui durent plus longtemps, suggérant que cette zone du cortex cérébral resterait actif aussi longtemps que dure une transition \citep{knapen2011role}. 
De cette manière, l'activation du cortex fronto-pariétal droit pendant la perception bistable pourrait refléter une réponse aux transitions perceptives, réponse associée à la construction d'un percept conscient issue d'une analyse de la scène auditive préalable ayant aboutie à une ségrégation réussie des flux auditifs entre la cible et le masqueur à travers le temps.  
Dans notre analyse, les résultats associant la zone cérébrale antéro-frontale droite à un rôle de centre attracteur vers lequel se dirige de l'information semblent soutenir cette idée que le cortex fronto-pariétal montre une activation en réponse aux fluctuations des transitions perceptives et des représentations des flux auditifs. 

%%%%%%%%%%%%%%%%%%%%%%%%%%%%%%%%%%%%%%%%%%%%%%%%%%%%%%%%%%%%%%%%%%%%%%%%%%%%%%%%%%%%%%%%%%%%%%%%%%%%%%%%%%%%%%%%%%%%%%%%%%%%%%%%%%%%%%%%%%%%%%%%%%%%%%%%%%%%%%%%%%%%%%%%%%%%%%%%%%%%%%%%%%%%%%%%%%%%%%%%%%%%%%%%%%%%%%%%%%%%%%%%%%%%%%%%%%%%%%%
%%%%%%%%%%%%%%%%%%%%%%%%%%%%%%%%%%%%%%%%%%%%%%%%%%%%%%%%%%%%%%%%%%%%%%%%%%%%%%%%%%%%%%%%%%%%%%%%%%%%%%%%%%%%%%%%%%%%%%%%%%%%%%%%%%%%%%%%%%%%%%%%%%%%%%%%%%%%%%%%%%%%%%%%%%%%%%%%%%%%%%%%%%%%%%%%%%%%%%%%%%%%%%%%%%%%%%%%%%%%%%%%%%%%%%%%%%%%%%%
%%%%%%%%%%%%%%%%%%%%%%%%%%%%%%%%%%%%%%%%%%%%%%%%%%%%%%%%%%%%%%%%%%%%%%%%%%%%%%%%%%%%%%%%%%%%%%%%%%%%%%%%%%%%%%%%%%%%%%%%%%%%%%%%%%%%%%%%%%%%%%%%%%%%%%%%%%%%%%%%%%%%%%%%%%%%%%%%%%%%%%%%%%%%%%%%%%%%%%%%%%%%%%%%%%%%%%%%%%%%%%%%%%%%%%%%%%%%%%%
%%%%%%%%%%%%%%%%%%%%%%%%%%%%%%%%%%%%%%%%%%%%%%%%%%%%%%%%%%%%%%%%%%%%%%%%%%%%%%%%%%%%%%%%%%%%%%%%%%%%%%%%%%%%%%%%%%%%%%%%%%%%%%%%%%%%%%%%%%%%%%%%%%%%%%%%%%%%%%%%%%%%%%%%%%%%%%%%%%%%%%%%%%%%%%%%%%%%%%%%%%%%%%%%%%%%%%%%%%%%%%%%%%%%%%%%%%%%%%%
%%%%%%%%%%%%%%%%%%%%%%%%%%%%%%%%%%%%%%%%%%%%%%%%%%%%%%%%%%%%%%%%%%%%%%%%%%%%%%%%%%%%%%%%%%%%%%%%%%%%%%%%%%%%%%%%%%%%%%%%%%%%%%%%%%%%%%%%%%%%%%%%%%%%%%%%%%%%%%%%%%%%%%%%%%%%%%%%%%%%%%%%%%%%%%%%%%%%%%%%%%%%%%%%%%%%%%%%%%%%%%%%%%%%%%%%%%%%%%%
%%%%%%%%%%%%%%%%%%%%%%%%%%%%%%%%%%%%%%%%%%%%%%%%%%%%%%%%%%%%%%%%%%%%%%%%%%%%%%%%%%%%%%%%%%%%%%%%%%%%%%%%%%%%%%%%%%%%%%%%%%%%%%%%%%%%%%%%%%%%%%%%%%%%%%%%%%%%%%%%%%%%%%%%%%%%%%%%%%%%%%%%%%%%%%%%%%%%%%%%%%%%%%%%%%%%%%%%%%%%%%%%%%%%%%%%%%%%%%%
%%%%%%%%%%%%%%%%%%%%%%%%%%%%%%%%%%%%%%%%%%%%%%%%%%%%%%%%%%%%%%%%%%%%%%%%%%%%%%%%%%%%%%%%%%%%%%%%%%%%%%%%%%%%%%%%%%%%%%%%%%%%%%%%%%%%%%%%%%%%%%%%%%%%%%%%%%%%%%%%%%%%%%%%%%%%%%%%%%%%%%%%%%%%%%%%%%%%%%%%%%%%%%%%%%%%%%%%%%%%%%%%%%%%%%%%%%%%%%%
%%%%%%%%%%%%%%%%%%%%%%%%%%%%%%%%%%%%%%%%%%%%%%%%%%%%%%%%%%%%%%%%%%%%%%%%%%%%%%%%%%%%%%%%%%%%%%%%%%%%%%%%%%%%%%%%%%%%%%%%%%%%%%%%%%%%%%%%%%%%%%%%%%%%%%%%%%%%%%%%%%%%%%%%%%%%%%%%%%%%%%%%%%%%%%%%%%%%%%%%%%%%%%%%%%%%%%%%%%%%%%%%%%%%%%%%%%%%%%%
%%%%%%%%%%%%%%%%%%%%%%%%%%%%%%%%%%%%%%%%%%%%%%%%%%%%%%%%%%%%%%%%%%%%%%%%%%%%%%%%%%%%%%%%%%%%%%%%%%%%%%%%%%%%%%%%%%%%%%%%%%%%%%%%%%%%%%%%%%%%%%%%%%%%%%%%%%%%%%%%%%%%%%%%%%%%%%%%%%%%%%%%%%%%%%%%%%%%%%%%%%%%%%%%%%%%%%%%%%%%%%%%%%%%%%%%%%%%%%%
%%%%%%%%%%%%%%%%%%%%%%%%%%%%%%%%%%%%%%%%%%%%%%%%%%%%%%%%%%%%%%%%%%%%%%%%%%%%%%%%%%%%%%%%%%%%%%%%%%%%%%%%%%%%%%%%%%%%%%%%%%%%%%%%%%%%%%%%%%%%%%%%%%%%%%%%%%%%%%%%%%%%%%%%%%%%%%%%%%%%%%%%%%%%%%%%%%%%%%%%%%%%%%%%%%%%%%%%%%%%%%%%%%%%%%%%%%%%%%%

%%%%%%%%%%%%%%%%%%%%%%%%%%%%%%%%%%%%%%%%%%%%%%%%%%%%%%%%%%%%%%%%%%%%%%%%%%%%%%%
\clearpage\newpage
\section{Intégration de l'information dans le MI}
\label{integrationinformationmasquageinformationnel}
%%%%%%%%%%%%%%%%%%%%%%%%%%%%%%%%%%%%%%%%%%%%%%%%%%%%%%%%%%%%%%%%%%%%%%%%%%%%%%%

Dans cette quatrième et dernière partie, nous avons étudié des mesures issues de la théorie de l'information intégrée (TII) pour caractériser la dynamique cérébrale associée à la perception consciente de la cible auditive. 
Parmi les méthodes capables de caractériser la dynamique cérébrale lors de la construction d'un percept auditif conscient, nous venons d'étudier celles usuellement employées dans la littérature pour caractériser les états de conscience. 
Les méthodes utilisées dans cette partie découlent d'un ensemble théorique développé pour aborder le problème des contenus de conscience et caractériser la dynamique de l'activité cérébrale lors de la conscience perceptive. 

Parmi ces méthodes, présentées dans le Chapitre \ref{chapitre2}, la TII propose des mesures permettant de quantifier l'intégration de l'information à des échelles cérébrales différentes. 
Parmi ces mesures notamment, la mesure $\Phi$ de la quantité d'information intégrée dans un système physique tient un rôle particulier. 
D'une part, la TII postule que $\Phi$ présente une relation d'identité avec la conscience et la décrit comme une information intégrée entre des éléments hautement différenciés mais irréductibles du système cérébral. 
D'autre part, elle prédit que la valeur de $\Phi$ estimée à partir des activités cérébrales représente le niveau de conscience à travers la diversité du répertoire d'états fonctionnels associés au système cérébral. 
Ainsi, plus un système est conscient, plus il intègre de l'information, plus il génère de l'information intégrée et par conséquent, plus son $\Phi$ augmente. 

Nous avons brièvement présenté plusieurs des mesures proposées dans la littérature et avons soulevé des problématiques liées à leur utilisation. 
Actuellement, les mesures de la TII soulèvent d'importantes questions sur la possibilité de les appliquer aux données réelles issues d'enregistrements neurophysiologiques.
Leur comportement, à l'exception des cas les plus simples, n'a pas été caractérisé de manière approfondie et malgré le nombre croissant d'études comparatives s'y intéressant \citep{barrett2011practical, barrett2019phi, haun2016contents, haun2017conscious, isler2018integrated, kim2018estimating, kim2019criticality, kitazono2018efficient, mediano2019measuring, oizumi2016measuring, seth2011causal, tegmark2016improved, toker2019information}, il y a un réel manque d'études visant spécifiquement à tester et mettre à l'épreuve cette théorie sur des données neuronales. 
À notre connaissance, aucune étude ne s'est intéressée à étudier les mécanismes d'intégration de l'information de la perception auditive consciente dans une situation de MI en s'appuyant sur la TII de la conscience. 
Nous avons donc mis à l'épreuve différentes mesures d'information intégrée et étudié leur décours temporel au cours de la construction de la perception auditive consciente dans le MI. 

Nous avons vu que les aires cérébrales temporales présentent une activité qui est associée à des traitements de l'information liée au stimulus sonore \citep{bidet2007mecanismes, micheyl2007role}. 
Les aires primaires du cortex auditif ont un rôle essentiel dans le décodage des informations sensorielles \citep{bidet2007mecanismes}. 
Les aires secondaires sont plutôt impliquées dans l'intégration des informations primaires et peuvent initier des processus d'identification et de stockage des informations, ou de focalisation attentionnelle sur une information pertinente \citep{hasselmann2017codage, lorenzi2016audition}. 
Les aires cérébrales plus centrales correspondent plutôt à des activités associées à l'intégration des informations en provenance de sources sensorielles multi-modales \citep{dehaene2006conscious, del2007brain}. 

Les travaux de \cite{giani2015detecting} ont montré, en utilisant une analyse par modélisation causale dynamique, que la composante ARN suscitée par la détection de la cible pouvait provenir de processus récursifs dans le cortex auditif. 
Cette analyse a permis d'indiquer que la composante ARN sous-tendant la ségrégation des flux auditifs était principalement due à des changements dans la connectivité intrinsèque des cortex auditifs. 
De plus, cette même analyse a permis de montrer que la composante P300 amplifiée de manière significative lorsque la cible a été détectée, reposait sur un traitement récurrent entre les cortex auditif et pariétal. 
Ainsi, dans le cadre de l'étude de la caractérisation de la dynamique temporelle de la perception auditive consciente, le décours temporel de la quantité d'information intégrée associée à ces zones cérébrales temporales, centrales et pariétales nous a semblé ici pertinent. 

Dans ce contexte, notre objectif principal était de savoir si des mesures d’information intégrée étaient susceptibles de caractériser la perception auditive consciente d’un flux de tonalités cible sous MI. 
Nous avons souhaité déterminer l'intégration d'information en utilisant des algorithmes issus de la littérature récente afin de calculer les mesures d'information intégrée sur les données EEG issues de la tâche de MI. 
Pour cela, nous avons étudié le décours temporel de l’intégration d’information pour les cibles perçues et non-perçues, avant et après la perception. 
À partir des prédictions de la TII, nous avons supposé que la perception auditive consciente d'un flux de tonalités cible intégré dans un masqueur multi-tonalités susciterait un niveau plus élevé d'information intégrée. 
Nous avons également supposé qu'une hausse progressive de la quantité d'information intégrée dans le temps serait observée pour les cibles détectées jusqu'à atteindre le report explicite conscient par le sujet lors de l'appui-bouton.
Ainsi, quelque soit la zone cérébrale considérée, nous avons fait l'hypothèse que les cibles perçues susciteraient davantage d'information intégrée que les cibles non-perçues et qu'elles exprimeraient une augmentation progressive de l'information intégrée jusqu'au report perceptif du sujet. 

%%%%%%%%%%%%%%%%%%%%%%%%%%%%%%%%%%%%%%%%%%%%%%%%%%%%%%%%%%%%%%%%%%%%%%%%%%%%%%%
\subsection{Mesures d'information intégrée}
\label{mesuresinfointegreepratique}
%%%%%%%%%%%%%%%%%%%%%%%%%%%%%%%%%%%%%%%%%%%%%%%%%%%%%%%%%%%%%%%%%%%%%%%%%%%%%%%

Les quatre mesures d'information intégrée utilisées ici pour étudier la dynamique de la perception auditive consciente sont celles présentées dans la Section \ref{integrationinformationmesures}. 
Elles sont : i) l'information intégrée basée sur le décodage $\Phi^*$, ii) l'information intégrée géométrique $\Phi^G$, iii) l'information intégrée stochastique $\Phi^H$ et iv) l'information intégrée par redondance ou «multi-information» $\Phi^{MI}$. 
Les algorithmes permettant le calcul de ces quatre mesures ont été rendus disponibles dans la toolbox Matlab «phitoolbox» \citep{oizumi2016measuring, toker2019information}. 
Nous présentons dans ce qui suit une description rapide permettant de comprendre l'implémentation pratique des algorithmes trouvés dans cette toolbox. 

Les mesures d'information intégrée visent à caractériser, comme la complexité neuronale présentée dans la Section \ref{integrationinformationmesures}, et dont elles sont issues, la différence d'information entre le système effectif et ses interactions et un système totalement indépendant. 
Cette différence vise à totalement quantifier l'intégration d'information dans un système organisé tel que le cerveau. 
Chaque mesure aborde le problème selon un point de vue théorique particulier lui donnant sa spécificité. 

%%%%%%%%%%%%%%%%%%%%%%%%%%%%%%%%%%%%%%%%%%%%%%%%%%%%%%%%%%%%%%%%%%%%%%%%%%%%%%%
\subsubsection*{Information intégrée basée sur le décodage}
\label{phistar}
%%%%%%%%%%%%%%%%%%%%%%%%%%%%%%%%%%%%%%%%%%%%%%%%%%%%%%%%%%%%%%%%%%%%%%%%%%%%%%%

L'information intégrée basée sur le décodage $\Phi^*$ aborde le problème en terme d'émetteur et récepteur en considérant leur situation : la première correspond à celle où le récepteur décode l'information du système à partir de la distribution de probabilité optimale correspondant à celle observée et la compare à la situation où le récepteur décode le message sur la base d'une distribution de probabilité issue d'un système dont les parties sont totalement indépendantes. 

Ainsi : 

\begin{equation} 
\Phi^* = IM(X_t;X_{t-\tau}) - IM^*(X_t;X_{t-\tau})
\end{equation}
avec $\tau$ le décalage temporel souhaité pour caractériser l'intégration d'information, $IM(X_t;X_{t-\tau})$ l'information mutuelle entre les états présents et les états passés décalés selon $\tau$ et $IM^*$ l'information de décodage issue du système aux parties indépendantes. 

Le formalisme complet pour le calcul de $\Phi^*$ est donné dans \cite{oizumi2016measuring}.

%%%%%%%%%%%%%%%%%%%%%%%%%%%%%%%%%%%%%%%%%%%%%%%%%%%%%%%%%%%%%%%%%%%%%%%%%%%%%%%
\subsubsection*{Information intégrée géométrique}
\label{phig}
%%%%%%%%%%%%%%%%%%%%%%%%%%%%%%%%%%%%%%%%%%%%%%%%%%%%%%%%%%%%%%%%%%%%%%%%%%%%%%%

L'information intégrée géométrique $\Phi^G$ utilise la géométrie de l'information qui est une application de la géométrie différentielle aux relations et structures de distributions de probabilité. 
Dans ce formalisme, la divergence de Kullback-Leibler est une mesure naturelle de la distance entre deux distributions de probabilité. 
L'information intégrée géométrique $\Phi^G$ est basée sur ce fait et vise à mesurer la distance entre la distribution de probabilité du système par rapport à celle d'un système totalement déconnecté. 

Ainsi : 

\begin{equation}
\Phi^G = \min \limits_q ~ D_{KL} \left[ p(X_t, X_{t-\tau}) ~ || ~ q(X_t, X_{t-\tau}) \right]
\end{equation}
où $D_{KL}[p||q]$ représente la divergence de Kullback-Leibler entre les deux distributions de probabilités jointes des états présents et passés du système entre les modèles connecté $p$ et déconnecté $q$.

Comme dans le cas de $\Phi^*$, les informations intégrées géométriques $\Phi^G$ ont une formulation analytique qui apparaît plus simple et plus rapide à calculer dans le cadre de l'hypothèse gaussienne \citep{oizumi2016unified}. 
Dans le cas gaussien, le recours à la géométrie de l'information n'est pas nécessaire pour minimiser $D_{KL}$ ; il n'y a donc pas de sens direct dans lequel cette mesure est «géométrique» pour des variables gaussiennes. 
Cela dit, parce que le cadre de la géométrie de l'information est nécessaire pour le calcul de cette mesure dans le cas non-gaussien, cette mesure est appelée «information intégrée géométrique» également dans le cas gaussien. 
L'information intégrée géométrique $\Phi^G$  peut être calculée avec des données gaussiennes en utilisant le cadre classique de la régression linéaire, comme de nombreuses mesures théoriques de l'information. 

Le formalisme complet pour le calcul de $\Phi^G$ est donné dans \cite{oizumi2016unified}.

%%%%%%%%%%%%%%%%%%%%%%%%%%%%%%%%%%%%%%%%%%%%%%%%%%%%%%%%%%%%%%%%%%%%%%%%%%%%%%%
\subsubsection*{Information intégrée stochastique}
\label{phih}
%%%%%%%%%%%%%%%%%%%%%%%%%%%%%%%%%%%%%%%%%%%%%%%%%%%%%%%%%%%%%%%%%%%%%%%%%%%%%%%

L'information intégrée stochastique $\Phi^H$ aborde le problème à partir des probabilités de transition entre un état et un autre dans le cas du système intégré par rapport à un système dont les état évoluent de façon totalement indépendante. 

Ainsi : 

\begin{equation}
\Phi^H = \sum_i H(M_{t-\tau}^i | M_t^i) - H(X_{t-\tau}|X_t)
\end{equation}
où $H(X_{t-\tau}|X_t)$ mesure la quantité d'information «perdue» de manière irréversible et $M^i$ sont les états du système. 

Le formalisme complet pour le calcul de $\Phi^H$ est donné dans \cite{kitazono2018efficient} et \cite{mediano2019measuring}. 

%%%%%%%%%%%%%%%%%%%%%%%%%%%%%%%%%%%%%%%%%%%%%%%%%%%%%%%%%%%%%%%%%%%%%%%%%%%%%%%
\subsubsection*{Information intégrée par redondance ou multi-information}
\label{phimi}
%%%%%%%%%%%%%%%%%%%%%%%%%%%%%%%%%%%%%%%%%%%%%%%%%%%%%%%%%%%%%%%%%%%%%%%%%%%%%%%

L'information intégrée par redondance $\Phi^{MI}$, également appelée «multi-information mutuelle» \citep{amari2001information, ay2015information} représente une généralisation multivariée de l'information mutuelle de Shannon \cite{mcgill1954multivariate, shannon1948}. 

Elle quantifie l'intégration d'information en abordant le problème par mesure de la quantité d'information partagée par les variables du système connecté par rapport à celle partagée lorsque toutes les interactions sont supprimées, c'est-à-dire lorsque les $M$ parties du système sont considérées comme indépendantes \citep{kitazono2018efficient} :

\begin{equation}
\Phi^{MI} = \sum_i~H(M_{t-\tau}^i,M_t^i)-H(X_{t-\tau},X_t)
\end{equation}
avec $H(X_{t-\tau},X_t)$ l'entropie conjointe. 

Le formalisme complet pour le calcul de $\Phi^{MI}$ est donné dans \cite{kitazono2018efficient}. 

%%%%%%%%%%%%%%%%%%%%%%%%%%%%%%%%%%%%%%%%%%%%%%%%%%%%%%%%%%%%%%%%%%%%%%%%%%%%%%%
\subsection{Procédure}
\label{procedureiit}
%%%%%%%%%%%%%%%%%%%%%%%%%%%%%%%%%%%%%%%%%%%%%%%%%%%%%%%%%%%%%%%%%%%%%%%%%%%%%%%

%%%%%%%%%%%%%%%%%%%%%%%%%%%%%%%%%%%%%%%%%%%%%%%%%%%%%%%%%%%%%%%%%%%%%%%%%%%%%%%
\subsubsection{Prétraitement et traitement}
\label{pretraitementiit}
%%%%%%%%%%%%%%%%%%%%%%%%%%%%%%%%%%%%%%%%%%%%%%%%%%%%%%%%%%%%%%%%%%%%%%%%%%%%%%%

Les mêmes données EEG issues de la tâche de MI (Section \ref{etude2materielmethode}) ont été utilisées dans le cadre de cette analyse. 
Les données EEG ont été reréférencées à la moyenne des électrodes, puis le signal EEG a été sous-échantillonné à $125$~Hz et des filtres non-causaux passe-bas ($80$~Hz) et passe-haut ($1$~Hz) ont été appliqués aux données. 
Les procédures de prétraitements ont été appliquées de la même manière que décrit précédemment (\textit{i.e.}, rejet des artefacts, ICA, autoreject, inspection visuelle). 
Pour permettre l'étude de la dynamique autour de la détection par une approche comparative, nous avons utilisé les temps de détection des sujets pour les détections correctes et la moyenne des temps de détection des sujets pour les 
détections manquées (\textit{i.e.}, $3.4$~s).

Le signal prétraité a ensuite été segmenté en fenêtres de $3$ secondes avant la détection $[-3s:0s]$ jusqu'à $3$ secondes après $[0s:+3s]$ et centré en utilisant la fenêtre temporelle totale. 
Ces fenêtres ont été respectivement étiquetées «Epoch Before» et «Epoch After» et il en résulte les quatre catégories de fenêtres : «Epoch Before Hit», «Epoch After Hit», «Epoch Before Miss» et «Epoch After Miss». 
Chaque fenêtre temporelle correspondant aux essais cible-détectée et aux essais cible-manquée sont d'une durée totale de $6$ secondes, soit un nombre d'échantillons de $750$ points. 
Pour chaque mesure d'information intégrée $\Phi$ , nous avons considéré un total de $65$ fenêtres autour de la référence : $27$ fenêtres avant et $38$ fenêtres après. 
Chacune de ces $65$ fenêtres temporelles consistait en la moyenne arithmétique de $10$ points de données, pour un total de $650$ points. 

Le calcul des mesures d'information intégrée nécessite de prendre en compte un décalage temporel $\tau$ dans les données de la série temporelle multivariée $X$. 
Intégrer un tel décalage permet d'obtenir les matrices de covariances entre les états présents ($X_t$) et les états passés ($X_{t-\tau}$) du système. 
En faisant varier le décalage temporel $\tau$, on obtient ensuite les décours temporels d’intégration de l’information pour chacun des sujets pour les cibles détectées et celles non-détectées, avant et après la référence. 
Cela permet in fine l'étude de la dynamique de l'intégration d'information à l'échelle cérébrale macroscopique. 

Néanmoins, le calcul des mesures issues de la théorie de l'information intégrée sur des données réelles neurophysiologiques se heurte à deux problèmes notables. 
Ces deux problèmes sont intrinséquement liés car ils sont tous deux associés à l'analyse de graphes. 
Le premier problème concerne la taille du graphe du réseau sous-jacent à l'étude, c'est-à-dire ici le nombre d'électrodes à prendre en compte dans le calcul des mesures d'information intégrée. 
Plus le nombre d'électrodes du réseau est important, plus le graphe augmente en densité, plus le nombre de connexions à quantifier est grand, plus les calculs sous-jacents sont nombreux et finalement plus le temps de calcul augmente. 
Le second problème est la manière dont le graphe est partitionné et la méthode de partitionnement sous-jacente utilisée, qui va nécessairement présenter un impact déterminant sur les temps de calcul. 
Dans la Section \ref{partitioninfominimaletheorique}, nous avons précisé que l’information intégrée se réfère à l'information minimale irréductible au sein du système pouvant être obtenue lors du partitionnement de celui-ci. 
Cela implique la division du système afin de rechercher la partition d'information minimale (PIM) pour laquelle les éléments du système sont le moins intégrés afin de calculer la mesure d’information résiduelle provoquée par le partitionnement. 

La recherche de la PIM est une étape fondamentale et obligatoire pour le calcul des mesures d'information intégrée et peut être formulée comme un problème d'optimisation. 
La PIM est la partition qui divise un système en sous-systèmes les moins interdépendants possible. 
Ceci, de sorte que la perte d'information causée par la suppression des interactions entre les sous-systèmes soit minimisée. 
La perte d'information est quantifiée par la mesure de l'information intégrée. 
Ainsi, la PIM, $\pi^{PIM}$, est définie comme une partition où l'information intégrée est minimisée. 
Dans notre étude, la recherche de partitions d'information minimale a été orienté uniquement sur les bi-partitions. 

Les différents pré-tests réalisés au préalable nous ont contraints à utiliser une faible fréquence d'échantillonnage ($125$~Hz) et deux clusters : temporal et sagittal. 
Le cluster temporal était composé de 6 électrodes d'intérêt (FT7, T7, TP7, FT8, T8 et TP8) tandis que le cluster sagittal de 4 électrodes (Fz, FCz, Cz et CPz). 
De cette manière, les quatre mesures d'informations intégrées $\Phi^*$, $\Phi^G$, $\Phi^H$ et $\Phi^{MI}$ ont été calculées sur les données issues de 6 électrodes du cluster temporal puis sur les données issues de 4 électrodes du cluster sagittal recueillies chez 20 sujets. 

%%%%%%%%%%%%%%%%%%%%%%%%%%%%%%%%%%%%%%%%%%%%%%%%%%%%%%%%%%%%%%%%%%%%%%%%%%%%%%%
\subsection{Analyses statistiques}
\label{analysesstattii}
%%%%%%%%%%%%%%%%%%%%%%%%%%%%%%%%%%%%%%%%%%%%%%%%%%%%%%%%%%%%%%%%%%%%%%%%%%%%%%%

Toutes les statistiques ont été réalisées avec le logiciel R \citep{Rlanguage2021}. 
Les mesures d'information intégrée ont été analysées à l'aide de modèles à effets mixtes avec la bibliothèque \texttt{lme} \citep{bates2007lme4}. 
Quatre facteurs ont été utilisés : Détection (Hits/Miss), Condition (Avant/Après la détection), Fenêtre ($65$, $27$ avant, $38$ après), Sujet ($20$). 
Les facteurs expérimentaux (Détection, Condition, Fenêtre) ont été traités comme des variables à effet fixe et le facteur Sujets Id. comme effet aléatoire. 

Dans un premier temps, nous avons effectué une analyse statistique de manière à voir les effets de la détection, de la condition et de leur interaction sur les différentes mesures d'information intégrée. 
Une analyse par modélisation linéaire à effets mixtes a ainsi été réalisée selon la formule R suivante : $\Phi$ $\sim$ Détection * Condition + $1|$Sujet. 
Dans un second temps, nous avons étudié les effets de la détection, de la fenêtre temporelle et de leur interaction sur les mesures d'information intégrée. 
Une analyse statistique a été réalisée avec des modèles linéaires à effets mixtes en utilisant la formule R : $\Phi$ $\sim$ Détection * Fenêtre + $1|$Sujet. 
Le premier modèle linéaire à effets mixtes est nommé «modèle simple» tandis que le deuxième est nommé «modèle fenêtré». 
Une analyse de variance a été effectuée sur le modèle à effets mixtes afin d'évaluer les effets de chaque facteur expérimental et de leurs interactions sur la valeur de la mesure donnée. 
Dans le cas d'effets statistiques de facteurs et de leurs interactions, nous avons étudié les comparaisons appariées en utilisant les moyennes marginales estimées implémentées dans la bibliothèque R \texttt{emmeans} (voir Section~\ref{chapitre4analyses}).

%%%%%%%%%%%%%%%%%%%%%%%%%%%%%%%%%%%%%%%%%%%%%%%%%%%%%%%%%%%%%%%%%%%%%%%%%%%%%%%
\subsection{Résultats}
\label{resultatstii}
%%%%%%%%%%%%%%%%%%%%%%%%%%%%%%%%%%%%%%%%%%%%%%%%%%%%%%%%%%%%%%%%%%%%%%%%%%%%%%%

En préambule, nous présentons les résultats associés à l'une des principales caractéristiques limitantes de la TII : le temps de calcul issu de son application à des données neuronales réelles. 
Les temps de calcul nécessaires pour obtenir les différentes mesures d'information intégrée ont été extrêmement longs. 
La Table~\ref{fig:table5tempsdecalculinfointegree} présente les correspondances pour chaque sujet entre le nombre de cibles détectées et non-détectées (\textit{i.e.}, nombre d'essais Hits et nombre d'essais Miss, respectivement) et le temps de calcul en nombre d'heures nécessaire pour obtenir l'ensemble des quatre mesures d'informations intégrées $\Phi^*$, $\Phi^G$, $\Phi^H$ et $\Phi^{MI}$ pour le cluster sagittal, comprenant les électrodes d'intérêt Fz, FCz, Cz et CPz\footnote{Les temps de calcul pour le cluster temporal n'ont malheureusement pas été enregistrées et ne sont donc pas reportées ici, cependant les mêmes ordres de grandeur ont été observés qualitativement.}.
Les mesures n'ont pas pu être calculées pour le Sujet n°$14$ du fait de son nombre de cibles non-détectées trop faible, ce qui nous a forcé à le retirer des analyses. 
Nous avons également choisi de reporter le ratio nombre d'heures par essai. 
Pour les cibles détectées, on reporte une moyenne de $1.72$ heure par essai (soit $103$ min) et un écart-type de $0.33$ heure par essai (soit $19.91$ min). 
Pour les cibles non-détectées, la moyenne est de $1.03$ heure par essai (soit $61.92$ min) et l'écart-type de $0.44$ heure par essai (soit $26.97$ min). 

\begin{table}[!t]
\centering
\footnotesize
\caption[Table des temps de calcul pour les différentes mesures d'informations intégrées.]{Table de correspondances pour chaque sujet entre le nombre de cibles détectées et non-détectées et le temps de calcul en nombre d'heures des mesures d'informations intégrées nécessaire pour le cluster sagittal (Fz, FCz, Cz et CPz). Le ratio nombre d'heures par essai est également indiqué. Les mesures n'ont pas été calculées pour le Sujet n°14 puisqu'il a été retiré des analyses du à son nombre de cibles non-détectées trop faible.}
\label{fig:table5tempsdecalculinfointegree}
\begin{tabular}{|l||*{7}{c|}}\hline
\backslashbox{\textbf{Sujet Id.}}{\textbf{Catégorie/Heures}} & 
\makebox[3em]{\textbf{nb Hits}} & \makebox[4em]{\textbf{nb heures}} & 
\makebox[5em]{\textbf{heure(s)/Hit}} & \makebox[3em]{\textbf{nb Miss}} & 
\makebox[4em]{\textbf{nb heures}} & \makebox[5em]{\textbf{heure(s)/Miss}} 
\\\hline\hline
1 & 125 & 260 & 2.08 & 31 & 26 & 0.83 \\\hline
2 & 93 & 141 & 1.51 & 40 & 43 & 1.07 \\\hline
3 & 96 & 115 & 1.19 & 35 & 33 & 0.94 \\\hline
4 & 105 & 216 & 2.05 & 44 & 48 & 1.09 \\\hline
5 & 120 & 264 & 2.2 & 34 & 31 & 1.9 \\\hline
6 & 97 & 204 & 2.1 & 22 & 15 & 0.68 \\\hline
7 & 63 & 84 & 1.33 & 81 & 160 & 1.97 \\\hline
8 & 132 & 232 & 1.75 & 28 & 24 & 0.85 \\\hline
9 & 69 & 112 & 1.62 & 31 & 26 & 0.83 \\\hline
10 & 150 & 288 & 1.92 & 10 & 2 & 0.24 \\\hline
11 & 113 & 240 & 2.12 & 44 & 45 & 1.02 \\\hline
12 & 82 & 96 & 1.17 & 78 & 124 & 1.58 \\\hline
13 & 91 & 168 & 1.84 & 61 & 14 & 0.22 \\\hline
14 & 136 & - & - & 1 & - & - \\\hline
15 & 120 & 168 & 1.4 & 31 & 26 & 0.83 \\\hline
16 & 123 & 240 & 1.95 & 37 & 36 & 0.97 \\\hline
17 & 68 & 98 & 1.44 & 92 & 146 & 1.58 \\\hline
18 & 120 & 170 & 1.41 & 40 & 43 & 1.07 \\\hline
19 & 117 & 240 & 2.05 & 37 & 33 & 0.89 \\\hline
20 & 78 & 120 & 1.53 & 38 & 40 & 1.05 \\\hline
Total & 2098 & 3456 & 33.66 & 815 & 915 & 19.6 \\\hline
Moyenne & 105 & 181 & 1.72 & 40 & 48 & 0.44 \\\hline
\end{tabular}
\end{table} 

%%%%%%%%%%%%%%%%%%%%%%%%%%%%%%%%%%%%%%%%%%%%%%%%%%%%%%%%%%%%%%%%%%%%%%%%%%%%%%%
\subsubsection{Cluster Temporal}
\label{resultatstiiclustertemporal}
%%%%%%%%%%%%%%%%%%%%%%%%%%%%%%%%%%%%%%%%%%%%%%%%%%%%%%%%%%%%%%%%%%%%%%%%%%%%%%%

Nous reportons les résultats pour chaque mesure d'information intégrée obtenus à partir des électrodes du cluster temporal (FT7, T7, TP7, FT8, T8 et TP8). 
La Figure~\ref{fig:figure5phimesuresdetectionconditiontemporal} présente la distribution des valeurs des quatre mesures d'information intégrée en fonction de la détection (cibles détectées, en rouge; cibles manquées, en bleu) et de la condition (avant la détection; après la détection). 
La Table~\ref{tab:table5statsmesuresIITclustertemporal} (Haut) reporte les résultats des analyses statistiques du modèle simple réalisées sur les valeurs des mesures d'information intégrées $\Phi^{*}$, $\Phi^{G}$, $\Phi^{H}$ et $\Phi^{MI}$. 

Nous reportons ensuite les résultats des analyses des mesures d'information intégrée en fonction du décalage temporel $\tau$ pour les électrodes du cluster temporal (FT7, T7, TP7, FT8, T8 et TP8). 
La Figure~\ref{fig:figure5dynamiquemesuresphitemporal} nous présente dans la colonne de gauche les moyennes et erreurs standards des valeurs des quatre mesures d'information intégrée en fonction du décalage temporel $\tau$ et de la détection (cibles détectées, en rouge; cibles manquées, en bleu). 
La Table~\ref{tab:table5statsmesuresIITclustertemporal} (Bas) reporte les résultats des analyses du modèle fenêtré réalisées sur les valeurs des mesures d'information intégrées $\Phi^{*}$, $\Phi^{G}$, $\Phi^{H}$ et $\Phi^{MI}$. 

\begin{figure*}[!t]
\centering
\textbf{Information intégrée par décodage $\Phi^{*}$~~~~~~~~~~~~~~~~~~~~Information intégrée géométrique $\Phi^{G}$}\par\medskip
\includegraphics[width=0.49\linewidth]{Figures/illustrations/Exp_EEG/Inf_Int/Phi_Star_temporal.jpeg}
\includegraphics[width=0.49\linewidth]{Figures/illustrations/Exp_EEG/Inf_Int/Phi_Geo_temporal.jpeg}
\textbf{Information intégrée stochastique $\Phi^{H}$~~~~~~~~~~~~~~~~Information intégrée par redondance $\Phi^{MI}$}\par\medskip
\includegraphics[width=0.49\linewidth]{Figures/illustrations/Exp_EEG/Inf_Int/Phi_H_temporal.jpeg}
\includegraphics[width=0.49\linewidth]{Figures/illustrations/Exp_EEG/Inf_Int/Phi_MI_temporal.jpeg}
\caption[Mesures d'information intégrée du cluster temporal pour la détection et la condition.]{Mesures d'information intégrée représentées ici pour le cluster temporal sous forme de violin-plot en fonction de la détection (cibles détectées, en rouge; cibles manquées, en bleu) et de la condition (avant la détection; après la détection) :  $\Phi^{*}$ (Haut Gauche) ; $\Phi^{G}$ (Haut Droite) ; $\Phi^{H}$ (Bas Gauche) et $\Phi^{MI}$ (Bas Droite).}
\label{fig:figure5phimesuresdetectionconditiontemporal}
\end{figure*}

Le fait que $100$ points de données temporels soient manquants correspond à la nécessité de disposer d'un certain nombre de valeurs pour le calcul des informations intégrées. 
L'insertion d'un décalage temporel $\tau$ par incrémentation progressive dans les calculs de $\Phi$ amène donc à une réduction de cette fenêtre là où le décalage est inséré. 
Par conséquent, cela provoque la dissymétrie dans la fenêtre temporelle observée sur les figures de dynamique temporelle des $\Phi$, comme l'intervalle des abssices va de $-2.15$ à $+3$ secondes . 
Comme le calcul des mesures était considéré de façon unidirectionnelle, c'est-à-dire de «Avant» la détection à «Après» la détection, c'est la partie initiale des données de la condition «Avant» qui représente les $100$ points manquants dans les fenêtres temporelles. 

Nous représentons également pour chaque mesure d'information intégrée $\Phi$, la différence Hit-Miss calculée au cours du temps ($\Delta = \Phi_{Hit} - \Phi_{Miss}$). 
Cette différence peut être observée dans la colonne de droite de la Figure~\ref{fig:figure5dynamiquemesuresphitemporal}. 
La différence de valeurs des informations intégrées entre les cibles détectées et les cibles non-détectées renseigne sur le mécanisme temporel d'intégration de l'information associé à la prise de conscience de la cible auditive : \\

\begin{figure*}[!t]
\centering
\textbf{Informations intégrées en fonction de $\tau$ ~~~~~~~~~~~~~~~~ $\Delta = \Phi_{Hit} - \Phi_{Miss}$ en fonction de $\tau$}\par\medskip
\includegraphics[width=0.49\linewidth]{Figures/illustrations/Exp_EEG/Inf_Int/Phi_Star_windows_temporal.jpeg}
\includegraphics[width=0.49\linewidth]{Figures/illustrations/Exp_EEG/Inf_Int/Phi_Star_windows_temporal_diff_hitmiss.jpeg}
\includegraphics[width=0.49\linewidth]{Figures/illustrations/Exp_EEG/Inf_Int/Phi_Geo_windows_temporal.jpeg}
\includegraphics[width=0.49\linewidth]{Figures/illustrations/Exp_EEG/Inf_Int/Phi_Geo_windows_temporal_diff_hitmiss.jpeg}
\includegraphics[width=0.49\linewidth]{Figures/illustrations/Exp_EEG/Inf_Int/Phi_H_windows_temporal.jpeg}
\includegraphics[width=0.49\linewidth]{Figures/illustrations/Exp_EEG/Inf_Int/Phi_H_windows_temporal_diff_hitmiss.jpeg}
\includegraphics[width=0.49\linewidth]{Figures/illustrations/Exp_EEG/Inf_Int/Phi_MI_windows_temporal_statspoints.jpeg}
\includegraphics[width=0.49\linewidth]{Figures/illustrations/Exp_EEG/Inf_Int/Phi_MI_windows_temporal_diff_hitmiss.jpeg}
\caption[Évolution de la dynamique autour de la détection des mesures d'information intégrée pour le cluster temporal.]{Évolution de la dynamique autour de la détection des mesures d'information intégrée pour le cluster temporal. Dans la colonne de gauche, les mesures apparaissent en fonction de $\tau$ et de la détection dans l'ordre suivant : $\Phi^{*}$, $\Phi^{G}$, $\Phi^{H}$ et $\Phi^{MI}$. Dans la colonne de droite, sont représentés pour chaque mesure de $\Phi$, son différence Hit-Miss calculé sur le temps ($\Delta = \Phi_{Hit} - \Phi_{Miss}$). Les barres verticales rouges représentent la référence temporelle. Les points noirs montrent les fenêtres temporelles exprimant une différence statistiquement significative entre les cibles détectées et les cibles manquées.}
\label{fig:figure5dynamiquemesuresphitemporal}
\end{figure*}

\begin{itemize}
\item[$\bullet$] Pour l'information intégrée basée sur le décodage $\Phi^{*}$, la différence est positive avant la référence et nous n'observons pas de variation forte au moment de la référence. 
Cependant, on voit une diminution progressive au cours du temps de la différence avec une inversion des valeurs, la différence devenant négative. 
Ainsi, alors que $\Phi^{*}_{Hits} > \Phi^{*}_{Miss}$ avant la référence temporelle, les valeurs de $\Phi^{*}_{Miss}$ semblent globalement dépasser celles de $\Phi^{*}_{Hits}$ après la référence. 
\item[$\bullet$] Pour l'information intégrée géométrique $\Phi^{G}$, on observe une instabilité du comportement avant la référence temporelle. 
Après la référence, elle présente un comportement similaire à celui de $\Phi^{*}$, c'est-à-dire des valeurs globalement supérieures de $\Phi^{G}_{Miss}$ par rapport à celles de $\Phi^{G}_{Hits}$.
\item[$\bullet$] Pour l'information intégrée stochastique $\Phi^{H}$, la différence est négative et faible avant la référence et diminue fortement à partir de celle-ci avant de retrouver une stabilité une seconde après la référence. 
On voit ici qu'il n'y a pas de changement dans l'ordre des valeurs : $\Phi^{H}_{Miss} > \Phi^{H}_{Hits}$ avant comme après la référence temporelle. 
On remarque néanmoins, que c'est au moment de la référence que les valeurs de $\Phi^{H}_{Miss} et \Phi^{H}_{Hits}$ semblent être quasiment équivalentes. 
\item[$\bullet$] Enfin, pour l'information intégrée par redondance $\Phi^{MI}$, la différence est positive et relativement stable avant la référence pour ensuite diminuer de façon importante à partir de la référence jusqu'à afficher des valeurs négatives. 
Dans le cluster temporal, on observe donc un changement dans l'ordre des valeurs de $\Phi^{MI}$ : $\Phi^{MI}_{Hits} > \Phi^{MI}_{Miss}$ avant la référence temporelle, puis le contraire. \\
\end{itemize}

\begin{table}[!t]
\centering
\scriptsize
\caption[Table des résultats des analyses statistiques pour les mesures de la TII du cluster temporal]{Tables des résultats des analyses statistiques par modélisation linéaire à effets mixtes pour le modèle simple (tables du haut) et pour le modèle fenêtré (tables du bas) réalisées sur les valeurs des mesures d'information intégrées $\Phi^{*}$, $\Phi^{G}$, $\Phi^{H}$ et $\Phi^{MI}$ issues du cluster temporal avant et après l'appui-bouton pour les cibles détectées et avant et après le temps moyen de la détection pour les cibles non-détectées. Les p-values significatives sont accompagnées d'une étoile (*) dans la colonne correspondante.}
\label{tab:table5statsmesuresIITclustertemporal}

\textbf{Modèle Simple}

\begin{multicols}{2}

% \begin{tabular}{lllll}
\begin{tabular}{|l|*{5}{c|}}
\hline
\textbf{$\Phi^{*}$} & \textbf{NumDF} & \textbf{DenDF} & \textbf{F value} & \textbf{Pr($>$F)} \\ 
\hline
Detection & 1 & 54 & 0.51 & 0.47 \\ 
\textit{Condition} & 1 & 54 & 580.20 & 0.00 * \\ 
Detection:Condition & 1 & 54 & 0.75 & 0.39 \\ 
\hline
& E & SE & $z$ & $p$ \\
\hline
miss - hits & -0.00 & 0.00 & -0.21 & 0.83 \\ 
\hline
\textit{After - Before} & -0.03 & 0.00 & -24.46 & 0.00 * \\ 
\hline
\end{tabular}

\vspace{2.55cm}

% \begin{tabular}{lllll}
\begin{tabular}{|l|*{5}{c|}}
\hline
\textbf{$\Phi^{H}$} & \textbf{NumDF} & \textbf{DenDF} & \textbf{F value} & \textbf{Pr($>$F)} \\ 
\hline
\textit{Detection} & 1 & 54 & 4.08 & 0.04 * \\ 
\textit{Condition} & 1 & 54 & 49.39 & 0.00 * \\ 
Detection:Condition & 1 & 54 & 1.01 & 0.31 \\ 
\hline
& E & SE & $z$ & $p$ \\
\hline
miss - hits & 0.02 & 0.01 & 1.49 & 0.14 \\ 
\hline
\textit{After - Before} & -0.07 & 0.01 & -6.90 & 0.00 * \\ 
\hline
\end{tabular}

% \begin{tabular}{lllll}
\begin{tabular}{|l|*{5}{c|}}
\hline
\textbf{$\Phi^{G}$} & \textbf{NumDF} & \textbf{DenDF} & \textbf{F value} & \textbf{Pr($>$F)} \\ 
\hline
Detection & 1 & 54 & 0.03 & 0.85 \\ 
\textit{Condition} & 1 & 54 & 541.93 & 0.00 *\\ 
Detection:Condition & 1 & 54 & 0.15 & 0.69 \\ 
\hline
& E & SE & $z$ & $p$ \\
\hline
miss - hits & 0.00 & 0.00 & 0.06 & 0.96 \\ 
\hline
\textit{After - Before} & -0.03 & 0.00 & -23.88 & 0.00 * \\ 
\hline
miss.Before - hits.Before & -0.00 & 0.00 & -0.15 & 1.00 \\ 
hits.After - hits.Before & -0.03 & 0.00 & -17.20 & 0.00 \\ 
miss.After - hits.Before & -0.03 & 0.00 & -16.78 & 0.00 \\ 
hits.After - miss.Before & -0.03 & 0.00 & -17.04 & 0.00 \\ 
miss.After - miss.Before & -0.03 & 0.00 & -16.63 & 0.00 \\ 
miss.After - hits.After & 0.00 & 0.00 & 0.41 & 1.00 \\ 
\hline
\end{tabular}

\vspace{0.5cm}

% \begin{tabular}{lllll}
\begin{tabular}{|l|*{5}{c|}}
\hline
\textbf{$\Phi^{MI}$} & \textbf{NumDF} & \textbf{DenDF} & \textbf{F value} & \textbf{Pr($>$F)} \\ 
\hline
\textit{Detection} & 1 & 54 & 4.29 & 0.04 * \\ 
\textit{Condition} & 1 & 54 & 69.34 & 0.00 * \\ 
\textit{Detection:Condition} & 1 & 54 & 11.03 & 0.00 * \\ 
\hline
& E & SE & $z$ & $p$ \\
\hline
miss - hits & -0.05 & 0.03 & -1.35 & 0.18 \\ 
\hline
\textit{After - Before} & -0.18 & 0.02 & -7.55 & 0.00 * \\ 
\hline
\textit{miss.Before - hits.Before} & -0.12 & 0.03 & -3.92 & 0.00 * \\ 
\textit{hits.After - hits.Before} & -0.26 & 0.03 & -8.46 & 0.00 * \\ 
\textit{miss.After - hits.Before} & -0.23 & 0.03 & -7.55 & 0.00 * \\ 
\textit{hits.After - miss.Before} & -0.14 & 0.03 & -4.55 & 0.00 * \\ 
\textit{miss.After - miss.Before} & -0.11 & 0.03 & -3.64 & 0.00 * \\ 
miss.After - hits.After & 0.03 & 0.03 & 0.91 & 1.00 \\ 
\hline
\end{tabular} 

\end{multicols}

\vspace{1cm}

\textbf{Modèle Fenêtré}

\begin{multicols}{2}

% \begin{tabular}{lllll}
\begin{tabular}{|l|*{5}{c|}}
\hline
\textbf{$\Phi^{*}$} & \textbf{NumDF} & \textbf{DenDF} & \textbf{F value} & \textbf{Pr($>$F)} \\ 
\hline
\textit{Detection} & 1 & 2322 & 7.58 & 0.00 * \\ 
\textit{Fenetre} & 64 & 2322 & 356.27 & 0.00 * \\ 
Detection:Fenetre & 64 & 2322 & 0.46 & 0.99 \\ 
\hline
& E & SE & $z$ & $p$ \\
\hline
miss - hits & -0.00 & 0.00 & -0.86 & 0.39 \\ 
\hline
\end{tabular}

\vspace{0.5cm}

% \begin{tabular}{lllll}
\begin{tabular}{|l|*{5}{c|}}
\hline
\textbf{$\Phi^{H}$} & \textbf{NumDF} & \textbf{DenDF} & \textbf{F value} & \textbf{Pr($>$F)} \\ 
\hline
\textit{Detection} & 1 & 2322 & 187.54 & 0.00 * \\ 
\textit{Fenetre} & 64 & 2322 & 42.46 & 0.00 * \\ 
Detection:Fenetre & 64 & 2322 & 0.79 & 0.89 \\ 
\hline
& E & SE & $z$ & $p$ \\
\hline
\textit{miss - hits} & 0.02 & 0.00 & 9.50 & 0.00 * \\ 
\hline
\end{tabular}

% \begin{tabular}{lllll}
\begin{tabular}{|l|*{5}{c|}}
\hline
\textbf{$\Phi^{G}$} & \textbf{NumDF} & \textbf{DenDF} & \textbf{F value} & \textbf{Pr($>$F)} \\ 
\hline
Detection & 1 & 2322 & 1.38 & 0.24 \\ 
\textit{Fenetre} & 64 & 2322 & 331.52 & 0.00 * \\ 
Detection:Fenetre & 64 & 2322 & 0.28 & 1.00 \\ 
\hline
& E & SE & $z$ & $p$ \\
\hline
miss - hits & 0.00 & 0.00 & 0.38 & 0.71 \\ 
\hline
\end{tabular}

\vspace{0.5cm}

% \begin{tabular}{lllll}
\begin{tabular}{|l|*{5}{c|}}
\hline
\textbf{$\Phi^{MI}$} & \textbf{NumDF} & \textbf{DenDF} & \textbf{F value} & \textbf{Pr($>$F)} \\ 
\hline
\textit{Detection} & 1 & 2322 & 98.95 & 0.00 * \\ 
\textit{Fenetre} & 64 & 2322 & 55.38 & 0.00 * \\ 
\textit{Detection:Fenetre} & 64 & 2322 & 8.54 & 0.00 * \\ 
\hline
& E & SE & $z$ & $p$ \\
\hline
\textit{miss - hits} & -0.03 & 0.01 & -6.15 & 0.00 * \\ 
\hline
\end{tabular}

\end{multicols}
\end{table}

\underline{Pour l'information intégrée basée sur le décodage $\Phi^{*}$}, le pouvoir explicatif total du modèle simple est substantiel ($R^2$ conditionnel $=0,93$) et la partie liée aux seuls effets fixes ($R^2$ marginal) est de $0,55$. 
L'analyse de variance montre les effets significatifs suivants :
\begin{itemize}
\item[$\bullet$] l'effet principal de la détection n'est pas significatif ($F(1)=0.51$, $p=0.479$, $\eta^2=9.31\times10^{-3}$) ; 
\item[$\bullet$] l'effet principal de la condition est significatif et important ($F(1)=580.20$, $p<0.001$, $\eta^2=0.91$) ; 
\item[$\bullet$] l'interaction entre la détection et la condition n'est pas significative ($F(1)=0.75$, $p=0.390$, $\eta^2=0.01$).
\end{itemize}
Les tests post-hoc pour le modèle simple montrent que les valeurs de $\Phi^{*}$ diminuent significativement après la référence.
Le pouvoir explicatif total du modèle fenêtré est substantiel ($R^2$ conditionnel $=0,93$) et la partie liée aux seuls effets fixes ($R^2$ marginal) est de $0,67$. 
L'analyse de variance montre les effets significatifs suivants : 
\begin{itemize}
\item[$\bullet$] l'effet principal de la détection est significatif et très faible ($F(1)=7.58$, $p=0.006$, $\eta^2=3.25\times10^{-3}$) ; 
\item[$\bullet$] l'effet principal de la fenêtre est significatif et important ($F(64)=356.27$, $p<0.001$, $\eta^2=0.91$) ; 
\item[$\bullet$] l'interaction entre la détection et la fenêtre n'est pas significative ($F(64)=0.46$, $p>0.999$, $\eta^2=0.01$). \\
\end{itemize}

\underline{Pour l'information intégrée géométrique $\Phi^{G}$}, le pouvoir explicatif total du modèle simple est substantiel ($R^2$ conditionnel $=0,92$) et la partie liée aux seuls effets fixes ($R^2$ marginal) est de $0,58$. 
L'analyse de variance montre les effets significatifs suivants :
\begin{itemize}
\item[$\bullet$] l'effet principal de la détection n'est pas significatif ($F(1)=0.03$, $p=0.859$, $\eta^2=5.93\times10^{-4}$) ; 
\item[$\bullet$] l'effet principal de la condition est significatif et important ($F(1)=541.93$, $p<0.001$, $\eta^2=0.91$) ; 
\item[$\bullet$] l'interaction entre la détection et la condition n'est pas significative ($F(1)=0.15$, $p=0.698$, $\eta^2=2.80\times10^{-3}$). 
\end{itemize}
Les tests post-hoc pour le modèle simple montrent que les valeurs de $\Phi^{G}$ diminuent significativement après la référence. 
Le pouvoir explicatif total du modèle fenêtré est substantiel ($R^2$ conditionnel $=0,92$) et la partie liée aux seuls effets fixes ($R^2$ marginal) est de $0,69$. 
L'analyse de variance montre les effets significatifs suivants : 
\begin{itemize}
\item[$\bullet$] l'effet principal de la détection n'est pas significatif ($F(1)=1.38$, $p=0.241$, $\eta^2=5.93\times10^{-4}$) ; 
\item[$\bullet$] l'effet principal de la fenêtre est significatif et important ($F(64)=331.52$, $p<0.001$, $\eta^2=0.90$) ; 
\item[$\bullet$] l'interaction entre la détection et la fenêtre n'est pas significative ($F(64)=0.28$, $p>0.999$, $\eta^2=7.74\times10^{-3}$). \\
\end{itemize}

\underline{Pour l'information intégrée stochastique $\Phi^{H}$}, le pouvoir explicatif total du modèle simple est substantiel ($R^2$ conditionnel $=0,78$) et la partie liée aux seuls effets fixes ($R^2$ marginal) est de $0,16$. 
L'analyse de variance montre les effets significatifs suivants :
\begin{itemize}
\item[$\bullet$] l'effet principal de la détection est significatif et moyen ($F(1)=4.08$, $p=0.048$, $\eta^2=0.07$) ; 
\item[$\bullet$] l'effet principal de la condition est significatif et important ($F(1)=49.39$, $p<0.001$, $\eta^2=0.48$) ; 
\item[$\bullet$] l'interaction entre la détection et la condition n'est pas significative ($F(1)=1.01$, $p=0.319$, $\eta^2=0.02$). 
\end{itemize}
Les tests post-hoc pour le modèle simple montrent que les valeurs de $\Phi^{H}$ diminuent significativement après la référence. 
Le pouvoir explicatif total du modèle fenêtré est substantiel ($R^2$ conditionnel $=0,83$) et la partie liée aux seuls effets fixes ($R^2$ marginal) est de $0,22$. 
L'analyse de variance montre les effets significatifs suivants : 
\begin{itemize}
\item[$\bullet$] l'effet principal de la détection est significatif et moyen ($F(1)=187.54$, $p<0.001$, $\eta^2=0.07$) ; 
\item[$\bullet$] l'effet principal de la fenêtre est significatif et important ($F(64)=42.46$, $p<0.001$, $\eta^2=0.54$) ; 
\item[$\bullet$] l'interaction entre la détection et la fenêtre n'est pas significative ($F(64)=0.79$, $p=0.892$, $\eta^2=0.02$). 
\end{itemize}
Les tests post-hoc pour le modèle fenêtré montrent que les valeurs de $\Phi^{H}$ augmentent significativement pour les cibles manquées comparativement aux cibles détectées. \\

\underline{Pour l'information intégrée par redondance $\Phi^{MI}$}, le pouvoir explicatif total du modèle simple est substantiel ($R^2$ conditionnel $=0,77$) et la partie liée aux seuls effets fixes ($R^2$ marginal) est de $0,26$. 
L'analyse de variance montre les effets significatifs suivants : 
\begin{itemize}
\item[$\bullet$] l'effet principal de la détection est significatif et moyen ($F(1)=4.29$, $p=0.043$, $\eta^2=0.07$) ; 
\item[$\bullet$] l'effet principal de la condition est significatif et important ($F(1)=69.34$, $p<0.001$, $\eta^2=0.56$) ;
\item[$\bullet$] l'interaction entre la détection et la condition est significative et importante ($F(1)=11.03$, $p=0.002$, $\eta^2=0.17$).
\end{itemize}
Les tests post-hoc pour le modèle simple (Table~\ref{tab:table5statsmesuresIITclustertemporal}) montrent que : 
\begin{itemize}
\item[$\bullet$] les valeurs de $\Phi^{MI}$ diminuent significativement après la référence ; 
\item[$\bullet$] les valeurs de $\Phi^{MI}$ sont significativement plus faibles avant la référence pour les cibles manquées comparativement aux cibles détectées ;
\item[$\bullet$] les valeurs de $\Phi^{MI}$ sont significativement plus faibles après la référence pour les cibles manquées comparativement à avant la référence. 
\end{itemize}
Le pouvoir explicatif total du modèle fenêtré est substantiel ($R^2$ conditionnel $=0,82$) et la partie liée aux seuls effets fixes ($R^2$ marginal) est de $0,30$. 
L'analyse de variance montre les effets significatifs suivants : 
\begin{itemize}
\item[$\bullet$] l'effet principal de la détection est significatif et faible ($F(1)=98.95$, $p<0.001$, $\eta^2=0.04$) ; 
\item[$\bullet$] l'effet principal de la fenêtre est significatif et important ($F(64)=55.38$, $p<0.001$, $\eta^2=0.60$) ;
\item[$\bullet$] l'interaction entre la détection et la fenêtre est significative et importante ($F(64)=8.54$, $p<0.001$, $\eta^2=0.19$). 
\end{itemize}
Les tests post-hoc pour le modèle fenêtré montrent que les valeurs de $\Phi^{MI}$ diminuent significativement pour les cibles manquées comparativement aux cibles détectées. \\

%%%%%%%%%%%%%%%%%%%%%%%%%%%%%%%%%%%%%%%%%%%%%%%%%%%%%%%%%%%%%%%%%%%%%%%%%%%%%%%
\subsubsection{Cluster Sagittal}
\label{resultatstiiclustersagittal}
%%%%%%%%%%%%%%%%%%%%%%%%%%%%%%%%%%%%%%%%%%%%%%%%%%%%%%%%%%%%%%%%%%%%%%%%%%%%%%%

Nous reportons désormais les résultats de chaque mesure d'information intégrée pour les électrodes du cluster sagittal (Fz, FCz, Cz et CPz) : la distribution des valeurs sur la Figure~\ref{fig:figure5phimesuresdetectionconditionsagittal}, les résultats pour le modèle simple sur la Table~\ref{tab:table5statsmesuresIITclustersagittal}, les moyennes et erreurs standards des valeurs en fonction du décalage temporel $\tau$ sur la Figure~\ref{fig:figure5dynamiquemesuresphisagittal} et les résultats des analyses pour le modèle fenêtré sur la Table~\ref{tab:table5statsmesuresIITclustersagittal}. 

\begin{figure*}[!t]
\centering
\textbf{Information intégrée par décodage $\Phi^{*}$~~~~~~~~~~~~~~~~~~~~Information intégrée géométrique $\Phi^{G}$}\par\medskip
\includegraphics[width=0.49\linewidth]{Figures/illustrations/Exp_EEG/Inf_Int/Phi_Star_sagittal.jpeg}
\includegraphics[width=0.49\linewidth]{Figures/illustrations/Exp_EEG/Inf_Int/Phi_Geo_sagittal.jpeg}
\textbf{Information intégrée stochastique $\Phi^{H}$~~~~~~~~~~~~~~~~Information intégrée par redondance $\Phi^{MI}$}\par\medskip
\includegraphics[width=0.49\linewidth]{Figures/illustrations/Exp_EEG/Inf_Int/Phi_H_sagittal.jpeg}
\includegraphics[width=0.49\linewidth]{Figures/illustrations/Exp_EEG/Inf_Int/Phi_MI_sagittal.jpeg}
\caption[Mesures d'information intégrée du cluster sagittal pour la détection et la condition.]{Mesures d'information intégrée représentées pour le cluster sagittal sous forme de violin-plot en fonction de la détection (cibles détectées, en rouge; cibles manquées, en bleu) et de la condition (avant la détection; après la détection) : $\Phi^{*}$ (Haut Gauche) ; $\Phi^{G}$ (Haut Droite) ; $\Phi^{H}$ (Bas Gauche) et $\Phi^{MI}$ (Bas Droite).}
\label{fig:figure5phimesuresdetectionconditionsagittal}
\end{figure*}

La différence Hit-Miss calculée au cours du temps ($\Delta = \Phi_{Hit} - \Phi_{Miss}$) est représentée pour chaque mesure d'information intégrée $\Phi$ dans la colonne de droite de la Figure~\ref{fig:figure5dynamiquemesuresphisagittal}. 
Un résultat notable est que les patterns dynamiques d'informations intégrées et leurs différences Hit-Miss calculées au cours du temps semblent présenter des comportements relativement similaires entre les clusters temporal et sagittal. 
En outre, des effets de bords sont nettement visibles aux extrêmités après la référence temporelle, notamment pour l'information intégrée basée sur le décodage $\Phi^{*}$ et l'information intégrée géométrique $\Phi^{G}$ dans le cluster temporal et pour ces deux mêmes ainsi que pour l'information intégrée stochastique $\Phi^{H}$ dans le cluster sagittal : \\

\begin{figure*}[!t]
\centering
\textbf{Informations intégrées en fonction de $\tau$ ~~~~~~~~~~~~~~~~ $\Delta = \Phi_{Hit} - \Phi_{Miss}$ en fonction de $\tau$}\par\medskip
\includegraphics[width=0.49\linewidth]{Figures/illustrations/Exp_EEG/Inf_Int/Phi_Star_windows_sagittal.jpeg}
\includegraphics[width=0.49\linewidth]{Figures/illustrations/Exp_EEG/Inf_Int/Phi_Star_windows_sagittal_diff_hitmiss.jpeg}
\includegraphics[width=0.49\linewidth]{Figures/illustrations/Exp_EEG/Inf_Int/Phi_Geo_windows_sagittal.jpeg}
\includegraphics[width=0.49\linewidth]{Figures/illustrations/Exp_EEG/Inf_Int/Phi_Geo_windows_sagittal_diff_hitmiss.jpeg}
\includegraphics[width=0.49\linewidth]{Figures/illustrations/Exp_EEG/Inf_Int/Phi_H_windows_sagittal.jpeg}
\includegraphics[width=0.49\linewidth]{Figures/illustrations/Exp_EEG/Inf_Int/Phi_H_windows_sagittal_diff_hitmiss.jpeg}
\includegraphics[width=0.49\linewidth]{Figures/illustrations/Exp_EEG/Inf_Int/Phi_MI_windows_sagittal_stats.jpeg}
\includegraphics[width=0.49\linewidth]{Figures/illustrations/Exp_EEG/Inf_Int/Phi_MI_windows_sagittal_diff_hitmiss.jpeg}
\caption[Évolution de la dynamique autour de la détection des mesures d'information intégrée pour le cluster sagittal.]{Évolution de la dynamique autour de la détection des mesures d'information intégrée pour le cluster sagittal. Dans la colonne de gauche, les mesures apparaissent en fonction de $\tau$ et de la détection dans l'ordre suivant : $\Phi^{*}$, $\Phi^{G}$, $\Phi^{H}$ et $\Phi^{MI}$. Dans la colonne de droite, sont représentés pour chaque mesure de $\Phi$, son différence Hit-Miss calculé sur le temps ($\Delta = \Phi_{Hit} - \Phi_{Miss}$). Les barres verticales rouges représentent la référence temporelle. Les points noirs montrent les fenêtres temporelles exprimant une différence statistiquement significative entre les cibles détectées et les cibles manquées.}
\label{fig:figure5dynamiquemesuresphisagittal}
\end{figure*}

\begin{itemize}
\item[$\bullet$] Pour l'information intégrée basée sur le décodage $\Phi^{*}$, la différence est positive au début pendant une seconde avant de diminuer fortement pour devenir négative et globalement le rester jusqu'à la fin. 
Ainsi, pendant une seconde avant la référence $\Phi^{*}_{Hits} > \Phi^{*}_{Miss}$, puis on observe un changement dans l'ordre des valeurs et $\Phi^{*}_{Hits} < \Phi^{*}_{Miss}$ quasiment jusqu'à la fin.
\item[$\bullet$] Pour l'information intégrée géométrique $\Phi^{G}$, la différence est positive seulement au début et négative tout le reste du décours temporel. 
De cette manière, $\Phi^{*}_{Miss} > \Phi^{*}_{Hits}$ pendant quasiment tout l'intervalle temporel étudié. 
\item[$\bullet$] Pour l'information intégrée stochastique $\Phi^{H}$, la différence est positive est très variable avant la référence temporelle. 
Elle affiche une tendance à la hausse jusqu'à arriver à la référence où la différence diminue fortement avant de présenter de nouveau une forte tendance à la hausse une seconde après la référence. 
Ici également, aucun changement dans l'ordre des valeurs de $\Phi^{H}$ n'est observé et $\Phi^{H}_{Hits} > \Phi^{H}_{Miss}$ avant et après la référence temporelle. 
\item[$\bullet$] Enfin, pour l'information intégrée par redondance $\Phi^{MI}$, la différence est positive et relativement stable avant la référence temporelle. 
Elle diminue ensuite fortement à partir de la référence jusqu'à environ une seconde où elle ne varie que très peu. 
Pour le cluster sagittal, contrairement au cluster temporal, on n'observe aucun changement dans l'ordre des valeurs de $\Phi^{MI}$ : $\Phi^{MI}_{Hits} > \Phi^{MI}_{Miss}$ avant et après la référence temporelle. 
Cependant, après la référence, les valeurs de $\Phi^{MI}_{Hits}$ tendent à se rapprocher progressivement de $\Phi^{MI}_{Miss}$. \\
\end{itemize}

\begin{table}
\centering
\scriptsize
\caption[Table des résultats des analyses statistiques pour les mesures de la TII du cluster sagittal]{Tables des résultats des analyses statistiques par modélisation linéaire à effets mixtes pour le modèle simple (tables du haut) et pour le modèle fenêtré (tables du bas) réalisées sur les valeurs des mesures d'information intégrées $\Phi^{*}$, $\Phi^{G}$, $\Phi^{H}$ et $\Phi^{MI}$ issues du cluster sagittal avant et après l'appui-bouton pour les cibles détectées et avant et après le temps moyen de la détection pour les cibles non-détectées. Les p-values significatives sont accompagnées d'une étoile (*) tandis que les tendances significatives d'un point (.) dans la colonne correspondante.}
\label{tab:table5statsmesuresIITclustersagittal}

\textbf{Modèle Simple}

\begin{multicols}{2}

% \begin{tabular}{lllll}
\begin{tabular}{|l|*{5}{c|}}
\hline
\textbf{$\Phi^{*}$} & \textbf{NumDF} & \textbf{DenDF} & \textbf{F value} & \textbf{Pr($>$F)} \\ 
\hline
Detection & 1 & 54 & 0.12 & 0.72 \\ 
\textit{Condition} & 1 & 54 & 525.25 & 0.00 * \\ 
Detection:Condition & 1 & 54 & 0.21 & 0.64 \\ 
\hline
& E & SE & $z$ & $p$ \\
\hline
miss - hits & 0.00 & 0.00 & 0.11 & 0.91 \\ 
\hline
\textit{After - Before} & -0.02 & 0.00 & -23.47 & 0.00 * \\ 
\hline
\end{tabular}

\vspace{0.5cm}

% \begin{tabular}{lllll}
\begin{tabular}{|l|*{5}{c|}}
\hline
\textbf{$\Phi^{H}$} & \textbf{NumDF} & \textbf{DenDF} & \textbf{F value} & \textbf{Pr($>$F)} \\ 
\hline
Detection & 1 & 54 & 2.26 & 0.13 \\ 
\textit{Condition} & 1 & 54 & 25.89 & 0.00 * \\ 
Detection:Condition & 1 & 54 & 0.00 & 0.94 \\ 
\hline
& E & SE & $z$ & $p$ \\
\hline
miss - hits & -0.01 & 0.01 & -1.27 & 0.20 \\ 
\hline
\textit{After - Before} & -0.05 & 0.01 & -5.12 & 0.00 * \\ 
\hline
\end{tabular}

% \begin{tabular}{lllll}
\begin{tabular}{|l|*{5}{c|}}
\hline
\textbf{$\Phi^{G}$} & \textbf{NumDF} & \textbf{DenDF} & \textbf{F value} & \textbf{Pr($>$F)} \\ 
\hline
Detection & 1 & 54 & 1.14 & 0.29 \\ 
\textit{Condition} & 1 & 54 & 443.73 & 0.00 * \\ 
Detection:Condition & 1 & 54 & 0.05 & 0.82 \\ 
\hline
& E & SE & $z$ & $p$ \\
\hline
miss - hits & 0.00 & 0.00 & 0.36 & 0.72 \\ 
\hline
\textit{After - Before} & -0.02 & 0.00 & -21.41 & 0.00 * \\ 
\hline
\end{tabular}

\vspace{0.5cm}

% \begin{tabular}{lllll}
\begin{tabular}{|l|*{5}{c|}}
\hline
\textbf{$\Phi^{MI}$} & \textbf{NumDF} & \textbf{DenDF} & \textbf{F value} & \textbf{Pr($>$F)} \\ 
\hline
\textit{Detection} & 1 & 54 & 13.67 & 0.00 * \\ 
\textit{Condition} & 1 & 54 & 28.02 & 0.00 * \\ 
Detection:Condition & 1 & 54 & 1.11 & 0.2960 \\ 
\hline
& E & SE & $z$ & $p$ \\
\hline
\textit{miss - hits} & -0.07 & 0.02 & -3.06 & 0.00 * \\ 
\hline
\textit{After - Before} & -0.10 & 0.02 & -4.82 & 0.00 * \\ 
\hline
\end{tabular}

\end{multicols}

\vspace{1cm}

\textbf{Modèle Fenêtré}

\begin{multicols}{2}

% \begin{tabular}{lllll}
\begin{tabular}{|l|*{5}{c|}}
\hline
\textbf{$\Phi^{*}$} & \textbf{NumDF} & \textbf{DenDF} & \textbf{F value} & \textbf{Pr($>$F)} \\ 
\hline
Detection & 1 & 2322 & 3.39 & 0.06 . \\ 
\textit{Fenetre} & 64 & 2322 & 277.68 & 0.00 * \\ 
Detection:Fenetre & 64 & 2322 & 0.34 & 1.00 \\ 
\hline
& E & SE & $z$ & $p$ \\
\hline
miss - hits & 0.00 & 0.00 & 0.64 & 0.52 \\ 
\hline
\end{tabular}

\vspace{0.5cm} 

% \begin{tabular}{lllll}
\begin{tabular}{|l|*{5}{c|}}
\hline
\textbf{$\Phi^{H}$} & \textbf{NumDF} & \textbf{DenDF} & \textbf{F value} & \textbf{Pr($>$F)} \\ 
\hline
\textit{Detection} & 1 & 2322 & 87.86 & 0.00 * \\ 
\textit{Fenetre} & 64 & 2322 & 24.43 & 0.00 * \\ 
Detection:Fenetre & 64 & 2322 & 0.04 & 1.00 \\ 
\hline
& E & SE & $z$ & $p$ \\
\hline
\textit{miss - hits} & -0.01 & 0.00 & -7.44 & 0.00 * \\ 
\hline
\end{tabular}

\vspace{0.5cm}

% \begin{tabular}{lllll}
\begin{tabular}{|l|*{5}{c|}}
\hline
\textbf{$\Phi^{G}$} & \textbf{NumDF} & \textbf{DenDF} & \textbf{F value} & \textbf{Pr($>$F)} \\ 
\hline
\textit{Detection} & 1 & 2322 & 23.47 & 0.00 * \\ 
\textit{Fenetre} & 64 & 2322 & 223.68 & 0.00 * \\ 
Detection:Fenetre & 64 & 2322.00 & 0.39 & 1.00 \\ 
\hline
& E & SE & $z$ & $p$ \\
\hline
miss - hits & 0.00 & 0.00 & 1.86 & 0.06 . \\ 
\hline
\end{tabular}

\vspace{0.5cm}

% \begin{tabular}{lllll}
\begin{tabular}{|l|*{5}{c|}}
\hline
\textbf{$\Phi^{MI}$} & \textbf{NumDF} & \textbf{DenDF} & \textbf{F value} & \textbf{Pr($>$F)} \\ 
\hline
\textit{Detection} & 1 & 2322 & 505.32 & 0.00 * \\ 
\textit{Fenetre} & 64 & 2322 & 23.61 & 0.00 * \\ 
Detection:Fenetre & 64 & 2322 & 0.88 & 0.73 \\ 
\hline
& E & SE & $z$ & $p$ \\
\hline
\textit{miss - hits} & -0.07 & 0.00 & -17.84 & 0.00 * \\ 
\hline
\end{tabular}

\end{multicols}

\end{table}

\underline{Pour l'information intégrée basée sur le décodage $\Phi^{*}$}, le pouvoir explicatif total du modèle est substantiel ($R^2$ conditionnel $=0,92$) et la partie liée aux seuls effets fixes ($R^2$ marginal) est de $0,55$. 
L'analyse de variance montre les effets significatifs suivants : 
\begin{itemize}
\item[$\bullet$] l'effet principal de la détection n'est pas significatif ($F(1)=0.12$, $p=0.728$, $\eta^2=2.25\times10^{-3}$) ; 
\item[$\bullet$] l'effet principal de la condition est significatif et important ($F(1)=525.25$, $p<0.001$, $\eta^2=0.91$) ; 
\item[$\bullet$] l'interaction entre la détection et la condition n'est pas significative ($F(1)=0.21$, $p=0.648$, $\eta^2=3.90\times10^{-3}$).
\end{itemize}
Le pouvoir explicatif total du modèle fenêtré est substantiel ($R^2$ conditionnel $=0,91$) et la partie liée aux seuls effets fixes ($R^2$ marginal) est de $0,66$. 
L'analyse de variance montre les effets significatifs suivants : 
\begin{itemize}
\item[$\bullet$] l'effet principal de la détection n'est pas significatif ($F(1)=3.39$, $p=0.066$, $\eta^2=1.46\times10^{-3}$) ; 
\item[$\bullet$] l'effet principal de la fenêtre est significatif et important ($F(64)=277.68$, $p<0.001$, $\eta^2=0.88$) ; 
\item[$\bullet$] l'interaction entre la détection et le fenêtre n'est pas significative ($F(64)=0.34$, $p>0.999$, $\eta^2=9.22\times10^{-3}$). \\
\end{itemize}

\underline{Pour l'information intégrée géométrique $\Phi^{G}$}, le pouvoir explicatif total du modèle est substantiel ($R^2$ conditionnel $=0,91$) et la partie liée aux seuls effets fixes ($R^2$ marginal) est de $0,53$. 
L'analyse de variance montre les effets significatifs suivants : 
\begin{itemize}
\item[$\bullet$] l'effet principal de la détection n'est pas significatif ($F(1)=1.14$, $p=0.291$, $\eta^2=0.02$) ; 
\item[$\bullet$] l'effet principal de la condition est significatif et important ($F(1)=443.73$, $p<0.001$, $\eta^2=0.89$) ; 
\item[$\bullet$] l'interaction entre la détection et la condition n'est pas significative ($F(1)=0.05$, $p=0.824$, $\eta^2=9.29\times10^{-4}$).
\end{itemize}
Le pouvoir explicatif total du modèle fenêtré est substantiel ($R^2$ conditionnel $=0,89$) et la partie liée aux seuls effets fixes ($R^2$ marginal) est de $0,63$. 
L'analyse de variance montre les effets significatifs suivants : 
\begin{itemize}
\item[$\bullet$] l'effet principal de la détection est significatif et faible ($F(1)=23.47$, $p<0.001$, $\eta^2=0.01$) ; 
\item[$\bullet$] l'effet principal de la fenêtre est significatif et important ($F(64)=223.68$, $p<0.001$, $\eta^2=0.86$) ; 
\item[$\bullet$] l'interaction entre la détection et le fenêtre n'est pas significative ($F(64)=0.39$, $p>0.999$, $\eta^2=0.01$). \\
\end{itemize}

\underline{Pour l'information intégrée stochastique $\Phi^{H}$}, le pouvoir explicatif total du modèle est substantiel ($R^2$ conditionnel $=0,74$) et la partie liée aux seuls effets fixes ($R^2$ marginal) est de $0,10$. 
L'analyse de variance montre les effets significatifs suivants : 
\begin{itemize}
\item[$\bullet$] l'effet principal de la détection n'est pas significatif ($F(1)=2.26$, $p=0.139$, $\eta^2=0.04$) ; 
\item[$\bullet$] l'effet principal de la condition est significatif et important ($F(1)=25.89$, $p<0.001$, $\eta^2=0.32$) ; 
\item[$\bullet$] l'interaction entre la détection et la condition n'est pas significative ($F(1)=4.83\times10^{-3}$, $p=0.945$, $\eta^2=8.95\times10^{-5}$). 
\end{itemize}
Le pouvoir explicatif total du modèle fenêtré est substantiel ($R^2$ conditionnel $=0,80$) et la partie liée aux seuls effets fixes ($R^2$ marginal) est de $0,13$. 
L'analyse de variance montre les effets significatifs suivants : 
\begin{itemize}
\item[$\bullet$] l'effet principal de la détection est significatif et faible ($F(1)=87.86$, $p<0.001$, $\eta^2=0.04$) ; 
\item[$\bullet$] l'effet principal de la fenêtre est significatif et important ($F(64)=24.43$, $p<0.001$, $\eta^2=0.40$) ; 
\item[$\bullet$] l'interaction entre la détection et le fenêtre n'est pas significative ($F(64)=0.04$, $p>0.999$, $\eta^2=1.06\times10^{-3}$). \\
\end{itemize}

\underline{Pour l'information intégrée par redondance $\Phi^{MI}$}, le pouvoir explicatif total du modèle est substantiel ($R^2$ conditionnel $=0,76$) et la partie liée aux seuls effets fixes ($R^2$ marginal) est de $0,14$. 
L'analyse de variance montre les effets significatifs suivants : 
\begin{itemize}
\item[$\bullet$] l'effet principal de la détection est significatif et important ($F(1)=13.67$, $p<0.001$, $\eta^2=0.20$) ; 
\item[$\bullet$] l'effet principal de la condition est significatif et important ($F(1)=28.02$, $p<0.001$, $\eta^2=0.34$) ; 
\item[$\bullet$] l'interaction entre la détection et la condition n'est pas significative ($F(1)=1.11$, $p=0.296$, $\eta^2=0.02$).
\end{itemize}
Le pouvoir explicatif total du modèle fenêtré est substantiel ($R^2$ conditionnel $=0,81$) et la partie liée aux seuls effets fixes ($R^2$ marginal) est de $0,16$. 
L'analyse de variance montre les effets significatifs suivants : 
\begin{itemize}
\item[$\bullet$] l'effet principal de la détection est significatif et important ($F(1)=505.32$, $p<0.001$, $\eta^2=0.18$) ; 
\item[$\bullet$] l'effet principal de la fenêtre est significatif et important ($F(64)= 23.61$, $p<0.001$, $\eta^2=0.39$) ; 
\item[$\bullet$] l'interaction entre la détection et le fenêtre n'est pas significative ($F(64)=0.88$, $p=0.734$, $\eta^2=0.02$). \\
\end{itemize}

Globalement, les tests post-hoc pour le modèle simple (Table~\ref{tab:table5statsmesuresIITclustersagittal}) ont indiqué que les valeurs diminuent significativement pour les cibles manquées comparativement aux cibles détectées uniquement pour l'information intégrée par redondance $\Phi^{MI}$. 
Ils ont cependant montré que les quatre mesures présentaient des valeurs significativement inférieures pour la condition après la référence. 
Au contraire, pour le modèle fenêtré, les comparaisons multiples de Tukey on indiqué que, à la fois les valeurs de $\Phi^{MI}$ et de $\Phi^{H}$ diminuent significativement pour les cibles manquées comparativement aux cibles détectées.

%%%%%%%%%%%%%%%%%%%%%%%%%%%%%%%%%%%%%%%%%%%%%%%%%%%%%%%%%%%%%%%%%%%%%%%%%%%%%%%
\subsection{Synthèse et discussion pour les mesures d'information intégrée}
\label{synthesediscussionmesurestii}
%%%%%%%%%%%%%%%%%%%%%%%%%%%%%%%%%%%%%%%%%%%%%%%%%%%%%%%%%%%%%%%%%%%%%%%%%%%%%%%

Dans cette section, nous avons abordé le problème de la prise de conscience perceptive en étudiant l'effet de la perception auditive sur l'intégration de l'information à l'échelle cérébrale macroscopique dans le MI. 
Afin de caractériser la dynamique de l'activité cérébrale lors de la perception d'une cible auditive, nous nous sommes basés sur une approche théorique de la conscience en utilisant la théorie de l'information intégrée et plusieurs des mesures qu'elle propose. 
Nous avons cherché à savoir si ces mesures d’information intégrée étaient susceptibles de caractériser la perception auditive consciente d’un flux de tonalités cible sous MI et avons étudié le décours temporel de l’intégration d’information pour les cibles perçues et non-perçues, avant et après la perception, en considérant un décalage temporel nécessaire pour le calcul des mesures. 
Puisque la TII prédit que la conscience d'un système croît avec son degré d'intégration de l'information et son niveau de génération d'information intégrée \citep{oizumi2014phenomenology, tononi2012integrated, tononi2015integrated, tononi2015consciousness}, nous avons supposé que la perception d'une cible auditive dans un masqueur générerait un niveau plus élevé d'information intégrée en comparaison à son absence et qu'une hausse progressive de la quantité d'information intégrée dans le temps serait observée jusqu'à atteindre le report explicite par le sujet lors de l'appui-bouton.

Des analyses par modèles linéaires à effets mixtes ont été réalisées sur les données cérébrales EEG provenant d'un cluster temporal et d'un cluster sagittal.
Tout d'abord, une différence entre les cibles détectées et les cibles non-détectées a été trouvée sur les valeurs des mesures d'information intégrée par redondance $\Phi^{MI}$ et stochastique $\Phi^{H}$ dans le cluster temporal et sur les valeurs de $\Phi^{MI}$ seulement dans le cluster sagittal. 
Une différence significative a également été trouvée entre les conditions avant la référence et après la référence temporelle sur les valeurs des mesures $\Phi^{MI}$ et $\Phi^{H}$ dans les clusters temporal et sagittal. 
L'effet de l'interaction entre la détection et la condition était significatif uniquement pour les valeurs de $\Phi^{MI}$ dans le cluster temporal et les valeurs de $\Phi^{MI}$ étaient significativement plus élevées pour les cibles détectées comparativement aux cibles manquées dans le cluster temporal.

La perception auditive consciente a augmenté l'intégration de l'information en augmentant l'information intégrée par redondance $\Phi^{MI}$ dans le cluster temporal. 
Cela signifie que la perception auditive consciente a augmenté la redondance de l'information au niveau des signaux d'activité cérébrale associés à l'aire temporale des deux hémisphères. 
Cette augmentation de l'intégration d'information à l'échelle du cortex auditif lors de la conscience perceptive d'une cible auditive pourrait être considérée comme un argument valable pour la théorie de l'information intégrée. 
En effet, ce résultat va dans le sens des hypothèses prédictives de la TII que la perception consciente d'un contenu spécifique augmenterait les valeurs de la mesure comparativement à l'absence de perception consciente. 
Néanmoins, il existe des discussions sur la validité de cette mesure en tant que mesure d'information intégrée car elle peut être négative et, de ce fait, ne remplit pas toutes les contraintes théoriques qui définissent une mesure d'information intégrée \citep{oizumi2016measuring}. 

Cette augmentation de la redondance d'information à l'échelle locale du cortex auditif lors de la conscience perceptive d'une cible auditive pourrait également représenter un argument pour la théorie du traitement récurrent. 
La théorie du traitement récurrent suppose que l'activité précoce dans les zones sensorielles primaires correspond étroitement à la conscience phénoménale et que le traitement en boucles récurrentes permet à un stimulus d'accéder à la conscience \citep{lamme2000distinct, lamme2003visual, lamme2006towards}. 
Si un degré plus élevé de redondance de l'information est observé au sein de la zone sensorielle temporale lors de la conscience perceptive d'une cible auditive, cela pourrait correspondre à une forme de traitement récurrent dans cette zone. 
Ce résultat est donc à associer aux résultats de l'étude de \cite{giani2015detecting}, suggérant que la détection de la cible auditive serait sous-tendue par des processus récurrents dans le cortex auditif et liés à la composante ARN. 
La redondance de l'information augmentée dans les cortex auditifs est un élément supplémentaire pour indiquer que des mécanismes sous-tendant la ségrégation des flux auditifs sont liés à des changements dans les processus récurrents des cortex auditifs. 

Ensuite, la détection et la fenêtre ont eu un effet significatif sur les valeurs des mesures d'information intégrée par redondance $\Phi^{MI}$, stochastique $\Phi^{H}$ et basée sur le décodage $\Phi^{*}$ dans les deux clusters. 
Les valeurs de $\Phi^{MI}$ étaient inférieures pour les cibles manquées comparativement aux cibles détectées à la fois dans le cluster temporal et dans le cluster sagittal. 
Pour ces deux clusters, la fenêtre avait un effet significatif pour les valeurs des quatre mesures d'information intégrée. 
Par contre, l'effet de l'interaction entre la détection et la fenêtre était significatif uniquement pour l'information intégrée par redondance $\Phi^{MI}$ dans le cluster temporal. 
Pour la mesure d'information intégrée $\Phi^{MI}$, les cibles détectées ont suscité des valeurs supérieures par rapport aux cibles manquées avant la référence dans le cluster temporal. 

La représentation des différences entre cibles perçues - cibles manquées a permis d'observer une chute importante de la différence entre les valeurs des cibles perçues et les valeurs des cibles non-perçues à partir de la référence temporelle pour $\Phi^{MI}$ et $\Phi^{H}$ dans le cluster temporal et pour $\Phi^{MI}$ seulement dans le cluster sagittal. 
Cette variation nette entre la perception consciente et son absence fournit une caractérisation spécifique de la dynamique de l'accès conscient de la cible auditive. 
Nous ne nous attendions pas à observer une telle variation dans l'évolution des valeurs, et nous pensons que ce résultat vient appuyer la nécéssité d'étudier plus spécifiquement ces différentes mesures. 
Cependant, les représentations des différences laissent apparaître d'importants effets de bords, présents pour les mesures $\Phi^{*}$ et $\Phi^{G}$ dans le cluster temporal et $\Phi^{H}$, $\Phi^{*}$ et $\Phi^{G}$ dans le cluster sagittal. 
Ces effets montrent que des efforts supplémentaires doivent être réalisés pour appréhender plus rigoureusement les comportements des différentes mesures. 

En soi, ces résultats permettent d'appuyer l'hypothèse de la TII selon laquelle la perception consciente d'un contenu spécifique augmente les valeurs de la mesure d'information intégrée comparativement à l'absence de perception consciente. 
Sur la base de ces observations, la conscience perceptive de la cible auditive a provoqué une augmentation de l'intégration d'information au sein du cortex temporal par rapport à l'absence de perception. 
Cependant, au lieu d'observer comme supposé, une hausse progressive de la quantité d'information intégrée dans le temps jusqu'à atteindre le report explicite conscient, nous avons trouvé au contraire, une diminution progressive de la quantité d'information intégrée jusqu'au report. 
Une fois le report atteint, nous avons mis en évidence une nette diminution de la quantité d'information intégrée sur le temps jusqu'à dépasser les valeurs d'information intégrée des cibles non-perçues. 

Le report perceptif du sujet a ainsi donné lieu à une chute de l'information intégrée, que l'on pourrait simplement considérer comme une diminution de l'intégration de l'information à l'échelle du cortex auditif une fois la cible perçue et reportée. 
En effet, on pourrait considérer cet «arrêt» de l'intégration de l'information au sein du cortex auditif comme un possible ralentissement du décodage des informations sensorielles et de l'intégration des informations primaires, puisque le cortex auditif, à travers les aires cérébrales temporales présentent une activité qui est associée à des traitements de l'information liée au stimulus sonore. 
Cependant, cela est à mettre en adéquation avec le fait qu'un pattern similaire est observé pour les deux clusters, bien que moins prononcé dans le cluster sagittal. 
On voit que pour le cluster temporal, cette chute est fortement prononcée et les valeurs des cibles détectées deviennent inférieures à celles des cibles manquées très rapidement après le report. 
Ce dépassement n'est toutefois pas observé dans le cluster sagittal mais on y voit tout de même une diminution nette des valeurs de $\Phi^{MI}$ pour les cibles détectées juste au moment du report. 
Néanmoins, la significativité est atteinte seulement pour le cluster temporal suggérant qu'un tel ralentissement puisse être observée à l'échelle cérébrale mais que son influence locale soit spécifique aux aires temporales du fait de leur rôle dans le traitement des sons. 

De plus, il est tout à fait possible que des mécanismes de réduction de l'incertitude à l'échelle cérébrale macroscopique soient mis en évidence ici par la diminution de l'information intégrée dans les zones temporales. 
En effet, on pourrait s'attendre à ce que la conscience perceptive de la cible suivie de son report par le sujet, provoque une diminution de la réduction de l'incertitude qui est associée aux traitements récurrents dans les zones où l'information sonore est traitée. 
Une fois que la cible a été perçue et que les informations statistiques sur son contenu aient été analysées et intégrées par le système, l'opération de réduction de l'incertitude associée aux stimulations entrantes se retrouverait réduite, ce qui pourrait être reflété par une diminution des valeurs de l'information intégrée par redondance $\Phi^{MI}$ dans le cluster temporal. 
Dans le cadre de l'activité d'un réseau large-échelle de transfert et d'intégration de l'information, on pourrait penser que ces patterns similaires dans la quantité de redondance d'information entre les zones temporales et sagittales sont un signe de traitements récurrents émergents à une échelle cérébrale globale. 
L'amplification des mécanismes de récurrence de l'information au sein du cortex temporal associée à l'augmentation des flux de transfert d'information en direction du cortex pariétal peuvent apparaître comme un signe de l'activation du réseau fronto-temporo-pariétal lors de la perception consciente. 

Globalement, ces résultats semblent apporter un argument expérimental à la fois pour la théorie de l'information intégrée et pour la théorie du traitement récurrent. 
Ils permettent de considérer comment les aires cérébrales temporales sont associées à une intégration de l'information du stimulus auditif sur la base de mesures spécifiques de quantification de l'intégration de l'information. 
Le cortex auditif peut être considéré comme un centre précurseur des représentations des informations sensorielles et ainsi se retrouver engagé dans une activation plus largement étendue d'un réseau fronto-temporo-pariétal impliquée dans les mécanismes de conscience perceptive. 
En outre, un lien particulièrement intéressant est soulevé ici dans la mesure où l'intégration des informations en provenance du stimulus auditif au sein du cortex auditif pourrait être associée dans ce réseau à une accumulation de l'évidence d'information au sein des aires pariétales proches, puisque représentant un centre informationnel attracteur des informations largement diffuse à l'échelle inter-hémisphérique. 
Cette accumulation d'évidence couplée à des mécanismes de traitement par intégration de l'information par redondance pourraient ainsi consister en une caractérisation distincte de la perception auditive consciente à l'échelle cérébrale macroscopique. 
En fait, des analyses supplémentaires seraient réalisées afin de savoir si l'augmentation de la quantité d'information intégrée par redondance est associée à une latéralisation hémisphérique comme celle observée précédemment avec la représentation de l'aire pariétale postérieure droite comme un centre où converge de l'information. 

Finalement, cette analyse nous permet d'apporter un argument supplémentaire sur la nécéssité d'étudier de manière plus rigoureuse et spécifique les comportements des différentes mesures issues de la TII dans des paradigmes expérimentaux variés, et vient ainsi s'ajouter au nombre croissant d'études de caractérisation de ces mesures \citep{barrett2011practical, barrett2019phi, haun2016contents, haun2017conscious, isler2018integrated, kim2018estimating, kim2019criticality, kitazono2018efficient, mediano2019measuring, oizumi2016measuring, seth2011causal, tegmark2016improved, toker2019information}. 

%%%%%%%%%%%%%%%%%%%%%%%%%%%%%%%%%%%%%%%%%%%%%%%%%%%%%%%%%%%%%%%%%%%%%%%%%%%%%%%
\newpage
\section{Conclusion}
\label{conclusionchapitre5}
%%%%%%%%%%%%%%%%%%%%%%%%%%%%%%%%%%%%%%%%%%%%%%%%%%%%%%%%%%%%%%%%%%%%%%%%%%%%%%%

L'objectif de cette seconde étude expérimentale était de mieux caractériser i) la dynamique de la prise de conscience perceptive dans la modalité auditive et ii) les corrélats neuronaux de l'accès à la conscience, soit par des marqueurs pragmatiques de la dynamique cérébrale associée à la prise de conscience auditive ou soit grâce à des mesures développées dans le cadre d'une théorie neuroscientifique de la conscience.
Nous avons scindé notre étude en quatre analyses afin de pouvoir établir un diagnostic assez large de la caractérisation de la dynamique cérébrale associée à la perception auditive consciente. 

Dans la première analyse, nous avons étudié deux corrélats neurophysiologiques issus de la littérature: l'un associé à la détection des tonalités de la cible (ARN) et l'autre usuellement associé au traitement haut-niveau des caractéristiques du stimulus et à son intégration consciente (P300). 
Nous avons reproduit ces deux corrélats dans notre paradigme expérimental de MI en enregistrant l'activité cérébrale EEG. 
La perception auditive consciente a suscité i) une composante ARN sur trois électrodes adjacentes à la zone temporale (C5, F6 et F7) et ii) une composante P300 sur quatre électrodes sagittales (FCz, Fz, CPz et Pz) approximativement $300$~ms après la première tonalité précédant le report perceptif. 

Dans la seconde analyse, nous avons étudié les corrélats neuroinformationnels que l'on peut extraire du signal EEG au moyen de mesures d'entropie et de complexité largement employées dans la littérature sur les états de conscience. 
Nous avons trouvé que les mesures d’entropies et de complexité permettent de discriminer significativement l’effet de la perception auditive consciente dans un cluster fronto-central dans le MI.
Dans ce cluster, la perception consciente auditive a suscité une hausse de valeurs significative dans une fenêtre d'environ $400$~ms après le report perceptif du sujet uniquement pour les valeurs de l'exposant de Hurst. 
La perception auditive consciente a donc été corrélée ici à une élévation du degré de complexité associé aux signaux cérébraux issus des aires fronto-centrales après le report perceptif.

Dans la troisième analyse, nous avons étudié le transfert informationnel par dépendance linéaire (corrélation et covariance) et non-linéaire (information mutuelle et entropie de transfert) entre les signaux EEG. 
Nous avons trouvé un couplage linéaire inter-hémisphérique significativement supérieur lors de la perception auditive consciente au niveau frontal pour la covariance et au niveau fronto-pariéto-occipital pour la corrélation. 
La perception auditive consciente a également été associée à un important cluster informationnel situé au niveau de l'aire pariétale arrière droite et localisé principalement au niveau des électrodes P6 et P8. 
L'entropie de transfert a permis de montrer spécifiquement qu'un flux massif d'information en provenance de la zone hémisphérique contra-latérale convergeait vers ce centre pariétal. 

Dans la quatrième analyse, nous avons étudié l'intégration de l'information au moyen de mesures issues de la théorie de l'information intégrée de la conscience. 
La perception auditive consciente a augmenté le degré de redondance de l'information ($\Phi^{MI}$) au sein des aires auditives, augmentant ainsi le niveau d'intégration de l'information associé aux zones cérébrales temporales. 
Nous avons cependant observé une diminution de ce degré de redondance de l'information jusqu'au report perceptif des cibles détectées. 
La conscience perceptive d'un contenu auditif spécifique a ainsi augmenté les valeurs d'information intégrée à l'échelle cérébrale temporale mais c'est une diminution de ces valeurs d'information intégrée qui a été observé au fur et à mesure du décours temporel de la construction du percept. 

Cette seconde étude expérimentale nous montre ainsi plusieurs approches méthodologiques pour une caractérisation pratique de la perception auditive consciente. 
La conscience perceptive d'une cible auditive dans le MI chez l'humain peut donc être corrélée à plusieurs aspects de l'activité cérébrale. 
Ces aspects consistent à quantifier i) l'amplitude de la forme d'onde négative évoquée au niveau des aires cérébrales temporales ; ii) la complexité des signaux cérébraux associés aux aires fronto-centrales ; iii) le transfert d'information en direction des aires pariétales ; et iv) le niveau de redondance de l'information exprimé dans les aires temporales.  
Dans l'ensemble, ces différents résultats corroborent l'hypothèse de l'activation d'un réseau fronto-temporo-pariétal associée à la perception consciente chez l'être humain. 

%%%%%%%%%%%%%%%%%%%%%%%%%%%%%%%%%%%%%%%%%%%%%%%%%%%%%%%%%%%%%%%%%%%%%%%%%%%%%%%
\clearpage\null\newpage
%%%%%%%%%%%%%%%%%%%%%%%%%%%%%%%%%%%%%%%%%%%%%%%%%%%%%%%%%%%%%%%%%%%%%%%%%%%%%%%
