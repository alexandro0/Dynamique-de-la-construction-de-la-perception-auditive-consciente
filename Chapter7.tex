%%%%%%%%%%%%%%%%%%%%%%%%%%%%%%%%%%%%%%%%%%%%%%%%%%%%%%%%%%%%%%%%%%%%%%%%%%%%%%%
\chapter*{Conclusion}
\label{chapitre8}
\addcontentsline{toc}{chapter}{Conclusion}
%%%%%%%%%%%%%%%%%%%%%%%%%%%%%%%%%%%%%%%%%%%%%%%%%%%%%%%%%%%%%%%%%%%%%%%%%%%%%%%

\noindent \hrulefill \\

Ce travail de thèse est une contribution de recherche afin de mettre en place une implémentation technologique de solutions pratiques pour répondre à la problématique de l'absence de perception d'alarme sonore par le pilote dans un cockpit. 
Nous avons cherché à caractériser la perception auditive consciente à travers ses aspects électrophysiologiques macroscopiques et corrélats informationnels.
Nous avons étudié les modifications spécifiques du contenu en information et en complexité, de la transmission et de l'intégration d'information à l'échelle cérébrale EEG lors de la perception auditive consciente. 
Nous proposons quelques éléments de réponses issus de ce travail aux problématiques posées en introduction. \\

\begin{itemize}
\item[$\bullet$] \textit{\textbf{Comment alerter un humain de son absence de réaction vis-à-vis d'une alerte présentée ?}} \\
\end{itemize}

Alerter un humain de son absence de réaction vis-à-vis d'une alarme sonore consiste à rendre compte d'une absence de la détection à un niveau comportemental. 
Il est donc nécessaire de déterminer en premier lieu l'échelle temporelle de réactivité auquel on souhaite que le sujet réponde. 
Dans le cadre de situations opérationnelles issues du milieu aéronautique, il est convenable de supposer pertinente une échelle temporelle inférieure à la seconde. 
Ainsi, l'étude de la dynamique de la détection et donc de la prise de conscience du stimulus cible doit cibler les contraintes qui jouent sur la construction du percept conscient à cette échelle cible. 
Notre première étude montre que les paramètres du stimulus sont des facteurs modulables sur lesquels on peut s'appuyer pour disposer d'une augmentation de la probabilité de ségrégation des flux auditifs. 
En effet, les caractéristiques du stimulus auditif sont primordiales pour favoriser ou non la ségrégation des flux auditifs. 
De ce fait, il est nécessaire d'opter pour des signaux dont les caractéristiques sonores sont suffisamment saillantes afin de favoriser une perception cohérente des flux auditifs sans pour autant qu'elles soient «trop» saillante au point d'engendrer du stress chez le sujet. 

Nous avons étudié une situation expérimentale qui nous a permis de modéliser dans une certaine mesure la situation opérationnelle au moyen de l'analyse des caractéristiques du signal sur la dynamique de l'accès conscient. 
De cette manière, une réponse pratique que l'on peut apporter à cette problématique repose intrinséquement sur l'objectif de la tâche à accomplir. 
Dans le cadre de l'absence de réaction vis-à-vis d'une alarme sonore, la question principale est de savoir quelle est la sortie comportementale de l'humain que l'on souhaite observer. 
Si cette absence de réaction implique un risque pour la sécurité, alors forcément l'alerte explicite sur cette absence de réaction doit être donnée à l'individu, de sorte que la prise de conscience de ce dernier lui permette un comportement adéquat pour la résolution du problème en cours. 
Ainsi, la capacité à pouvoir alerter un humain de son absence de réaction est liée à un accès conscient d'une information pertinente à la tâche dans une fenêtre temporelle spécifique. 
Un premier niveau de réponse consiste à expliciter à l'humain son absence de perception en utilisant un langage qui lui est connu. 
Par exemple, si une minute après le début de présentation de l'alarme auditive dans le cockpit, aucune réponse n'est apportée par le pilote, alors le système automatisé prononcera clairement l'action à effectuer au pilote dans l'immédiat. 
Un deuxième niveau consiste en l'utilisation de systèmes technologiques d'interface homme-machine en boucle fermé pour pouvoir aborder le problème spécifiquement à l'échelle de la dynamique du signal cérébral afin d'utiliser des méthodes de neurofeedbacks. \\

\begin{itemize}
\item[$\bullet$] \textit{\textbf{Comment la dynamique du signal EEG macroscopique peut-elle caractériser l'accès conscient d'un signal auditif d'intérêt chez l'humain ? }} \\
\end{itemize}

Caractériser l'accès conscient d'une alarme sonore au moyen de la dynamique d'un signal EEG macroscopique consiste à s'intéresser aux moyens de sa caractérisation. 
Ainsi, caractériser le signal EEG lors de phénomènes de perception consciente commence d'abord par déterminer les différentes approches permettant une telle caractérisation. 
Notre travail offre un ensemble d'approches qui permettent de fournir des moyens de caractériser le signal EEG lors de la perception consciente d'un signal auditif cible. 
Nous avons abordé la situation en considérant une première approche consistant à étudier directement le signal EEG au moyen de ses caractéristiques statistiques, définissant ainsi des mesures d'information et de complexité. 
Une deuxième approche nous a permis d'étudier le signal EEG en se basant sur un modèle théorique de la conscience, visant spécifiquement à la quantifier au moyen de mesures «adéquates». 

Notre seconde étude montre que ces deux approches sont utiles dans leur manière à caractériser le signal EEG macroscopique lors de la perception d'une cible auditive et qu'elles rendent chacune à leur manière une information pertinente à apporter dans la discrimination de l'accès conscient ou de l'absence de cet accès vis-à-vis d'un signal auditif présenté. 
Ainsi, la dynamique du signal EEG macroscopique peut caractériser la conscience perceptive d'un signal auditif chez l'humain de par l'information qu'il est possible d'en extraire. 
Le signal EEG, comme signal neuronal fluctuant dans le temps, reflète la variation du potentiel électrique diffus à la surface du crâne et est sensible aux variations de stimulations de l'environnement externe. 
Cette variation de l'activité cérébrale macroscopique génère du contenu en information et en complexité sur le cerveau humain, qu'il est possible d'utiliser pour renseigner et caractériser la conscience perceptive associée à ces variations de stimulations de l'environnement. 
De cette façon, la dynamique temporelle associée à ces variations de contenus et précisément les variations propres en réaction au signal d'intérêt est à même d'être utilisée dans un cadre de système en boucle fermé dans lequel l'absence de telles variations viendrait alimenter la génération de neurofeedbacks. 
Par conséquent, une solution à l'aide au diagnostic de l'absence de réaction vis-à-vis d'une alarme auditive dans le cockpit peut être basée, dans le cadre de ces systèmes à neurofeedbacks, sur une détermination des potentiels biomarqueurs utilisables pour indiquer l'absence d'accès conscient. \\

\begin{itemize}
\item[$\bullet$] \textit{\textbf{Quel(s) biomarqueur(s) associé(s) au signal EEG serai(en)t utilisable(s) dans le cadre d'une exploitation pratique du diagnostic de la perception auditive consciente chez l'humain ?}} \\
\end{itemize}

Sélectionner un ou des biomarqueurs EEG pour une exploitation pratique du diagnostic de la perception auditive consciente consiste à en trouver un ou plusieurs qui indique(nt) de manière fiable, pertinente et robuste le phénomène. 
De tels biomarqueurs doivent donc être en mesure de rendre une information discriminante sur le fait que le sujet a ou non détecté la cible auditive. 
La conscience perceptive d'une alarme sonore représente une organisation perceptive cohérente à l'échelle des systèmes auditifs périphérique et central chez le sujet, qui est fondée sur une ségrégation aboutie des flux en provenance de la cible et de l'environnement sonore. 
Ainsi, un biomarqueur pertinent serait en mesure de pouvoir indiquer ce phénomène de ségrégation sur la dynamique du signal EEG dans le sens où il permettrait de discriminer la ségrégation de l'absence de ségrégation des différents flux auditifs. 

Notre seconde étude nous a permis de déterminer que l'exposant de Hurst, en tant que mesure de complexité du signal neuronal rapide à calculer, est à même de renseigner sur la ségrégation des flux de la cible et de l'environnement sonore puisqu'il a permis de discriminer, sur la base de l'étude de la dynamique du signal neuronal, les cibles détectées des cibles non-détectées. 
De plus, la mesure d'information intégrée par redondance a également montré son potentiel d'utilisation à discriminer la perception auditive consciente de l'absence de perception. 
La perception auditive consciente a provoqué une chute rapide des valeurs de la mesure d'information intégrée par redondance dans les cortex auditifs immédiatement après le report perceptif du sujet, suggérant l'utilisation potentielle d'une telle mesure pour indiquer et caractériser la conscience perceptive d'une cible auditive. 
Par conséquent, notre travail a permis de rendre compte d'au moins deux marqueurs du signal neuronal --- l'exposant de Hurst, comme mesure de complexité, et l'information intégrée par redondance, comme mesure d'intégration de l'information --- qui sont pertinents dans le cadre d'une utilisation pratique du diagnostic de la présence ou de l'absence de l'accès conscient d'un signal auditif. 
Une telle exploitation pratique basée sur des systèmes à boucle fermée par retour de neurofeedbacks pourrait intégrer ce genre de biomarqueurs et ainsi améliorer le diagnostic de l'absence de perception des alarmes sonores par le pilote dans un cockpit. \\

\begin{itemize}
\item[$\bullet$] \textit{\textbf{Comment la perception consciente d'un signal auditif peut-elle être liée à des variations de la dynamique des traitements informationnels à l'échelle cérébrale macroscopique ? }} \\
\end{itemize}

Finalement, le lien entre la conscience perceptive et la richesse dans les variations de traitements de l'information à l'échelle cérébrale macroscopique a pu être étudié dans ce manuscrit spécifiquement au moyen des théories de l'information et de l'information intégrée. 
Nous apportons de nouvelles preuves supplémentaires que la conscience perceptive peut être associée à une mise en activation du réseau fronto-temporo-pariétal dans le cadre du masquage informationnel chez l'être humain. 
Plus précisément, nous avons montré que la perception consciente d'une cible auditive est soutenue par d'importants transferts d'information inter-hémisphériques, notamment en direction d'une zone cérébrale pariétale postérieure. 
Les zones temporales et fronto-centrales ont également montré des activations significatives lors de la perception consciente auditive, notamment au niveau de l'amplitude des formes d'ondes dans le lobe temporal et au niveau de l'information et de la complexité associées aux signaux neuronaux issus des aires fronto-centrales latérales. 
De plus, une intégration d'information supérieure est également observable dans les cortex auditifs lorsque l'individu prend conscience d'un signal auditif comparativement à lorsqu'il n'en prend pas conscience. 
Ainsi, la conscience perceptive d'une cible auditive chez le sujet humain sain génère une multitude de traitements de l'information qui peuvent être abordés sous des prismes théoriques différents pour rendre compte de mécanismes de récurrences ou d'intégration de l'information associés à l'organisation perceptive cohérente des flux en provenance de signaux auditifs distincts. 

%%%%%%%%%%%%%%%%%%%%%%%%%%%%%%%%%%%%%%%%%%%%%%%%%%%%%%%%%%%%%%%%%%%%%%%%%%%%%%%
\clearpage\null\newpage
%%%%%%%%%%%%%%%%%%%%%%%%%%%%%%%%%%%%%%%%%%%%%%%%%%%%%%%%%%%%%%%%%%%%%%%%%%%%%%%
